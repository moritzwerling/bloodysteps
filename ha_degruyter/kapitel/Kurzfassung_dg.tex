\chapter*{Kurzfassung}
%\markboth{Kurzfassung}{Kurzfassung}
%\addcontentsline{toc}{chapter}{Kurzfassung}
Die Automatisierung des Fahrens in unterschiedlichen Ausprägungsstufen ist eines der wichtigsten Zukunftsthemen der Automobilbranche. Durch große Fortschritte in der Sensorik und neue Methoden der Maschinellen Wahrnehmung können Fahrzeuge immer genauer ihre Umgebung erfassen. Damit besteht die nächste große Herausforderung darin, die gewonnenen Informationen zu sicheren, energieeffizienten und komfortablen Fahrmanövern zu verarbeiten, die durch eine optimale Ansteuerung von Lenkung, Antrieb und Bremse robust realisiert werden müssen. Eingangs werden hierzu überblicksartig die unterschiedlichen Aufgabenstellungen und die in der Praxis vorzufindenden Randbedingungen erörtert. Dies resultiert in der Problemformulierung der sog.\ Optimalen Steuerung, zu deren Berechnung im Anschluss drei grundsätzlich unterschiedliche Wege beschrieben werden. Schließlich werden die wesentlichen Eigenschaften der Optimierungsmethoden gegenübergestellt und ergänzend Empfehlungen zu deren vorteilhaften Kombinationen gegeben. Zahlreiche Praxisbeispiele erläutern die Verfahren, wodurch parallel ein tiefer Einblick in die Herausforderungen aktueller Forschungssysteme gegeben wird.

	%Aktive Fahreingriffe verfolgen immer ein konkretes Ziel, sei es die Erhöhung der Verkehrsicherheit oder des Fahrkomforts. In den meisten Fällen lässt das Erreichen des Ziels jedoch Raum für zusätzliche Anforderungen. Die stärkste von allen besteht darin, dass sich während des Eingriffs das Fahrzeug "`optimal"' verhält. Aus mathematischer Sicht ist hierzu ein genau zu definierendes Gütekriterium erforderlich \cite{foellingeroptimal}, anhand dessen das Fahrzeugverhalten bewertet werden kann. So kann bei einem automatischen Ausweichmanöver die Erhöhung des Sicherheitsabstands zu einem Hindernis nur mit Anstieg ...
%Dises Buch richtet sich hauptsächlich an... 




% Vereinfachte Zusammenfassung Methoden: 
% Innenstadt, kombinierte Manöver, unstrukturiert aber ohne kombinatorischen Anteil-> direkte Methode
% Autobahn: strukturierte Umgebung mit elementaren Manöver -> indirekte Methoden
% Parkplatz, unstrukturierte Umgebung: kombinatorisches Problem -> dynamische Programmierung