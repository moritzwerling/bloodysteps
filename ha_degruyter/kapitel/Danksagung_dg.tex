\chapter*{Danksagung}
%\markboth{Danksagung}{Danksagung}
%\addcontentsline{toc}{chapter}{Danksagung}
Das vorliegende Buch stellt im Wesentlichen meine Habilitationsschrift "`Optimale Fahreingriffe
für Sicherheits- und Komfortsysteme"' dar, eingereicht beim Karlsruher Instituts für Technologie (KIT). Das Habilitationsvorhaben wurde am 1.\ Juni 2016 erfolgreich abgeschlossen. \\
Die Inhalte entstanden im Rahmen meiner Tätigkeit als Entwicklungsingenieur in der BMW Forschung und Technik GmbH im wissenschaftlichen Austausch mit dem KIT. Während dieser spannenden fünf Jahre wurde mir die Möglichkeit geboten, an herausfordernden Themenstellungen der Fahrerassistenz und des automatisierten Fahrens zu arbeiten. Für die dabei gewährten zeitlichen und finanziellen Freiheiten, interessante Themenstellungen eigenständig zu identifizieren, theoretisch zu durchdringen, technisch umzusetzen, praktisch zu demonstrieren und schließlich frei zu publizieren, möchte ich mich bei meinen Vorgesetzten ganz herzlich bedanken. \\
Unvergesslich macht diese Zeit jedoch erst die tägliche Zusammenarbeit mit gleichermaßen begeisterten, kompetenten und hilfsbereiten Kollegen, den Ingenieuren, Technikern, Doktoranden und Studenten. An dieser Stelle seien Dr.\ Philipp Reinisch, Benjamin Gutjahr, Michael Heidingsfeld, Udo Rietschel, Arne Purschwitz, Dr.\ Sebastian Gnatzig, Dr.\ Peter Zahn, Lawrence Louis, Yves Pilat, Dr.\ Georg Tanzmeister, Martin Friedl, Stefan Galler, Martin Medler, Andreas Lawitzky und Dr.\ Daniel Althoff genannt. \\

Auf akademischer Seite gilt mein besonderer Dank Herrn Prof.\ Dr.\ habil.\ Georg Bretthauer, ohne den es nicht zu dieser Arbeit gekommen wäre. Er war nicht nur Hauptgutachter, sondern ein ständiger Motivator, ein aufrichtiger Mentor, ein geschickter Arrangeur und ein gründlicher Korrektor. Weiter danke ich Herrn Prof.\ Dr.\ Christoph Stiller für die Übernahme des Korreferats, den regelmäßigen fachlichen Austausch und die tatkräftige Unterstützung bei meiner Vorlesung "`Verhaltensgenerierung für Fahrzeuge"' an seinem Lehrstuhl. Ebenso spreche ich meinen Dank für die Übernahme des Korreferats Herrn Prof.\ Dr.\ Dieter Ammon und Herrn Prof.\ Dr.\ Roland Kasper aus. \\
Auch möchte ich Herrn PD Dr.\ Lutz Gröll für die gemeinsame Betreuung von Studenten und Doktoranden bedanken sowie für die zurückliegende Wissensvermittlung während meiner Institutstätigkeit, die die Grundlage zu dieser Arbeit darstellt. \\
Mein Dank gilt auch Herrn Prof.\ Dr.\ Matthias Althoff für den wissenschaftlichen Austausch sowie die gründliche Durchsicht der Arbeit und seine wertvollen Anmerkungen. \\

Zu guter Letzt bedanke ich mich bei meiner Frau Anja für die vielfältige Unterstützung, vor allem für das entgegengebrachte Verständnis hinsichtlich zahlreicher Wochenenden und Urlaubstage, die in die Arbeit geflossen sind. \\

\hspace{.5cm}

\noindent München, im Juni 2016 \hfill \textit{Moritz Werling}

