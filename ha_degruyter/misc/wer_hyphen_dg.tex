
%****************************************************************************
%                                                                           *
%  Projekt: DISSERTATION                                                    *
%           Trennregeln                                                     *
%                                                                           *
%                                                                           *
%                                                                           *
%                                                                           *
%  Autor:   Hz                                                              *
%                                                                           *
%  Datum:   10.02.2004                                                      *
%                                                                           *
%****************************************************************************

\hyphenation{
 al-ge-bra-isch
 al-ge-bra-isch-en
 As-sis-tenz-sys-tem
 As-sis-tenz-funk-tion 
 Aus-weich-tra-jek-torie
 Au-to-ma-ti-sie-rungs-tech-nik
 asymp-to-tisch
 Kom-fort-sys-te-men
 Kom-fort-sys-te-me
 Au-to-ma-ti-sie-rungs-tech-nik
 Abs-zis-sen-wer-te
 au�en
 frame-work
 -be-schleu-ni-gung
 Ab-tast-ras-ter
 Ab-tast-ras-ters
 An-iso-tro-pie
 an-iso-trop
 Auf-l\"{o}-sung
 Be-we-gungs-kos-ten
 Be-zugs-ebe-ne
 Bild-erfas-sung
 Bild-serie
 Diplom-arbeit
 Dif-feren-tia-tion
 Dif-fe-ren-tial-glei-chungs-sys-tem
 E/A-Li-neari-sierung
 Ein-zel-en-tro-pien
 Ele-va-tion
 Fah-rer-as-sis-tenz-sys-tem
 Fah-rer-as-sis-tenz-sys-te-me
 Feh-ler-ma�e
 Feh-ler-ma�es
 gleicher-ma�en
 Ge-fah-ren-brems-sys-te-me
 His-to-gramm
 His-to-gram-me
 His-to-gramms
 His-to-gram-men
 Identi-fi-ka-tions-systems
 IEEE
 In-te-gra-tor-er-wei-ter-ung
 Ein-park-as-sis-tent
 Fah-rer-as-sis-tenz
 Fahr-zeug-ab-stand
 Fahr-zeug-um-ris-se
 Fahr-zeug-um-ris-ses
 %Ent-wick-lungs-in-ge-nieure
 Kon-zen-tra-ti-ons-f�-hig-keit
 Kons-tel-la-ti-on
 Kons-tel-la-ti-onen
 Kos-ten
 Kos-ten-ab-sch�tz-ung 
 Kurs-winkel-�n-derungs-rate
 Leistungs-f�-hig-keit
 maxi-mieren
 Me-tall-ober-fl\"{a}-che
 pa-ral-lel
 pa-ral-lelen
 Pla-teau
 Pla-teaus
 Pro-fil-da-ten
 po-ten-tiel-le
 Pro-jek-tion
 Punkt-kos-ten
 Re-flektanz-eigen-schaf-ten
 Rie-fen-be-reich
 Schwer-punkts-tra-jek-to-rie
 Sich-er-heits-sys-te-me
 Sich-er-heits-sys-te-men 
 Sig-nal
 Sig-nals
 Sig-na-le
 Sig-na-len
 Sig-nal-mo-dell
 Sig-nal-mo-dells
 Sig-nal-mo-del-le
 Sig-na-tur
 Sig-na-turen
 Sig-na-tur-ver-gleich
 Sig-na-tur-ver-glei-che
 Soft-ware-tech-nik-en
 St\"{o}r-ein-fl\"{u}s-se
 Sta-bi-li-sierungs-stra-tegien
 Stell-gr��en-spr�nge
 Stell-auf-wand
 Sys-tem-aus-gang
 Sys-tem-ein-gang
 Sys-tem-ein-gangs
 Sys-tem-ei-gen-schaf-ten
 Sys-tem-dy-na-mik
 Sys-tem
 Sys-te-me
 Sys-tems
 Sys-tem-ak-ti-vie-rung
 Sys-tem-zu-stand
 Sys-tem-zu-st�n-de
 Stra�en-ver-kehr
 Stra�en-rand
 Stra�e
 system-ein-gangs-nahe
 System-ord-nung
 System-ein-gang
 Low-level-Sta-bi-li-sierungs-stra-tegien
 Teil-sys-tem
 Tex-tur-in-homo-ge-ni-t\"{a}-ten
 tex-tur-orien-tiert
 tex-tur-orien-tier-te
 tex-tur-orien-tier-ten
 tex-tur-orien-tier-tes
 Tra-jek-torien-be-rech-nung
 Tra-jek-torien-Op-ti-mie-rungs-auf-gab-en
 Tracking-re-gel-ung
 �ber-f�h-rungs-kos-ten
 Ver-gleichs-pro-ze-dur
 Low-level-Re-ge-lungs-stra-tegien
 Kol-lisions-frei-heit
 Ro-tations-in-varianz
 Zu-stands-tra-jek-torien
 Pr�-dik-tions-mo-dells
 Funk-tions-wei-se
 Sys-tem
 Sys-tem-sta-bi-li-t�t
 Ge-schwin-dig-keits-mess-sys-te-me
 Sys-tem-aus-fall
 sys-te-ma-tisch
 R�ck-fall-ebene
 Schwing-ung
 Re-ge-lung-en
 Vor-fahrts-re-ge-lung-en
 Op-ti-mal-steu-er-ung
 Stell-glied-an-zahl
 Sen-sor-in-for-ma-ti-on-en
 Fahr-er-as-sis-tenz-Li-te-ra-tur
 Grund-idee
 Ein-schr�n-kung-en
 Kr�m-mungs-soll-vor-gabe
 Fu�-g�ng-er-be-we-gung 
 Kos-ten-funk-tio-nal
 Vor-w�rts-be-we-gung 
 }

% Optimierungsvariablen und "~methoden.