
%\chapter*{Symbolverzeichnis}
%\markboth{Symbolverzeichnis}{Symbolverzeichnis}
%\addcontentsline{toc}{chapter}{Symbolverzeichnis}
\chapter*{Abkürzungen und Notationen}
\markboth{}{}%{Abkürzungen und Notationen}{Abkürzungen und Notationen}
\addcontentsline{toc}{chapter}{Abkürzungen und Notationen}

% \arraystretch kann man nach belieben anpassen.
% Am besten am Dateiende wieder auf Normalma{\ss} zur\"{u}ck \"{a}ndern.
\renewcommand{\arraystretch}{1.18}
\setlength{\tabcolsep}{0cm}

% ******************************************************************************************
\section*{Abk\"{u}rzungen}
\vspace{-4mm}
\begin{longtable}{p{26mm}p{89mm}}
ABS & Antiblockiersystem \\
ACC & Adaptive Cruise Control \\
ASIL & Automotive Safety Integrity Levels \\
CAN & Controller Area Network \\
ASR & Antischlupfregelung \\
CLF & Control-Lyapunov-Funktion \\
C2X & Car-to-X (Fahrzeug-Kommunikation) \\
DARPA & Defense Advanced Research Projects Agency \\
EHB & elektro-hydraulische Bremse \\
ESP & Elektronisches Stabilitätsprogramm \\
EPS & electric power steering (elektrische Servolenkung) \\
FAS	& Fahrerassistenzsysteme \\
GPS & Global Positioning System \\
ICS & Inevitable Collision States \\
KD  & Kickdown \\
LQ  & linear quadratisch \\
LQR & Linear Quadratic Regulator \\
MEM & mikro-elektro-mechanisch \\
MPC & model predictive control (modellprädiktive Regelung) \\
PID & proportional-integral-derivative (Regler) \\
THZ & Tandemhauptbremszylinder \\
TTB & time-to-brake \\
TTC & time-to-collision \\
TTR & time-to-react\\
TTS & time-to-steer \\
SQP & sequentielles quadratisches Programm \\
\end{longtable}

\vspace{1cm}

% ******************************************************************************************
\section*{Notationsvereinbarungen}
\vspace{-4mm}
\begin{longtable}{p{26mm}p{89mm}}
Skalare         & nicht fett, kursiv: $a$, $b$, $c$, $A$, $B$, $C$ \ldots\\
Vektoren        & klein, fett, kursiv: $\vektor{a}$, $\vektor{b}$, $\vektor{c}$\ldots\\
Matrizen        & groß, fett, kursiv, $\mmatrix{A}$, $\mmatrix{B}$, $\mmatrix{C}$, \ldots\\
Mengen          & kalligraphisch, gro{\ss}: $\mathcal{A}$, $\mathcal{B}$, $\mathcal{C}$, \ldots\\
%Konstanten, Bezeichner & nicht kursiv: $\textrm{a}$, $\textrm{b}$, $\textrm{c}$, \ldots
$()^\ast$ 	& Optimale Lösung für die Variable\\
$\bar{()}$ 	& interne Optimierungsvariable\\
$()_d$				& Variable zur Beschreibung des Sollwerts \\
%$()_t, ()_n$	& tangentiale \bzw normale Komponente \\
$\dot{()}$	& Zeitableitung \\
${()'}$			& Wegableitung \\
$:=$        & Definition \\
${\rm argmax}$ &  Argument mit dem maximalen Wert \\
${\rm argmin}$ &  Argument mit dem minimalen Wert \\
$\max$ & Maximum \\
$\min$ & Minimum \\
\end{longtable}

%\section*{Allgemeine Symbole}
%\vspace{-4mm}
%\begin{longtable}{p{26mm}p{89mm}}
%$:=$ & Definition \\
%$\det(\cdot)$ & Determinante \\
%$\|\cdot\|$	& Euklidische Norm \\
%$\vektor{a}^\T, \mmatrix{A}^\T$		& Transponierte des Vektors $\vektor{a}$ \bzw der Matrix $\mmatrix{A}$ \\
%$y$ 			& Systemausgang \\
%$u$				& Stellgröße \\
%$l$ 				& Abstand der Fahrzeugachsen \\
%$l_h, l_v$ 	& Abstand der hinteren und vorderen Fahrzeugachse zum Schwerpunkt \\
%$m$					& Fahrzeugmasse \\
%$\varsigma$	& Sollfahrtrichtung \\
%$\psi$			& Fahrzeugorientierung \\
%$r$					& Gierrate \\
%$\theta$		& Kurswinkel \\
%$\kappa$		& Krümmung \\
%$\kappa'$		& bogenbezogene Krümmungsänderung \\
%$v$					& Geschwindigkeit \\
%$a$					& Beschleunigung \\
%$\beta$		& Schwimmwinkel \\
%$\vektor{n}_x, \vektor{t}_x$ & durch $x$ spezifizierter Normal- und Tangentialvektor \\
%$\mathcal{T}_x$ & durch $x$ spezifizierte Trajektorie
%\end{longtable}

%\section*{Symbole im zweiten Kapitel} % allgemein stab
%\vspace{-4mm}
%\begin{longtable}{p{26mm}p{89mm}}
%$J$					& Kostenfunktional \\
%$\vektor{x}$					& Zustandsvektor \\
%$s$					& zurückgelegte Wegstrecke \\
%$\gamma$		& Kurvenparameter \\
%$\Gamma$		& Sollkurve \\
%
%\end{longtable}
%\newpage
%
%\section*{Symbole im dritten Kapitel} % TP
%\vspace{-4mm}
%\begin{longtable}{p{26mm}p{89mm}}
%$\Gamma$ & Referenzkurve \\
%$\vektor{x}$  & geplante Trajektorie \\
%$c_0, \ldots, c_5$ & Polynomkoeffizienten \\
%$s$					& zurückgelegte Wegstrecke des Fußpunkts \\
%$s_\text{ref}$ & Referenztrajektorie von $s$ \\
%$d$					& Euklidischer Abstand zur Referenzkurve \\
%$d_\text{ref}$ & Referenztrajektorie von $d$ \\
%$J_x$		& durch $x$ spezifiziertes Kostenfunktional \\
%$f_0, h_0$ & Energiefunktion und Endkosten \\
%$\vektor{z}$ 	& Zielmannigfaltigkeit \\
%$\tau$ & Ankunftszeitpunkt auf Zielmannigfaltigkeit \\
%$k_x$	& durch $x$ spezifizierter Wichtungsfaktor \\
%%$k_\tau, k_\sigma, k_\nu, k_s, k_{\xi_1}, k_s$ & Wichtungsfaktoren \\
%$\xi_{\text{ref}}$ & Referenztrajektorie von $\xi$ \\
%$\delta_i, \sigma_i, \nu_i$ & Endabstände zur Referenztrajektorie\\
%$\smallDelta t$ & Anstiegsdauer des Sicherheitsabstands \\
%
%
%\end{longtable}
%
%\section*{Symbole im vierten Kapitel} % Regelung
%\vspace{-4mm}
%\begin{longtable}{p{26mm}p{89mm}}
%$s$					& zurückgelegte Wegstrecke \\
%$q, Q$			& generalisierte Koordinate und Kraft \\
%$\xi_1$			& fiktive Regelstellgröße \\
%$e_{x}$			& Trackingfehler \bzgl $x$ \\
%$x_1, x_2$ 	& Koordinaten des Referenzpunkts \\
%%$v$					& absolute Geschwindigkeit des Referenzpunkts \\
%$\lambda$		& Vorausschaulänge \bzgl Referenzpunkt \\
%$\delta, \delta_L$		& Lenkwinkel und Lenkradwinkel \\
%$\gamma$		& Verteilungsverhältnis der Antriebskraft \\
%$u_1, u_2, u_{\dot \delta}$	& reale Systemeingänge \\
%$w_1, w_2$	& virtuelle Systemeingänge \\
%$L$ & Lagrangefunktion \\
%$T$ & kinematische Energie \\
%$\mmatrix{R}$ & Rotationsmatrix \\
%$\varsigma$ & Sollfahrtrichtung \\
%$\kappa_\delta$ & fiktiver Systemeingang \\
%$[\mathcal{L}_f\vektor{g}](\vektor{x})$ & erste Lie-Ableitung von $\vektor{g}(\vektor{x})$ \bzgl $\vektor{f}(\vektor{x})$  \\
%$[\mathcal{L}^2_f\vektor{g}](\vektor{x})$ & zweite Lie-Ableitung von $\vektor{g}(\vektor{x})$ \bzgl $\vektor{f}(\vektor{x})$  \\
%$\tilde{\vektor{z}}_t, \tilde{\vektor{z}}_n$ & Zustände der exakt e/a-linearisierten Teilsysteme \\
%$\zeta$	& Dämpfungsgrad \\
%$ c_v, c_h$ & Seitensteifigkeit der Vorder- und Hinterachse \\
%$v_s$ & Geschwindigkeitsschwellwert für Reglerumschaltung \\
%
%$A, B, C$ & Parameter der Reifen-Fahrbahn-Paarung \\
%$\alpha$ 	& Reifenschräglaufwinkel \\
%$\rho$			& relativer Grad \\
%$\vektor{s}$			& vektorieller Reifenschlupf \\
%$\vektor{F}$			& vektorielle Reifenkraft \\
%$F_x$			& durch $x$ spezifizierte Reifenkraftkomponente \\
%$c_x$			& durch $x$ spezifizierter Schwellwertparameter \\
%$a_x$			& durch $x$ spezifizierte Beschleunigungskomponente \\
%$\vektor{x}$				& Zustandsvektor \\
%%$\vektor{f}$				& Zeitliche Änderung des Zustandsvektors \\
%$k_1, k_2, k_3, k_4$ & Reglerparameter beim kinematischen Einspurmodell \\
%$k_{t0}, k_{t1}, k_{n0}, k_{n1}$ & Reglerparameter beim dynamischen Einspurmodell \\
%$V, V_c$	& Lyapunov-Funktionskandidaten \\
%$J$ 				& Fahrzeugdrehträgheit \\
%
%\end{longtable}
%
%\section*{Symbole im fünften Kapitel} 
%\vspace{-4mm}
%\begin{longtable}{p{26mm}p{89mm}}
%$F_M$ & gangnormiertes Motorkennfeld \\
%$i$ & aktueller Gang \\
%$k$ & gangabhängige Antriebsübersetzung \\
%$F_W$ & Fahrwiderstand \\
%$b$	& Bremskonstante \\
%$p_{\text{Bremse}}$ & Bremsdruck \\
%$\phi_{\text{Gas}}$ & Drosselklappenstellung \\
%\end{longtable}
%
%
%\renewcommand{\arraystretch}{1}
%\setlength\tabcolsep{6\lineskip}
