%****************************************************************************
%                                                                           *
%  Projekt: DISSERTATION                                                    *
%           Befehle, Makros etc.                                            *
%                                                                           *
%                                                                           *
%                                                                           *
%                                                                           *
%  Autor:   Werling                                                         *
%                                                                           *
%  Datum:   19.09.2008                                                      *
%                                                                           *
%****************************************************************************

\newcommand{\dhe}{d.\,h.\ }
\newcommand{\uU}{u.\,U.\ }
\newcommand{\Dhe}{D.\,h.\ }
\newcommand{\eV}{e.\,V.\ }
\newcommand{\zB}{z.\,B.\ }
\newcommand{\s}{s.\ }
\newcommand{\oae}{o.\,\"{a}.\ }
%\newcommand{\uU}{u.\,U.\ }
\newcommand{\ua}{u.\,a.\ }
%\newcommand{\ia}{i.\,a.\ }
%\newcommand{\iA}{i.\,A.\ }
\newcommand{\iA}{i.\,Allg.\ }
%\newcommand{\og}{o.\,g.\ }
\newcommand{\bzw}{bzw.\ }
\newcommand{\vgl}{vgl.\ }
\newcommand{\etc}{etc.\ }
\newcommand{\evtl}{evtl.\ }
\newcommand{\ca}{ca.\ }
\newcommand{\sog}{sog.\ }
\newcommand{\ggf}{ggf.\ }
\newcommand{\insbes}{insbes.\ }
\newcommand{\engl}[1]{engl. \textit{#1}}
\newcommand{\mindex}[1]{#1\index{#1}}

\newcommand{\bzgl}{bzgl.\ }
\newcommand{\zzgl}{zzgl.\ }
\newcommand{\hk}[1]{\glqq #1\grqq}
%\newcommand{\EALin}{{E/A-Linearisierung}}
% Englisch
\newcommand{\wrt}{w.r.\,t.\ }
\newcommand{\eg}{e.\,g.\ }



% Einheiten
%\newcommand{\nm}{\,\text{nm}}
%\newcommand{\mum}{\,\mu\text{m}}
%\newcommand{\mm}{\,\text{mm}}
%\newcommand{\ms}{\,\text{ms}}
%\newcommand{\s}{\,\text{s}}
%\newcommand{\mn}{\,\text{min}}
%\newcommand{\m}{\,\text{m}}


\DeclareMathOperator{\falt}{\ast} %
\DeclareMathOperator{\falttwod}{\ast\ast} %
\newcommand{\korr}{\circledast} %
\newcommand{\korrtwod}{\circledast\circledast} %
\DeclareMathOperator{\definition}{\;\raisebox{0.07ex}{\ensuremath{:}}\!\!=\;} %
\newcommand{\vektor}[1]{\boldsymbol{#1}}
\newcommand{\mmatrix}[1]{\boldsymbol{#1}}
\newcommand{\bs}[1]{\boldsymbol{#1}}
\newcommand{\grad}{\ensuremath{^\circ}}
\newcommand{\Bilin}[1]{ \text{Bilin}\!\left\{#1\right\} }
\newcommand{\supp}[1]{ \text{supp}\!\left\{#1\right\} }
\newcommand{\rect}[1]{ \text{rect}\!\left(#1\right) }
\newcommand{\norm}[1]{\left\lVert#1\right\rVert}
\newcommand{\betrag}[1]{\left|#1\right|}
\newcommand{\transp}[1]{#1^\textrm{T}}
\newcommand{\menge}[1]{\mathcal{#1}}
\newcommand{\operator}[1]{\mathscr{#1}}
%\newcommand{\eqref}[1]{(\ref{#1})}
\newcommand{\glref}[1]{\eqref{#1}}  %eigentlich \"{u}berfl\"{u}ssig...
\newcommand{\var}{\textrm{var}}
\newcommand{\const}{\textrm{const.}}
\newcommand{\kursiv}[1]{\emph{#1}}
%\newcommand{\sollsein}{\mbox{$\;\stackrel{!}{=}\;$}}
\DeclareMathOperator{\sollsein}{\;\stackrel{!}{=}\;} %
\newcommand{\phase}[1]{\angle\!\left\{#1\right\}}
\newcommand{\fourier}[1]{\operator{F}\!\left\{#1\right\}}
\newcommand{\fuerdiegilt}{\;|\;}
\newcommand{\fueralle}{\forall\;}
\newcommand{\median}[1]{\underset{#1}{\text{Median}}}

\newcommand{\EXP}[1]{\exp\!\left(#1\right)}
\newcommand{\GAMMA}[1]{\Gamma\!\left(#1\right)}
\newcommand{\EWO}[1]{\textrm{E}\!\left\{#1\right\}}
\newcommand{\J}{\textrm{j}}
\newcommand{\argmax}[2]{\arg \max_{#1}\!\left\{#2\right\}}
\newcommand{\MIN}[2]{\min_{#1}\!\left\{#2\right\}}
\newcommand{\MAX}[2]{\max_{#1}\!\left\{#2\right\}}
\newcommand{\mittelwert}[1]{\overline{#1}}

\newcommand{\smallDelta}{{\scriptstyle {\Delta}}}

%\newtheorem{note}{Notiz f�r mich:}
\newtheorem{todo}{Todo:}


% ----------- Vereinbarungen ----------------
% Satzzeichen nach Formel:  \;.
% Integrale:                \int_{i=0}^{\infty} f(x) \; \textrm{d} x
% modulo:                   a \bmod b
% Bilder:                   114,5 breit (\textwidth=115mm)
% Koordinatensysteme:       \{ \vektor{e}_\xi, \vektor{e}_\eta \}
% Kommazahlen:              1{,}5   (beseitigt zus\"{a}tzlichen Zwischenraum nach dem ,)
% Gedankenstrich:           ---
% $2\frac{1}{2}$D-Informationen
% Beschriftungen in Bildern klein: "gemitteltes Profil"
% subfigure-Beschriftungen gro{\ss}:  (a)~Reine Translation
% cases:
%   \rect{x} = \begin{cases} %
%   1 \;, & x \in \left[-\frac{1}{2},\frac{1}{2}\right] \;, \\
%   0 \;, & \text{sonst} \;.
%   \end{cases}

% weitergehend
% im Folgenden
% im Wesentlichen
% zum einen - zum anderen
% 4--fach, $i$--tes Objekt $180\grad$--periodisch
\newcommand{\lat}{\text{lat}}
\newcommand{\lon}{\text{lon}}
\newcommand{\vl}{v_\lambda}
\newcommand{\betal}{\beta_\lambda}
\newcommand{\dl}{d_\lambda}
%\newcommand{\dpsi}{\Delta \psi} Veraltet!
\newcommand{\dtheta}{\,{\scriptstyle \!\Delta}\theta}
\newcommand{\dotdtheta}{{\scriptstyle \Delta} \dot\theta}
\newcommand{\ddotdtheta}{{\scriptstyle \Delta} \ddot\theta}

\newcommand  {\diff}[2]{\frac{{\rm d}#1}{{\rm d}#2}}
\newcommand {\ddiff}[2]{\frac{{\rm d}^2#1}{{\rm d}#2^2}}
\newcommand{\dddiff}[2]{\frac{{\rm d}^3#1}{{\rm d}#2^3}}
\newcommand{\entsp}{\mathrel{\widehat{=}}}

\newcommand  {\pardiff}[2]{\frac{\partial #1}{\partial#2}}


\newcommand{\sRA}{s_{_{\text {R\!A}}}}
\newcommand{\dRA}{d_{_{\text {R\!A}}}}
\newcommand{\vRA}{v_{_{\text {R\!A}}}}
\newcommand{\rRA}{r_{_{\text {R\!A}}}}
\newcommand{\kappaRA}{\kappa_{_{\text {R\!A}}}}
\newcommand{\CoG}{\text{CoG}}
\newcommand{\SP}{\text{SP}}

\newcommand{\thetad}{{\theta_{\!d}}}

% Ableitungsoperatoren
\newcommand  {\Prime}[1]{\diff{}  {\sigma}{\left(#1\right)}}
\newcommand {\PPrime}[1]{\ddiff{} {\sigma}{\left(#1\right)}}
\newcommand{\PPPrime}[1]{\dddiff{}{\sigma}{\left(#1\right)}}

\renewcommand{\Dot}[1] {\diff{}{t}\left(#1\right)}
\newcommand{\DDot}[1] {\ddiff{}{t}\left(#1\right)}
\newcommand{\DDDot}[1]{\dddiff{}{t}\left(#1\right)}

\newcommand{\ued}{u_{1,d}}

\newcommand{\scp}[2]{\langle#1,#2\rangle}

\newcommand{\vect}{{\vec t}}
\newcommand{\vecn}{{\vec n}}
\newcommand{\vecy}{{\vec y}}

% Minimierung
\newcommand{\argmin}[1]{\begin{array}[t]{c}{\rm argmin}\\ [-1.25ex] \scriptstyle { #1 } \\ \end{array}}

\newcommand{\sgn}[1]{{\,\rm sgn(#1)\,}}
\renewcommand{\vector}[1]{\left(\begin{array}{c}#1\end{array}\right)}

\newcommand{\vecarray}[1]{\left[\begin{array}{c} #1  \end{array}\right]}
\newcommand{\mtrx}[2]{\left[\begin{array}{#1} #2  \end{array}\right]}

\newcommand{\cog}{\text{\tiny \!CoG}}

\newtheorem{theorem}{Theorem}
\newtheorem{satz}{Satz}[chapter]
%\newtheorem{mydef}{Definition}
\newtheorem{proposition}{Proposition}
\newtheorem{lemma}{Lemma}
\newtheorem{remark}{Remark}
\newtheorem{beweis}{Beweis}

\newcommand{\vecx}{{\vec{x}}}
\newcommand{\vecr}{{\vec{r}}}
\newcommand{\veca}{{\vec{a}}}
\newcommand{\vecv}{{\vec{v}}}

\newcommand{\opt}{{\text{ \rm opt}}}


    % Debuginformationen
\usepackage[normalem]{ulem}
\newcommand{\TODO}[1]{{\color{red} [\,\textbf{#1}\,]}}
\newcommand{\stich}[1]{{\color{blue} \textit{#1}}}
\newcommand{\TODOlist}[1]{{\color{red}\textbf{\begin{enumerate}#1\end{enumerate}}}}
\newcommand{\expl}[1]{{\color{OliveGreen}\textbf{[Hinweis]}\\ #1 }}
\newcommand{\notes}[1]{}

\newcommand{\T}{{{\rm T}}}

%\newcommand{\myplus}{
%\begin{minipage}{10mm}
 %\begin{center}
 	%\includegraphics[width = 0.3cm]{plus}
 	%\end{center}
 %\end{minipage}
%}
%\newcommand{\myminus}{
%\begin{minipage}{10mm}
 %\begin{center}
 	%\includegraphics[width = 0.3cm]{minus}
 	%\end{center}
 %\end{minipage}
%}

\newcommand{\ND}{^0}
\newcommand{\xuND}{\vektor{x}\ND\!\!\!,\,\vektor{u}\ND}
\newcommand{\Traj}{\mathcal{T}}


\theoremstyle{remark}
\newtheorem{bemerkung}{Bemerkung}
\newtheorem{modell}{Modell}

\theoremstyle{definition}
\newtheorem{mydef}{Definition}[chapter]

\newcommand{\negspace}{\!\!\!\!\!\!}
\newcommand{\bldx}{\boldsymbol x}
\newcommand{\Lie}{\mathcal{L}}

\newcommand{\zitat}[2]{
\begin{flushright}
``\emph{#1}'' \\
\vspace{.3cm}
#2
\end{flushright}
}

\newcommand{\abb}[1]{Abb.\,\ref{#1}}
\newcommand{\abschn}[1]{Abschn.\,\ref{#1}}
\newcommand{\kap}[1]{Kap.\,\ref{#1}}