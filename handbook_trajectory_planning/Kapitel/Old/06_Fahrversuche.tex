\section{Fahrversuche}
Zur Demonstration der Leistungsfähigkeit der vorgestellten Regelungsstruktur werden im Folgenden verschiedene,  repräsentative Fahrmanöver vorgestellt.  Die Umsetzung erfolgt auf einem Fahrzeug der BMW 3er Reihe.  Das Fahrzeug ist mit Umfeld- und Referenzsensorik ausgerüstet. Zur Berechnung der Sollvorgaben wurde die Trajektorienplanung entsprechend \cite{Rathgeber2015b} umgesetzt. 

Zunächst wird das Verhalten des Gesamtsystems bei einem kombinierten Quer-Längsmanöver vorgestellt.  Abb.~\ref{abb_LQ_Bremsung} zeigt die geplanten und gemessenen Zustände. 
Durch die Trajektorienplanung wird eine dynamische Verzögerung bei gleichzeitigem Spurwechsel gefordert.        
 \begin{figure}[thpb]
 	 \centering
	 \setlength\figureheight{5.5cm} 
	 \setlength\figurewidth{10.5cm}
	  % This file was created by matlab2tikz v0.5.0 running on MATLAB 7.11.1.
%Copyright (c) 2008--2014, Nico Schlömer <nico.schloemer@gmail.com>
%All rights reserved.
%Minimal pgfplots version: 1.3
%
%The latest updates can be retrieved from
%  http://www.mathworks.com/matlabcentral/fileexchange/22022-matlab2tikz
%where you can also make suggestions and rate matlab2tikz.
%
\begin{tikzpicture}

\begin{axis}[%
width=0.410625\figurewidth,
height=0.264706\figureheight,
at={(0.540296\figurewidth,0\figureheight)},
scale only axis,
separate axis lines,
every outer x axis line/.append style={black},
every x tick label/.append style={font=\color{black}},
xmin=11,
xmax=20,
xlabel={$t$ [s]},
xlabel near ticks,
xmajorgrids,
every outer y axis line/.append style={black},
every y tick label/.append style={font=\color{black}},
ymin=-0.03,
ymax=0.03,
ylabel near ticks,
ylabel={$\kappa\text{ [1/m]}$},
ymajorgrids
]
\addplot [color=light-gray,solid,forget plot,line width=1.0]
  table[row sep=crcr]{%
10.975	-0.0008\\
10.995	-0.0008\\
11.015	-0.0008\\
11.035	-0.0008\\
11.055	-0.0008\\
11.075	-0.0009\\
11.095	-0.0008\\
11.115	-0.0008\\
11.135	-0.0009\\
11.155	-0.0007\\
11.175	-0.0007\\
11.195	-0.0007\\
11.215	-0.0006\\
11.235	-0.0007\\
11.255	-0.0007\\
11.275	-0.0007\\
11.295	-0.0008\\
11.315	-0.0007\\
11.335	-0.0005\\
11.355	-0.0008\\
11.375	-0.0008\\
11.395	-0.0005\\
11.415	-0.0007\\
11.435	-0.0006\\
11.455	-0.0005\\
11.475	-0.0003\\
11.495	-0.0003\\
11.515	-0.0004\\
11.535	-0.0002\\
11.555	-0.0003\\
11.575	-0.0002\\
11.595	0\\
11.615	0\\
11.635	-0.0001\\
11.655	0\\
11.675	0\\
11.695	0.0001\\
11.715	0.0001\\
11.735	0\\
11.755	0.0001\\
11.775	0.0002\\
11.795	0.0001\\
11.815	0.0002\\
11.835	0.0003\\
11.855	0.0002\\
11.875	0.0002\\
11.895	0.0002\\
11.915	0.0001\\
11.935	0\\
11.955	0.0001\\
11.975	0.0001\\
11.995	0.0001\\
12.015	0.0001\\
12.035	0.0002\\
12.055	0.0003\\
12.075	0.0005\\
12.095	0.0006\\
12.115	0.0007\\
12.135	0.0008\\
12.155	0.0007\\
12.175	0.0007\\
12.195	0.0008\\
12.215	0.0008\\
12.235	0.0008\\
12.255	0.0007\\
12.275	0.0006\\
12.295	0.0006\\
12.315	0.0006\\
12.335	0.0005\\
12.355	0.0007\\
12.375	0.0007\\
12.395	0.0006\\
12.415	0.0006\\
12.435	0.0006\\
12.455	0.0005\\
12.475	0.0005\\
12.495	0.0006\\
12.515	0.0007\\
12.535	0.0005\\
12.555	0.0006\\
12.575	0.0006\\
12.595	0.0005\\
12.615	0.0005\\
12.635	0.0006\\
12.655	0.0005\\
12.675	0.0006\\
12.695	0.0005\\
12.715	0.0005\\
12.735	0.0005\\
12.755	0.0005\\
12.775	0.0006\\
12.795	0.0006\\
12.815	0.0005\\
12.835	0.0006\\
12.855	0.0005\\
12.875	0.0005\\
12.895	0.0006\\
12.915	0.0005\\
12.935	0.0004\\
12.955	0.0005\\
12.975	0.0005\\
12.995	0.0003\\
13.015	0.0004\\
13.035	0.0004\\
13.055	0.0004\\
13.075	0.0003\\
13.095	0.0003\\
13.115	0.0003\\
13.135	0.0003\\
13.155	0.0004\\
13.175	0.0003\\
13.195	0.0003\\
13.215	0.0002\\
13.235	0.0002\\
13.255	0.0003\\
13.275	0.0005\\
13.295	0.0005\\
13.315	0.0008\\
13.335	0.0013\\
13.355	0.0018\\
13.375	0.0027\\
13.395	0.004\\
13.415	0.0053\\
13.435	0.0068\\
13.455	0.0084\\
13.475	0.01\\
13.495	0.0114\\
13.515	0.0128\\
13.535	0.0139\\
13.555	0.0147\\
13.575	0.0153\\
13.595	0.0157\\
13.615	0.016\\
13.635	0.0162\\
13.655	0.0165\\
13.675	0.0166\\
13.695	0.0168\\
13.715	0.0172\\
13.735	0.0175\\
13.755	0.0176\\
13.775	0.018\\
13.795	0.0184\\
13.815	0.0188\\
13.835	0.019\\
13.855	0.0194\\
13.875	0.0196\\
13.895	0.0198\\
13.915	0.02\\
13.935	0.0202\\
13.955	0.0204\\
13.975	0.0206\\
13.995	0.0206\\
14.015	0.0208\\
14.035	0.0207\\
14.055	0.0205\\
14.075	0.0205\\
14.095	0.0203\\
14.115	0.0198\\
14.135	0.0191\\
14.155	0.0183\\
14.175	0.0174\\
14.195	0.0159\\
14.215	0.0145\\
14.235	0.0128\\
14.255	0.011\\
14.275	0.0096\\
14.295	0.0079\\
14.315	0.0066\\
14.335	0.0054\\
14.355	0.0042\\
14.375	0.0034\\
14.395	0.0028\\
14.415	0.0021\\
14.435	0.0015\\
14.455	0.0012\\
14.475	0.0008\\
14.495	0.0002\\
14.515	-0.0004\\
14.535	-0.001\\
14.555	-0.0019\\
14.575	-0.003\\
14.595	-0.0043\\
14.615	-0.0055\\
14.635	-0.0069\\
14.655	-0.0085\\
14.675	-0.01\\
14.695	-0.0116\\
14.715	-0.0132\\
14.735	-0.0147\\
14.755	-0.0161\\
14.775	-0.0173\\
14.795	-0.0183\\
14.815	-0.0193\\
14.835	-0.0198\\
14.855	-0.0204\\
14.875	-0.0208\\
14.895	-0.0215\\
14.915	-0.0217\\
14.935	-0.0219\\
14.955	-0.0225\\
14.975	-0.0228\\
14.995	-0.0228\\
15.015	-0.0233\\
15.035	-0.0236\\
15.055	-0.0236\\
15.075	-0.024\\
15.095	-0.0245\\
15.115	-0.0247\\
15.135	-0.0249\\
15.155	-0.0252\\
15.175	-0.0259\\
15.195	-0.0258\\
15.215	-0.026\\
15.235	-0.0262\\
15.255	-0.0262\\
15.275	-0.026\\
15.295	-0.0262\\
15.315	-0.0262\\
15.335	-0.0257\\
15.355	-0.0252\\
15.375	-0.0249\\
15.395	-0.0241\\
15.415	-0.023\\
15.435	-0.0223\\
15.455	-0.0214\\
15.475	-0.0197\\
15.495	-0.0186\\
15.515	-0.0175\\
15.535	-0.016\\
15.555	-0.0152\\
15.575	-0.0147\\
15.595	-0.014\\
15.615	-0.0137\\
15.635	-0.0135\\
15.655	-0.0132\\
15.675	-0.0126\\
15.695	-0.012\\
15.715	-0.0107\\
15.735	-0.0094\\
15.755	-0.0075\\
15.775	-0.0052\\
15.795	-0.0033\\
15.815	-0.0009\\
15.835	0.001\\
15.855	0.0023\\
15.875	0.0032\\
15.895	0.0038\\
15.915	0.0038\\
15.935	0.0034\\
15.955	0.0031\\
15.975	0.0026\\
15.995	0.0023\\
16.015	0.002\\
16.035	0.0018\\
16.055	0.0019\\
16.075	0.0024\\
16.095	0.0025\\
16.115	0.0024\\
16.135	0.0029\\
16.155	0.0029\\
16.175	0.0026\\
16.195	0.002\\
16.215	0.0012\\
16.235	0.0005\\
16.255	-0.001\\
16.275	-0.0024\\
16.295	-0.003\\
16.315	-0.0042\\
16.335	-0.0044\\
16.355	-0.0042\\
16.375	-0.0044\\
16.395	-0.0037\\
16.415	-0.0019\\
16.435	-0.0015\\
16.455	-0.0015\\
16.475	-0.0008\\
16.495	0.0002\\
16.515	-0.0003\\
16.535	-0.0009\\
16.555	-0.0007\\
16.575	-0.0004\\
16.595	-0.0004\\
16.615	-0.0015\\
16.635	-0.0012\\
16.655	-0.0006\\
16.675	-0.0012\\
16.695	-0.0012\\
16.715	0\\
16.735	0\\
16.755	0\\
16.775	0\\
16.795	0\\
16.815	0\\
16.835	0\\
16.855	0\\
16.875	0\\
16.895	0\\
16.915	0\\
16.935	0\\
16.955	0\\
16.975	0\\
16.995	0\\
17.015	0\\
17.035	0\\
17.055	0\\
17.075	0\\
17.095	0\\
17.115	0\\
17.135	0\\
17.155	0\\
17.175	0\\
17.195	0\\
17.215	0\\
17.235	0\\
17.255	0\\
17.275	0\\
17.295	0\\
17.315	0\\
17.335	0\\
17.355	0\\
17.375	0\\
17.395	0\\
17.415	0\\
17.435	0\\
17.455	0\\
17.475	0\\
17.495	0\\
17.515	0\\
17.535	0\\
17.555	0\\
17.575	0\\
17.595	0\\
17.615	0\\
17.635	0\\
17.655	0\\
17.675	0\\
17.695	0\\
17.715	0\\
17.735	0\\
17.755	0\\
17.775	0\\
17.795	0\\
17.815	0\\
17.835	0\\
17.855	0\\
17.875	0\\
17.895	0\\
17.915	0\\
17.935	0\\
17.955	0\\
17.975	0\\
17.995	0\\
18.015	0\\
18.035	0\\
18.055	0\\
18.075	0\\
18.095	0\\
18.115	0\\
18.135	0\\
18.155	0\\
18.175	0\\
18.195	0\\
18.215	0\\
18.235	0\\
18.255	0\\
18.275	0\\
18.295	0\\
18.315	0\\
18.335	0\\
18.355	0\\
18.375	0\\
18.395	0\\
18.415	0\\
18.435	0\\
18.455	0\\
18.475	0\\
18.495	0\\
18.515	0\\
18.535	0\\
18.555	0\\
18.575	0\\
18.595	0\\
18.615	0\\
18.635	0\\
18.655	0\\
18.675	0\\
18.695	0\\
18.715	0\\
18.735	0\\
18.755	0\\
18.775	0\\
18.795	0\\
18.815	0\\
18.835	0\\
18.855	0\\
18.875	0\\
18.895	0\\
18.915	0\\
18.935	0\\
18.955	0\\
18.975	0\\
18.995	0\\
19.015	0\\
19.035	0\\
19.055	0\\
19.075	0\\
19.095	0\\
19.115	0\\
19.135	0\\
19.155	0\\
19.175	0\\
19.195	0\\
19.215	0\\
19.235	0\\
19.255	0\\
19.275	0\\
19.295	0\\
19.315	0\\
19.335	0\\
19.355	0\\
19.375	0\\
19.395	0\\
19.415	0\\
19.435	0\\
19.455	0\\
19.475	0\\
19.495	0\\
19.515	0\\
19.535	0\\
19.555	0\\
19.575	0\\
19.595	0\\
19.615	0\\
19.635	0\\
19.655	0\\
19.675	0\\
19.695	0\\
19.715	0\\
19.735	0\\
19.755	0\\
19.775	0\\
19.795	0\\
19.815	0\\
19.835	0\\
19.855	0\\
19.875	0\\
19.895	0\\
19.915	0\\
19.935	0\\
19.955	0\\
19.975	0\\
};
\addplot [color=black,solid,forget plot, line width=1.0]
  table[row sep=crcr]{%
10.975	9.38148847495768e-009\\
10.995	9.45987199685305e-009\\
11.015	9.09139252769364e-009\\
11.035	8.36938784942731e-009\\
11.055	7.37805505579558e-009\\
11.075	6.19277695790288e-009\\
11.095	4.88053730762772e-009\\
11.115	3.50029649709427e-009\\
11.135	2.10335149297691e-009\\
11.155	7.33718696910302e-010\\
11.175	-5.7150401078232e-010\\
11.195	-1.78167269826446e-009\\
11.215	-2.87234236395761e-009\\
11.235	-3.82485643157793e-009\\
11.255	-4.62588234384498e-009\\
11.275	-5.26714361015479e-009\\
11.295	-5.74497738270452e-009\\
11.315	-6.05993655256043e-009\\
11.335	-6.21655527055509e-009\\
11.355	-6.22276630224405e-009\\
11.375	-6.089700299583e-009\\
11.395	-5.83114090346726e-009\\
11.415	-5.46328760009374e-009\\
11.435	-5.00431474037555e-009\\
11.455	-4.47412329407371e-009\\
11.475	-3.89363785657793e-009\\
11.495	-3.28475491251368e-009\\
11.515	-2.66987343344738e-009\\
11.535	-2.07136618968207e-009\\
11.555	-1.51148671356793e-009\\
11.575	-1.01163233345858e-009\\
11.595	-5.9256316520262e-010\\
11.615	-2.73234351810814e-010\\
11.635	-5.02802521840096e-011\\
11.655	1.0052545601491e-010\\
11.675	1.92129354092963e-010\\
11.695	2.36154013011358e-010\\
11.715	2.42959846685764e-010\\
11.735	2.21706972225455e-010\\
11.755	1.80417222828133e-010\\
11.775	1.26035973324612e-010\\
11.795	6.44934730620328e-011\\
11.815	7.68546847256663e-013\\
11.835	-6.10522535304803e-011\\
11.855	-1.17711299041368e-010\\
11.875	-1.66716279670354e-010\\
11.895	-2.06281686176979e-010\\
11.915	-2.35265029679965e-010\\
11.935	-2.53104648351155e-010\\
11.955	-2.59759103116153e-010\\
11.975	-2.5564203531836e-010\\
11.995	-2.41569431125299e-010\\
12.015	-2.18682585928498e-010\\
12.035	-1.88401946821237e-010\\
12.055	-1.52353199500688e-010\\
12.075	-1.12311389355302e-010\\
12.095	-7.01404004321837e-011\\
12.115	-2.77242812041223e-011\\
12.135	1.30846470811075e-011\\
12.155	5.05420427732162e-011\\
12.175	8.30556595721177e-011\\
12.195	1.09265513303924e-010\\
12.215	1.28075119953941e-010\\
12.235	1.3877214655178e-010\\
12.255	1.41022513111544e-010\\
12.275	1.34977709564943e-010\\
12.295	1.2130875470806e-010\\
12.315	1.01290115184227e-010\\
12.335	7.68618224622486e-011\\
12.355	5.06567392200008e-011\\
12.375	2.61007378904443e-011\\
12.395	7.09933414139163e-012\\
12.415	-5.93231236703518e-012\\
12.435	-1.4052081546978e-011\\
12.455	-1.8210407140562e-011\\
12.475	-1.92552605804419e-011\\
12.495	-1.79371465597322e-011\\
12.515	-1.49141619193438e-011\\
12.535	-1.07569066501445e-011\\
12.555	-5.95353800980636e-012\\
12.575	-9.148091355618e-013\\
12.595	4.02102916935432e-012\\
12.615	8.58317757146398e-012\\
12.635	1.25631883368671e-011\\
12.655	1.58100806751937e-011\\
12.675	1.82254870917387e-011\\
12.695	1.97582974287291e-011\\
12.715	2.03998033743158e-011\\
12.735	2.01786851117269e-011\\
12.755	1.91564975687841e-011\\
12.775	1.7421722051103e-011\\
12.795	1.50856653557963e-011\\
12.815	1.22766198248914e-011\\
12.835	9.13556209847233e-012\\
12.855	5.8112104538155e-012\\
12.875	2.45400639913018e-012\\
12.895	-7.87358839151459e-013\\
12.915	-3.7730495811017e-012\\
12.935	-6.37401893419098e-012\\
12.955	-8.4805764841156e-012\\
12.975	-1.00036420119798e-011\\
12.995	-1.088417984213e-011\\
13.015	-1.10936555081098e-011\\
13.035	-1.06421095893983e-011\\
13.055	-9.58222824004595e-012\\
13.075	-8.01295401559043e-012\\
13.095	-6.08831301346369e-012\\
13.115	-4.01692871673798e-012\\
13.135	-2.07200944769836e-012\\
13.155	7.61524279369041e-005\\
13.175	0.000294439669232816\\
13.195	0.000640104874037206\\
13.215	0.00109905668068677\\
13.235	0.00165786920115352\\
13.255	0.00230377702973783\\
13.275	0.00302466470748186\\
13.295	0.00380906090140343\\
13.315	0.00464611640200019\\
13.335	0.00552559411153197\\
13.355	0.00643784459680319\\
13.375	0.00737379351630807\\
13.395	0.00832490902394056\\
13.415	0.00928320270031691\\
13.435	0.010241175070405\\
13.455	0.0111918346956372\\
13.475	0.0121286436915398\\
13.495	0.0130455186590552\\
13.515	0.0139368204399943\\
13.535	0.0147973271086812\\
13.555	0.0156222078949213\\
13.575	0.0164070446044207\\
13.595	0.0171477943658829\\
13.615	0.017840787768364\\
13.635	0.0184827297925949\\
13.655	0.0190706793218851\\
13.675	0.0196020118892193\\
13.695	0.0200745072215796\\
13.715	0.0204862281680107\\
13.735	0.0208355896174908\\
13.755	0.0211213044822216\\
13.775	0.0213424079120159\\
13.795	0.0214982330799103\\
13.815	0.0215883702039719\\
13.835	0.0216127261519432\\
13.855	0.0215714070945978\\
13.875	0.0214648135006428\\
13.895	0.0212935190647841\\
13.915	0.0208893418312073\\
13.935	0.0205252654850483\\
13.955	0.0200681891292334\\
13.975	0.0195259526371956\\
13.995	0.0189060214906931\\
14.015	0.0182154532521963\\
14.035	0.0174609180539846\\
14.055	0.0166487339884043\\
14.075	0.0157848298549652\\
14.095	0.0148747498169541\\
14.115	0.0139237055554986\\
14.135	0.0129365483298898\\
14.155	0.0119178323075175\\
14.175	0.0108717568218708\\
14.195	0.0098022473976016\\
14.215	0.0087129594758153\\
14.235	0.00760728446766734\\
14.255	0.0064883790910244\\
14.275	0.00535917980596423\\
14.295	0.00422243913635612\\
14.315	0.00308071915060282\\
14.335	0.00193643930833787\\
14.355	0.0007918716291897\\
14.375	-0.000350823480403051\\
14.395	-0.00148959539365023\\
14.415	-0.00262247584760189\\
14.435	-0.00374756054952741\\
14.455	-0.0048630191013217\\
14.475	-0.00596705172210932\\
14.495	-0.00705791264772415\\
14.515	-0.00813384726643562\\
14.535	-0.00919315312057734\\
14.555	-0.0102341193705797\\
14.575	-0.0112550351768732\\
14.595	-0.012254199013114\\
14.615	-0.0132298897951841\\
14.635	-0.0141804125159979\\
14.655	-0.0151040479540825\\
14.675	-0.015894278883934\\
14.695	-0.0167499110102654\\
14.715	-0.0175765585154295\\
14.735	-0.0183720029890537\\
14.755	-0.0191340837627649\\
14.775	-0.0198606867343187\\
14.795	-0.020549725741148\\
14.815	-0.0211991760879755\\
14.835	-0.0218070726841688\\
14.855	-0.0223715417087078\\
14.875	-0.0228907614946365\\
14.895	-0.0233629811555147\\
14.915	-0.0237866062670946\\
14.935	-0.0241600964218378\\
14.955	-0.0244820471853018\\
14.975	-0.0247511677443981\\
14.995	-0.0249663032591343\\
15.015	-0.0251264497637749\\
15.035	-0.0252307299524546\\
15.055	-0.0252784211188555\\
15.075	-0.0252689998596907\\
15.095	-0.0252020508050919\\
15.115	-0.0250773653388023\\
15.135	-0.0248949136584997\\
15.155	-0.0246548224240541\\
15.175	-0.0243574250489473\\
15.195	-0.0240032058209181\\
15.215	-0.0235928744077683\\
15.235	-0.0231273099780083\\
15.255	-0.0226075984537601\\
15.275	-0.0220349989831448\\
15.295	-0.0214109793305397\\
15.315	-0.0207372177392244\\
15.335	-0.0200155545026064\\
15.355	-0.0192480850964785\\
15.375	-0.0184370670467615\\
15.395	-0.0175849869847298\\
15.415	-0.0166945457458496\\
15.435	-0.0154487006366253\\
15.455	-0.0145300589501858\\
15.475	-0.0135937612503767\\
15.495	-0.0126383891329169\\
15.515	-0.011663313023746\\
15.535	-0.010668683797121\\
15.555	-0.00965546444058418\\
15.575	-0.00862538628280163\\
15.595	-0.00758100813254714\\
15.615	-0.00652569532394409\\
15.635	-0.00546360807493329\\
15.655	-0.0043997117318213\\
15.675	-0.00333974603563547\\
15.695	-0.00229020859114826\\
15.715	-0.00125833577476442\\
15.735	-0.000252038182225078\\
15.755	0.000720137672033161\\
15.775	0.00164908799342811\\
15.795	0.00252524018287659\\
15.815	0.00333866570144892\\
15.835	0.00407921243458986\\
15.855	0.00473667168989778\\
15.875	0.00530098797753453\\
15.895	0.00576248578727245\\
15.915	0.00611215457320213\\
15.935	0.00634199054911733\\
15.955	0.00644539948552847\\
15.975	0.00641764560714364\\
15.995	0.00625641411170363\\
16.015	0.00596244959160686\\
16.035	0.00554029876366258\\
16.055	0.00499916216358542\\
16.075	0.00435391720384359\\
16.095	0.00362631003372371\\
16.115	0.00284624588675797\\
16.135	0.00205346406437457\\
16.155	0.00129925995133817\\
16.175	0.000648667337372899\\
16.195	0.000141284268465824\\
16.215	-0.00021642267529387\\
16.235	-0.000460295588709414\\
16.255	-0.000604157743509859\\
16.275	-0.000661558704450727\\
16.295	-0.000645733904093504\\
16.315	-0.000569559924770147\\
16.335	-0.000445504294475541\\
16.355	-0.000285565882222727\\
16.375	-0.000101219455245882\\
16.395	9.66589068411849e-005\\
16.415	0.00029786306549795\\
16.435	0.000492932158522308\\
16.455	0.000673242728225887\\
16.475	0.000831100915092975\\
16.495	0.000959834142122418\\
16.515	0.00105388753581792\\
16.535	0.00110893463715911\\
16.555	0.00112198665738106\\
16.575	0.00109147059265524\\
16.595	0.00101734348572791\\
16.615	0.000901181541848928\\
16.635	0.000746206089388579\\
16.655	0.000557409832254052\\
16.675	0.000341464794473723\\
16.695	0.000106784333183896\\
16.715	-0.00013668391329702\\
16.735	-0.000377511547412723\\
16.755	-0.000603243242949247\\
16.775	-0.000800598354544491\\
16.795	-0.000956356467213482\\
16.815	-0.00105803308542818\\
16.835	-0.00088266673265025\\
16.855	-0.000694614427629858\\
16.875	-0.000171450956258923\\
16.895	0.000124329992104322\\
16.915	0.000416566093917936\\
16.935	0.000693605747073889\\
16.955	0.000946863379795104\\
16.975	0.00117084791418165\\
16.995	0.00136221456341445\\
17.015	0.00151936302427202\\
17.035	0.00164265569765121\\
17.055	0.00173280166927725\\
17.075	0.00179165427107364\\
17.095	0.00182142097037286\\
17.115	0.00182470050640404\\
17.135	0.00180423317942768\\
17.155	0.00176282704342157\\
17.175	0.00170313054695725\\
17.195	0.00162822415586561\\
17.215	0.00154031591955572\\
17.235	0.00144191144499928\\
17.255	0.00133530143648386\\
17.275	0.00122233352158219\\
17.295	0.00110503996256739\\
17.315	0.000984836951829493\\
17.335	0.000862893648445606\\
17.355	0.000741032534278929\\
17.375	0.000619568105321378\\
17.395	0.000500011083204299\\
17.415	0.000383059930754825\\
17.435	0.00026850585709326\\
17.455	0.00015788254677318\\
17.475	5.05603638885077e-005\\
17.495	-5.21098299941514e-005\\
17.515	-0.000150760897668079\\
17.535	-0.000244750815909356\\
17.555	-0.000334092968842015\\
17.575	-0.000419403077103198\\
17.595	-0.000500212074257433\\
17.615	-0.000576465739868581\\
17.635	-0.000575385871343315\\
17.655	-0.00064448470948264\\
17.675	-0.000707569241058081\\
17.695	-0.00076569092925638\\
17.715	-0.000819005013909191\\
17.735	-0.000868120812810957\\
17.755	-0.000912855379283428\\
17.775	-0.000954058952629566\\
17.795	-0.000991941662505269\\
17.815	-0.00102669955231249\\
17.835	-0.00105864589568228\\
17.855	-0.00108768558129668\\
17.875	-0.00111473130527884\\
17.895	-0.00113939878065139\\
17.915	-0.0011621720623225\\
17.935	-0.00118291424587369\\
17.955	-0.00120184221304953\\
17.975	-0.00121960090473294\\
17.995	-0.00123547448311001\\
18.015	-0.00125019589904696\\
18.035	-0.00126339041162282\\
18.055	-0.00127584952861071\\
18.075	-0.00128708721604198\\
18.095	-0.00129735225345939\\
18.115	-0.0013067782856524\\
18.135	-0.00131519662681967\\
18.155	-0.00132293126080185\\
18.175	-0.00133029115386307\\
18.195	-0.00133671355433762\\
18.215	-0.00134250754490495\\
18.235	-0.00134759407956153\\
18.255	-0.00135236175265163\\
18.275	-0.00135672418400645\\
18.295	-0.00136059720534831\\
18.315	-0.00136417197063565\\
18.335	-0.00136708980426192\\
18.355	-0.0013699937844649\\
18.375	-0.0013724333839491\\
18.395	-0.00137243268545717\\
18.415	-0.00137459125835449\\
18.435	-0.00137656240258366\\
18.455	-0.001378258690238\\
18.475	-0.00137968303170055\\
18.495	-0.00138092611450702\\
18.515	-0.00138216628693044\\
18.535	-0.00138305022846907\\
18.555	-0.00138375617098063\\
18.575	-0.00138437317218632\\
18.595	-0.00138507748488337\\
18.615	-0.00138534139841795\\
18.635	-0.00138569297268987\\
18.655	-0.00138604443054646\\
18.675	-0.00138613232411444\\
18.695	-0.00138630776200444\\
18.715	-0.00138630776200444\\
18.735	-0.00138630776200444\\
18.755	-0.00138621998485178\\
18.775	-0.00138613232411444\\
18.795	-0.00138613232411444\\
18.815	-0.00138604443054646\\
18.835	-0.0013858686434105\\
18.855	-0.00138560507912189\\
18.875	-0.00138542940840125\\
18.895	-0.00138534139841795\\
18.915	-0.00138516549486667\\
18.935	-0.00138490158133209\\
18.955	-0.00138481345493346\\
18.975	-0.00138472556136549\\
18.995	-0.00138428504578769\\
19.015	-0.00138419691938907\\
19.035	-0.0013839325401932\\
19.055	-0.00138375617098063\\
19.075	-0.00138366792816669\\
19.095	-0.00138366792816669\\
19.115	-0.0013834914425388\\
19.135	-0.00138322694692761\\
19.155	-0.0013834914425388\\
19.175	-0.0013834914425388\\
19.195	-0.00138357968535274\\
19.215	-0.00138357968535274\\
19.235	-0.00138366792816669\\
19.255	-0.00138366792816669\\
19.275	-0.00138375605456531\\
19.295	-0.00138384429737926\\
19.315	-0.00138393242377788\\
19.335	-0.00138402066659182\\
19.355	-0.00138402066659182\\
19.375	-0.00138428504578769\\
19.395	-0.00138437328860164\\
19.415	-0.00138454942498356\\
19.435	-0.00138463743496686\\
19.455	-0.00138481345493346\\
19.475	-0.00138507748488337\\
19.495	-0.00138516549486667\\
19.515	-0.00138525338843465\\
19.535	-0.00138551718555391\\
19.555	-0.00138569285627455\\
19.575	-0.00138578074984252\\
19.595	-0.00138595653697848\\
19.615	-0.00138621998485178\\
19.635	-0.00138648319989443\\
19.655	-0.00138648319989443\\
19.675	-0.00138674629852176\\
19.695	-0.00138692161999643\\
19.715	-0.00138718460220844\\
19.735	-0.0013874473515898\\
19.755	-0.0013874473515898\\
19.775	-0.00138771010097116\\
19.795	-0.00138788507319987\\
19.815	-0.00138806004542857\\
19.835	-0.00138814747333527\\
19.855	-0.00138840975705534\\
19.875	-0.00138849706854671\\
19.895	-0.00138875923585147\\
19.915	-0.00138867180794477\\
19.935	-0.00138884654734284\\
19.955	-0.00138902117032558\\
19.975	-0.001389195676893\\
};
\end{axis}

\begin{axis}[%
width=0.410625\figurewidth,
height=0.264706\figureheight,
at={(0.540296\figurewidth,0.367647\figureheight)},
scale only axis,
separate axis lines,
every outer x axis line/.append style={black},
every x tick label/.append style={font=\color{black}},
xmin=11,
xmax=20,
xmajorgrids,
every outer y axis line/.append style={black},
every y tick label/.append style={font=\color{black}},
ymin=-0.05,
ymax=0.3,
ylabel={$\theta{}\text{ [rad]}$},
ylabel near ticks,
ymajorgrids
]
\addplot [color=light-gray,solid,forget plot,line width=1.0]
  table[row sep=crcr]{%
10.975	-0.00483439303934574\\
10.995	-0.00451913895085454\\
11.015	-0.00403015315532684\\
11.035	-0.00371382758021355\\
11.055	-0.00340464920736849\\
11.075	-0.0031281728297472\\
11.095	-0.00268081738613546\\
11.115	-0.00234167650341988\\
11.135	-0.00200298940762877\\
11.155	-0.00164385139942169\\
11.175	-0.00127162481658161\\
11.195	-0.000931398419197649\\
11.215	-0.000425654667196795\\
11.235	-0.00014473038027063\\
11.255	9.24975829548202e-005\\
11.275	0.00036603314219974\\
11.295	0.00071013969136402\\
11.315	0.0010577297070995\\
11.335	0.00139422703068703\\
11.355	0.0017333080759272\\
11.375	0.00219001923687756\\
11.395	0.0022976640611887\\
11.415	0.00261506601236761\\
11.435	0.00293107074685395\\
11.455	0.00319629046134651\\
11.475	0.00344069371931255\\
11.495	0.00352165382355452\\
11.515	0.00377931469120085\\
11.535	0.0038671309594065\\
11.555	0.0039545614272356\\
11.575	0.00424557365477085\\
11.595	0.00440585799515247\\
11.615	0.00455497624352574\\
11.635	0.00463511003181338\\
11.655	0.00471678376197815\\
11.675	0.00477723404765129\\
11.695	0.00483727781102061\\
11.715	0.00499625317752361\\
11.735	0.00516080670058727\\
11.755	0.00527735473588109\\
11.775	0.00537290750071406\\
11.795	0.00534263486042619\\
11.815	0.00550834601745009\\
11.835	0.00563527410849929\\
11.855	0.00553830340504646\\
11.875	0.0056325402110815\\
11.895	0.00574886985123158\\
11.915	0.0058641224168241\\
11.935	0.00595522578805685\\
11.955	0.00587058207020164\\
11.975	0.00590058974921703\\
11.995	0.00593192968517542\\
12.015	0.005979357752949\\
12.035	0.00602993601933122\\
12.055	0.00606499426066875\\
12.075	0.00592184765264392\\
12.095	0.00575958611443639\\
12.115	0.00558505300432444\\
12.135	0.00542569765821099\\
12.155	0.00528155546635389\\
12.175	0.00514543056488037\\
12.195	0.00502065755426884\\
12.215	0.00471238838508725\\
12.235	0.00453785574063659\\
12.255	0.00436954759061337\\
12.275	0.00427515711635351\\
12.295	0.00418374128639698\\
12.315	0.00401425687596202\\
12.335	0.00401425687596202\\
12.355	0.00372830522246659\\
12.375	0.00381141202524304\\
12.395	0.0036651911213994\\
12.415	0.00349065847694874\\
12.435	0.00333178066648543\\
12.455	0.0031892757397145\\
12.475	0.00305238901637495\\
12.495	0.00292894314043224\\
12.515	0.00296705961227417\\
12.535	0.00279252673499286\\
12.555	0.00262482278048992\\
12.575	0.0025304276496172\\
12.595	0.00243900623172522\\
12.615	0.00226892810314894\\
12.635	0.00209439499303699\\
12.655	0.00191986211575568\\
12.675	0.00191986211575568\\
12.695	0.00174532923847437\\
12.715	0.00157079636119306\\
12.735	0.00139626336749643\\
12.755	0.00122173049021512\\
12.775	0.0012336039217189\\
12.795	0.00109147350303829\\
12.815	0.000952567032072693\\
12.835	0.000828756135888398\\
12.855	0.000701412267517298\\
12.875	0.000735987967345864\\
12.895	0.000594562268815935\\
12.915	0.000468334998004138\\
12.935	0.000340828002663329\\
12.955	0.000349065900081769\\
12.975	0.000174532979144715\\
12.995	5.48260068171658e-005\\
13.015	0.000137932162033394\\
13.035	4.30011061480773e-011\\
13.055	3.78349990226567e-011\\
13.075	-0.000150637730257586\\
13.095	-8.89875518623739e-005\\
13.115	-0.000214108091313392\\
13.135	-0.000192654260899872\\
13.155	-0.000349065841874108\\
13.175	-0.000349065841874108\\
13.195	-0.000523598748259246\\
13.215	-0.000523598748259246\\
13.235	-0.000509376986883581\\
13.255	-0.000638434430584311\\
13.275	-0.00044242013245821\\
13.295	-0.000214132131077349\\
13.315	0.000294335419312119\\
13.335	0.000916192075237632\\
13.355	0.0018915687687695\\
13.375	0.00297170225530863\\
13.395	0.00420343596488237\\
13.415	0.00571291660889983\\
13.435	0.00715364469215274\\
13.455	0.00875583384186029\\
13.475	0.0103405928239226\\
13.495	0.012110099196434\\
13.515	0.0138591974973679\\
13.535	0.0159659888595343\\
13.555	0.0184932127594948\\
13.575	0.0211956650018692\\
13.595	0.024395115673542\\
13.615	0.0280760042369366\\
13.635	0.0320195220410824\\
13.655	0.0362177491188049\\
13.675	0.0410008020699024\\
13.695	0.0459900014102459\\
13.715	0.0513581037521362\\
13.735	0.0569324344396591\\
13.755	0.0629786103963852\\
13.775	0.0689269378781319\\
13.795	0.0751102790236473\\
13.815	0.0814620926976204\\
13.835	0.08797487616539\\
13.855	0.0944609865546227\\
13.875	0.101039662957191\\
13.895	0.107544586062431\\
13.915	0.114099718630314\\
13.935	0.120607577264309\\
13.955	0.127007782459259\\
13.975	0.133139878511429\\
13.995	0.138508960604668\\
14.015	0.144446417689323\\
14.035	0.14996574819088\\
14.055	0.155373007059097\\
14.075	0.160490453243256\\
14.095	0.165264219045639\\
14.115	0.169773995876312\\
14.135	0.173960700631142\\
14.155	0.177855342626572\\
14.175	0.181625500321388\\
14.195	0.185052931308746\\
14.215	0.188465446233749\\
14.235	0.191673189401627\\
14.255	0.194837927818298\\
14.275	0.197772860527039\\
14.295	0.200471863150597\\
14.315	0.203103393316269\\
14.335	0.205316424369812\\
14.355	0.207117453217506\\
14.375	0.208682402968407\\
14.395	0.209825798869133\\
14.415	0.210565850138664\\
14.435	0.210889026522636\\
14.455	0.210599526762962\\
14.475	0.209887966513634\\
14.495	0.208614647388458\\
14.515	0.207015544176102\\
14.535	0.205214142799377\\
14.555	0.202833577990532\\
14.575	0.200071692466736\\
14.595	0.197319120168686\\
14.615	0.194094255566597\\
14.635	0.190908581018448\\
14.655	0.187416553497314\\
14.675	0.183977425098419\\
14.695	0.180281788110733\\
14.715	0.176548466086388\\
14.735	0.172778606414795\\
14.755	0.166822656989098\\
14.775	0.162805020809174\\
14.795	0.158624708652496\\
14.815	0.154465854167938\\
14.835	0.150020197033882\\
14.855	0.145302042365074\\
14.875	0.140546351671219\\
14.895	0.135574862360954\\
14.915	0.130569651722908\\
14.935	0.125341147184372\\
14.955	0.119939461350441\\
14.975	0.114401087164879\\
14.995	0.108973294496536\\
15.015	0.103290572762489\\
15.035	0.0977189987897873\\
15.055	0.0919091105461121\\
15.075	0.0862571224570274\\
15.095	0.0806166529655457\\
15.115	0.0750076100230217\\
15.135	0.0694537907838821\\
15.155	0.0641543194651604\\
15.175	0.0588194951415062\\
15.195	0.0537868142127991\\
15.215	0.0487291887402534\\
15.235	0.0438425615429878\\
15.255	0.0391476228833199\\
15.275	0.0346642807126045\\
15.295	0.0304115638136864\\
15.315	0.0262331068515778\\
15.335	0.0223206169903278\\
15.355	0.0185136273503304\\
15.375	0.0149959102272987\\
15.395	0.0114402994513512\\
15.415	0.00803345441818237\\
15.435	0.00493914633989334\\
15.455	0.00183527916669846\\
15.475	-0.00108802318572998\\
15.495	-0.0039878822863102\\
15.515	-0.00684348307549953\\
15.535	-0.00927583687007427\\
15.555	-0.0115190958604217\\
15.575	-0.0135813420638442\\
15.595	-0.0154619161039591\\
15.615	-0.0169806070625782\\
15.635	-0.0181384738534689\\
15.655	-0.0190950985997915\\
15.675	-0.0196885839104652\\
15.695	-0.0201043952256441\\
15.715	-0.0203469023108482\\
15.735	-0.0204577334225178\\
15.755	-0.0205946899950504\\
15.775	-0.0205855667591095\\
15.795	-0.0206108465790749\\
15.815	-0.0206425711512566\\
15.835	-0.0207072347402573\\
15.855	-0.0205147508531809\\
15.875	-0.020250340923667\\
15.895	-0.0199508983641863\\
15.915	-0.0194827634841204\\
15.935	-0.0186694879084826\\
15.955	-0.0177136063575745\\
15.975	-0.0166447069495916\\
15.995	-0.0154935643076897\\
16.015	-0.0141174532473087\\
16.035	-0.0128977093845606\\
16.055	-0.0115175386890769\\
16.075	-0.0101832151412964\\
16.095	-0.00909971352666616\\
16.115	-0.00794643722474575\\
16.135	-0.00709790922701359\\
16.155	-0.00640112720429897\\
16.175	-0.00569774629548192\\
16.195	-0.00534668657928705\\
16.215	-0.00483251176774502\\
16.235	-0.00466508185490966\\
16.255	-0.00464842421934009\\
16.275	-0.00440300488844514\\
16.295	-0.00406999187543988\\
16.315	-0.00397666776552796\\
16.335	-0.00375404581427574\\
16.355	-0.00354636879637837\\
16.375	-0.00334203150123358\\
16.395	-0.00312494090758264\\
16.415	-0.00306154508143663\\
16.435	-0.00297144637443125\\
16.455	-0.00284127844497561\\
16.475	-0.00284184189513326\\
16.495	-0.00280879624187946\\
16.515	-0.0027437093667686\\
16.535	-0.00247504422441125\\
16.555	-0.00235604192130268\\
16.575	-0.0022172552999109\\
16.595	-0.00206443434581161\\
16.615	-0.00172920909244567\\
16.635	-0.00156700145453215\\
16.655	-0.00140952889341861\\
16.675	-0.00108825869392604\\
16.695	-0.000957716139964759\\
16.715	-0.000673596689011902\\
16.735	-0.00058883655583486\\
16.755	-0.000531715981196612\\
16.775	-0.000503859599120915\\
16.795	-0.000505547621287405\\
16.815	-0.000505547621287405\\
16.835	-0.000417621631640941\\
16.855	-0.000496391323395073\\
16.875	-0.000417910632677376\\
16.895	-0.000525287061464041\\
16.915	-8.07306496426463e-005\\
16.935	1.14298309199512e-005\\
16.955	9.65120852924883e-005\\
16.975	0.00026870277361013\\
16.995	0.000293723860522732\\
17.015	0.000518253014888614\\
17.035	0.000590036390349269\\
17.055	0.000680043362081051\\
17.075	0.000784516043495387\\
17.095	0.000899647304322571\\
17.115	0.00102229556068778\\
17.135	0.0013236126396805\\
17.155	0.00145188614260405\\
17.175	0.00157917651813477\\
17.195	0.00170338503085077\\
17.215	0.00182311749085784\\
17.235	0.00193687668070197\\
17.255	0.00204391032457352\\
17.275	0.0021429134067148\\
17.295	0.00223397486843169\\
17.315	0.00231667188927531\\
17.335	0.00239088176749647\\
17.355	0.00245679961517453\\
17.375	0.00251450552605093\\
17.395	0.00256440578959882\\
17.415	0.00243241246789694\\
17.435	0.00246781785972416\\
17.455	0.00249677244573832\\
17.475	0.00251960474997759\\
17.495	0.00253686006180942\\
17.515	0.00272366707213223\\
17.535	0.00273133371956646\\
17.555	0.00273494375869632\\
17.575	0.00256038340739906\\
17.595	0.00255717989057302\\
17.615	0.00255121383816004\\
17.635	0.00254289503209293\\
17.655	0.00253250193782151\\
17.675	0.00252040289342403\\
17.695	0.00250693154521286\\
17.715	0.00250693899579346\\
17.735	0.0026669108774513\\
17.755	0.00265163998119533\\
17.775	0.00263576535508037\\
17.795	0.00261956127360463\\
17.815	0.00260313437320292\\
17.835	0.00258682668209076\\
17.855	0.00257058744318783\\
17.875	0.00255455123260617\\
17.895	0.00253884098492563\\
17.915	0.00269803544506431\\
17.935	0.00268329726532102\\
17.955	0.00266884570010006\\
17.975	0.00265501835383475\\
17.995	0.00281620514579117\\
18.015	0.0028035375289619\\
18.035	0.00279152928851545\\
18.055	0.00277985306456685\\
18.075	0.00276906392537057\\
18.095	0.00275874719955027\\
18.115	0.00274923513643444\\
18.135	0.00274001387879252\\
18.155	0.00273148808628321\\
18.175	0.00272352178581059\\
18.195	0.00289058452472091\\
18.215	0.00288378307595849\\
18.235	0.00287742260843515\\
18.255	0.00287126936018467\\
18.275	0.00286581553518772\\
18.295	0.00286082830280066\\
18.315	0.00285639520734549\\
18.335	0.00285219307988882\\
18.355	0.00284830760210752\\
18.375	0.00284482445567846\\
18.395	0.00284158159047365\\
18.415	0.00283891428261995\\
18.435	0.00283624138683081\\
18.455	0.00283398199826479\\
18.475	0.0028339815326035\\
18.495	0.00283196941018105\\
18.515	0.00283012259751558\\
18.535	0.00282852537930012\\
18.555	0.00282717868685722\\
18.575	0.00282599870115519\\
18.595	0.00282481871545315\\
18.615	0.00282397447153926\\
18.635	0.00282329926267266\\
18.655	0.00282270787283778\\
18.675	0.00282203173264861\\
18.695	0.00282177794724703\\
18.715	0.00282143987715244\\
18.735	0.00282110134139657\\
18.755	0.00282101659104228\\
18.775	0.00282084755599499\\
18.795	0.00282084755599499\\
18.815	0.00299538066610694\\
18.835	0.00299546518363059\\
18.855	0.00299554970115423\\
18.875	0.00299554970115423\\
18.895	0.00299563445150852\\
18.915	0.00282127084210515\\
18.935	0.00282152462750673\\
18.955	0.00264716031961143\\
18.975	0.00247271219268441\\
18.995	0.0024728812277317\\
19.015	0.00247313478030264\\
19.035	0.00247321929782629\\
19.055	0.00247330381534994\\
19.075	0.00247372640296817\\
19.095	0.00247381092049181\\
19.115	0.00247406447306275\\
19.135	0.0024742332752794\\
19.155	0.00247431755997241\\
19.175	0.00247431755997241\\
19.195	0.00247448636218905\\
19.215	0.00230020703747869\\
19.235	0.00229995348490775\\
19.255	0.00229995348490775\\
19.275	0.0022998689673841\\
19.295	0.0022998689673841\\
19.315	0.0022997846826911\\
19.335	0.0022997846826911\\
19.355	0.00247423304244876\\
19.375	0.00247414852492511\\
19.395	0.00247406424023211\\
19.415	0.00247397972270846\\
19.435	0.00247397972270846\\
19.455	0.00247372617013752\\
19.475	0.00247364165261388\\
19.495	0.00229893974028528\\
19.515	0.00229885522276163\\
19.535	0.00229868618771434\\
19.555	0.0022984326351434\\
19.575	0.00229834811761975\\
19.595	0.00229826360009611\\
19.615	0.00229800981469452\\
19.635	0.00229784077964723\\
19.655	0.00229775602929294\\
19.675	0.00229758699424565\\
19.695	0.00229733320884407\\
19.715	0.00229707942344248\\
19.735	0.00229707942344248\\
19.755	0.00229682540521026\\
19.775	0.00229665613733232\\
19.795	0.00229640235193074\\
19.815	0.00229614833369851\\
19.835	0.00229614833369851\\
19.855	0.00229589408263564\\
19.875	0.00229572458192706\\
19.895	0.00229555531404912\\
19.915	0.00229547056369483\\
19.935	0.00229521631263196\\
19.955	0.00229513156227767\\
19.975	0.0022948773112148\\
};
\addplot [color=black,solid,forget plot, line width=1.0]
  table[row sep=crcr]{%
10.975	3.46754680524697e-010\\
10.995	1.44974465765557e-009\\
11.015	3.25399618361644e-009\\
11.035	5.52293144551186e-009\\
11.055	8.05564681627402e-009\\
11.075	1.06839941338421e-008\\
11.095	1.32697790533598e-008\\
11.115	1.57020636493144e-008\\
11.135	1.78945285256304e-008\\
11.155	1.97830427595136e-008\\
11.175	2.13232258516882e-008\\
11.195	2.24881766541785e-008\\
11.215	2.32663115440346e-008\\
11.235	2.36592594404783e-008\\
11.255	2.36799291286616e-008\\
11.275	2.33505907942799e-008\\
11.295	2.27011582865089e-008\\
11.315	2.17676365821262e-008\\
11.335	2.05902832561833e-008\\
11.355	1.92125071407645e-008\\
11.375	1.76792820383298e-008\\
11.395	1.60359672207733e-008\\
11.415	1.43272664843153e-008\\
11.435	1.25962005270708e-008\\
11.455	1.08831512690699e-008\\
11.475	9.22468057495962e-009\\
11.495	7.6536554871609e-009\\
11.515	6.19784623623332e-009\\
11.535	4.88015716726409e-009\\
11.555	3.71706554425089e-009\\
11.575	2.71956723807421e-009\\
11.595	1.89264603989159e-009\\
11.615	1.23470134116843e-009\\
11.635	7.38190231164282e-010\\
11.655	3.89295401470591e-010\\
11.675	1.68577152237503e-010\\
11.695	5.10353877214431e-011\\
11.715	7.93279504585076e-012\\
11.735	1.64258051604804e-011\\
11.755	5.83026960043753e-011\\
11.775	1.18762763512414e-010\\
11.795	1.86056850481897e-010\\
11.815	2.5114593737996e-010\\
11.835	3.07375541686028e-010\\
11.855	3.50168588569844e-010\\
11.875	3.76735420637431e-010\\
11.895	3.85799170388168e-010\\
11.915	3.77341241586393e-010\\
11.935	3.52361889666142e-010\\
11.955	3.12657011392048e-010\\
11.975	2.60617555314369e-010\\
11.995	1.99038258097417e-010\\
12.015	1.30949889820542e-010\\
12.035	5.94634411155148e-011\\
12.055	-1.23633872237128e-011\\
12.075	-8.16371692469176e-011\\
12.095	-1.45750370106335e-010\\
12.115	-2.02443811714303e-010\\
12.135	-2.4988178193297e-010\\
12.155	-2.86711460129041e-010\\
12.175	-3.12079195818882e-010\\
12.195	-3.25653642940793e-010\\
12.215	-3.27625898632888e-010\\
12.235	-3.18698512025151e-010\\
12.255	-3.00009989073757e-010\\
12.275	-2.73145311924239e-010\\
12.295	-2.39982533845051e-010\\
12.315	-2.02728653309059e-010\\
12.335	-1.63673977149337e-010\\
12.355	-1.25157662012043e-010\\
12.375	-8.93987106564964e-011\\
12.395	-5.83528503295128e-011\\
12.415	-3.35415480112733e-011\\
12.435	-1.58179771364564e-011\\
12.455	-5.22792789961479e-012\\
12.475	-7.68054511050137e-013\\
12.495	-7.31446207855729e-013\\
12.515	-3.60931206797033e-012\\
12.535	-8.17167600947188e-012\\
12.555	-1.34382192873428e-011\\
12.575	-1.8650530772546e-011\\
12.595	-2.32457161158939e-011\\
12.615	-2.68314103574196e-011\\
12.635	-2.91621934933595e-011\\
12.655	-3.01173391781262e-011\\
12.675	-2.96801160037941e-011\\
12.695	-2.79182683915469e-011\\
12.715	-2.49658054646273e-011\\
12.735	-2.10067414851967e-011\\
12.755	-1.62595631403306e-011\\
12.775	-1.09634142042569e-011\\
12.795	-5.36542685281027e-012\\
12.815	2.89968759724021e-013\\
12.835	5.77073762039748e-012\\
12.855	1.08663061187952e-011\\
12.875	1.5393767857641e-011\\
12.895	1.92034669416197e-011\\
12.915	2.21832621716267e-011\\
12.935	2.42616968182396e-011\\
12.955	2.54090689461028e-011\\
12.975	2.5635727915474e-011\\
12.995	2.49952870901016e-011\\
13.015	2.35754263333554e-011\\
13.035	2.15005530740386e-011\\
13.055	1.89174995113284e-011\\
13.075	1.60011257716031e-011\\
13.095	1.29337599658053e-011\\
13.115	9.90005386697357e-012\\
13.135	7.07806426317181e-012\\
13.155	4.62441239815203e-012\\
13.175	2.66072363794279e-012\\
13.195	1.25568284069233e-012\\
13.215	4.15522949165012e-013\\
13.235	7.11089478500071e-006\\
13.255	5.54653197468724e-005\\
13.275	0.000182478921487927\\
13.295	0.000421557226218283\\
13.315	0.000802277296315879\\
13.335	0.00135056767612696\\
13.355	0.00208888668566942\\
13.375	0.00303639750927687\\
13.395	0.00420913891866803\\
13.415	0.00562019506469369\\
13.435	0.00727985799312592\\
13.455	0.0091957887634635\\
13.475	0.0113731659948826\\
13.495	0.013814851641655\\
13.515	0.0165215060114861\\
13.535	0.0194917637854815\\
13.555	0.0227223467081785\\
13.575	0.0262081921100616\\
13.595	0.029942600056529\\
13.615	0.033917348831892\\
13.635	0.0381227880716324\\
13.655	0.0425479970872402\\
13.675	0.047180887311697\\
13.695	0.0520082786679268\\
13.715	0.0570160485804081\\
13.735	0.0621892027556896\\
13.755	0.0675120204687119\\
13.775	0.0729681327939034\\
13.795	0.0785406082868576\\
13.815	0.084212027490139\\
13.835	0.0899646505713463\\
13.855	0.0957804247736931\\
13.875	0.101641118526459\\
13.895	0.107528433203697\\
13.915	0.113423995673656\\
13.935	0.119309455156326\\
13.955	0.125166684389114\\
13.975	0.130977541208267\\
13.995	0.136406496167183\\
14.015	0.142031162977219\\
14.035	0.147535338997841\\
14.055	0.152894884347916\\
14.075	0.15808779001236\\
14.095	0.163094058632851\\
14.115	0.167895764112473\\
14.135	0.172476902604103\\
14.155	0.17682321369648\\
14.175	0.180922091007233\\
14.195	0.184762582182884\\
14.215	0.18833515048027\\
14.235	0.191631868481636\\
14.255	0.194645926356316\\
14.275	0.19737184047699\\
14.295	0.19980525970459\\
14.315	0.201943024992943\\
14.335	0.20378290116787\\
14.355	0.205323666334152\\
14.375	0.206564962863922\\
14.395	0.207507327198982\\
14.415	0.208152055740356\\
14.435	0.208501085639\\
14.455	0.208557337522507\\
14.475	0.208324044942856\\
14.495	0.207805350422859\\
14.515	0.207005798816681\\
14.535	0.205930560827255\\
14.555	0.204585358500481\\
14.575	0.20297634601593\\
14.595	0.201110184192657\\
14.615	0.198994070291519\\
14.635	0.196635484695435\\
14.655	0.194042444229126\\
14.675	0.191223278641701\\
14.695	0.188186764717102\\
14.715	0.184941962361336\\
14.735	0.18149833381176\\
14.755	0.176788702607155\\
14.775	0.17297925055027\\
14.795	0.169001191854477\\
14.815	0.164864420890808\\
14.835	0.160579264163971\\
14.855	0.156156376004219\\
14.875	0.151606753468513\\
14.895	0.146941632032394\\
14.915	0.14217247068882\\
14.935	0.137311056256294\\
14.955	0.132369220256805\\
14.975	0.127359047532082\\
14.995	0.122292689979076\\
15.015	0.117182396352291\\
15.035	0.112040415406227\\
15.055	0.106879092752934\\
15.075	0.101710684597492\\
15.095	0.0965473502874374\\
15.115	0.091401219367981\\
15.135	0.0862842500209808\\
15.155	0.0812081769108772\\
15.175	0.0761845707893372\\
15.195	0.0712247714400291\\
15.215	0.0663397610187531\\
15.235	0.0615402534604073\\
15.255	0.0568365901708603\\
15.275	0.0522387251257896\\
15.295	0.0477561727166176\\
15.315	0.0433980152010918\\
15.335	0.0391728430986404\\
15.355	0.035088736563921\\
15.375	0.0311532374471426\\
15.395	0.0273733232170343\\
15.415	0.0237553585320711\\
15.435	0.0203051194548607\\
15.455	0.0170277412980795\\
15.475	0.0139276962727308\\
15.495	0.0110087618231773\\
15.515	0.0081846984103322\\
15.535	0.00565952435135841\\
15.555	0.00331616448238492\\
15.575	0.00115431612357497\\
15.595	-0.000826266419608146\\
15.615	-0.00262590730562806\\
15.635	-0.00424513639882207\\
15.655	-0.00568480882793665\\
15.675	-0.00694622006267309\\
15.695	-0.00803122390061617\\
15.715	-0.00894230883568525\\
15.735	-0.00968271773308516\\
15.755	-0.0102565139532089\\
15.775	-0.0106686530634761\\
15.795	-0.0109250554814935\\
15.815	-0.0110326334834099\\
15.835	-0.0109993601217866\\
15.855	-0.0108342589810491\\
15.875	-0.0105474162846804\\
15.895	-0.0101499818265438\\
15.915	-0.00965411495417356\\
15.935	-0.00907294452190399\\
15.955	-0.00842047110199928\\
15.975	-0.00771148782223463\\
15.995	-0.00696138385683298\\
16.015	-0.00618606200441718\\
16.035	-0.00540165742859244\\
16.055	-0.00462430529296398\\
16.075	-0.00386987696401775\\
16.095	-0.00315359351225197\\
16.115	-0.00248968880623579\\
16.135	-0.00189089181367308\\
16.155	-0.00136796792503446\\
16.175	-0.0009290108573623\\
16.195	-0.000578948180191219\\
16.215	-0.000318634876748547\\
16.235	-0.000143963086884469\\
16.255	-4.50333936896641e-005\\
16.275	-5.89983619647683e-006\\
16.295	-1.06570314528653e-005\\
16.315	-4.50731531600468e-005\\
16.335	-9.76831361185759e-005\\
16.355	-0.000158896058565006\\
16.375	-0.000220867965254001\\
16.395	-0.000277374056167901\\
16.415	-0.00032368052052334\\
16.435	-0.000356416014255956\\
16.455	-0.000373441784176975\\
16.475	-0.000373723421944305\\
16.495	-0.0003572006826289\\
16.515	-0.000324657128658146\\
16.535	-0.000277590937912464\\
16.555	-0.000218089873669669\\
16.575	-0.000148696504766122\\
16.595	-7.22860640962608e-005\\
16.615	8.06006573839113e-006\\
16.635	8.91638774191961e-005\\
16.655	0.000167900157975964\\
16.675	0.000241268819081597\\
16.695	0.000306540110614151\\
16.715	0.000361333339242265\\
16.735	0.000403713405830786\\
16.755	0.000432273693149909\\
16.775	0.000446201884187758\\
16.795	0.000445357873104513\\
16.815	0.000445357873104513\\
16.835	0.000402054429287091\\
16.855	0.000362669583410025\\
16.875	0.000314643431920558\\
16.895	0.000260955217527226\\
16.915	0.000395966984797269\\
16.935	0.000354780757334083\\
16.955	0.000310055416775867\\
16.975	0.000308884307742119\\
16.995	0.00032139485119842\\
17.015	0.000346392975188792\\
17.035	0.000382284662919119\\
17.055	0.00042728814878501\\
17.075	0.000479524489492178\\
17.095	0.00053709011990577\\
17.115	0.000598414218984544\\
17.135	0.000661806319840252\\
17.155	0.000725943071302027\\
17.175	0.000789588259067386\\
17.195	0.000851692515425384\\
17.215	0.00091155874542892\\
17.235	0.000968438340350986\\
17.255	0.00102195516228676\\
17.275	0.0010714567033574\\
17.295	0.00111698743421584\\
17.315	0.00115833594463766\\
17.335	0.00119544088374823\\
17.355	0.00122839980758727\\
17.375	0.00125725276302546\\
17.395	0.00128220289479941\\
17.415	0.00130347267258912\\
17.435	0.00132117536850274\\
17.455	0.00133565266150981\\
17.475	0.00134706881362945\\
17.495	0.00135569646954536\\
17.515	0.00136183353606611\\
17.535	0.00136566685978323\\
17.555	0.00136747187934816\\
17.575	0.00136745814234018\\
17.595	0.00136585638392717\\
17.615	0.00136287335772067\\
17.635	0.00135871395468712\\
17.655	0.00135351740755141\\
17.675	0.00134746788535267\\
17.695	0.00134073221124709\\
17.715	0.00134073593653739\\
17.735	0.00133345543872565\\
17.755	0.00132581999059767\\
17.775	0.00131788267754018\\
17.795	0.00130978063680232\\
17.815	0.00130156718660146\\
17.835	0.00129341334104538\\
17.855	0.00128529372159392\\
17.875	0.00127727561630309\\
17.895	0.00126942049246281\\
17.915	0.0012617512838915\\
17.935	0.00125438219401985\\
17.955	0.00124715641140938\\
17.975	0.00124024273827672\\
17.995	0.00123356969561428\\
18.015	0.00122723588719964\\
18.035	0.00122123176697642\\
18.055	0.00121539365500212\\
18.075	0.00120999908540398\\
18.095	0.00120484072249383\\
18.115	0.00120008469093591\\
18.135	0.00119547406211495\\
18.155	0.0011912111658603\\
18.175	0.00118722801562399\\
18.195	0.00118349294643849\\
18.215	0.00118009210564196\\
18.235	0.00117691198829561\\
18.255	0.00117383524775505\\
18.275	0.0011711084516719\\
18.295	0.00116861483547837\\
18.315	0.00116639828775078\\
18.335	0.00116429722402245\\
18.355	0.0011623544851318\\
18.375	0.00116061291191727\\
18.395	0.00115899136289954\\
18.415	0.00115765770897269\\
18.435	0.00115632126107812\\
18.455	0.00115519156679511\\
18.475	0.00115519133396447\\
18.495	0.00115418538916856\\
18.515	0.00115326186642051\\
18.535	0.00115246325731277\\
18.555	0.00115178991109133\\
18.575	0.00115120003465563\\
18.595	0.00115060992538929\\
18.615	0.00115018791984767\\
18.635	0.00114985019899905\\
18.655	0.00114955450408161\\
18.675	0.00114921643398702\\
18.695	0.00114908965770155\\
18.715	0.00114892050623894\\
18.735	0.00114875135477632\\
18.755	0.00114870897959918\\
18.775	0.00114862446207553\\
18.795	0.00114862446207553\\
18.815	0.00114862446207553\\
18.835	0.00114866672083735\\
18.855	0.00114870897959918\\
18.875	0.00114870897959918\\
18.895	0.00114875135477632\\
18.915	0.00114883598871529\\
18.935	0.00114896288141608\\
18.955	0.00114904728252441\\
18.975	0.00114908965770155\\
18.995	0.0011491741752252\\
19.015	0.00114930095151067\\
19.035	0.00114934321027249\\
19.055	0.00114938546903431\\
19.075	0.00114959676284343\\
19.095	0.00114963902160525\\
19.115	0.00114976579789072\\
19.135	0.00114985019899905\\
19.155	0.00114989234134555\\
19.175	0.00114989234134555\\
19.195	0.00114997674245387\\
19.215	0.00115010351873934\\
19.235	0.00114997674245387\\
19.255	0.00114997674245387\\
19.275	0.00114993448369205\\
19.295	0.00114993448369205\\
19.315	0.00114989234134555\\
19.335	0.00114989234134555\\
19.355	0.00114985008258373\\
19.375	0.0011498078238219\\
19.395	0.0011497656814754\\
19.415	0.00114972342271358\\
19.435	0.00114972342271358\\
19.455	0.00114959664642811\\
19.475	0.00114955438766629\\
19.495	0.00114946987014264\\
19.515	0.00114942761138082\\
19.535	0.00114934309385717\\
19.555	0.0011492163175717\\
19.575	0.00114917405880988\\
19.595	0.00114913180004805\\
19.615	0.00114900490734726\\
19.635	0.00114892038982362\\
19.655	0.00114887801464647\\
19.675	0.00114879349712282\\
19.695	0.00114866660442203\\
19.715	0.00114853971172124\\
19.735	0.00114853971172124\\
19.755	0.00114841270260513\\
19.775	0.00114832806866616\\
19.795	0.00114820117596537\\
19.815	0.00114807416684926\\
19.835	0.00114807416684926\\
19.855	0.00114794704131782\\
19.875	0.00114786229096353\\
19.895	0.00114777765702456\\
19.915	0.00114773528184742\\
19.935	0.00114760815631598\\
19.955	0.00114756578113884\\
19.975	0.0011474386556074\\
};
\end{axis}

\begin{axis}[%
width=0.410625\figurewidth,
height=0.264706\figureheight,
at={(0.540296\figurewidth,0.735294\figureheight)},
scale only axis,
separate axis lines,
every outer x axis line/.append style={black},
every x tick label/.append style={font=\color{black}},
xmin=11,
xmax=20,
xmajorgrids,
every outer y axis line/.append style={black},
every y tick label/.append style={font=\color{black}},
ymin=1,
ymax=6,
ylabel={$y$ [m]},
ylabel near ticks,
ymajorgrids
]
\addplot [color=light-gray,solid,forget plot,line width=1.0]
  table[row sep=crcr]{%
10.975	1.88448309898376\\
10.995	1.88198530673981\\
11.015	1.87989151477814\\
11.035	1.88133335113525\\
11.055	1.88074922561646\\
11.075	1.87928700447083\\
11.095	1.87605631351471\\
11.115	1.87543427944183\\
11.135	1.87308669090271\\
11.155	1.86944460868835\\
11.175	1.86716306209564\\
11.195	1.86639821529388\\
11.215	1.86944985389709\\
11.235	1.86734127998352\\
11.255	1.86359107494354\\
11.275	1.86056315898895\\
11.295	1.86197769641876\\
11.315	1.87416565418243\\
11.335	1.87080204486847\\
11.355	1.86937379837036\\
11.375	1.86870431900024\\
11.395	1.86803483963013\\
11.415	1.86923038959503\\
11.435	1.8685622215271\\
11.455	1.86925482749939\\
11.475	1.86794412136078\\
11.495	1.86677980422974\\
11.515	1.86283087730408\\
11.535	1.87251234054565\\
11.555	1.8726886510849\\
11.575	1.87416100502014\\
11.595	1.87264466285706\\
11.615	1.87314414978027\\
11.635	1.87248837947845\\
11.655	1.87353038787842\\
11.675	1.8705837726593\\
11.695	1.87747728824615\\
11.715	1.87817287445068\\
11.735	1.87724053859711\\
11.755	1.87839925289154\\
11.775	1.87776947021484\\
11.795	1.8779137134552\\
11.815	1.8755886554718\\
11.835	1.88564395904541\\
11.855	1.88481259346008\\
11.875	1.8865202665329\\
11.895	1.88760018348694\\
11.915	1.89003562927246\\
11.935	1.88846278190613\\
11.955	1.88875257968903\\
11.975	1.89743912220001\\
11.995	1.89814352989197\\
12.015	1.89676558971405\\
12.035	1.89720582962036\\
12.055	1.89777278900146\\
12.075	1.89887118339539\\
12.095	1.90021562576294\\
12.115	1.89871096611023\\
12.135	1.90873229503632\\
12.155	1.90695405006409\\
12.175	1.90813779830933\\
12.195	1.90953624248505\\
12.215	1.91233575344086\\
12.235	1.91125762462616\\
12.255	1.91203129291534\\
12.275	1.91099488735199\\
12.295	1.91180837154388\\
12.315	1.90966844558716\\
12.335	1.90926241874695\\
12.355	1.91989660263062\\
12.375	1.92344999313354\\
12.395	1.92257690429688\\
12.415	1.92216467857361\\
12.435	1.92118954658508\\
12.455	1.92331218719482\\
12.475	1.9231595993042\\
12.495	1.92421174049377\\
12.515	1.92358243465424\\
12.535	1.92355823516846\\
12.555	1.9253762960434\\
12.575	1.92482030391693\\
12.595	1.92613220214844\\
12.615	1.92315089702606\\
12.635	1.92180585861206\\
12.655	1.92251300811768\\
12.675	1.92609918117523\\
12.695	1.92706429958344\\
12.715	1.92959606647491\\
12.735	1.92878210544586\\
12.755	1.9284690618515\\
12.775	1.92756998538971\\
12.795	1.93058109283447\\
12.815	1.93067693710327\\
12.835	1.92945241928101\\
12.855	1.92687201499939\\
12.875	1.92013847827911\\
12.895	1.92128574848175\\
12.915	1.923788189888\\
12.935	1.924360871315\\
12.955	1.92391300201416\\
12.975	1.9210946559906\\
12.995	1.92061746120453\\
13.015	1.92312157154083\\
13.035	1.92385613918304\\
13.055	1.92662966251373\\
13.075	1.92757058143616\\
13.095	1.93057012557983\\
13.115	1.93034601211548\\
13.135	1.92826187610626\\
13.155	1.91511619091034\\
13.175	1.91723847389221\\
13.195	1.91746401786804\\
13.215	1.9202972650528\\
13.235	1.92121815681458\\
13.255	1.92381453514099\\
13.275	1.92450058460236\\
13.295	1.92705261707306\\
13.315	1.92781364917755\\
13.335	1.92791068553925\\
13.355	1.91476666927338\\
13.375	1.91581594944\\
13.395	1.91812992095947\\
13.415	1.92310988903046\\
13.435	1.92714238166809\\
13.455	1.9321436882019\\
13.475	1.94005215167999\\
13.495	1.94733560085297\\
13.515	1.95748448371887\\
13.535	1.95749878883362\\
13.555	1.97138690948486\\
13.575	1.98528957366943\\
13.595	1.99100577831268\\
13.615	2.00853824615479\\
13.635	2.01752400398254\\
13.655	2.02513241767883\\
13.675	2.04696702957153\\
13.695	2.06399059295654\\
13.715	2.07401847839355\\
13.735	2.09700918197632\\
13.755	2.12280559539795\\
13.775	2.15115022659302\\
13.795	2.17078471183777\\
13.815	2.19525694847107\\
13.835	2.23262166976929\\
13.855	2.26113224029541\\
13.875	2.29453873634338\\
13.895	2.32945466041565\\
13.915	2.35785293579102\\
13.935	2.39892911911011\\
13.955	2.44260907173157\\
13.975	2.4796085357666\\
13.995	2.53012537956238\\
14.015	2.57188677787781\\
14.035	2.61767196655273\\
14.055	2.66427731513977\\
14.075	2.7154004573822\\
14.095	2.76724028587341\\
14.115	2.81168508529663\\
14.135	2.86825656890869\\
14.155	2.91531157493591\\
14.175	2.97604250907898\\
14.195	3.02879190444946\\
14.215	3.08138418197632\\
14.235	3.13549542427063\\
14.255	3.19095420837402\\
14.275	3.24758958816528\\
14.295	3.30522561073303\\
14.315	3.36349987983704\\
14.335	3.42057943344116\\
14.355	3.47977352142334\\
14.375	3.5274338722229\\
14.395	3.5850613117218\\
14.415	3.6427538394928\\
14.435	3.70023488998413\\
14.455	3.75742173194885\\
14.475	3.81412959098816\\
14.495	3.8602728843689\\
14.515	3.91327381134033\\
14.535	3.97693061828613\\
14.555	4.02781629562378\\
14.575	4.07744026184082\\
14.595	4.12566661834717\\
14.615	4.17239379882813\\
14.635	4.22731590270996\\
14.655	4.27059650421143\\
14.675	4.31193065643311\\
14.695	4.35933446884155\\
14.715	4.39654302597046\\
14.735	4.43960237503052\\
14.755	4.48046541213989\\
14.775	4.51934909820557\\
14.795	4.55589056015015\\
14.815	4.59159517288208\\
14.835	4.62661266326904\\
14.855	4.65717220306396\\
14.875	4.69429922103882\\
14.895	4.71586751937866\\
14.915	4.7464804649353\\
14.935	4.77322006225586\\
14.955	4.79759359359741\\
14.975	4.81936407089233\\
14.995	4.83790636062622\\
15.015	4.86352682113647\\
15.035	4.87673711776733\\
15.055	4.89603996276855\\
15.075	4.91472005844116\\
15.095	4.92910575866699\\
15.115	4.94057703018188\\
15.135	4.95135164260864\\
15.155	4.96805286407471\\
15.175	4.97597312927246\\
15.195	4.98948240280151\\
15.215	5.00273180007935\\
15.235	5.00895166397095\\
15.255	5.01462650299072\\
15.275	5.01653575897217\\
15.295	5.01646995544434\\
15.315	5.02431011199951\\
15.335	5.02056264877319\\
15.355	5.024986743927\\
15.375	5.02604246139526\\
15.395	5.02559614181519\\
15.415	5.0255069732666\\
15.435	5.02243041992188\\
15.455	5.01816368103027\\
15.475	5.02257251739502\\
15.495	5.01534748077393\\
15.515	5.02030372619629\\
15.535	5.02239847183228\\
15.555	5.0236554145813\\
15.575	5.01254844665527\\
15.595	5.00997591018677\\
15.615	5.00679159164429\\
15.635	5.00306844711304\\
15.655	5.0066556930542\\
15.675	5.00157737731934\\
15.695	4.99440002441406\\
15.715	4.98694658279419\\
15.735	4.98923301696777\\
15.755	4.97982311248779\\
15.775	4.97208213806152\\
15.795	4.97234773635864\\
15.815	4.97307014465332\\
15.835	4.96625280380249\\
15.855	4.96565008163452\\
15.875	4.96699237823486\\
15.895	4.95888948440552\\
15.915	4.95923900604248\\
15.935	4.95976638793945\\
15.955	4.96048784255981\\
15.975	4.95159578323364\\
15.995	4.95192623138428\\
16.015	4.94998931884766\\
16.035	4.94826936721802\\
16.055	4.94852018356323\\
16.075	4.94544887542725\\
16.095	4.93451547622681\\
16.115	4.93377351760864\\
16.135	4.93140316009521\\
16.155	4.93094921112061\\
16.175	4.93866539001465\\
16.195	4.93663120269775\\
16.215	4.93303489685059\\
16.235	4.93133211135864\\
16.255	4.92785501480103\\
16.275	4.92619371414185\\
16.295	4.93257141113281\\
16.315	4.92922496795654\\
16.335	4.92672252655029\\
16.355	4.92230319976807\\
16.375	4.92959499359131\\
16.395	4.92515563964844\\
16.415	4.9305214881897\\
16.435	4.92606592178345\\
16.455	4.93113279342651\\
16.475	4.92559337615967\\
16.495	4.92805147171021\\
16.515	4.93227815628052\\
16.535	4.92490482330322\\
16.555	4.92738485336304\\
16.575	4.9298734664917\\
16.595	4.92253255844116\\
16.615	4.92326354980469\\
16.635	4.92628335952759\\
16.655	4.92748165130615\\
16.675	4.92069387435913\\
16.695	4.92190837860107\\
16.715	4.9231333732605\\
16.735	4.92436122894287\\
16.755	4.9237585067749\\
16.775	4.92499208450317\\
16.795	4.92438983917236\\
16.815	4.92296981811523\\
16.835	4.92545032501221\\
16.855	4.92503118515015\\
16.875	4.92460870742798\\
16.895	4.92230653762817\\
16.915	4.91883993148804\\
16.935	4.91654634475708\\
16.955	4.92597389221191\\
16.975	4.92367029190063\\
16.995	4.9213662147522\\
17.015	4.91906261444092\\
17.035	4.92658424377441\\
17.055	4.92414379119873\\
17.075	4.91972494125366\\
17.095	4.92682504653931\\
17.115	4.92232179641724\\
17.135	4.91958618164063\\
17.155	4.92493057250977\\
17.175	4.92043590545654\\
17.195	4.92578458786011\\
17.215	4.9212965965271\\
17.235	4.9266505241394\\
17.255	4.92216300964355\\
17.275	4.9257550239563\\
17.295	4.92126893997192\\
17.315	4.92486238479614\\
17.335	4.9203782081604\\
17.355	4.92396879196167\\
17.375	4.91771793365479\\
17.395	4.92307281494141\\
17.415	4.92666244506836\\
17.435	4.92040872573853\\
17.455	4.9239935874939\\
17.475	4.91769599914551\\
17.495	4.92139387130737\\
17.515	4.92508983612061\\
17.535	4.9171290397644\\
17.555	4.92082023620605\\
17.575	4.92450904846191\\
17.595	4.92637586593628\\
17.615	4.92022800445557\\
17.635	4.92209148406982\\
17.655	4.92394971847534\\
17.675	4.92762994766235\\
17.695	4.91965293884277\\
17.715	4.92147541046143\\
17.735	4.92332983016968\\
17.755	4.92700386047363\\
17.775	4.91902256011963\\
17.795	4.92087125778198\\
17.815	4.92271947860718\\
17.835	4.92456531524658\\
17.855	4.92640972137451\\
17.875	4.92825412750244\\
17.895	4.91844129562378\\
17.915	4.92028093338013\\
17.935	4.92212152481079\\
17.955	4.92395877838135\\
17.975	4.9257984161377\\
17.995	4.92581176757813\\
18.015	4.92764711380005\\
18.035	4.91964960098267\\
18.055	4.91966104507446\\
18.075	4.92149448394775\\
18.095	4.92150402069092\\
18.115	4.92333507537842\\
18.135	4.92334461212158\\
18.155	4.92517471313477\\
18.175	4.92518138885498\\
18.195	4.92701005935669\\
18.215	4.92701768875122\\
18.235	4.92884588241577\\
18.255	4.91901779174805\\
18.275	4.91902256011963\\
18.295	4.92084980010986\\
18.315	4.92085361480713\\
18.335	4.92085647583008\\
18.355	4.92268180847168\\
18.375	4.92268562316895\\
18.395	4.92268848419189\\
18.415	4.92451238632202\\
18.435	4.92451429367065\\
18.455	4.92451620101929\\
18.475	4.92633867263794\\
18.495	4.92634057998657\\
18.515	4.92634153366089\\
18.535	4.92634344100952\\
18.555	4.92816638946533\\
18.575	4.9183349609375\\
18.595	4.91833591461182\\
18.615	4.92015933990479\\
18.635	4.92015933990479\\
18.655	4.9201602935791\\
18.675	4.92198324203491\\
18.695	4.92198324203491\\
18.715	4.92198324203491\\
18.735	4.92380571365356\\
18.755	4.92380571365356\\
18.775	4.92380666732788\\
18.795	4.92562913894653\\
18.815	4.92562913894653\\
18.835	4.92745018005371\\
18.855	4.91761779785156\\
18.875	4.91761779785156\\
18.895	4.91943979263306\\
18.915	4.91943979263306\\
18.935	4.92126226425171\\
18.955	4.92126226425171\\
18.975	4.92308473587036\\
18.995	4.92308473587036\\
19.015	4.92490577697754\\
19.035	4.92490577697754\\
19.055	4.92672824859619\\
19.075	4.92672824859619\\
19.095	4.92672824859619\\
19.115	4.91871786117554\\
19.135	4.91871690750122\\
19.155	4.92056226730347\\
19.175	4.92056226730347\\
19.195	4.92242622375488\\
19.215	4.92242622375488\\
19.235	4.92428970336914\\
19.255	4.92428970336914\\
19.275	4.92428970336914\\
19.295	4.92615365982056\\
19.315	4.92615365982056\\
19.335	4.92801761627197\\
19.355	4.92801761627197\\
19.375	4.92801761627197\\
19.395	4.92988157272339\\
19.415	4.92005681991577\\
19.435	4.92005681991577\\
19.455	4.9219217300415\\
19.475	4.9219217300415\\
19.495	4.9219217300415\\
19.515	4.92378568649292\\
19.535	4.92378568649292\\
19.555	4.92378568649292\\
19.575	4.92378568649292\\
19.595	4.92564916610718\\
19.615	4.92565011978149\\
19.635	4.92565011978149\\
19.655	4.92565011978149\\
19.675	4.92751407623291\\
19.695	4.92751407623291\\
19.715	4.92751502990723\\
19.735	4.92751502990723\\
19.755	4.91955423355103\\
19.775	4.91955423355103\\
19.795	4.91955423355103\\
19.815	4.91955423355103\\
19.835	4.91955423355103\\
19.855	4.92141914367676\\
19.875	4.92141914367676\\
19.895	4.92141914367676\\
19.915	4.92141914367676\\
19.935	4.92141914367676\\
19.955	4.92141914367676\\
19.975	4.92328310012817\\
};
\addplot [color=black,solid,forget plot, line width=1.0]
  table[row sep=crcr]{%
10.975	1.87499988079071\\
10.995	1.87499988079071\\
11.015	1.87499988079071\\
11.035	1.875\\
11.055	1.87499988079071\\
11.075	1.87499988079071\\
11.095	1.87499988079071\\
11.115	1.87500011920929\\
11.135	1.87499988079071\\
11.155	1.875\\
11.175	1.87499988079071\\
11.195	1.875\\
11.215	1.87499976158142\\
11.235	1.87499988079071\\
11.255	1.875\\
11.275	1.875\\
11.295	1.875\\
11.315	1.875\\
11.335	1.87499988079071\\
11.355	1.87499988079071\\
11.375	1.875\\
11.395	1.875\\
11.415	1.875\\
11.435	1.875\\
11.455	1.87499988079071\\
11.475	1.87500011920929\\
11.495	1.87500011920929\\
11.515	1.875\\
11.535	1.875\\
11.555	1.875\\
11.575	1.875\\
11.595	1.875\\
11.615	1.875\\
11.635	1.875\\
11.655	1.875\\
11.675	1.875\\
11.695	1.875\\
11.715	1.875\\
11.735	1.875\\
11.755	1.875\\
11.775	1.875\\
11.795	1.875\\
11.815	1.875\\
11.835	1.875\\
11.855	1.875\\
11.875	1.875\\
11.895	1.875\\
11.915	1.875\\
11.935	1.875\\
11.955	1.875\\
11.975	1.875\\
11.995	1.875\\
12.015	1.875\\
12.035	1.87499988079071\\
12.055	1.875\\
12.075	1.875\\
12.095	1.875\\
12.115	1.875\\
12.135	1.875\\
12.155	1.875\\
12.175	1.875\\
12.195	1.875\\
12.215	1.875\\
12.235	1.87500011920929\\
12.255	1.87499988079071\\
12.275	1.875\\
12.295	1.87500011920929\\
12.315	1.875\\
12.335	1.87500011920929\\
12.355	1.875\\
12.375	1.875\\
12.395	1.875\\
12.415	1.875\\
12.435	1.875\\
12.455	1.875\\
12.475	1.875\\
12.495	1.875\\
12.515	1.875\\
12.535	1.875\\
12.555	1.875\\
12.575	1.875\\
12.595	1.875\\
12.615	1.875\\
12.635	1.875\\
12.655	1.875\\
12.675	1.875\\
12.695	1.875\\
12.715	1.875\\
12.735	1.875\\
12.755	1.875\\
12.775	1.875\\
12.795	1.875\\
12.815	1.875\\
12.835	1.875\\
12.855	1.875\\
12.875	1.875\\
12.895	1.875\\
12.915	1.875\\
12.935	1.875\\
12.955	1.875\\
12.975	1.875\\
12.995	1.875\\
13.015	1.875\\
13.035	1.875\\
13.055	1.875\\
13.075	1.875\\
13.095	1.875\\
13.115	1.87499988079071\\
13.135	1.87499988079071\\
13.155	1.87500762939453\\
13.175	1.87503838539124\\
13.195	1.87511932849884\\
13.215	1.87528586387634\\
13.235	1.8755806684494\\
13.255	1.87605357170105\\
13.275	1.87676024436951\\
13.295	1.87776076793671\\
13.315	1.87911951541901\\
13.335	1.88090395927429\\
13.355	1.88318395614624\\
13.375	1.88603103160858\\
13.395	1.88951766490936\\
13.415	1.8937166929245\\
13.435	1.89870071411133\\
13.455	1.90454149246216\\
13.475	1.91130948066711\\
13.495	1.91907262802124\\
13.515	1.92789769172668\\
13.535	1.93784773349762\\
13.555	1.94898295402527\\
13.575	1.96135997772217\\
13.595	1.97503197193146\\
13.615	1.99004745483398\\
13.635	2.00645112991333\\
13.655	2.02428269386292\\
13.675	2.04357719421387\\
13.695	2.06436467170715\\
13.715	2.086669921875\\
13.735	2.11051249504089\\
13.755	2.13590741157532\\
13.775	2.16286373138428\\
13.795	2.1913845539093\\
13.815	2.22146844863892\\
13.835	2.25310754776001\\
13.855	2.28628993034363\\
13.875	2.32099771499634\\
13.895	2.35720682144165\\
13.915	2.39488768577576\\
13.935	2.43400263786316\\
13.955	2.47450566291809\\
13.975	2.5163459777832\\
13.995	2.55946612358093\\
14.015	2.6038031578064\\
14.035	2.64928913116455\\
14.055	2.69585275650024\\
14.075	2.7434184551239\\
14.095	2.791907787323\\
14.115	2.84123992919922\\
14.135	2.89133167266846\\
14.155	2.94209718704224\\
14.175	2.99345088005066\\
14.195	3.04530501365662\\
14.215	3.09757232666016\\
14.235	3.15016508102417\\
14.255	3.20299530029297\\
14.275	3.25597596168518\\
14.295	3.30902004241943\\
14.315	3.36204361915588\\
14.335	3.41496157646179\\
14.355	3.46769237518311\\
14.375	3.52015590667725\\
14.395	3.57227301597595\\
14.415	3.62396931648254\\
14.435	3.67516922950745\\
14.455	3.72580337524414\\
14.475	3.77580308914185\\
14.495	3.8251039981842\\
14.515	3.87364268302917\\
14.535	3.92136168479919\\
14.555	3.9682035446167\\
14.575	4.01411771774292\\
14.595	4.05905485153198\\
14.615	4.1029691696167\\
14.635	4.14581727981567\\
14.655	4.18756341934204\\
14.675	4.22817134857178\\
14.695	4.26760864257813\\
14.715	4.30584907531738\\
14.735	4.34286594390869\\
14.755	4.37863922119141\\
14.775	4.41315078735352\\
14.795	4.44638538360596\\
14.815	4.47833251953125\\
14.835	4.50898456573486\\
14.855	4.53833723068237\\
14.875	4.56638860702515\\
14.895	4.59314298629761\\
14.915	4.61860275268555\\
14.935	4.64277791976929\\
14.955	4.66567945480347\\
14.975	4.68732070922852\\
14.995	4.7077202796936\\
15.015	4.72689580917358\\
15.035	4.74486970901489\\
15.055	4.76166677474976\\
15.075	4.77731370925903\\
15.095	4.7918381690979\\
15.115	4.80527257919312\\
15.135	4.81764698028564\\
15.155	4.82899761199951\\
15.175	4.83935928344727\\
15.195	4.8487696647644\\
15.215	4.85726642608643\\
15.235	4.86488962173462\\
15.255	4.87167882919312\\
15.275	4.87767505645752\\
15.295	4.88292026519775\\
15.315	4.88745594024658\\
15.335	4.89132452011108\\
15.355	4.89456796646118\\
15.375	4.89722967147827\\
15.395	4.8993501663208\\
15.415	4.90097188949585\\
15.435	4.90213584899902\\
15.455	4.90288114547729\\
15.475	4.90324735641479\\
15.495	4.90327262878418\\
15.515	4.90299224853516\\
15.535	4.90244245529175\\
15.555	4.9016580581665\\
15.575	4.90067338943481\\
15.595	4.89951992034912\\
15.615	4.89822816848755\\
15.635	4.89682865142822\\
15.655	4.895348072052\\
15.675	4.89381456375122\\
15.695	4.89225149154663\\
15.715	4.89068365097046\\
15.735	4.88913011550903\\
15.755	4.88761043548584\\
15.775	4.88614273071289\\
15.795	4.88474130630493\\
15.815	4.88341856002808\\
15.835	4.8821849822998\\
15.855	4.88104820251465\\
15.875	4.88001489639282\\
15.895	4.87908983230591\\
15.915	4.87827205657959\\
15.935	4.87756252288818\\
15.955	4.87695741653442\\
15.975	4.8764533996582\\
15.995	4.87604284286499\\
16.015	4.87571811676025\\
16.035	4.87546968460083\\
16.055	4.87528896331787\\
16.075	4.87516355514526\\
16.095	4.87508296966553\\
16.115	4.87503528594971\\
16.135	4.87501239776611\\
16.155	4.87500286102295\\
16.175	4.87500095367432\\
16.195	4.87500047683716\\
16.215	4.87499713897705\\
16.235	4.87499094009399\\
16.255	4.87497854232788\\
16.275	4.87496089935303\\
16.295	4.87493848800659\\
16.315	4.87491035461426\\
16.335	4.87487983703613\\
16.355	4.87484836578369\\
16.375	4.87481546401978\\
16.395	4.87478542327881\\
16.415	4.87475681304932\\
16.435	4.87473249435425\\
16.455	4.87471342086792\\
16.475	4.87469863891602\\
16.495	4.87469053268433\\
16.515	4.87468862533569\\
16.535	4.87469148635864\\
16.555	4.87470102310181\\
16.575	4.87471532821655\\
16.595	4.87473297119141\\
16.615	4.874755859375\\
16.635	4.87478017807007\\
16.655	4.87480688095093\\
16.675	4.87483406066895\\
16.695	4.8748607635498\\
16.715	4.87488651275635\\
16.735	4.87491130828857\\
16.755	4.87493181228638\\
16.775	4.87495088577271\\
16.795	4.87496614456177\\
16.815	4.87497758865356\\
16.835	4.87346315383911\\
16.855	4.8734827041626\\
16.875	4.87351512908936\\
16.895	4.87352991104126\\
16.915	4.87354421615601\\
16.935	4.87355899810791\\
16.955	4.87357473373413\\
16.975	4.87359189987183\\
16.995	4.873610496521\\
17.015	4.8736310005188\\
17.035	4.87365341186523\\
17.055	4.87367725372314\\
17.075	4.87370204925537\\
17.095	4.87372875213623\\
17.115	4.87375593185425\\
17.135	4.87378597259521\\
17.155	4.87381553649902\\
17.175	4.87384605407715\\
17.195	4.87387704849243\\
17.215	4.87390804290771\\
17.235	4.87393999099731\\
17.255	4.87397193908691\\
17.275	4.87400245666504\\
17.295	4.87403345108032\\
17.315	4.87406349182129\\
17.335	4.87409353256226\\
17.355	4.87412309646606\\
17.375	4.87415075302124\\
17.395	4.87417793273926\\
17.415	4.87420463562012\\
17.435	4.87423038482666\\
17.455	4.87425518035889\\
17.475	4.87427854537964\\
17.495	4.87430047988892\\
17.515	4.87432289123535\\
17.535	4.874342918396\\
17.555	4.8743634223938\\
17.575	4.87438154220581\\
17.595	4.87439918518066\\
17.615	4.8744158744812\\
17.635	4.8744158744812\\
17.655	4.87443208694458\\
17.675	4.87444686889648\\
17.695	4.87446117401123\\
17.715	4.87447452545166\\
17.735	4.87448740005493\\
17.755	4.87449932098389\\
17.775	4.87451028823853\\
17.795	4.87452125549316\\
17.815	4.87453126907349\\
17.835	4.87453985214233\\
17.855	4.87454891204834\\
17.875	4.87455654144287\\
17.895	4.87456512451172\\
17.915	4.87457180023193\\
17.935	4.87457847595215\\
17.955	4.87458467483521\\
17.975	4.8745903968811\\
17.995	4.874596118927\\
18.015	4.87460088729858\\
18.035	4.87460517883301\\
18.055	4.87460994720459\\
18.075	4.87461376190186\\
18.095	4.87461709976196\\
18.115	4.87462043762207\\
18.135	4.87462425231934\\
18.155	4.87462711334229\\
18.175	4.87462949752808\\
18.195	4.87463188171387\\
18.215	4.87463426589966\\
18.235	4.87463617324829\\
18.255	4.87463760375977\\
18.275	4.87463903427124\\
18.295	4.87464094161987\\
18.315	4.87464237213135\\
18.335	4.87464332580566\\
18.355	4.87464427947998\\
18.375	4.8746452331543\\
18.395	4.8746452331543\\
18.415	4.87464618682861\\
18.435	4.87464666366577\\
18.455	4.87464761734009\\
18.475	4.87464809417725\\
18.495	4.8746485710144\\
18.515	4.87464904785156\\
18.535	4.87464952468872\\
18.555	4.87464952468872\\
18.575	4.87465000152588\\
18.595	4.87465047836304\\
18.615	4.87465047836304\\
18.635	4.87465047836304\\
18.655	4.87465047836304\\
18.675	4.87465047836304\\
18.695	4.8746509552002\\
18.715	4.8746509552002\\
18.735	4.8746509552002\\
18.755	4.87465047836304\\
18.775	4.87465047836304\\
18.795	4.87465047836304\\
18.815	4.87465047836304\\
18.835	4.87465047836304\\
18.855	4.87465047836304\\
18.875	4.87465047836304\\
18.895	4.87465047836304\\
18.915	4.87465047836304\\
18.935	4.87465000152588\\
18.955	4.87465000152588\\
18.975	4.87465000152588\\
18.995	4.87465000152588\\
19.015	4.87465000152588\\
19.035	4.87465000152588\\
19.055	4.87464952468872\\
19.075	4.87464952468872\\
19.095	4.87464952468872\\
19.115	4.87464952468872\\
19.135	4.87464952468872\\
19.155	4.87464952468872\\
19.175	4.87464952468872\\
19.195	4.87464952468872\\
19.215	4.87464952468872\\
19.235	4.87464952468872\\
19.255	4.87464952468872\\
19.275	4.87464952468872\\
19.295	4.87464952468872\\
19.315	4.87464952468872\\
19.335	4.87464952468872\\
19.355	4.87464952468872\\
19.375	4.87465000152588\\
19.395	4.87465000152588\\
19.415	4.87465000152588\\
19.435	4.87465000152588\\
19.455	4.87465000152588\\
19.475	4.87465000152588\\
19.495	4.87465000152588\\
19.515	4.87465000152588\\
19.535	4.87465047836304\\
19.555	4.87465047836304\\
19.575	4.87465047836304\\
19.595	4.87465047836304\\
19.615	4.87465047836304\\
19.635	4.8746509552002\\
19.655	4.8746509552002\\
19.675	4.8746509552002\\
19.695	4.8746509552002\\
19.715	4.8746509552002\\
19.735	4.8746509552002\\
19.755	4.8746509552002\\
19.775	4.87465143203735\\
19.795	4.87465143203735\\
19.815	4.87465143203735\\
19.835	4.87465143203735\\
19.855	4.87465143203735\\
19.875	4.87465143203735\\
19.895	4.87465143203735\\
19.915	4.87465143203735\\
19.935	4.87465143203735\\
19.955	4.87465143203735\\
19.975	4.87465190887451\\
};
\addplot [color=black,dashed,forget plot, line width=1.0]
  table[row sep=crcr]{%
10.975	1.875\\
10.995	1.875\\
11.015	1.875\\
11.035	1.875\\
11.055	1.875\\
11.075	1.875\\
11.095	1.875\\
11.115	1.875\\
11.135	1.875\\
11.155	1.875\\
11.175	1.875\\
11.195	1.875\\
11.215	1.875\\
11.235	1.875\\
11.255	1.875\\
11.275	1.875\\
11.295	1.875\\
11.315	1.875\\
11.335	1.875\\
11.355	1.875\\
11.375	1.875\\
11.395	1.875\\
11.415	1.875\\
11.435	1.875\\
11.455	1.875\\
11.475	1.875\\
11.495	1.875\\
11.515	1.875\\
11.535	1.875\\
11.555	1.875\\
11.575	1.875\\
11.595	1.875\\
11.615	1.875\\
11.635	1.875\\
11.655	1.875\\
11.675	1.875\\
11.695	1.875\\
11.715	1.875\\
11.735	1.875\\
11.755	1.875\\
11.775	1.875\\
11.795	1.875\\
11.815	1.875\\
11.835	1.875\\
11.855	1.875\\
11.875	1.875\\
11.895	1.875\\
11.915	1.875\\
11.935	1.875\\
11.955	1.875\\
11.975	1.875\\
11.995	1.875\\
12.015	1.875\\
12.035	1.875\\
12.055	1.875\\
12.075	1.875\\
12.095	1.875\\
12.115	1.875\\
12.135	1.875\\
12.155	1.875\\
12.175	1.875\\
12.195	1.875\\
12.215	1.875\\
12.235	1.875\\
12.255	1.875\\
12.275	1.875\\
12.295	1.875\\
12.315	1.875\\
12.335	1.875\\
12.355	1.875\\
12.375	1.875\\
12.395	1.875\\
12.415	1.875\\
12.435	1.875\\
12.455	1.875\\
12.475	1.875\\
12.495	1.875\\
12.515	1.875\\
12.535	1.875\\
12.555	1.875\\
12.575	1.875\\
12.595	1.875\\
12.615	1.875\\
12.635	1.875\\
12.655	1.875\\
12.675	1.875\\
12.695	1.875\\
12.715	1.875\\
12.735	1.875\\
12.755	1.875\\
12.775	1.875\\
12.795	1.875\\
12.815	1.875\\
12.835	1.875\\
12.855	1.875\\
12.875	1.875\\
12.895	1.875\\
12.915	1.875\\
12.935	1.875\\
12.955	1.875\\
12.975	1.875\\
12.995	4.875\\
13.015	4.875\\
13.035	4.875\\
13.055	4.875\\
13.075	4.875\\
13.095	4.875\\
13.115	4.875\\
13.135	4.875\\
13.155	4.875\\
13.175	4.875\\
13.195	4.875\\
13.215	4.875\\
13.235	4.875\\
13.255	4.875\\
13.275	4.875\\
13.295	4.875\\
13.315	4.875\\
13.335	4.875\\
13.355	4.875\\
13.375	4.875\\
13.395	4.875\\
13.415	4.875\\
13.435	4.875\\
13.455	4.875\\
13.475	4.875\\
13.495	4.875\\
13.515	4.875\\
13.535	4.875\\
13.555	4.875\\
13.575	4.875\\
13.595	4.875\\
13.615	4.875\\
13.635	4.875\\
13.655	4.875\\
13.675	4.875\\
13.695	4.875\\
13.715	4.875\\
13.735	4.875\\
13.755	4.875\\
13.775	4.875\\
13.795	4.875\\
13.815	4.875\\
13.835	4.875\\
13.855	4.875\\
13.875	4.875\\
13.895	4.875\\
13.915	4.875\\
13.935	4.875\\
13.955	4.875\\
13.975	4.875\\
13.995	4.875\\
14.015	4.875\\
14.035	4.875\\
14.055	4.875\\
14.075	4.875\\
14.095	4.875\\
14.115	4.875\\
14.135	4.875\\
14.155	4.875\\
14.175	4.875\\
14.195	4.875\\
14.215	4.875\\
14.235	4.875\\
14.255	4.875\\
14.275	4.875\\
14.295	4.875\\
14.315	4.875\\
14.335	4.875\\
14.355	4.875\\
14.375	4.875\\
14.395	4.875\\
14.415	4.875\\
14.435	4.875\\
14.455	4.875\\
14.475	4.875\\
14.495	4.875\\
14.515	4.875\\
14.535	4.875\\
14.555	4.875\\
14.575	4.875\\
14.595	4.875\\
14.615	4.875\\
14.635	4.875\\
14.655	4.875\\
14.675	4.875\\
14.695	4.875\\
14.715	4.875\\
14.735	4.875\\
14.755	4.875\\
14.775	4.875\\
14.795	4.875\\
14.815	4.875\\
14.835	4.875\\
14.855	4.875\\
14.875	4.875\\
14.895	4.875\\
14.915	4.875\\
14.935	4.875\\
14.955	4.875\\
14.975	4.875\\
14.995	4.875\\
15.015	4.875\\
15.035	4.875\\
15.055	4.875\\
15.075	4.875\\
15.095	4.875\\
15.115	4.875\\
15.135	4.875\\
15.155	4.875\\
15.175	4.875\\
15.195	4.875\\
15.215	4.875\\
15.235	4.875\\
15.255	4.875\\
15.275	4.875\\
15.295	4.875\\
15.315	4.875\\
15.335	4.875\\
15.355	4.875\\
15.375	4.875\\
15.395	4.875\\
15.415	4.875\\
15.435	4.875\\
15.455	4.875\\
15.475	4.875\\
15.495	4.875\\
15.515	4.875\\
15.535	4.875\\
15.555	4.875\\
15.575	4.875\\
15.595	4.875\\
15.615	4.875\\
15.635	4.875\\
15.655	4.875\\
15.675	4.875\\
15.695	4.875\\
15.715	4.875\\
15.735	4.875\\
15.755	4.875\\
15.775	4.875\\
15.795	4.875\\
15.815	4.875\\
15.835	4.875\\
15.855	4.875\\
15.875	4.875\\
15.895	4.875\\
15.915	4.875\\
15.935	4.875\\
15.955	4.875\\
15.975	4.875\\
15.995	4.875\\
16.015	4.875\\
16.035	4.875\\
16.055	4.875\\
16.075	4.875\\
16.095	4.875\\
16.115	4.875\\
16.135	4.875\\
16.155	4.875\\
16.175	4.875\\
16.195	4.875\\
16.215	4.875\\
16.235	4.875\\
16.255	4.875\\
16.275	4.875\\
16.295	4.875\\
16.315	4.875\\
16.335	4.875\\
16.355	4.875\\
16.375	4.875\\
16.395	4.875\\
16.415	4.875\\
16.435	4.875\\
16.455	4.875\\
16.475	4.875\\
16.495	4.875\\
16.515	4.875\\
16.535	4.875\\
16.555	4.875\\
16.575	4.875\\
16.595	4.875\\
16.615	4.875\\
16.635	4.875\\
16.655	4.875\\
16.675	4.875\\
16.695	4.875\\
16.715	4.875\\
16.735	4.875\\
16.755	4.875\\
16.775	4.875\\
16.795	4.875\\
16.815	4.875\\
16.835	4.875\\
16.855	4.875\\
16.875	4.875\\
16.895	4.875\\
16.915	4.875\\
16.935	4.875\\
16.955	4.875\\
16.975	4.875\\
16.995	4.875\\
17.015	4.875\\
17.035	4.875\\
17.055	4.875\\
17.075	4.875\\
17.095	4.875\\
17.115	4.875\\
17.135	4.875\\
17.155	4.875\\
17.175	4.875\\
17.195	4.875\\
17.215	4.875\\
17.235	4.875\\
17.255	4.875\\
17.275	4.875\\
17.295	4.875\\
17.315	4.875\\
17.335	4.875\\
17.355	4.875\\
17.375	4.875\\
17.395	4.875\\
17.415	4.875\\
17.435	4.875\\
17.455	4.875\\
17.475	4.875\\
17.495	4.875\\
17.515	4.875\\
17.535	4.875\\
17.555	4.875\\
17.575	4.875\\
17.595	4.875\\
17.615	4.875\\
17.635	4.875\\
17.655	4.875\\
17.675	4.875\\
17.695	4.875\\
17.715	4.875\\
17.735	4.875\\
17.755	4.875\\
17.775	4.875\\
17.795	4.875\\
17.815	4.875\\
17.835	4.875\\
17.855	4.875\\
17.875	4.875\\
17.895	4.875\\
17.915	4.875\\
17.935	4.875\\
17.955	4.875\\
17.975	4.875\\
17.995	4.875\\
18.015	4.875\\
18.035	4.875\\
18.055	4.875\\
18.075	4.875\\
18.095	4.875\\
18.115	4.875\\
18.135	4.875\\
18.155	4.875\\
18.175	4.875\\
18.195	4.875\\
18.215	4.875\\
18.235	4.875\\
18.255	4.875\\
18.275	4.875\\
18.295	4.875\\
18.315	4.875\\
18.335	4.875\\
18.355	4.875\\
18.375	4.875\\
18.395	4.875\\
18.415	4.875\\
18.435	4.875\\
18.455	4.875\\
18.475	4.875\\
18.495	4.875\\
18.515	4.875\\
18.535	4.875\\
18.555	4.875\\
18.575	4.875\\
18.595	4.875\\
18.615	4.875\\
18.635	4.875\\
18.655	4.875\\
18.675	4.875\\
18.695	4.875\\
18.715	4.875\\
18.735	4.875\\
18.755	4.875\\
18.775	4.875\\
18.795	4.875\\
18.815	4.875\\
18.835	4.875\\
18.855	4.875\\
18.875	4.875\\
18.895	4.875\\
18.915	4.875\\
18.935	4.875\\
18.955	4.875\\
18.975	4.875\\
18.995	4.875\\
19.015	4.875\\
19.035	4.875\\
19.055	4.875\\
19.075	4.875\\
19.095	4.875\\
19.115	4.875\\
19.135	4.875\\
19.155	4.875\\
19.175	4.875\\
19.195	4.875\\
19.215	4.875\\
19.235	4.875\\
19.255	4.875\\
19.275	4.875\\
19.295	4.875\\
19.315	4.875\\
19.335	4.875\\
19.355	4.875\\
19.375	4.875\\
19.395	4.875\\
19.415	4.875\\
19.435	4.875\\
19.455	4.875\\
19.475	4.875\\
19.495	4.875\\
19.515	4.875\\
19.535	4.875\\
19.555	4.875\\
19.575	4.875\\
19.595	4.875\\
19.615	4.875\\
19.635	4.875\\
19.655	4.875\\
19.675	4.875\\
19.695	4.875\\
19.715	4.875\\
19.735	4.875\\
19.755	4.875\\
19.775	4.875\\
19.795	4.875\\
19.815	4.875\\
19.835	4.875\\
19.855	4.875\\
19.875	4.875\\
19.895	4.875\\
19.915	4.875\\
19.935	4.875\\
19.955	4.875\\
19.975	4.875\\
};
\end{axis}

\begin{axis}[%
width=0.410625\figurewidth,
height=0.264706\figureheight,
at={(0\figurewidth,0\figureheight)},
scale only axis,
every outer x axis line/.append style={black},
every x tick label/.append style={font=\color{black}},
xmin=11,
xmax=20,
xlabel={$t$ [s]},
xlabel near ticks,
xmajorgrids,
every outer y axis line/.append style={black},
every y tick label/.append style={font=\color{black}},
ymin=-5,
ymax=2,
ylabel={$a_x$ [m/s$^2$]},
ylabel near ticks,
ymajorgrids,
axis x line*=bottom,
axis y line*=left
]
\addplot [color=light-gray,solid,forget plot,line width=1.0]
  table[row sep=crcr]{%
10.975	-0.0919999999999987\\
10.995	-0.169999999999999\\
11.015	-0.125999999999999\\
11.035	-0.0899999999999987\\
11.055	-0.119999999999999\\
11.075	-0.119999999999999\\
11.095	-0.0999999999999986\\
11.115	-0.0199999999999986\\
11.135	-0.157999999999999\\
11.155	-0.109999999999999\\
11.175	-0.0219999999999986\\
11.195	-0.0559999999999986\\
11.215	-0.0999999999999986\\
11.235	-0.107999999999999\\
11.255	-0.00799999999999865\\
11.275	-0.0539999999999986\\
11.295	-0.0839999999999986\\
11.315	-0.00799999999999865\\
11.335	0.0660000000000013\\
11.355	-0.0339999999999986\\
11.375	0.120000000000001\\
11.395	0.178000000000001\\
11.415	0.140000000000001\\
11.435	0.128000000000001\\
11.455	0.0480000000000014\\
11.475	0.136000000000001\\
11.495	0.210000000000001\\
11.515	0.0560000000000014\\
11.535	0.0660000000000013\\
11.555	0.0500000000000014\\
11.575	0.0760000000000014\\
11.595	0.120000000000001\\
11.615	0.0520000000000014\\
11.635	0.144000000000001\\
11.655	-0.0739999999999987\\
11.675	0.0660000000000013\\
11.695	0.0800000000000013\\
11.715	-0.0399999999999986\\
11.735	0.0100000000000014\\
11.755	-0.0579999999999987\\
11.775	0.0100000000000014\\
11.795	-0.00999999999999865\\
11.815	-0.0399999999999986\\
11.835	-0.0459999999999986\\
11.855	-0.0339999999999986\\
11.875	-0.0759999999999987\\
11.895	-0.123999999999999\\
11.915	0.00600000000000135\\
11.935	-0.0259999999999986\\
11.955	-0.0439999999999987\\
11.975	-0.0459999999999986\\
11.995	-0.0179999999999986\\
12.015	0.00400000000000135\\
12.035	-0.0179999999999986\\
12.055	-0.00599999999999865\\
12.075	0.00600000000000135\\
12.095	1.35308431126191e-015\\
12.115	-0.0219999999999986\\
12.135	0.134000000000001\\
12.155	0.0640000000000014\\
12.175	0.0180000000000014\\
12.195	0.0380000000000014\\
12.215	0.0980000000000014\\
12.235	0.130000000000001\\
12.255	-0.00799999999999865\\
12.275	0.0680000000000014\\
12.295	0.0700000000000014\\
12.315	0.0260000000000014\\
12.335	0.0380000000000014\\
12.355	0.0460000000000014\\
12.375	0.0620000000000014\\
12.395	-0.00799999999999865\\
12.415	0.0200000000000014\\
12.435	0.0120000000000014\\
12.455	0.0320000000000014\\
12.475	0.0120000000000014\\
12.495	0.0160000000000014\\
12.515	-0.00999999999999865\\
12.535	0.0460000000000014\\
12.555	-0.0199999999999986\\
12.575	0.0120000000000014\\
12.595	0.0420000000000014\\
12.615	-0.00599999999999865\\
12.635	-0.0339999999999986\\
12.655	-0.0319999999999986\\
12.675	0.0420000000000014\\
12.695	-0.0299999999999986\\
12.715	-0.00599999999999865\\
12.735	-0.00599999999999865\\
12.755	0.0500000000000014\\
12.775	-0.0499999999999987\\
12.795	-0.0439999999999987\\
12.815	0.0500000000000014\\
12.835	1.35308431126191e-015\\
12.855	0.0120000000000014\\
12.875	-0.0319999999999986\\
12.895	-0.00399999999999865\\
12.915	-0.0179999999999986\\
12.935	-0.0459999999999986\\
12.955	0.0320000000000014\\
12.975	0.00800000000000135\\
12.995	0.00200000000000135\\
13.015	0.0420000000000014\\
13.035	0.0600000000000014\\
13.055	0.0660000000000013\\
13.075	-0.0539999999999986\\
13.095	0.0540000000000014\\
13.115	0.0760000000000014\\
13.135	0.0460000000000014\\
13.155	0.0660000000000013\\
13.175	0.0380000000000014\\
13.195	0.0960000000000014\\
13.215	0.0540000000000014\\
13.235	0.0100000000000014\\
13.255	-0.00799999999999865\\
13.275	0.0840000000000014\\
13.295	-0.0379999999999986\\
13.315	0.0440000000000014\\
13.335	0.0800000000000013\\
13.355	-0.0359999999999986\\
13.375	-0.0559999999999986\\
13.395	-0.115999999999999\\
13.415	-0.123999999999999\\
13.435	-0.215999999999999\\
13.455	-0.201999999999999\\
13.475	-0.285999999999999\\
13.495	-0.295999999999999\\
13.515	-0.249999999999999\\
13.535	-0.353999999999999\\
13.555	-0.391999999999999\\
13.575	-0.371999999999999\\
13.595	-0.427999999999999\\
13.615	-0.469999999999999\\
13.635	-0.525999999999999\\
13.655	-0.587999999999999\\
13.675	-0.841999999999999\\
13.695	-0.829999999999999\\
13.715	-0.767999999999999\\
13.735	-1.014\\
13.755	-1.124\\
13.775	-1.058\\
13.795	-1.166\\
13.815	-1.33\\
13.835	-1.45\\
13.855	-1.414\\
13.875	-1.43\\
13.895	-1.62\\
13.915	-1.884\\
13.935	-1.728\\
13.955	-1.964\\
13.975	-2.006\\
13.995	-2.108\\
14.015	-2.06\\
14.035	-2.136\\
14.055	-2.202\\
14.075	-2.194\\
14.095	-2.11\\
14.115	-2.384\\
14.135	-2.388\\
14.155	-2.206\\
14.175	-2.172\\
14.195	-2.314\\
14.215	-2.236\\
14.235	-2.364\\
14.255	-2.314\\
14.275	-2.43\\
14.295	-2.674\\
14.315	-2.426\\
14.335	-2.514\\
14.355	-2.686\\
14.375	-2.646\\
14.395	-2.684\\
14.415	-2.692\\
14.435	-2.928\\
14.455	-2.872\\
14.475	-3.006\\
14.495	-3\\
14.515	-2.928\\
14.535	-3.176\\
14.555	-3.078\\
14.575	-2.99\\
14.595	-3.226\\
14.615	-3.212\\
14.635	-3.344\\
14.655	-3.272\\
14.675	-3.366\\
14.695	-3.35\\
14.715	-3.458\\
14.735	-3.514\\
14.755	-3.442\\
14.775	-3.606\\
14.795	-3.382\\
14.815	-3.768\\
14.835	-3.662\\
14.855	-3.452\\
14.875	-3.684\\
14.895	-3.586\\
14.915	-3.506\\
14.935	-3.374\\
14.955	-3.448\\
14.975	-3.656\\
14.995	-3.7\\
15.015	-3.898\\
15.035	-3.7\\
15.055	-3.74\\
15.075	-3.868\\
15.095	-3.982\\
15.115	-3.95\\
15.135	-3.99\\
15.155	-3.978\\
15.175	-4.176\\
15.195	-4.034\\
15.215	-4.402\\
15.235	-4.304\\
15.255	-4.204\\
15.275	-4.35\\
15.295	-4.244\\
15.315	-4.344\\
15.335	-4.206\\
15.355	-4.298\\
15.375	-4.192\\
15.395	-4.05\\
15.415	-4.288\\
15.435	-4.46\\
15.455	-4.206\\
15.475	-4.226\\
15.495	-4.068\\
15.515	-4.018\\
15.535	-3.97\\
15.555	-4.12\\
15.575	-4.216\\
15.595	-4\\
15.615	-4.284\\
15.635	-4.11\\
15.655	-3.934\\
15.675	-4.284\\
15.695	-3.996\\
15.715	-4.238\\
15.735	-4.328\\
15.755	-4.096\\
15.775	-4.442\\
15.795	-4.384\\
15.815	-4.222\\
15.835	-4.488\\
15.855	-4.334\\
15.875	-4.52\\
15.895	-4.426\\
15.915	-4.416\\
15.935	-4.428\\
15.955	-4.332\\
15.975	-4.594\\
15.995	-4.384\\
16.015	-4.508\\
16.035	-4.724\\
16.055	-4.322\\
16.075	-4.612\\
16.095	-4.41\\
16.115	-4.358\\
16.135	-4.536\\
16.155	-4.214\\
16.175	-4.354\\
16.195	-4.374\\
16.215	-4.306\\
16.235	-4.184\\
16.255	-4.532\\
16.275	-4.082\\
16.295	-4.168\\
16.315	-4.142\\
16.335	-3.722\\
16.355	-4.286\\
16.375	-4.386\\
16.395	-3.942\\
16.415	-3.958\\
16.435	-4.156\\
16.455	-4.05\\
16.475	-3.92\\
16.495	-3.792\\
16.515	-3.604\\
16.535	-3.974\\
16.555	-3.666\\
16.575	-3.648\\
16.595	-3.78\\
16.615	-3.67\\
16.635	-3.834\\
16.655	-3.644\\
16.675	-3.638\\
16.695	-3.756\\
16.715	-3.612\\
16.735	-3.716\\
16.755	-3.698\\
16.775	-3.624\\
16.795	-3.576\\
16.815	-3.534\\
16.835	-3.568\\
16.855	-3.378\\
16.875	-3.424\\
16.895	-3.45\\
16.915	-3.522\\
16.935	-3.384\\
16.955	-3.414\\
16.975	-3.386\\
16.995	-3.222\\
17.015	-3.212\\
17.035	-2.958\\
17.055	-2.91\\
17.075	-2.944\\
17.095	-2.868\\
17.115	-2.866\\
17.135	-2.726\\
17.155	-2.606\\
17.175	-2.578\\
17.195	-2.626\\
17.215	-2.534\\
17.235	-2.374\\
17.255	-2.308\\
17.275	-2.28\\
17.295	-2.19\\
17.315	-2.122\\
17.335	-2.078\\
17.355	-1.952\\
17.375	-1.984\\
17.395	-1.972\\
17.415	-1.878\\
17.435	-1.886\\
17.455	-1.866\\
17.475	-1.81\\
17.495	-1.806\\
17.515	-1.844\\
17.535	-1.662\\
17.555	-1.62\\
17.575	-1.606\\
17.595	-1.494\\
17.615	-1.506\\
17.635	-1.418\\
17.655	-1.38\\
17.675	-1.398\\
17.695	-1.318\\
17.715	-1.172\\
17.735	-1.184\\
17.755	-1.246\\
17.775	-1.294\\
17.795	-1.222\\
17.815	-1.114\\
17.835	-1.036\\
17.855	-1.106\\
17.875	-1.116\\
17.895	-0.955999999999999\\
17.915	-0.871999999999999\\
17.935	-0.829999999999999\\
17.955	-0.781999999999999\\
17.975	-0.767999999999999\\
17.995	-0.807999999999999\\
18.015	-0.721999999999999\\
18.035	-0.775999999999999\\
18.055	-0.821999999999999\\
18.075	-0.771999999999999\\
18.095	-0.845999999999999\\
18.115	-0.747999999999999\\
18.135	-0.721999999999999\\
18.155	-0.707999999999999\\
18.175	-0.599999999999999\\
18.195	-0.597999999999999\\
18.215	-0.619999999999999\\
18.235	-0.519999999999999\\
18.255	-0.515999999999999\\
18.275	-0.423999999999999\\
18.295	-0.327999999999999\\
18.315	-0.251999999999999\\
18.335	-0.121999999999999\\
18.355	-0.187999999999999\\
18.375	-0.147999999999999\\
18.395	-0.169999999999999\\
18.415	-0.173999999999999\\
18.435	-0.235999999999999\\
18.455	-0.239999999999999\\
18.475	-0.111999999999999\\
18.495	-0.0419999999999987\\
18.515	-0.0819999999999986\\
18.535	0.0220000000000014\\
18.555	0.0580000000000014\\
18.575	0.0760000000000014\\
18.595	0.0340000000000014\\
18.615	0.0280000000000014\\
18.635	0.0120000000000014\\
18.655	0.0260000000000014\\
18.675	0.0440000000000014\\
18.695	0.0380000000000014\\
18.715	0.0840000000000014\\
18.735	0.158000000000001\\
18.755	0.128000000000001\\
18.775	0.0820000000000014\\
18.795	0.152000000000001\\
18.815	0.178000000000001\\
18.835	0.292000000000001\\
18.855	0.248000000000001\\
18.875	0.246000000000001\\
18.895	0.260000000000001\\
18.915	0.224000000000001\\
18.935	0.190000000000001\\
18.955	0.182000000000001\\
18.975	0.0920000000000014\\
18.995	0.0880000000000014\\
19.015	0.0540000000000014\\
19.035	0.0760000000000014\\
19.055	0.00200000000000135\\
19.075	0.0420000000000014\\
19.095	-0.0859999999999986\\
19.115	1.35308431126191e-015\\
19.135	-0.121999999999999\\
19.155	-0.0879999999999986\\
19.175	-0.135999999999999\\
19.195	-0.185999999999999\\
19.215	-0.173999999999999\\
19.235	-0.197999999999999\\
19.255	-0.225999999999999\\
19.275	-0.231999999999999\\
19.295	-0.267999999999999\\
19.315	-0.273999999999999\\
19.335	-0.283999999999999\\
19.355	-0.291999999999999\\
19.375	-0.235999999999999\\
19.395	-0.261999999999999\\
19.415	-0.239999999999999\\
19.435	-0.223999999999999\\
19.455	-0.155999999999999\\
19.475	-0.155999999999999\\
19.495	-0.169999999999999\\
19.515	-0.161999999999999\\
19.535	-0.147999999999999\\
19.555	-0.135999999999999\\
19.575	-0.171999999999999\\
19.595	-0.129999999999999\\
19.615	-0.143999999999999\\
19.635	-0.133999999999999\\
19.655	-0.161999999999999\\
19.675	-0.139999999999999\\
19.695	-0.123999999999999\\
19.715	-0.195999999999999\\
19.735	-0.203999999999999\\
19.755	-0.179999999999999\\
19.775	-0.141999999999999\\
19.795	-0.171999999999999\\
19.815	-0.167999999999999\\
19.835	-0.195999999999999\\
19.855	-0.117999999999999\\
19.875	-0.157999999999999\\
19.895	-0.183999999999999\\
19.915	-0.187999999999999\\
19.935	-0.147999999999999\\
19.955	-0.193999999999999\\
19.975	-0.185999999999999\\
};
\addplot [color=black,solid,forget plot, line width=1.0]
  table[row sep=crcr]{%
10.975	-9.05977958609583e-006\\
10.995	-8.93743435881333e-006\\
11.015	-8.77298134582816e-006\\
11.035	-8.57042687130161e-006\\
11.055	-8.33367539598839e-006\\
11.075	-8.06652042228961e-006\\
11.095	-7.77265267970506e-006\\
11.115	-7.45565330362297e-006\\
11.135	-7.11900020178291e-006\\
11.155	-6.76606441629701e-006\\
11.175	-6.40010784991318e-006\\
11.195	-6.02429099672008e-006\\
11.215	-5.64166612093686e-006\\
11.235	-5.2551758926711e-006\\
11.255	-4.8676620281185e-006\\
11.275	-4.48185528512113e-006\\
11.295	-4.10038501286181e-006\\
11.315	-3.72577073903813e-006\\
11.335	-3.36042671733594e-006\\
11.355	-3.00666101793468e-006\\
11.375	-2.66667507275997e-006\\
11.395	-2.34256572184677e-006\\
11.415	-2.03632112061314e-006\\
11.435	-1.74982994849415e-006\\
11.455	-1.48485742101911e-006\\
11.475	-1.2430842843969e-006\\
11.495	-1.0260698672937e-006\\
11.515	-8.35276011912356e-007\\
11.535	-6.72050873617991e-007\\
11.555	-5.37643359166395e-007\\
11.575	-4.33192042237351e-007\\
11.595	-3.59728687726601e-007\\
11.615	-3.18179985470124e-007\\
11.635	-2.92908566734695e-007\\
11.655	-2.09070748269369e-007\\
11.675	-1.10159895427842e-007\\
11.695	1.92814453292556e-009\\
11.715	1.25346232948687e-007\\
11.735	2.58296182664708e-007\\
11.755	3.99028408537561e-007\\
11.775	5.45842283372622e-007\\
11.795	6.97085908996087e-007\\
11.815	8.51156130465824e-007\\
11.835	1.00649867817992e-006\\
11.855	1.16160822472011e-006\\
11.875	1.31502758904389e-006\\
11.895	1.46534921441344e-006\\
11.915	1.6112137473101e-006\\
11.935	1.75131128798967e-006\\
11.955	1.88437945780606e-006\\
11.975	2.00920567294816e-006\\
11.995	2.12462691706605e-006\\
12.015	2.22952689910016e-006\\
12.035	2.32283969126001e-006\\
12.055	2.40354734160064e-006\\
12.075	2.47068146563834e-006\\
12.095	2.52332210948225e-006\\
12.115	2.56059729508706e-006\\
12.135	2.58168597611075e-006\\
12.155	2.58581280832004e-006\\
12.175	2.57225497080071e-006\\
12.195	2.54033375313156e-006\\
12.215	2.48942455982615e-006\\
12.235	2.41894736063841e-006\\
12.255	2.3283730570256e-006\\
12.275	2.21722143578518e-006\\
12.295	2.08505889531807e-006\\
12.315	1.93150344784954e-006\\
12.335	1.75621892140043e-006\\
12.355	1.55892132625013e-006\\
12.375	1.33937328428146e-006\\
12.395	1.36369339998055e-006\\
12.415	1.19350147542718e-006\\
12.435	1.0464524393683e-006\\
12.455	9.20694787964749e-007\\
12.475	8.1441942256788e-007\\
12.495	7.25859479189239e-007\\
12.515	6.53290101126913e-007\\
12.535	5.95028609495785e-007\\
12.555	5.49434730601206e-007\\
12.575	5.14909800131136e-007\\
12.595	4.8989772949426e-007\\
12.615	4.72884778446314e-007\\
12.635	4.62398617173676e-007\\
12.655	4.57009804222253e-007\\
12.675	4.55330507520557e-007\\
12.695	4.56015612826377e-007\\
12.715	4.57761558436687e-007\\
12.735	4.59306789934999e-007\\
12.755	4.59433550759059e-007\\
12.775	4.56963874739813e-007\\
12.795	4.50763877779536e-007\\
12.815	4.39740261981569e-007\\
12.835	4.22843328351519e-007\\
12.855	3.99064020939477e-007\\
12.875	3.67436939541221e-007\\
12.895	3.27038549130521e-007\\
12.915	2.76985673508534e-007\\
12.935	2.16440255940142e-007\\
12.955	1.4460346164924e-007\\
12.975	6.07216179560055e-008\\
12.995	-3.59202907418421e-008\\
13.015	-1.45991805311496e-007\\
13.035	-2.70123109658016e-007\\
13.055	-4.08900149295732e-007\\
13.075	-5.62867171538528e-007\\
13.095	-7.32526700630842e-007\\
13.115	-9.18340560929209e-007\\
13.135	-1.12072314095713e-006\\
13.155	-0.00149561651051044\\
13.175	-0.00592940207570791\\
13.195	-0.0132278660312295\\
13.215	-0.0233172234147787\\
13.235	-0.0361243225634098\\
13.255	-0.0515766330063343\\
13.275	-0.069602258503437\\
13.295	-0.0901299342513084\\
13.315	-0.113089017570019\\
13.335	-0.138409465551376\\
13.355	-0.166021928191185\\
13.375	-0.195857658982277\\
13.395	-0.22784848511219\\
13.415	-0.2619269490242\\
13.435	-0.298026114702225\\
13.455	-0.336079835891724\\
13.475	-0.376022398471832\\
13.495	-0.417788833379745\\
13.515	-0.461314767599106\\
13.535	-0.506536543369293\\
13.555	-0.553390920162201\\
13.575	-0.601815521717072\\
13.595	-0.651748538017273\\
13.615	-0.703128576278687\\
13.635	-0.75589519739151\\
13.655	-0.809988260269165\\
13.675	-0.86534857749939\\
13.695	-0.921917378902435\\
13.715	-0.979636549949646\\
13.735	-1.0384486913681\\
13.755	-1.09829699993134\\
13.775	-1.15912532806396\\
13.795	-1.22087800502777\\
13.815	-1.28350007534027\\
13.835	-1.34693717956543\\
13.855	-1.41113567352295\\
13.875	-1.47604250907898\\
13.895	-1.5416054725647\\
13.915	-1.54106330871582\\
13.935	-1.60563099384308\\
13.955	-1.66973340511322\\
13.975	-1.73335933685303\\
13.995	-1.79649722576141\\
14.015	-1.85913634300232\\
14.035	-1.92126548290253\\
14.055	-1.98287355899811\\
14.075	-2.04395008087158\\
14.095	-2.10448503494263\\
14.115	-2.16446709632874\\
14.135	-2.22388696670532\\
14.155	-2.28273439407349\\
14.175	-2.34099936485291\\
14.195	-2.39867234230042\\
14.215	-2.45574402809143\\
14.235	-2.51220464706421\\
14.255	-2.5680456161499\\
14.275	-2.62325739860535\\
14.295	-2.67783141136169\\
14.315	-2.73175883293152\\
14.335	-2.78503131866455\\
14.355	-2.8376407623291\\
14.375	-2.88957834243774\\
14.395	-2.94083666801453\\
14.415	-2.99140763282776\\
14.435	-3.04128336906433\\
14.455	-3.09045720100403\\
14.475	-3.13892078399658\\
14.495	-3.18666768074036\\
14.515	-3.23369026184082\\
14.535	-3.27998232841492\\
14.555	-3.32553720474243\\
14.575	-3.37034773826599\\
14.595	-3.41440796852112\\
14.615	-3.45771193504334\\
14.635	-3.50025343894959\\
14.655	-3.54202675819397\\
14.675	-3.54206109046936\\
14.695	-3.58316135406494\\
14.715	-3.62354636192322\\
14.735	-3.6632080078125\\
14.755	-3.70213842391968\\
14.775	-3.74032974243164\\
14.795	-3.77777528762817\\
14.815	-3.81446766853333\\
14.835	-3.85039901733398\\
14.855	-3.88556337356567\\
14.875	-3.9199538230896\\
14.895	-3.95356464385986\\
14.915	-3.98638820648193\\
14.935	-4.01842021942139\\
14.955	-4.04965305328369\\
14.975	-4.08008289337158\\
14.995	-4.10970306396484\\
15.015	-4.13850879669189\\
15.035	-4.16649532318115\\
15.055	-4.19365835189819\\
15.075	-4.21999216079712\\
15.095	-4.24549198150635\\
15.115	-4.27015495300293\\
15.135	-4.29397678375244\\
15.155	-4.31695365905762\\
15.175	-4.33908081054688\\
15.195	-4.36035680770874\\
15.215	-4.38077735900879\\
15.235	-4.40034008026123\\
15.255	-4.41904067993164\\
15.275	-4.43687772750854\\
15.295	-4.45384979248047\\
15.315	-4.46995258331299\\
15.335	-4.4851861000061\\
15.355	-4.49954652786255\\
15.375	-4.51303482055664\\
15.395	-4.52564668655396\\
15.415	-4.53738307952881\\
15.435	-4.53742980957031\\
15.455	-4.54842615127563\\
15.475	-4.55863046646118\\
15.495	-4.56803750991821\\
15.515	-4.5766429901123\\
15.535	-4.58444166183472\\
15.555	-4.59142875671387\\
15.575	-4.59760189056396\\
15.595	-4.60295629501343\\
15.615	-4.60748767852783\\
15.635	-4.6111946105957\\
15.655	-4.61407375335693\\
15.675	-4.61612176895142\\
15.695	-4.61733675003052\\
15.715	-4.61771631240845\\
15.735	-4.61725807189941\\
15.755	-4.61596202850342\\
15.775	-4.61382579803467\\
15.795	-4.61084842681885\\
15.815	-4.60703039169312\\
15.835	-4.60236978530884\\
15.855	-4.59686660766602\\
15.875	-4.59052181243896\\
15.895	-4.58333492279053\\
15.915	-4.57530689239502\\
15.935	-4.56643867492676\\
15.955	-4.55673122406006\\
15.975	-4.54618692398071\\
15.995	-4.53480672836304\\
16.015	-4.52259254455566\\
16.035	-4.50954675674438\\
16.055	-4.49567174911499\\
16.075	-4.48097133636475\\
16.095	-4.46544742584229\\
16.115	-4.44910383224487\\
16.135	-4.43194437026978\\
16.155	-4.41397333145142\\
16.175	-4.39519453048706\\
16.195	-4.39526462554932\\
16.215	-4.37588787078857\\
16.235	-4.35583639144897\\
16.255	-4.3351092338562\\
16.275	-4.31370067596436\\
16.295	-4.2916088104248\\
16.315	-4.26883029937744\\
16.335	-4.24536418914795\\
16.355	-4.22120761871338\\
16.375	-4.19636011123657\\
16.395	-4.17082071304321\\
16.415	-4.14458990097046\\
16.435	-4.11766624450684\\
16.455	-4.09005212783813\\
16.475	-4.06174850463867\\
16.495	-4.03275632858276\\
16.515	-4.0030779838562\\
16.535	-3.97271633148193\\
16.555	-3.94167447090149\\
16.575	-3.90995502471924\\
16.595	-3.87756323814392\\
16.615	-3.84450316429138\\
16.635	-3.81077933311462\\
16.655	-3.77639770507813\\
16.675	-3.74136400222778\\
16.695	-3.7056839466095\\
16.715	-3.66936540603638\\
16.735	-3.63241505622864\\
16.755	-3.59484076499939\\
16.775	-3.55665040016174\\
16.795	-3.51785349845886\\
16.815	-3.47845792770386\\
16.835	-3.47816753387451\\
16.855	-3.43734431266785\\
16.875	-3.35254406929016\\
16.895	-3.30867528915405\\
16.915	-3.26390218734741\\
16.935	-3.21828055381775\\
16.955	-3.1718635559082\\
16.975	-3.12470436096191\\
16.995	-3.07685518264771\\
17.015	-3.02836728096008\\
17.035	-2.9792914390564\\
17.055	-2.92967677116394\\
17.075	-2.87957286834717\\
17.095	-2.82902669906616\\
17.115	-2.77808570861816\\
17.135	-2.7267963886261\\
17.155	-2.67520356178284\\
17.175	-2.62335228919983\\
17.195	-2.57128596305847\\
17.215	-2.51904726028442\\
17.235	-2.46667790412903\\
17.255	-2.41422009468079\\
17.275	-2.36171245574951\\
17.295	-2.30919599533081\\
17.315	-2.2567081451416\\
17.335	-2.20428681373596\\
17.355	-2.15196895599365\\
17.375	-2.0997908115387\\
17.395	-2.04778671264648\\
17.415	-1.99599158763886\\
17.435	-1.94443845748901\\
17.455	-1.89316046237946\\
17.475	-1.84218871593475\\
17.495	-1.79155421257019\\
17.515	-1.74128699302673\\
17.535	-1.69141614437103\\
17.555	-1.641970038414\\
17.575	-1.59297609329224\\
17.595	-1.54446077346802\\
17.615	-1.49645006656647\\
17.635	-1.49644577503204\\
17.655	-1.44895195960999\\
17.675	-1.40200424194336\\
17.695	-1.35562694072723\\
17.715	-1.30984282493591\\
17.735	-1.26467418670654\\
17.755	-1.22014236450195\\
17.775	-1.17626786231995\\
17.795	-1.13307023048401\\
17.815	-1.09056806564331\\
17.835	-1.04877936840057\\
17.855	-1.00772082805634\\
17.875	-0.967408657073975\\
17.895	-0.927857935428619\\
17.915	-0.88908314704895\\
17.935	-0.851097464561462\\
17.955	-0.813913524150848\\
17.975	-0.777543008327484\\
17.995	-0.741996645927429\\
18.015	-0.707284212112427\\
18.035	-0.673414885997772\\
18.055	-0.64039671421051\\
18.075	-0.608236908912659\\
18.095	-0.576941847801209\\
18.115	-0.546517133712769\\
18.135	-0.516967236995697\\
18.155	-0.488295942544937\\
18.175	-0.460505992174149\\
18.195	-0.433599412441254\\
18.215	-0.407577246427536\\
18.235	-0.382439702749252\\
18.255	-0.35818612575531\\
18.275	-0.334814935922623\\
18.295	-0.312323659658432\\
18.315	-0.290709018707275\\
18.335	-0.269966721534729\\
18.355	-0.250091761350632\\
18.375	-0.231078147888184\\
18.395	-0.231161996722221\\
18.415	-0.213236182928085\\
18.435	-0.196280241012573\\
18.455	-0.180260673165321\\
18.475	-0.165144473314285\\
18.495	-0.150899276137352\\
18.515	-0.137493252754211\\
18.535	-0.124895162880421\\
18.555	-0.113074257969856\\
18.575	-0.102000430226326\\
18.595	-0.0916441082954407\\
18.615	-0.0819762796163559\\
18.635	-0.0729684978723526\\
18.655	-0.0645928904414177\\
18.675	-0.0568221211433411\\
18.695	-0.0496294721961021\\
18.715	-0.0429887436330318\\
18.735	-0.0368742905557156\\
18.755	-0.0312611013650894\\
18.775	-0.0261246375739574\\
18.795	-0.0214409995824099\\
18.815	-0.017186788842082\\
18.835	-0.0133392242714763\\
18.855	-0.00987607426941395\\
18.875	-0.00677564879879355\\
18.895	-0.00401686737313867\\
18.915	-0.00157910655252635\\
18.935	0.000557513965759426\\
18.955	0.0024124956689775\\
18.975	0.00400464795529842\\
18.995	0.00535226985812187\\
19.015	0.00647306302562356\\
19.035	0.00738427508622408\\
19.055	0.00810238253325224\\
19.075	0.00864345859736204\\
19.095	0.00902307592332363\\
19.115	0.00925599690526724\\
19.135	0.00935660861432552\\
19.155	0.00938588101416826\\
19.175	0.00944953877478838\\
19.195	0.00945121329277754\\
19.215	0.0093950517475605\\
19.235	0.00928509514778852\\
19.255	0.00912528857588768\\
19.275	0.00891947373747826\\
19.295	0.00867139361798763\\
19.315	0.00838468782603741\\
19.335	0.00806289818137884\\
19.355	0.00770946452394128\\
19.375	0.007327726110816\\
19.395	0.00692092208191752\\
19.415	0.00649218959733844\\
19.435	0.006044568028301\\
19.455	0.00558099197223783\\
19.475	0.00510429963469505\\
19.495	0.00461722770705819\\
19.515	0.0041224081069231\\
19.535	0.00362237775698304\\
19.555	0.00311956787481904\\
19.575	0.00261631491594017\\
19.595	0.0021148503292352\\
19.615	0.00161730556283146\\
19.635	0.00112571287900209\\
19.655	0.000642000639345497\\
19.675	0.000168003825820051\\
19.695	-0.000294553290586919\\
19.715	-0.000744036515243351\\
19.735	-0.00117892667185515\\
19.755	-0.00159778445959091\\
19.775	-0.00199929252266884\\
19.795	-0.00238221953622997\\
19.815	-0.0027454411610961\\
19.835	-0.00308793340809643\\
19.855	-0.00340876635164022\\
19.875	-0.0037071225233376\\
19.895	-0.00398227805271745\\
19.915	-0.00397617928683758\\
19.935	-0.00421075290068984\\
19.955	-0.00441252486780286\\
19.975	-0.00458314456045628\\
};
\end{axis}

\begin{axis}[%
width=0.410625\figurewidth,
height=0.264706\figureheight,
at={(0\figurewidth,0.367647\figureheight)},
scale only axis,
every outer x axis line/.append style={black},
every x tick label/.append style={font=\color{black}},
xmin=11,
xmax=20,
xmajorgrids,
every outer y axis line/.append style={black},
every y tick label/.append style={font=\color{black}},
ymin=0,
ymax=60,
ylabel={$v_x$ [km/h]},
ylabel near ticks,
ymajorgrids,
axis x line*=bottom,
axis y line*=left,
legend style={legend cell align=left,align=left,draw=black,legend columns=-1, at={(0.3,-2.2)},anchor=west}
]
\addplot [color=light-gray,solid,line width=1.0]
  table[row sep=crcr,y expr={\thisrowno{1}*3.6}]{%
10.975	13.8888888888889\\
10.995	13.8671875\\
11.015	13.8802083333333\\
11.035	13.8758680555556\\
11.055	13.8671875\\
11.075	13.8758680555556\\
11.095	13.8715277777778\\
11.115	13.8671875\\
11.135	13.8671875\\
11.155	13.8541666666667\\
11.175	13.8628472222222\\
11.195	13.8585069444444\\
11.215	13.8671875\\
11.235	13.8498263888889\\
11.255	13.8541666666667\\
11.275	13.8541666666667\\
11.295	13.828125\\
11.315	13.8454861111111\\
11.335	13.8541666666667\\
11.355	13.8368055555556\\
11.375	13.8498263888889\\
11.395	13.8541666666667\\
11.415	13.8498263888889\\
11.435	13.8585069444444\\
11.455	13.8368055555556\\
11.475	13.8585069444444\\
11.495	13.8628472222222\\
11.515	13.8628472222222\\
11.535	13.8541666666667\\
11.555	13.8541666666667\\
11.575	13.8585069444444\\
11.595	13.8585069444444\\
11.615	13.8585069444444\\
11.635	13.8628472222222\\
11.655	13.8671875\\
11.675	13.8715277777778\\
11.695	13.8758680555556\\
11.715	13.8585069444444\\
11.735	13.8585069444444\\
11.755	13.8628472222222\\
11.775	13.8671875\\
11.795	13.8628472222222\\
11.815	13.8585069444444\\
11.835	13.8628472222222\\
11.855	13.8628472222222\\
11.875	13.8454861111111\\
11.895	13.8498263888889\\
11.915	13.8411458333333\\
11.935	13.8411458333333\\
11.955	13.8454861111111\\
11.975	13.8454861111111\\
11.995	13.8498263888889\\
12.015	13.8498263888889\\
12.035	13.8585069444444\\
12.055	13.8585069444444\\
12.075	13.8585069444444\\
12.095	13.8671875\\
12.115	13.8585069444444\\
12.135	13.8671875\\
12.155	13.8671875\\
12.175	13.8585069444444\\
12.195	13.8671875\\
12.215	13.8671875\\
12.235	13.8585069444444\\
12.255	13.8715277777778\\
12.275	13.8585069444444\\
12.295	13.8715277777778\\
12.315	13.8671875\\
12.335	13.8671875\\
12.355	13.8715277777778\\
12.375	13.8628472222222\\
12.395	13.8585069444444\\
12.415	13.8715277777778\\
12.435	13.8671875\\
12.455	13.8758680555556\\
12.475	13.8628472222222\\
12.495	13.8671875\\
12.515	13.8758680555556\\
12.535	13.8715277777778\\
12.555	13.8758680555556\\
12.575	13.8758680555556\\
12.595	13.8715277777778\\
12.615	13.8715277777778\\
12.635	13.8715277777778\\
12.655	13.8802083333333\\
12.675	13.8888888888889\\
12.695	13.8802083333333\\
12.715	13.8802083333333\\
12.735	13.8845486111111\\
12.755	13.8802083333333\\
12.775	13.8715277777778\\
12.795	13.8585069444444\\
12.815	13.8715277777778\\
12.835	13.8758680555556\\
12.855	13.8715277777778\\
12.875	13.8715277777778\\
12.895	13.8671875\\
12.915	13.8715277777778\\
12.935	13.8628472222222\\
12.955	13.8628472222222\\
12.975	13.8715277777778\\
12.995	13.8715277777778\\
13.015	13.8758680555556\\
13.035	13.8888888888889\\
13.055	13.8715277777778\\
13.075	13.8715277777778\\
13.095	13.8671875\\
13.115	13.8628472222222\\
13.135	13.8758680555556\\
13.155	13.8845486111111\\
13.175	13.8888888888889\\
13.195	13.8932291666667\\
13.215	13.8932291666667\\
13.235	13.8932291666667\\
13.255	13.8888888888889\\
13.275	13.8845486111111\\
13.295	13.9019097222222\\
13.315	13.8888888888889\\
13.335	13.9019097222222\\
13.355	13.9105902777778\\
13.375	13.8888888888889\\
13.395	13.9019097222222\\
13.415	13.8932291666667\\
13.435	13.8932291666667\\
13.455	13.90625\\
13.475	13.8975694444444\\
13.495	13.9019097222222\\
13.515	13.8888888888889\\
13.535	13.8628472222222\\
13.555	13.8715277777778\\
13.575	13.8541666666667\\
13.595	13.8454861111111\\
13.615	13.8237847222222\\
13.635	13.8324652777778\\
13.655	13.828125\\
13.675	13.8107638888889\\
13.695	13.7717013888889\\
13.715	13.7760416666667\\
13.735	13.7456597222222\\
13.755	13.7196180555556\\
13.775	13.7065972222222\\
13.795	13.7022569444444\\
13.815	13.6805555555556\\
13.835	13.6284722222222\\
13.855	13.6197916666667\\
13.875	13.5894097222222\\
13.895	13.5373263888889\\
13.915	13.515625\\
13.935	13.4809027777778\\
13.955	13.4244791666667\\
13.975	13.3940972222222\\
13.995	13.3463541666667\\
14.015	13.3376736111111\\
14.035	13.2595486111111\\
14.055	13.2508680555556\\
14.075	13.1770833333333\\
14.095	13.1597222222222\\
14.115	13.1032986111111\\
14.135	13.0555555555556\\
14.155	13.0425347222222\\
14.175	12.9557291666667\\
14.195	12.9296875\\
14.215	12.8819444444444\\
14.235	12.8515625\\
14.255	12.7777777777778\\
14.275	12.7387152777778\\
14.295	12.6649305555556\\
14.315	12.6432291666667\\
14.335	12.5737847222222\\
14.355	12.5477430555556\\
14.375	12.4739583333333\\
14.395	12.4175347222222\\
14.415	12.3611111111111\\
14.435	12.2829861111111\\
14.455	12.2352430555556\\
14.475	12.1527777777778\\
14.495	12.1050347222222\\
14.515	12.0529513888889\\
14.535	11.9921875\\
14.555	11.9357638888889\\
14.575	11.8489583333333\\
14.595	11.796875\\
14.615	11.7447916666667\\
14.635	11.6623263888889\\
14.655	11.5972222222222\\
14.675	11.5277777777778\\
14.695	11.4670138888889\\
14.715	11.3975694444444\\
14.735	11.328125\\
14.755	11.2673611111111\\
14.775	11.2109375\\
14.795	11.1414930555556\\
14.815	11.0460069444444\\
14.835	10.9548611111111\\
14.855	10.8984375\\
14.875	10.8506944444444\\
14.895	10.78125\\
14.915	10.7291666666667\\
14.935	10.6597222222222\\
14.955	10.5772569444444\\
14.975	10.5034722222222\\
14.995	10.4427083333333\\
15.015	10.3125\\
15.035	10.2604166666667\\
15.055	10.2083333333333\\
15.075	10.1128472222222\\
15.095	10.0390625\\
15.115	9.95225694444444\\
15.135	9.86979166666667\\
15.155	9.79600694444444\\
15.175	9.71354166666667\\
15.195	9.64409722222222\\
15.215	9.54427083333333\\
15.235	9.44878472222222\\
15.255	9.37065972222222\\
15.275	9.30989583333333\\
15.295	9.21006944444444\\
15.315	9.12326388888889\\
15.335	9.03645833333333\\
15.355	8.93229166666667\\
15.375	8.8671875\\
15.395	8.7890625\\
15.415	8.70225694444444\\
15.435	8.61111111111111\\
15.455	8.51128472222222\\
15.475	8.43315972222222\\
15.495	8.36371527777778\\
15.515	8.29861111111111\\
15.535	8.22482638888889\\
15.555	8.12065972222222\\
15.575	8.02951388888889\\
15.595	7.93836805555556\\
15.615	7.86892361111111\\
15.635	7.77777777777778\\
15.655	7.70833333333333\\
15.675	7.62152777777778\\
15.695	7.53038194444444\\
15.715	7.44791666666667\\
15.735	7.35243055555556\\
15.755	7.27864583333333\\
15.775	7.17013888888889\\
15.795	7.09635416666667\\
15.815	7.02256944444444\\
15.835	6.92274305555556\\
15.855	6.83159722222222\\
15.875	6.73177083333333\\
15.895	6.65798611111111\\
15.915	6.57552083333333\\
15.935	6.484375\\
15.955	6.40625\\
15.975	6.29774305555556\\
15.995	6.21961805555556\\
16.015	6.14149305555556\\
16.035	6.03298611111111\\
16.055	5.95052083333333\\
16.075	5.87673611111111\\
16.095	5.79427083333333\\
16.115	5.69010416666667\\
16.135	5.59895833333333\\
16.155	5.52517361111111\\
16.175	5.43836805555556\\
16.195	5.34288194444444\\
16.215	5.25607638888889\\
16.235	5.18229166666667\\
16.255	5.09982638888889\\
16.275	5.00868055555556\\
16.295	4.93489583333333\\
16.315	4.83940972222222\\
16.335	4.765625\\
16.355	4.69618055555556\\
16.375	4.57899305555556\\
16.395	4.50520833333333\\
16.415	4.43576388888889\\
16.435	4.36197916666667\\
16.455	4.27083333333333\\
16.475	4.20138888888889\\
16.495	4.13628472222222\\
16.515	4.04947916666667\\
16.535	3.97135416666667\\
16.555	3.89756944444444\\
16.575	3.82378472222222\\
16.595	3.74565972222222\\
16.615	3.68055555555556\\
16.635	3.58940972222222\\
16.655	3.51996527777778\\
16.675	3.45920138888889\\
16.695	3.36371527777778\\
16.715	3.29861111111111\\
16.735	3.22482638888889\\
16.755	3.15538194444444\\
16.775	3.09461805555556\\
16.795	3.01649305555556\\
16.815	2.93836805555556\\
16.835	2.87326388888889\\
16.855	2.79947916666667\\
16.875	2.74305555555556\\
16.895	2.67361111111111\\
16.915	2.58246527777778\\
16.935	2.52604166666667\\
16.955	2.4609375\\
16.975	2.39583333333333\\
16.995	2.33940972222222\\
17.015	2.25694444444444\\
17.035	2.20486111111111\\
17.055	2.14409722222222\\
17.075	2.08333333333333\\
17.095	2.01822916666667\\
17.115	1.95746527777778\\
17.135	1.90104166666667\\
17.155	1.84027777777778\\
17.175	1.78385416666667\\
17.195	1.73177083333333\\
17.215	1.6796875\\
17.235	1.62326388888889\\
17.255	1.57552083333333\\
17.275	1.53645833333333\\
17.295	1.48871527777778\\
17.315	1.44097222222222\\
17.335	1.39322916666667\\
17.355	1.36284722222222\\
17.375	1.31510416666667\\
17.395	1.27170138888889\\
17.415	1.22829861111111\\
17.435	1.18489583333333\\
17.455	1.14583333333333\\
17.475	1.12847222222222\\
17.495	1.08506944444444\\
17.515	1.04600694444444\\
17.535	1.02430555555556\\
17.555	0.985243055555556\\
17.575	0.950520833333333\\
17.595	0.915798611111111\\
17.615	0.889756944444444\\
17.635	0.859375\\
17.655	0.824652777777778\\
17.675	0.794270833333333\\
17.695	0.768229166666667\\
17.715	0.7421875\\
17.735	0.716145833333333\\
17.755	0.698784722222222\\
17.775	0.672743055555556\\
17.795	0.651041666666667\\
17.815	0.629340277777778\\
17.835	0.603298611111111\\
17.855	0.577256944444444\\
17.875	0.555555555555556\\
17.895	0.533854166666667\\
17.915	0.529513888888889\\
17.935	0.5078125\\
17.955	0.490451388888889\\
17.975	0.473090277777778\\
17.995	0.455729166666667\\
18.015	0.451388888888889\\
18.035	0.434027777777778\\
18.055	0.416666666666667\\
18.075	0.403645833333333\\
18.095	0.386284722222222\\
18.115	0.373263888888889\\
18.135	0.360243055555556\\
18.155	0.3515625\\
18.175	0.338541666666667\\
18.195	0.325520833333333\\
18.215	0.3125\\
18.235	0.303819444444444\\
18.255	0.295138888888889\\
18.275	0.282118055555556\\
18.295	0.2734375\\
18.315	0.264756944444444\\
18.335	0.260416666666667\\
18.355	0.251736111111111\\
18.375	0.243055555555556\\
18.395	0.234375\\
18.415	0.230034722222222\\
18.435	0.221354166666667\\
18.455	0.217013888888889\\
18.475	0.212673611111111\\
18.495	0.203993055555556\\
18.515	0.199652777777778\\
18.535	0.1953125\\
18.555	0.190972222222222\\
18.575	0.186631944444444\\
18.595	0.182291666666667\\
18.615	0.173611111111111\\
18.635	0.169270833333333\\
18.655	0.169270833333333\\
18.675	0.169270833333333\\
18.695	0.169270833333333\\
18.715	0.169270833333333\\
18.735	0.169270833333333\\
18.755	0.164930555555556\\
18.775	0.169270833333333\\
18.795	0.169270833333333\\
18.815	0.173611111111111\\
18.835	0.182291666666667\\
18.855	0.177951388888889\\
18.875	0.182291666666667\\
18.895	0.182291666666667\\
18.915	0.182291666666667\\
18.935	0.182291666666667\\
18.955	0.190972222222222\\
18.975	0.208333333333333\\
18.995	0.208333333333333\\
19.015	0.208333333333333\\
19.035	0.217013888888889\\
19.055	0.217013888888889\\
19.075	0.217013888888889\\
19.095	0.217013888888889\\
19.115	0.225694444444444\\
19.135	0.234375\\
19.155	0.238715277777778\\
19.175	0.243055555555556\\
19.195	0.243055555555556\\
19.215	0.243055555555556\\
19.235	0.243055555555556\\
19.255	0.243055555555556\\
19.275	0.243055555555556\\
19.295	0.234375\\
19.315	0.221354166666667\\
19.335	0.217013888888889\\
19.355	0.217013888888889\\
19.375	0.217013888888889\\
19.395	0.217013888888889\\
19.415	0.212673611111111\\
19.435	0.203993055555556\\
19.455	0.203993055555556\\
19.475	0.203993055555556\\
19.495	0.1953125\\
19.515	0.190972222222222\\
19.535	0.190972222222222\\
19.555	0.182291666666667\\
19.575	0.182291666666667\\
19.595	0.182291666666667\\
19.615	0.177951388888889\\
19.635	0.173611111111111\\
19.655	0.173611111111111\\
19.675	0.160590277777778\\
19.695	0.160590277777778\\
19.715	0.160590277777778\\
19.735	0.151909722222222\\
19.755	0.151909722222222\\
19.775	0.151909722222222\\
19.795	0.151909722222222\\
19.815	0.147569444444444\\
19.835	0.147569444444444\\
19.855	0.143229166666667\\
19.875	0.138888888888889\\
19.895	0.138888888888889\\
19.915	0.134548611111111\\
19.935	0.138888888888889\\
19.955	0.121527777777778\\
19.975	0.121527777777778\\
};
\addlegendentry{gemessene Gr\"o\ss en};
\addplot [color=black,solid, line width=1.0]
  table[row sep=crcr,y expr={\thisrowno{1}*3.6}]{%
10.975	13.8888893127441\\
10.995	13.8888893127441\\
11.015	13.8888893127441\\
11.035	13.8888893127441\\
11.055	13.8888883590698\\
11.075	13.8888883590698\\
11.095	13.8888883590698\\
11.115	13.8888883590698\\
11.135	13.8888883590698\\
11.155	13.8888883590698\\
11.175	13.8888874053955\\
11.195	13.8888874053955\\
11.215	13.8888874053955\\
11.235	13.8888874053955\\
11.255	13.8888883590698\\
11.275	13.8888864517212\\
11.295	13.8888864517212\\
11.315	13.8888864517212\\
11.335	13.8888864517212\\
11.355	13.8888864517212\\
11.375	13.8888864517212\\
11.395	13.8888854980469\\
11.415	13.8888854980469\\
11.435	13.8888864517212\\
11.455	13.8888864517212\\
11.475	13.8888864517212\\
11.495	13.8888845443726\\
11.515	13.8888845443726\\
11.535	13.8888854980469\\
11.555	13.8888854980469\\
11.575	13.8888854980469\\
11.595	13.8888864517212\\
11.615	13.8888854980469\\
11.635	13.8888854980469\\
11.655	13.8888854980469\\
11.675	13.8888854980469\\
11.695	13.8888864517212\\
11.715	13.8888854980469\\
11.735	13.8888845443726\\
11.755	13.8888845443726\\
11.775	13.8888845443726\\
11.795	13.8888845443726\\
11.815	13.8888845443726\\
11.835	13.8888845443726\\
11.855	13.8888845443726\\
11.875	13.8888845443726\\
11.895	13.8888845443726\\
11.915	13.8888845443726\\
11.935	13.8888845443726\\
11.955	13.8888845443726\\
11.975	13.8888845443726\\
11.995	13.8888845443726\\
12.015	13.8888845443726\\
12.035	13.8888845443726\\
12.055	13.8888845443726\\
12.075	13.8888845443726\\
12.095	13.8888845443726\\
12.115	13.8888845443726\\
12.135	13.8888845443726\\
12.155	13.8888854980469\\
12.175	13.8888854980469\\
12.195	13.8888854980469\\
12.215	13.8888854980469\\
12.235	13.8888845443726\\
12.255	13.8888845443726\\
12.275	13.8888845443726\\
12.295	13.8888845443726\\
12.315	13.8888845443726\\
12.335	13.8888864517212\\
12.355	13.8888864517212\\
12.375	13.8888854980469\\
12.395	13.8888854980469\\
12.415	13.8888854980469\\
12.435	13.8888864517212\\
12.455	13.8888854980469\\
12.475	13.8888864517212\\
12.495	13.8888864517212\\
12.515	13.8888864517212\\
12.535	13.8888864517212\\
12.555	13.8888864517212\\
12.575	13.8888864517212\\
12.595	13.8888864517212\\
12.615	13.8888864517212\\
12.635	13.8888864517212\\
12.655	13.8888864517212\\
12.675	13.8888864517212\\
12.695	13.8888864517212\\
12.715	13.8888864517212\\
12.735	13.8888864517212\\
12.755	13.8888864517212\\
12.775	13.8888864517212\\
12.795	13.8888864517212\\
12.815	13.8888864517212\\
12.835	13.8888864517212\\
12.855	13.8888864517212\\
12.875	13.8888874053955\\
12.895	13.8888874053955\\
12.915	13.8888874053955\\
12.935	13.8888874053955\\
12.955	13.8888874053955\\
12.975	13.8888874053955\\
12.995	13.8888864517212\\
13.015	13.8888864517212\\
13.035	13.8888874053955\\
13.055	13.8888864517212\\
13.075	13.8888864517212\\
13.095	13.8888864517212\\
13.115	13.8888874053955\\
13.135	13.8888864517212\\
13.155	13.8888864517212\\
13.175	13.8888864517212\\
13.195	13.8888864517212\\
13.215	13.8888854980469\\
13.235	13.8888854980469\\
13.255	13.8888874053955\\
13.275	13.8888864517212\\
13.295	13.888876914978\\
13.315	13.8888082504272\\
13.335	13.8886213302612\\
13.355	13.8882608413696\\
13.375	13.8876705169678\\
13.395	13.8867969512939\\
13.415	13.8855905532837\\
13.435	13.883996963501\\
13.455	13.8819694519043\\
13.475	13.8794574737549\\
13.495	13.8764171600342\\
13.515	13.872802734375\\
13.535	13.8685693740845\\
13.555	13.8636741638184\\
13.575	13.858078956604\\
13.595	13.8517398834229\\
13.615	13.8446216583252\\
13.635	13.8366870880127\\
13.655	13.827898979187\\
13.675	13.8182229995728\\
13.695	13.8076267242432\\
13.715	13.7960767745972\\
13.735	13.7835445404053\\
13.755	13.7699975967407\\
13.775	13.7554092407227\\
13.795	13.7397527694702\\
13.815	13.7230014801025\\
13.835	13.7051296234131\\
13.855	13.6861162185669\\
13.875	13.6659364700317\\
13.895	13.6445713043213\\
13.915	13.6219997406006\\
13.935	13.5982007980347\\
13.955	13.5731592178345\\
13.975	13.5468549728394\\
13.995	13.5192756652832\\
14.015	13.4904050827026\\
14.035	13.4602308273315\\
14.055	13.4602336883545\\
14.075	13.4287643432617\\
14.095	13.3960113525391\\
14.115	13.3619804382324\\
14.135	13.326681137085\\
14.155	13.2901220321655\\
14.175	13.2523183822632\\
14.195	13.2132759094238\\
14.215	13.1730060577393\\
14.235	13.1315202713013\\
14.255	13.0888299942017\\
14.275	13.0449466705322\\
14.295	12.9998788833618\\
14.315	12.9536409378052\\
14.335	12.9062433242798\\
14.355	12.8576974868774\\
14.375	12.8080177307129\\
14.395	12.7572145462036\\
14.415	12.7053003311157\\
14.435	12.6522874832153\\
14.455	12.5981903076172\\
14.475	12.5430212020874\\
14.495	12.4867944717407\\
14.515	12.4295196533203\\
14.535	12.3712158203125\\
14.555	12.3118925094604\\
14.575	12.2515640258789\\
14.595	12.1902456283569\\
14.615	12.1279497146606\\
14.635	12.0646934509277\\
14.655	12.0004892349243\\
14.675	11.9353513717651\\
14.695	11.8692951202393\\
14.715	11.8023338317871\\
14.735	11.7344846725464\\
14.755	11.6657629013062\\
14.775	11.5961828231812\\
14.795	11.5257577896118\\
14.815	11.5257577896118\\
14.835	11.4545059204102\\
14.855	11.3824367523193\\
14.875	11.3095674514771\\
14.895	11.2359132766724\\
14.915	11.16148853302\\
14.935	11.0863056182861\\
14.955	11.0103816986084\\
14.975	10.9337320327759\\
14.995	10.856369972229\\
15.015	10.7783136367798\\
15.035	10.6995782852173\\
15.055	10.6201763153076\\
15.075	10.5401277542114\\
15.095	10.4594449996948\\
15.115	10.3781461715698\\
15.135	10.2962474822998\\
15.155	10.2137641906738\\
15.175	10.1307125091553\\
15.195	10.0471096038818\\
15.215	9.96297073364258\\
15.235	9.87831592559814\\
15.255	9.79315757751465\\
15.275	9.70751476287842\\
15.295	9.62140369415283\\
15.315	9.5348424911499\\
15.335	9.44784545898438\\
15.355	9.36043357849121\\
15.375	9.27262115478516\\
15.395	9.18442630767822\\
15.415	9.09586620330811\\
15.435	9.00695705413818\\
15.455	8.91771697998047\\
15.475	8.82816410064697\\
15.495	8.73831653594971\\
15.515	8.64818954467773\\
15.535	8.55780124664307\\
15.555	8.46716976165771\\
15.575	8.4671688079834\\
15.595	8.37630939483643\\
15.615	8.28523731231689\\
15.635	8.19396877288818\\
15.655	8.10252094268799\\
15.675	8.01091003417969\\
15.695	7.91914844512939\\
15.715	7.82725715637207\\
15.735	7.73525047302246\\
15.755	7.64314413070679\\
15.775	7.55095672607422\\
15.795	7.4587025642395\\
15.815	7.36639881134033\\
15.835	7.27406358718872\\
15.855	7.18171072006226\\
15.875	7.08936023712158\\
15.895	6.99702644348145\\
15.915	6.9047269821167\\
15.935	6.81247901916504\\
15.955	6.72029829025269\\
15.975	6.62820291519165\\
15.995	6.53620910644531\\
16.015	6.44433403015137\\
16.035	6.35259389877319\\
16.055	6.26100587844849\\
16.075	6.16958713531494\\
16.095	6.0783543586731\\
16.115	5.98732328414917\\
16.135	5.89651298522949\\
16.155	5.80593776702881\\
16.175	5.71561479568481\\
16.195	5.62556123733521\\
16.215	5.53579330444336\\
16.235	5.44632816314697\\
16.255	5.35718154907227\\
16.275	5.26836967468262\\
16.295	5.17990875244141\\
16.315	5.09181594848633\\
16.335	5.09181547164917\\
16.355	5.00410223007202\\
16.375	4.91678428649902\\
16.395	4.82987356185913\\
16.415	4.74338436126709\\
16.435	4.65733003616333\\
16.455	4.5717248916626\\
16.475	4.48658180236816\\
16.495	4.40191459655762\\
16.515	4.31773853302002\\
16.535	4.23406553268433\\
16.555	4.15090942382813\\
16.575	4.06828594207764\\
16.595	3.98620796203613\\
16.615	3.90468859672546\\
16.635	3.82374215126038\\
16.655	3.74338269233704\\
16.675	3.66362380981445\\
16.695	3.58447861671448\\
16.715	3.5059609413147\\
16.735	3.4280846118927\\
16.755	3.35086274147034\\
16.775	3.2743091583252\\
16.795	3.19843578338623\\
16.815	3.12325739860535\\
16.835	3.0487859249115\\
16.855	2.97503471374512\\
16.875	2.90201592445374\\
16.895	2.82974243164063\\
16.915	2.75822615623474\\
16.935	2.68748021125793\\
16.955	2.61751627922058\\
16.975	2.61751818656921\\
16.995	2.54836130142212\\
17.015	2.41254949569702\\
17.035	2.34593629837036\\
17.055	2.28020882606506\\
17.075	2.2153856754303\\
17.095	2.15148282051086\\
17.115	2.08851623535156\\
17.135	2.02649927139282\\
17.155	1.9654461145401\\
17.175	1.90536856651306\\
17.195	1.84627830982208\\
17.215	1.78818488121033\\
17.235	1.73109829425812\\
17.255	1.67502641677856\\
17.275	1.61997699737549\\
17.295	1.56595647335052\\
17.315	1.51297056674957\\
17.335	1.46102380752563\\
17.355	1.41012012958527\\
17.375	1.36026263237\\
17.395	1.31145358085632\\
17.415	1.26369416713715\\
17.435	1.21698522567749\\
17.455	1.17132616043091\\
17.475	1.12671637535095\\
17.495	1.08315408229828\\
17.515	1.04063677787781\\
17.535	0.99916136264801\\
17.555	0.958723962306976\\
17.575	0.919320225715637\\
17.595	0.880944728851318\\
17.615	0.843591749668121\\
17.635	0.8072549700737\\
17.655	0.771927237510681\\
17.675	0.737600922584534\\
17.695	0.704267799854279\\
17.715	0.671919107437134\\
17.735	0.640545666217804\\
17.755	0.610137343406677\\
17.775	0.610137462615967\\
17.795	0.580684363842011\\
17.815	0.552175760269165\\
17.835	0.524600386619568\\
17.855	0.497946679592133\\
17.875	0.472202569246292\\
17.895	0.447355508804321\\
17.915	0.423392534255981\\
17.935	0.400300264358521\\
17.955	0.37806510925293\\
17.975	0.356672793626785\\
17.995	0.336109071969986\\
18.015	0.316359013319016\\
18.035	0.297407627105713\\
18.055	0.279239505529404\\
18.075	0.261839002370834\\
18.095	0.245190262794495\\
18.115	0.229277014732361\\
18.135	0.214083001017571\\
18.155	0.199591591954231\\
18.175	0.185786023736\\
18.195	0.172649294137955\\
18.215	0.160164415836334\\
18.235	0.148314043879509\\
18.255	0.137080922722816\\
18.275	0.126447573304176\\
18.295	0.116396404802799\\
18.315	0.106909863650799\\
18.335	0.0979702994227409\\
18.355	0.0895599946379662\\
18.375	0.0816613063216209\\
18.395	0.0742565244436264\\
18.415	0.0673279836773872\\
18.435	0.0608580857515335\\
18.455	0.0548292435705662\\
18.475	0.0492239147424698\\
18.495	0.0440247692167759\\
18.515	0.0392144918441772\\
18.535	0.0392139367759228\\
18.555	0.0347715951502323\\
18.575	0.0306780226528645\\
18.595	0.0269141457974911\\
18.615	0.0234615728259087\\
18.635	0.0203025620430708\\
18.655	0.0174200087785721\\
18.675	0.014797443524003\\
18.695	0.0124190216884017\\
18.715	0.0102694947272539\\
18.735	0.00833422131836414\\
18.755	0.00659914221614599\\
18.775	0.00505077093839645\\
18.795	0.0036761867813766\\
18.815	0.00246302154846489\\
18.835	0.00139944907277822\\
18.855	0.000474163156468421\\
18.875	0\\
18.895	0\\
18.915	0\\
18.935	0\\
18.955	0\\
18.975	0\\
18.995	0\\
19.015	0\\
19.035	0\\
19.055	0\\
19.075	0\\
19.095	0\\
19.115	0\\
19.135	0\\
19.155	0\\
19.175	0\\
19.195	0\\
19.215	0\\
19.235	0\\
19.255	0\\
19.275	0\\
19.295	0.0001865327503765\\
19.315	0.000374993775039911\\
19.335	0.00056410109391436\\
19.355	0.00075265666237101\\
19.375	0.000939544523134828\\
19.395	0.00112372823059559\\
19.415	0.00130424927920103\\
19.435	0.00148022535722703\\
19.455	0.0016508474946022\\
19.475	0.00181537866592407\\
19.495	0.00197315239347517\\
19.515	0.00212356867268682\\
19.535	0.00226609432138503\\
19.555	0.00240025972016156\\
19.575	0.00252565601840615\\
19.595	0.00264193606562912\\
19.615	0.00274880859069526\\
19.635	0.00284603890031576\\
19.655	0.00293344631791115\\
19.675	0.00301090045832098\\
19.695	0.00307832285761833\\
19.715	0.00313568045385182\\
19.735	0.00318298698402941\\
19.755	0.00322030088864267\\
19.775	0.00324771925806999\\
19.795	0.00326538179069757\\
19.815	0.00327346450649202\\
19.835	0.00327217835001647\\
19.855	0.0032617689576\\
19.875	0.00324251386336982\\
19.895	0.00321471923962235\\
19.915	0.00317871849983931\\
19.935	0.00313487066887319\\
19.955	0.00308356084860861\\
19.975	0.00302519206888974\\
};
\addlegendentry{geplante Gr\"o\ss en};

\addplot [color=black,dashed, line width=1.0]
  table[row sep=crcr,y expr={\thisrowno{1}*3.6}]{%
10.975	13.8888888888889\\
10.995	13.8888888888889\\
11.015	13.8888888888889\\
11.035	13.8888888888889\\
11.055	13.8888888888889\\
11.075	13.8888888888889\\
11.095	13.8888888888889\\
11.115	13.8888888888889\\
11.135	13.8888888888889\\
11.155	13.8888888888889\\
11.175	13.8888888888889\\
11.195	13.8888888888889\\
11.215	13.8888888888889\\
11.235	13.8888888888889\\
11.255	13.8888888888889\\
11.275	13.8888888888889\\
11.295	13.8888888888889\\
11.315	13.8888888888889\\
11.335	13.8888888888889\\
11.355	13.8888888888889\\
11.375	13.8888888888889\\
11.395	13.8888888888889\\
11.415	13.8888888888889\\
11.435	13.8888888888889\\
11.455	13.8888888888889\\
11.475	13.8888888888889\\
11.495	13.8888888888889\\
11.515	13.8888888888889\\
11.535	13.8888888888889\\
11.555	13.8888888888889\\
11.575	13.8888888888889\\
11.595	13.8888888888889\\
11.615	13.8888888888889\\
11.635	13.8888888888889\\
11.655	13.8888888888889\\
11.675	13.8888888888889\\
11.695	13.8888888888889\\
11.715	13.8888888888889\\
11.735	13.8888888888889\\
11.755	13.8888888888889\\
11.775	13.8888888888889\\
11.795	13.8888888888889\\
11.815	13.8888888888889\\
11.835	13.8888888888889\\
11.855	13.8888888888889\\
11.875	13.8888888888889\\
11.895	13.8888888888889\\
11.915	13.8888888888889\\
11.935	13.8888888888889\\
11.955	13.8888888888889\\
11.975	13.8888888888889\\
11.995	13.8888888888889\\
12.015	13.8888888888889\\
12.035	13.8888888888889\\
12.055	13.8888888888889\\
12.075	13.8888888888889\\
12.095	13.8888888888889\\
12.115	13.8888888888889\\
12.135	13.8888888888889\\
12.155	13.8888888888889\\
12.175	13.8888888888889\\
12.195	13.8888888888889\\
12.215	13.8888888888889\\
12.235	13.8888888888889\\
12.255	13.8888888888889\\
12.275	13.8888888888889\\
12.295	13.8888888888889\\
12.315	13.8888888888889\\
12.335	13.8888888888889\\
12.355	13.8888888888889\\
12.375	13.8888888888889\\
12.395	13.8888888888889\\
12.415	13.8888888888889\\
12.435	13.8888888888889\\
12.455	13.8888888888889\\
12.475	13.8888888888889\\
12.495	13.8888888888889\\
12.515	13.8888888888889\\
12.535	13.8888888888889\\
12.555	13.8888888888889\\
12.575	13.8888888888889\\
12.595	13.8888888888889\\
12.615	13.8888888888889\\
12.635	13.8888888888889\\
12.655	13.8888888888889\\
12.675	13.8888888888889\\
12.695	13.8888888888889\\
12.715	13.8888888888889\\
12.735	13.8888888888889\\
12.755	13.8888888888889\\
12.775	13.8888888888889\\
12.795	13.8888888888889\\
12.815	13.8888888888889\\
12.835	13.8888888888889\\
12.855	13.8888888888889\\
12.875	13.8888888888889\\
12.895	13.8888888888889\\
12.915	13.8888888888889\\
12.935	13.8888888888889\\
12.955	13.8888888888889\\
12.975	13.8888888888889\\
12.995	0\\
13.015	0\\
13.035	0\\
13.055	0\\
13.075	0\\
13.095	0\\
13.115	0\\
13.135	0\\
13.155	0\\
13.175	0\\
13.195	0\\
13.215	0\\
13.235	0\\
13.255	0\\
13.275	0\\
13.295	0\\
13.315	0\\
13.335	0\\
13.355	0\\
13.375	0\\
13.395	0\\
13.415	0\\
13.435	0\\
13.455	0\\
13.475	0\\
13.495	0\\
13.515	0\\
13.535	0\\
13.555	0\\
13.575	0\\
13.595	0\\
13.615	0\\
13.635	0\\
13.655	0\\
13.675	0\\
13.695	0\\
13.715	0\\
13.735	0\\
13.755	0\\
13.775	0\\
13.795	0\\
13.815	0\\
13.835	0\\
13.855	0\\
13.875	0\\
13.895	0\\
13.915	0\\
13.935	0\\
13.955	0\\
13.975	0\\
13.995	0\\
14.015	0\\
14.035	0\\
14.055	0\\
14.075	0\\
14.095	0\\
14.115	0\\
14.135	0\\
14.155	0\\
14.175	0\\
14.195	0\\
14.215	0\\
14.235	0\\
14.255	0\\
14.275	0\\
14.295	0\\
14.315	0\\
14.335	0\\
14.355	0\\
14.375	0\\
14.395	0\\
14.415	0\\
14.435	0\\
14.455	0\\
14.475	0\\
14.495	0\\
14.515	0\\
14.535	0\\
14.555	0\\
14.575	0\\
14.595	0\\
14.615	0\\
14.635	0\\
14.655	0\\
14.675	0\\
14.695	0\\
14.715	0\\
14.735	0\\
14.755	0\\
14.775	0\\
14.795	0\\
14.815	0\\
14.835	0\\
14.855	0\\
14.875	0\\
14.895	0\\
14.915	0\\
14.935	0\\
14.955	0\\
14.975	0\\
14.995	0\\
15.015	0\\
15.035	0\\
15.055	0\\
15.075	0\\
15.095	0\\
15.115	0\\
15.135	0\\
15.155	0\\
15.175	0\\
15.195	0\\
15.215	0\\
15.235	0\\
15.255	0\\
15.275	0\\
15.295	0\\
15.315	0\\
15.335	0\\
15.355	0\\
15.375	0\\
15.395	0\\
15.415	0\\
15.435	0\\
15.455	0\\
15.475	0\\
15.495	0\\
15.515	0\\
15.535	0\\
15.555	0\\
15.575	0\\
15.595	0\\
15.615	0\\
15.635	0\\
15.655	0\\
15.675	0\\
15.695	0\\
15.715	0\\
15.735	0\\
15.755	0\\
15.775	0\\
15.795	0\\
15.815	0\\
15.835	0\\
15.855	0\\
15.875	0\\
15.895	0\\
15.915	0\\
15.935	0\\
15.955	0\\
15.975	0\\
15.995	0\\
16.015	0\\
16.035	0\\
16.055	0\\
16.075	0\\
16.095	0\\
16.115	0\\
16.135	0\\
16.155	0\\
16.175	0\\
16.195	0\\
16.215	0\\
16.235	0\\
16.255	0\\
16.275	0\\
16.295	0\\
16.315	0\\
16.335	0\\
16.355	0\\
16.375	0\\
16.395	0\\
16.415	0\\
16.435	0\\
16.455	0\\
16.475	0\\
16.495	0\\
16.515	0\\
16.535	0\\
16.555	0\\
16.575	0\\
16.595	0\\
16.615	0\\
16.635	0\\
16.655	0\\
16.675	0\\
16.695	0\\
16.715	0\\
16.735	0\\
16.755	0\\
16.775	0\\
16.795	0\\
16.815	0\\
16.835	0\\
16.855	0\\
16.875	0\\
16.895	0\\
16.915	0\\
16.935	0\\
16.955	0\\
16.975	0\\
16.995	0\\
17.015	0\\
17.035	0\\
17.055	0\\
17.075	0\\
17.095	0\\
17.115	0\\
17.135	0\\
17.155	0\\
17.175	0\\
17.195	0\\
17.215	0\\
17.235	0\\
17.255	0\\
17.275	0\\
17.295	0\\
17.315	0\\
17.335	0\\
17.355	0\\
17.375	0\\
17.395	0\\
17.415	0\\
17.435	0\\
17.455	0\\
17.475	0\\
17.495	0\\
17.515	0\\
17.535	0\\
17.555	0\\
17.575	0\\
17.595	0\\
17.615	0\\
17.635	0\\
17.655	0\\
17.675	0\\
17.695	0\\
17.715	0\\
17.735	0\\
17.755	0\\
17.775	0\\
17.795	0\\
17.815	0\\
17.835	0\\
17.855	0\\
17.875	0\\
17.895	0\\
17.915	0\\
17.935	0\\
17.955	0\\
17.975	0\\
17.995	0\\
18.015	0\\
18.035	0\\
18.055	0\\
18.075	0\\
18.095	0\\
18.115	0\\
18.135	0\\
18.155	0\\
18.175	0\\
18.195	0\\
18.215	0\\
18.235	0\\
18.255	0\\
18.275	0\\
18.295	0\\
18.315	0\\
18.335	0\\
18.355	0\\
18.375	0\\
18.395	0\\
18.415	0\\
18.435	0\\
18.455	0\\
18.475	0\\
18.495	0\\
18.515	0\\
18.535	0\\
18.555	0\\
18.575	0\\
18.595	0\\
18.615	0\\
18.635	0\\
18.655	0\\
18.675	0\\
18.695	0\\
18.715	0\\
18.735	0\\
18.755	0\\
18.775	0\\
18.795	0\\
18.815	0\\
18.835	0\\
18.855	0\\
18.875	0\\
18.895	0\\
18.915	0\\
18.935	0\\
18.955	0\\
18.975	0\\
18.995	0\\
19.015	0\\
19.035	0\\
19.055	0\\
19.075	0\\
19.095	0\\
19.115	0\\
19.135	0\\
19.155	0\\
19.175	0\\
19.195	0\\
19.215	0\\
19.235	0\\
19.255	0\\
19.275	0\\
19.295	0\\
19.315	0\\
19.335	0\\
19.355	0\\
19.375	0\\
19.395	0\\
19.415	0\\
19.435	0\\
19.455	0\\
19.475	0\\
19.495	0\\
19.515	0\\
19.535	0\\
19.555	0\\
19.575	0\\
19.595	0\\
19.615	0\\
19.635	0\\
19.655	0\\
19.675	0\\
19.695	0\\
19.715	0\\
19.735	0\\
19.755	0\\
19.775	0\\
19.795	0\\
19.815	0\\
19.835	0\\
19.855	0\\
19.875	0\\
19.895	0\\
19.915	0\\
19.935	0\\
19.955	0\\
19.975	0\\
};
\addlegendentry{Zielgr\"o\ss en};

\end{axis}

\begin{axis}[%
width=0.410625\figurewidth,
height=0.264706\figureheight,
at={(0\figurewidth,0.735294\figureheight)},
scale only axis,
separate axis lines,
every outer x axis line/.append style={black},
every x tick label/.append style={font=\color{black}},
xmin=11,
xmax=20,
xmajorgrids,
every outer y axis line/.append style={black},
every y tick label/.append style={font=\color{black}},
ymin=380,
ymax=480,
ylabel={$x$ [m]},
ylabel near ticks,
ymajorgrids
]
\addplot [color=light-gray,solid,forget plot,line width=1.0]
  table[row sep=crcr]{%
10.975	398.669372558594\\
10.995	398.947143554688\\
11.015	399.224487304688\\
11.035	399.502075195313\\
11.055	399.779602050781\\
11.075	400.056945800781\\
11.095	400.33447265625\\
11.115	400.611877441406\\
11.135	400.889221191406\\
11.155	401.166564941406\\
11.175	401.443664550781\\
11.195	401.720916748047\\
11.215	401.998077392578\\
11.235	402.275421142578\\
11.255	402.552429199219\\
11.275	402.829528808594\\
11.295	403.106597900391\\
11.315	403.383148193359\\
11.335	403.660064697266\\
11.355	403.937164306641\\
11.375	404.213897705078\\
11.395	404.490875244141\\
11.415	404.767974853516\\
11.435	405.044952392578\\
11.455	405.322143554688\\
11.475	405.598876953125\\
11.495	405.876037597656\\
11.515	406.153289794922\\
11.535	406.430541992188\\
11.555	406.707641601563\\
11.575	406.984710693359\\
11.595	407.261901855469\\
11.615	407.5390625\\
11.635	407.816223144531\\
11.655	408.093475341797\\
11.675	408.370819091797\\
11.695	408.648254394531\\
11.715	408.92578125\\
11.735	409.202941894531\\
11.755	409.480133056641\\
11.775	409.757385253906\\
11.795	410.034729003906\\
11.815	410.311981201172\\
11.835	410.589141845703\\
11.855	410.866394042969\\
11.875	411.143676757813\\
11.895	411.420562744141\\
11.915	411.697570800781\\
11.935	411.974395751953\\
11.955	412.251220703125\\
11.975	412.528137207031\\
11.995	412.805023193359\\
12.015	413.08203125\\
12.035	413.359039306641\\
12.055	413.636199951172\\
12.075	413.913360595703\\
12.095	414.190551757813\\
12.115	414.467895507813\\
12.135	414.745056152344\\
12.155	415.022399902344\\
12.175	415.299743652344\\
12.195	415.576904296875\\
12.215	415.854248046875\\
12.235	416.131591796875\\
12.255	416.408752441406\\
12.275	416.686187744141\\
12.295	416.96337890625\\
12.315	417.240783691406\\
12.335	417.518127441406\\
12.355	417.795471191406\\
12.375	418.072906494141\\
12.395	418.350158691406\\
12.415	418.627349853516\\
12.435	418.90478515625\\
12.455	419.18212890625\\
12.475	419.459625244141\\
12.495	419.736877441406\\
12.515	420.014221191406\\
12.535	420.291748046875\\
12.555	420.569183349609\\
12.575	420.846710205078\\
12.595	421.124206542969\\
12.615	421.401641845703\\
12.635	421.679077148438\\
12.655	421.956512451172\\
12.675	422.234100341797\\
12.695	422.511901855469\\
12.715	422.789489746094\\
12.735	423.067108154297\\
12.755	423.344787597656\\
12.775	423.622406005859\\
12.795	423.899841308594\\
12.815	424.177001953125\\
12.835	424.454437255859\\
12.855	424.73193359375\\
12.875	425.009368896484\\
12.895	425.286804199219\\
12.915	425.564147949219\\
12.935	425.841583251953\\
12.955	426.118835449219\\
12.975	426.396087646484\\
12.995	426.673522949219\\
13.015	426.950958251953\\
13.035	427.228485107422\\
13.055	427.506256103516\\
13.075	427.78369140625\\
13.095	428.061096191406\\
13.115	428.338439941406\\
13.135	428.61572265625\\
13.155	428.893218994141\\
13.175	429.170928955078\\
13.195	429.448699951172\\
13.215	429.7265625\\
13.235	430.004425048828\\
13.255	430.282287597656\\
13.275	430.56005859375\\
13.295	430.837768554688\\
13.315	431.115783691406\\
13.335	431.393585205078\\
13.355	431.671600341797\\
13.375	431.949829101563\\
13.395	432.227600097656\\
13.415	432.505645751953\\
13.435	432.783508300781\\
13.455	433.061370849609\\
13.475	433.339508056641\\
13.495	433.617462158203\\
13.515	433.895477294922\\
13.535	434.173278808594\\
13.555	434.450531005859\\
13.575	434.727966308594\\
13.595	435.005035400391\\
13.615	435.281951904297\\
13.635	435.558410644531\\
13.655	435.835083007813\\
13.675	436.111633300781\\
13.695	436.387847900391\\
13.715	436.663269042969\\
13.735	436.938812255859\\
13.755	437.213714599609\\
13.775	437.488098144531\\
13.795	437.762237548828\\
13.815	438.036285400391\\
13.835	438.309906005859\\
13.855	438.582458496094\\
13.875	438.854858398438\\
13.895	439.126647949219\\
13.915	439.397399902344\\
13.935	439.667694091797\\
13.955	439.937316894531\\
13.975	440.205810546875\\
13.995	440.473693847656\\
14.015	440.740631103516\\
14.035	441.007385253906\\
14.055	441.272583007813\\
14.075	441.53759765625\\
14.095	441.801116943359\\
14.115	442.064331054688\\
14.135	442.326385498047\\
14.155	442.587493896484\\
14.175	442.848358154297\\
14.195	443.107452392578\\
14.215	443.366058349609\\
14.235	443.623687744141\\
14.255	443.880737304688\\
14.275	444.136291503906\\
14.295	444.391052246094\\
14.315	444.644348144531\\
14.335	444.897216796875\\
14.355	445.148712158203\\
14.375	445.399658203125\\
14.395	445.649139404297\\
14.415	445.897491455078\\
14.435	446.144714355469\\
14.455	446.390350341797\\
14.475	446.635070800781\\
14.495	446.878112792969\\
14.515	447.120239257813\\
14.535	447.361297607422\\
14.555	447.601135253906\\
14.575	447.83984375\\
14.595	448.076812744141\\
14.615	448.312774658203\\
14.635	448.547668457031\\
14.655	448.780914306641\\
14.675	449.012847900391\\
14.695	449.243408203125\\
14.715	449.472747802734\\
14.735	449.70068359375\\
14.755	449.92724609375\\
14.775	450.152618408203\\
14.795	450.376831054688\\
14.815	450.599639892578\\
14.835	450.820587158203\\
14.855	451.039672851563\\
14.875	451.257629394531\\
14.895	451.474639892578\\
14.915	451.690277099609\\
14.935	451.904846191406\\
14.955	452.118041992188\\
14.975	452.32958984375\\
14.995	452.539672851563\\
15.015	452.74853515625\\
15.035	452.954772949219\\
15.055	453.159973144531\\
15.075	453.364135742188\\
15.095	453.56640625\\
15.115	453.767181396484\\
15.135	453.966247558594\\
15.155	454.163635253906\\
15.175	454.359558105469\\
15.195	454.553833007813\\
15.215	454.746704101563\\
15.235	454.937591552734\\
15.255	455.126556396484\\
15.275	455.31396484375\\
15.295	455.500183105469\\
15.315	455.684387207031\\
15.335	455.866851806641\\
15.355	456.047576904297\\
15.375	456.226226806641\\
15.395	456.403564453125\\
15.415	456.579345703125\\
15.435	456.753387451172\\
15.455	456.925598144531\\
15.475	457.095825195313\\
15.495	457.264495849609\\
15.515	457.431762695313\\
15.535	457.597747802734\\
15.555	457.762237548828\\
15.575	457.924652099609\\
15.595	458.085235595703\\
15.615	458.244018554688\\
15.635	458.401397705078\\
15.655	458.556945800781\\
15.675	458.711120605469\\
15.695	458.863555908203\\
15.715	459.01416015625\\
15.735	459.163116455078\\
15.755	459.310150146484\\
15.775	459.455718994141\\
15.795	459.59912109375\\
15.815	459.741058349609\\
15.835	459.881500244141\\
15.855	460.019958496094\\
15.875	460.156585693359\\
15.895	460.291229248047\\
15.915	460.424377441406\\
15.935	460.555908203125\\
15.955	460.685577392578\\
15.975	460.813720703125\\
15.995	460.939666748047\\
16.015	461.064056396484\\
16.035	461.186889648438\\
16.055	461.307556152344\\
16.075	461.426574707031\\
16.095	461.544097900391\\
16.115	461.659973144531\\
16.135	461.773773193359\\
16.155	461.885772705078\\
16.175	461.996276855469\\
16.195	462.105041503906\\
16.215	462.211883544922\\
16.235	462.317016601563\\
16.255	462.420654296875\\
16.275	462.522644042969\\
16.295	462.622833251953\\
16.315	462.721527099609\\
16.335	462.818328857422\\
16.355	462.913635253906\\
16.375	463.007537841797\\
16.395	463.09912109375\\
16.415	463.189239501953\\
16.435	463.277954101563\\
16.455	463.365203857422\\
16.475	463.450622558594\\
16.495	463.534637451172\\
16.515	463.617370605469\\
16.535	463.698364257813\\
16.555	463.777770996094\\
16.575	463.855743408203\\
16.595	463.932189941406\\
16.615	464.007110595703\\
16.635	464.080718994141\\
16.655	464.152526855469\\
16.675	464.222930908203\\
16.695	464.292114257813\\
16.715	464.359375\\
16.735	464.425354003906\\
16.755	464.489837646484\\
16.775	464.552947998047\\
16.795	464.614837646484\\
16.815	464.675170898438\\
16.835	464.733947753906\\
16.855	464.791412353516\\
16.875	464.847381591797\\
16.895	464.902252197266\\
16.915	464.955718994141\\
16.935	465.007385253906\\
16.955	465.057891845703\\
16.975	465.107116699219\\
16.995	465.155029296875\\
17.015	465.201812744141\\
17.035	465.246948242188\\
17.055	465.291046142578\\
17.075	465.333953857422\\
17.095	465.375610351563\\
17.115	465.415985107422\\
17.135	465.455108642578\\
17.155	465.493133544922\\
17.175	465.529937744141\\
17.195	465.565612792969\\
17.215	465.600250244141\\
17.235	465.633850097656\\
17.255	465.666320800781\\
17.275	465.697814941406\\
17.295	465.728546142578\\
17.315	465.758331298828\\
17.335	465.787139892578\\
17.355	465.815002441406\\
17.375	465.84228515625\\
17.395	465.868591308594\\
17.415	465.894012451172\\
17.435	465.918579101563\\
17.455	465.942260742188\\
17.475	465.965179443359\\
17.495	465.987762451172\\
17.515	466.009460449219\\
17.535	466.030395507813\\
17.555	466.050872802734\\
17.575	466.070587158203\\
17.595	466.089569091797\\
17.615	466.10791015625\\
17.635	466.125701904297\\
17.655	466.142883300781\\
17.675	466.159362792969\\
17.695	466.175262451172\\
17.715	466.190612792969\\
17.735	466.205474853516\\
17.755	466.219787597656\\
17.775	466.233764648438\\
17.795	466.247222900391\\
17.815	466.26025390625\\
17.835	466.272827148438\\
17.855	466.284881591797\\
17.875	466.296447753906\\
17.895	466.307556152344\\
17.915	466.318237304688\\
17.935	466.328826904297\\
17.955	466.338989257813\\
17.975	466.348785400391\\
17.995	466.358245849609\\
18.015	466.367370605469\\
18.035	466.376403808594\\
18.055	466.385070800781\\
18.075	466.393402099609\\
18.095	466.401489257813\\
18.115	466.409210205078\\
18.135	466.416656494141\\
18.155	466.423858642578\\
18.175	466.430908203125\\
18.195	466.437683105469\\
18.215	466.444183349609\\
18.235	466.450439453125\\
18.255	466.456512451172\\
18.275	466.46240234375\\
18.295	466.468048095703\\
18.315	466.473510742188\\
18.335	466.478820800781\\
18.355	466.484039306641\\
18.375	466.489074707031\\
18.395	466.493927001953\\
18.415	466.498596191406\\
18.435	466.503204345703\\
18.455	466.507629394531\\
18.475	466.511993408203\\
18.495	466.516235351563\\
18.515	466.520324707031\\
18.535	466.524291992188\\
18.555	466.528198242188\\
18.575	466.532043457031\\
18.595	466.535766601563\\
18.615	466.539398193359\\
18.635	466.542877197266\\
18.655	466.546264648438\\
18.675	466.549652099609\\
18.695	466.553039550781\\
18.715	466.556427001953\\
18.735	466.559814453125\\
18.755	466.563201904297\\
18.775	466.566497802734\\
18.795	466.569885253906\\
18.815	466.573272705078\\
18.835	466.576721191406\\
18.855	466.580383300781\\
18.875	466.583953857422\\
18.895	466.587585449219\\
18.915	466.591247558594\\
18.935	466.594879150391\\
18.955	466.598510742188\\
18.975	466.602355957031\\
18.995	466.606506347656\\
19.015	466.610687255859\\
19.035	466.614837646484\\
19.055	466.619171142578\\
19.075	466.62353515625\\
19.095	466.627868652344\\
19.115	466.632202148438\\
19.135	466.63671875\\
19.155	466.641418457031\\
19.175	466.646179199219\\
19.195	466.651031494141\\
19.215	466.655914306641\\
19.235	466.660766601563\\
19.255	466.665618896484\\
19.275	466.670471191406\\
19.295	466.675354003906\\
19.315	466.680023193359\\
19.335	466.684448242188\\
19.355	466.688812255859\\
19.375	466.693145751953\\
19.395	466.697479248047\\
19.415	466.701812744141\\
19.435	466.706085205078\\
19.455	466.710144042969\\
19.475	466.714233398438\\
19.495	466.718322753906\\
19.515	466.722229003906\\
19.535	466.726043701172\\
19.555	466.729858398438\\
19.575	466.733520507813\\
19.595	466.737152099609\\
19.615	466.740783691406\\
19.635	466.744354248047\\
19.655	466.747833251953\\
19.675	466.751312255859\\
19.695	466.754516601563\\
19.715	466.757720947266\\
19.735	466.760925292969\\
19.755	466.763977050781\\
19.775	466.767028808594\\
19.795	466.770050048828\\
19.815	466.773101806641\\
19.835	466.776031494141\\
19.855	466.778991699219\\
19.875	466.781860351563\\
19.895	466.784637451172\\
19.915	466.787414550781\\
19.935	466.790100097656\\
19.955	466.792877197266\\
19.975	466.795318603516\\
};
\addplot [color=black,solid,forget plot, line width=1.0]
  table[row sep=crcr]{%
10.975	396.200775146484\\
10.995	396.478515625\\
11.015	396.725433349609\\
11.035	397.003204345703\\
11.055	397.281005859375\\
11.075	397.558776855469\\
11.095	397.836547851563\\
11.115	398.114318847656\\
11.135	398.392120361328\\
11.155	398.669891357422\\
11.175	398.947662353516\\
11.195	399.225433349609\\
11.215	399.503204345703\\
11.235	399.781005859375\\
11.255	400.058776855469\\
11.275	400.336547851563\\
11.295	400.614318847656\\
11.315	400.892120361328\\
11.335	401.169891357422\\
11.355	401.447662353516\\
11.375	401.725433349609\\
11.395	402.003204345703\\
11.415	402.281005859375\\
11.435	402.558776855469\\
11.455	402.836547851563\\
11.475	403.114318847656\\
11.495	403.392120361328\\
11.515	403.669891357422\\
11.535	403.947662353516\\
11.555	404.225433349609\\
11.575	404.503204345703\\
11.595	404.781005859375\\
11.615	405.058776855469\\
11.635	405.336547851563\\
11.655	405.614318847656\\
11.675	405.892120361328\\
11.695	406.169891357422\\
11.715	406.447662353516\\
11.735	406.725433349609\\
11.755	407.003204345703\\
11.775	407.262481689453\\
11.795	407.540252685547\\
11.815	407.818054199219\\
11.835	408.095825195313\\
11.855	408.373596191406\\
11.875	408.6513671875\\
11.895	408.929168701172\\
11.915	409.206939697266\\
11.935	409.484710693359\\
11.955	409.762481689453\\
11.975	410.040252685547\\
11.995	410.318054199219\\
12.015	410.595825195313\\
12.035	410.873596191406\\
12.055	411.1513671875\\
12.075	411.429168701172\\
12.095	411.706939697266\\
12.115	411.984710693359\\
12.135	412.262481689453\\
12.155	412.540252685547\\
12.175	412.818054199219\\
12.195	413.095825195313\\
12.215	413.373596191406\\
12.235	413.6513671875\\
12.255	413.929168701172\\
12.275	414.206939697266\\
12.295	414.484710693359\\
12.315	414.762481689453\\
12.335	415.040252685547\\
12.355	415.318054199219\\
12.375	415.595825195313\\
12.395	415.873596191406\\
12.415	416.1513671875\\
12.435	416.429138183594\\
12.455	416.706939697266\\
12.475	416.984710693359\\
12.495	417.262481689453\\
12.515	417.540252685547\\
12.535	417.7958984375\\
12.555	418.073669433594\\
12.575	418.351470947266\\
12.595	418.629241943359\\
12.615	418.907012939453\\
12.635	419.184783935547\\
12.655	419.462585449219\\
12.675	419.740356445313\\
12.695	420.018127441406\\
12.715	420.2958984375\\
12.735	420.573669433594\\
12.755	420.851470947266\\
12.775	421.129241943359\\
12.795	421.407012939453\\
12.815	421.684783935547\\
12.835	421.962585449219\\
12.855	422.240356445313\\
12.875	422.518127441406\\
12.895	422.7958984375\\
12.915	423.073669433594\\
12.935	423.351470947266\\
12.955	423.629241943359\\
12.975	423.907012939453\\
12.995	424.184783935547\\
13.015	424.462585449219\\
13.035	424.740356445313\\
13.055	425.018127441406\\
13.075	425.2958984375\\
13.095	425.573669433594\\
13.115	425.851470947266\\
13.135	426.129241943359\\
13.155	426.407012939453\\
13.175	426.684783935547\\
13.195	426.962554931641\\
13.215	427.240356445313\\
13.235	427.518127441406\\
13.255	427.7958984375\\
13.275	428.073669433594\\
13.295	428.3388671875\\
13.315	428.616638183594\\
13.335	428.894439697266\\
13.355	429.172210693359\\
13.375	429.449951171875\\
13.395	429.727691650391\\
13.415	430.005432128906\\
13.435	430.283111572266\\
13.455	430.560760498047\\
13.475	430.838409423828\\
13.495	431.115936279297\\
13.515	431.393463134766\\
13.535	431.670867919922\\
13.555	431.948181152344\\
13.575	432.225402832031\\
13.595	432.502532958984\\
13.615	432.779510498047\\
13.635	433.056304931641\\
13.655	433.332885742188\\
13.675	433.609375\\
13.695	433.885681152344\\
13.715	434.161682128906\\
13.735	434.437469482422\\
13.755	434.713043212891\\
13.775	434.98828125\\
13.795	435.263244628906\\
13.815	435.537872314453\\
13.835	435.812133789063\\
13.855	436.086059570313\\
13.875	436.359619140625\\
13.895	436.632720947266\\
13.915	436.905364990234\\
13.935	437.177551269531\\
13.955	437.449249267578\\
13.975	437.720489501953\\
13.995	437.991149902344\\
14.015	438.261260986328\\
14.035	438.53076171875\\
14.055	438.851959228516\\
14.075	439.120849609375\\
14.095	439.389099121094\\
14.115	439.656677246094\\
14.135	439.923553466797\\
14.155	440.189727783203\\
14.175	440.455169677734\\
14.195	440.719848632813\\
14.215	440.983703613281\\
14.235	441.246734619141\\
14.255	441.508972167969\\
14.275	441.770294189453\\
14.295	442.030731201172\\
14.315	442.290283203125\\
14.335	442.548858642578\\
14.355	442.806518554688\\
14.375	443.063171386719\\
14.395	443.318817138672\\
14.415	443.573425292969\\
14.435	443.827056884766\\
14.455	444.079559326172\\
14.475	444.330963134766\\
14.495	444.581268310547\\
14.515	444.830413818359\\
14.535	445.078430175781\\
14.555	445.325256347656\\
14.575	445.570922851563\\
14.595	445.815307617188\\
14.615	446.058532714844\\
14.635	446.300445556641\\
14.655	446.541076660156\\
14.675	446.780456542969\\
14.695	447.018493652344\\
14.715	447.255187988281\\
14.735	447.490539550781\\
14.755	447.724578857422\\
14.775	447.957214355469\\
14.795	448.188446044922\\
14.815	448.544006347656\\
14.835	448.773773193359\\
14.855	449.002197265625\\
14.875	449.229095458984\\
14.895	449.454528808594\\
14.915	449.678497314453\\
14.935	449.901000976563\\
14.955	450.121978759766\\
14.975	450.341430664063\\
14.995	450.559326171875\\
15.015	450.775634765625\\
15.035	450.990447998047\\
15.055	451.203643798828\\
15.075	451.415222167969\\
15.095	451.625244140625\\
15.115	451.833648681641\\
15.135	452.040374755859\\
15.155	452.245483398438\\
15.175	452.448913574219\\
15.195	452.650695800781\\
15.215	452.850799560547\\
15.235	453.049194335938\\
15.255	453.245910644531\\
15.275	453.44091796875\\
15.295	453.634216308594\\
15.315	453.825775146484\\
15.335	454.015594482422\\
15.355	454.203704833984\\
15.375	454.390045166016\\
15.395	454.574584960938\\
15.415	454.757385253906\\
15.435	454.938415527344\\
15.455	455.117645263672\\
15.475	455.295104980469\\
15.495	455.470794677734\\
15.515	455.644653320313\\
15.535	455.816711425781\\
15.555	455.986999511719\\
15.575	456.396453857422\\
15.595	456.564910888672\\
15.615	456.731506347656\\
15.635	456.896301269531\\
15.655	457.059295654297\\
15.675	457.220428466797\\
15.695	457.379730224609\\
15.715	457.537200927734\\
15.735	457.692840576172\\
15.755	457.846557617188\\
15.775	457.99853515625\\
15.795	458.148620605469\\
15.815	458.296905517578\\
15.835	458.443267822266\\
15.855	458.587860107422\\
15.875	458.730560302734\\
15.895	458.871398925781\\
15.915	459.010437011719\\
15.935	459.147644042969\\
15.955	459.282928466797\\
15.975	459.416442871094\\
15.995	459.548065185547\\
16.015	459.677825927734\\
16.035	459.805847167969\\
16.055	459.931976318359\\
16.075	460.056274414063\\
16.095	460.178771972656\\
16.115	460.299407958984\\
16.135	460.418273925781\\
16.155	460.535278320313\\
16.175	460.650512695313\\
16.195	460.763916015625\\
16.215	460.875549316406\\
16.235	460.9853515625\\
16.255	461.093353271484\\
16.275	461.199615478516\\
16.295	461.304077148438\\
16.315	461.406829833984\\
16.335	461.876495361328\\
16.355	461.977416992188\\
16.375	462.07666015625\\
16.395	462.174133300781\\
16.415	462.269866943359\\
16.435	462.363861083984\\
16.455	462.456176757813\\
16.475	462.546752929688\\
16.495	462.635589599609\\
16.515	462.722808837891\\
16.535	462.808319091797\\
16.555	462.892150878906\\
16.575	462.974334716797\\
16.595	463.054901123047\\
16.615	463.1337890625\\
16.635	463.211120605469\\
16.655	463.286773681641\\
16.675	463.360870361328\\
16.695	463.433288574219\\
16.715	463.504211425781\\
16.735	463.573577880859\\
16.755	463.641357421875\\
16.775	463.707580566406\\
16.795	463.772308349609\\
16.815	463.835540771484\\
16.835	463.897277832031\\
16.855	463.95751953125\\
16.875	464.016296386719\\
16.895	464.073577880859\\
16.915	464.129486083984\\
16.935	464.183929443359\\
16.955	464.236968994141\\
16.975	464.605987548828\\
16.995	464.657684326172\\
17.015	464.826019287109\\
17.035	464.873626708984\\
17.055	464.919860839844\\
17.075	464.964813232422\\
17.095	465.008483886719\\
17.115	465.050903320313\\
17.135	465.092071533203\\
17.155	465.131958007813\\
17.175	465.170684814453\\
17.195	465.208190917969\\
17.215	465.244537353516\\
17.235	465.279724121094\\
17.255	465.313751220703\\
17.275	465.346710205078\\
17.295	465.378570556641\\
17.315	465.409423828125\\
17.335	465.439117431641\\
17.355	465.467834472656\\
17.375	465.495544433594\\
17.395	465.522247314453\\
17.415	465.548004150391\\
17.435	465.572784423828\\
17.455	465.596649169922\\
17.475	465.619659423828\\
17.495	465.641723632813\\
17.515	465.662994384766\\
17.535	465.683380126953\\
17.555	465.702911376953\\
17.575	465.721740722656\\
17.595	465.739715576172\\
17.615	465.757019042969\\
17.635	465.773498535156\\
17.655	465.789306640625\\
17.675	465.804382324219\\
17.695	465.818756103516\\
17.715	465.832550048828\\
17.735	465.845703125\\
17.755	465.858215332031\\
17.775	466.083099365234\\
17.795	466.095031738281\\
17.815	466.106323242188\\
17.835	466.117095947266\\
17.855	466.127319335938\\
17.875	466.137054443359\\
17.895	466.146209716797\\
17.915	466.154907226563\\
17.935	466.163146972656\\
17.955	466.170928955078\\
17.975	466.178283691406\\
17.995	466.185150146484\\
18.015	466.191711425781\\
18.035	466.197845458984\\
18.055	466.203643798828\\
18.075	466.209045410156\\
18.095	466.214080810547\\
18.115	466.218902587891\\
18.135	466.223297119141\\
18.155	466.227447509766\\
18.175	466.231231689453\\
18.195	466.23486328125\\
18.215	466.238189697266\\
18.235	466.241271972656\\
18.255	466.244140625\\
18.275	466.246734619141\\
18.295	466.249145507813\\
18.315	466.25146484375\\
18.335	466.253509521484\\
18.355	466.25537109375\\
18.375	466.257019042969\\
18.395	466.258575439453\\
18.415	466.260009765625\\
18.435	466.261291503906\\
18.455	466.262481689453\\
18.475	466.263458251953\\
18.495	466.264434814453\\
18.515	466.265258789063\\
18.535	466.479644775391\\
18.555	466.480377197266\\
18.575	466.481048583984\\
18.595	466.481628417969\\
18.615	466.482116699219\\
18.635	466.482543945313\\
18.655	466.482971191406\\
18.675	466.483276367188\\
18.695	466.483520507813\\
18.715	466.483734130859\\
18.735	466.483978271484\\
18.755	466.484069824219\\
18.775	466.484191894531\\
18.795	466.484313964844\\
18.815	466.484344482422\\
18.835	466.484405517578\\
18.855	466.484405517578\\
18.875	466.484405517578\\
18.895	466.484375\\
18.915	466.484344482422\\
18.935	466.484344482422\\
18.955	466.484313964844\\
18.975	466.484252929688\\
18.995	466.484161376953\\
19.015	466.484100341797\\
19.035	466.484069824219\\
19.055	466.484008789063\\
19.075	466.483917236328\\
19.095	466.48388671875\\
19.115	466.483856201172\\
19.135	466.483703613281\\
19.155	466.483673095703\\
19.175	466.483581542969\\
19.195	466.483520507813\\
19.215	466.483489990234\\
19.235	466.483489990234\\
19.255	466.483428955078\\
19.275	466.483337402344\\
19.295	466.627868652344\\
19.315	466.627868652344\\
19.335	466.627899169922\\
19.355	466.627899169922\\
19.375	466.6279296875\\
19.395	466.6279296875\\
19.415	466.627960205078\\
19.435	466.627990722656\\
19.455	466.628021240234\\
19.475	466.628051757813\\
19.495	466.628051757813\\
19.515	466.628143310547\\
19.535	466.628173828125\\
19.555	466.628234863281\\
19.575	466.628265380859\\
19.595	466.628326416016\\
19.615	466.62841796875\\
19.635	466.628448486328\\
19.655	466.628479003906\\
19.675	466.628570556641\\
19.695	466.628631591797\\
19.715	466.628662109375\\
19.735	466.628723144531\\
19.755	466.628814697266\\
19.775	466.62890625\\
19.795	466.62890625\\
19.815	466.628997802734\\
19.835	466.629058837891\\
19.855	466.629150390625\\
19.875	466.629241943359\\
19.895	466.629241943359\\
19.915	466.629333496094\\
19.935	466.62939453125\\
19.955	466.629455566406\\
19.975	466.629486083984\\
};
\end{axis}
\end{tikzpicture}% 
      \begin{center}
       		\caption{gekoppeltes Quer-Längsmanöver zum Ausweichen mit gleichzeitigem Abbremsen in den Stillstand}
     		 \label{abb_LQ_Bremsung}
       \end{center}
   \end{figure}   
Das Manöver findet ohne Fahrereingriffe statt. Aus einer Anfangsgeschwindigkeit von 50 km/h wird in den Stillstand abgebremst und gleichzeitig eine Querablage von 3 m relativ zur Ausgangslage aufgebaut. Die geplante Trajektorie wird dabei durch die Regelung mit minimalen Regelfehlern umgesetzt.
 
In zwei weiteren Versuchen soll auf die speziellen Eigenschaften des verwendeten Störgrößenbeobachters eingegangen werden.  Abb.~\ref{abb_messung_fahrer} zeigt dazu das Ergebnis einer Messung mit Fahrerinteraktion.  
\begin{figure}[thpb]
	 \centering 
		\begin{minipage}[t]{0.45\linewidth} 
			\centering
			\setlength\figureheight{6cm} 
			\setlength\figurewidth{4cm}
			% This file was created by matlab2tikz v0.5.0 running on MATLAB 7.11.1.
%Copyright (c) 2008--2014, Nico Schlömer <nico.schloemer@gmail.com>
%All rights reserved.
%Minimal pgfplots version: 1.3
%
%The latest updates can be retrieved from
%  http://www.mathworks.com/matlabcentral/fileexchange/22022-matlab2tikz
%where you can also make suggestions and rate matlab2tikz.
%
\begin{tikzpicture}

\begin{axis}[%
width=0.95092\figurewidth,
height=0.264706\figureheight,
at={(0\figurewidth,0\figureheight)},
scale only axis,
every outer x axis line/.append style={black},
every x tick label/.append style={font=\color{black}},
xmin=8,
xmax=26,
xlabel={$t$ [s]},
xlabel near ticks,
xmajorgrids,
every outer y axis line/.append style={black},
every y tick label/.append style={font=\color{black}},
ymin=-4,
ymax=1,
ylabel={$\Delta d$ [m]},
ylabel near ticks,
ymajorgrids,
axis x line*=bottom,
axis y line*=left
]
\addplot [color=black,solid,forget plot, line width=1.0]
  table[row sep=crcr]{%
7.97475	0.0192920030020045\\
7.99575	0.0102367716335747\\
8.01475	0.0110092538667868\\
8.03575	0.011781736100001\\
8.05475	0.0125501715600205\\
8.07575	0.0151647864525835\\
8.09475	0.0159329770925436\\
8.11575	0.01669703151512\\
8.13475	0.0174610859375766\\
8.15575	0.0182249062484012\\
8.17475	0.00915654519675346\\
8.19575	0.0117624464489969\\
8.21475	0.012521670864388\\
8.23575	0.0132769360025944\\
8.25475	0.0140322011406795\\
8.27575	0.0147867451413837\\
8.29475	0.0173832225006865\\
8.31575	0.0181337531261532\\
8.33475	0.0188832196892919\\
8.35575	0.00980052475061166\\
8.37475	0.0105459985186775\\
8.39575	0.0112901102613603\\
8.41475	0.0138758527014375\\
8.43575	0.0146158843271005\\
8.45475	0.0153541018621643\\
8.47575	0.0160882508131697\\
8.49475	0.0168223997639592\\
8.51575	0.0193997926535481\\
8.53475	0.0102993277161327\\
8.55575	0.0110271061548572\\
8.57375	0.0117520816171885\\
8.59575	0.0124729618927266\\
8.61475	0.0131938421681128\\
8.63575	0.0157565387842329\\
8.65475	0.0164699603677905\\
8.67675	0.0171833819509222\\
8.69475	0.00806444217416979\\
8.71575	0.00876981802949306\\
8.73475	0.0094751938846005\\
8.75575	0.012020635037103\\
8.77475	0.0127173572975749\\
8.79575	0.0134140795578039\\
8.81475	0.0141052538968958\\
8.83575	0.0147926992069012\\
8.85475	0.00565173100270755\\
8.87575	0.00817716069143071\\
8.89475	0.00885469501134528\\
8.91575	0.00953222933121145\\
8.93475	0.0102024196375838\\
8.95575	0.0108694020412856\\
8.97475	0.0133804537655133\\
8.99575	0.0140389928246996\\
9.01475	0.00486623155230204\\
9.03575	0.00552201727996504\\
9.05475	0.00616840049492673\\
9.07575	0.00865589500303621\\
9.09475	0.00929983792249578\\
9.11575	0.00993314004957613\\
9.13475	0.0105645944822528\\
9.15575	0.00136739935407659\\
9.17375	0.00383011706586966\\
9.19575	0.00444843927862326\\
9.21475	0.00506676149119922\\
9.23575	0.00567199934432505\\
9.25475	0.0062765485550873\\
9.27575	0.00872364729132125\\
9.29475	-0.000515037747212066\\
9.31575	7.51008405006637e-005\\
9.33475	0.000664518005116577\\
9.35575	0.00123961155809793\\
9.37475	0.00181470511131954\\
9.39575	0.00423014050066373\\
9.41475	-0.00503939526456687\\
9.43575	-0.00447997802309041\\
9.45475	-0.00392293966673307\\
9.47675	-0.00337982743659238\\
9.49475	-0.000995429495345146\\
9.51575	-0.0102848643832631\\
9.53475	-0.00975868412290559\\
9.55575	-0.00923250386043284\\
9.57475	-0.00871060503098464\\
9.59575	-0.00820198280384155\\
9.61475	-0.0175225274037616\\
9.63575	-0.0151789618361451\\
9.65475	-0.0146885240495598\\
9.67575	-0.0141980862597304\\
9.69475	-0.0235434009507118\\
9.71575	-0.0230717760581602\\
9.73475	-0.0226001511605602\\
9.75575	-0.0319653511451543\\
9.77375	-0.0296739289811105\\
9.79575	-0.0292217489946247\\
9.81475	-0.0386079725665911\\
9.83575	-0.0381758742489913\\
9.85475	-0.0475732646165619\\
9.87575	-0.047151279090647\\
9.89475	-0.056569480963645\\
9.91575	-0.0561581070252895\\
9.93475	-0.0655876800888771\\
9.95575	-0.0651976807178221\\
9.97475	-0.0746373469668731\\
9.99575	-0.0724225970444881\\
10.01475	-0.0818843890215435\\
10.03575	-0.0913461809906604\\
10.05475	-0.0909927048841155\\
10.07575	-0.100477292672049\\
10.09475	-0.109961880445796\\
10.11575	-0.109632565237939\\
10.13475	-0.119140627592706\\
10.15575	-0.128648689926209\\
10.17475	-0.1381743299189\\
10.19575	-0.147706554277979\\
10.21475	-0.147408725886988\\
10.23575	-0.156960123763747\\
10.25475	-0.166517206393632\\
10.27675	-0.176074288982596\\
10.29475	-0.185652315639278\\
10.31575	-0.195234961359317\\
10.33475	-0.20481760702509\\
10.35575	-0.2144230411913\\
10.37475	-0.224031962288131\\
10.39575	-0.233640883315251\\
10.41475	-0.253104996718574\\
10.43575	-0.262740911495657\\
10.45475	-0.272376826182709\\
10.47575	-0.282039554130982\\
10.49475	-0.291703185347273\\
10.51575	-0.311197973581052\\
10.53475	-0.320890047066243\\
10.55575	-0.330582120432267\\
10.57475	-0.350107003759502\\
10.59575	-0.361659788278184\\
10.61475	-0.371381029993247\\
10.63575	-0.390936143416423\\
10.65475	-0.400687276579357\\
10.67575	-0.420269376028087\\
10.69475	-0.430025205561287\\
10.71475	-0.449638038938924\\
10.73475	-0.459419778120489\\
10.75575	-0.479039176212142\\
10.77475	-0.500512935507703\\
10.79575	-0.510325982400716\\
10.81475	-0.529978044168297\\
10.83575	-0.549654439173525\\
10.85475	-0.559499476242842\\
10.87575	-0.579185656528794\\
10.89475	-0.600722876367171\\
10.91575	-0.620432055059132\\
10.93475	-0.640152470662356\\
10.95575	-0.659895064270446\\
10.97475	-0.669806025462607\\
10.99575	-0.689561958408783\\
11.01475	-0.711165100630625\\
11.03575	-0.730941704673773\\
11.05475	-0.750733172437128\\
11.07675	-0.770544345480322\\
11.09475	-0.79218128524242\\
11.11575	-0.812008866823357\\
11.13475	-0.831855123353148\\
11.15575	-0.851701379367192\\
11.17475	-0.871566396030516\\
11.19575	-0.893272395489316\\
11.21475	-0.913154201904336\\
11.23575	-0.933056342833352\\
11.25475	-0.952974114182806\\
11.27575	-0.984547632700013\\
11.29475	-1.00448745725449\\
11.31575	-1.02444154891247\\
11.33475	-1.0462182208303\\
11.35475	-1.06619592189789\\
11.37475	-1.08618662730424\\
11.39575	-1.10617733193876\\
11.41475	-1.13784775437816\\
11.43575	-1.15787530092756\\
11.45475	-1.17790284663518\\
11.47575	-1.19795801176336\\
11.49475	-1.2296756398578\\
11.51575	-1.24974018539452\\
11.53475	-1.26983396288923\\
11.55575	-1.29175488931589\\
11.57475	-1.32168963983831\\
11.59575	-1.34182206582847\\
11.61475	-1.36196080712434\\
11.63575	-1.38391801145529\\
11.65475	-1.41392231048168\\
11.67575	-1.43409810331625\\
11.69475	-1.45609153298669\\
11.71575	-1.47630088689379\\
11.73475	-1.50634717758935\\
11.75575	-1.52837670744671\\
11.77475	-1.54862432872019\\
11.79575	-1.56887376685756\\
11.81475	-1.59895745266051\\
11.83575	-1.62105852151793\\
11.85475	-1.64134440618276\\
11.87675	-1.67146618309826\\
11.89475	-1.69360254590208\\
11.91575	-1.71392452943821\\
11.93475	-1.73424984078035\\
11.95475	-1.76625525867257\\
11.97475	-1.78661292199255\\
11.99575	-1.80697532772751\\
12.01475	-1.82736818211669\\
12.03575	-1.84957383300225\\
12.05475	-1.87980710507533\\
12.07575	-1.90023458904491\\
12.09475	-1.92247409725301\\
12.11575	-1.94290884222469\\
12.13475	-1.97320492351693\\
12.15575	-1.9936664059275\\
12.17475	-2.01594764898375\\
12.19575	-2.03644243433667\\
12.21475	-2.05693721840334\\
12.23575	-2.08727710876642\\
12.25475	-2.10961423002262\\
12.27575	-2.13014155102055\\
12.29475	-2.15068016230599\\
12.31575	-2.17304828307835\\
12.33475	-2.19360731072144\\
12.35575	-2.2240135836519\\
12.37475	-2.24460342408276\\
12.39575	-2.26700166805599\\
12.41475	-2.28760448660658\\
12.43575	-2.30822417951819\\
12.45475	-2.32884387123218\\
12.47575	-2.35128491191237\\
12.49475	-2.37193343526057\\
12.51575	-2.39258195745427\\
12.53475	-2.41324523091628\\
12.55475	-2.43572797727108\\
12.57475	-2.45640424539533\\
12.59575	-2.47709564631878\\
12.61475	-2.49779851398592\\
12.63575	-2.52030726455211\\
12.65475	-2.54102552138361\\
12.67675	-2.56175377648349\\
12.69475	-2.57264633907058\\
12.71575	-2.59339054826137\\
12.73475	-2.61594769527943\\
12.75575	-2.63670006151126\\
12.77475	-2.65746834493582\\
12.79575	-2.66840760041216\\
12.81475	-2.69098701046027\\
12.83575	-2.71177779189219\\
12.85475	-2.73257429607969\\
12.87575	-2.7435348286929\\
12.89475	-2.76434699986978\\
12.91575	-2.78516339455605\\
12.93475	-2.79794708821252\\
12.95575	-2.8187785416546\\
12.97475	-2.83961328095823\\
12.99575	-2.85061189249117\\
13.01475	-2.87146128972502\\
13.03575	-2.8824765582225\\
13.05475	-2.90513059133957\\
13.07575	-2.91615946939531\\
13.09475	-2.93702596191593\\
13.11575	-2.94805619703663\\
13.13475	-2.95909911763904\\
13.15575	-2.97997886414001\\
13.17475	-2.99102244337412\\
13.19575	-3.00207723180894\\
13.21775	-3.02477006214636\\
13.23675	-3.03582524638322\\
13.25775	-3.04688945853148\\
13.27775	-3.057953670488\\
13.29775	-3.07885497400228\\
13.31775	-3.08992659667009\\
13.33775	-3.10099821920998\\
13.35775	-3.1120705934499\\
13.37775	-3.1231475276514\\
13.39775	-3.13422446177905\\
13.41775	-3.15513844893325\\
13.43775	-3.16621850913712\\
13.45775	-3.17729856930805\\
13.47875	-3.18837883547162\\
13.49775	-3.19945974990672\\
13.51775	-3.21054066433434\\
13.53775	-3.22162114442986\\
13.55775	-3.23270055588459\\
13.57775	-3.24377996733927\\
13.59775	-3.25485805002761\\
13.61775	-3.26593351668579\\
13.63775	-3.27700898333019\\
13.65775	-3.28808196061831\\
13.67775	-3.2991509571586\\
13.69775	-3.31021995364729\\
13.71775	-3.31144864369203\\
13.73775	-3.32250856309354\\
13.75775	-3.33356848237861\\
13.77775	-3.34462279538858\\
13.79775	-3.35567095174258\\
13.81775	-3.36671910788491\\
13.83675	-3.36792337319047\\
13.85775	-3.37895700585227\\
13.87775	-3.38999063817455\\
13.89775	-3.40101440245903\\
13.91775	-3.41203068196234\\
13.93775	-3.41321076126536\\
13.95775	-3.42421463657679\\
13.97775	-3.43521067238948\\
13.99775	-3.43637059089851\\
14.01775	-3.44554787674929\\
14.03775	-3.45652072608961\\
14.05775	-3.46749357447888\\
14.07775	-3.46861146856658\\
14.09775	-3.47955814584902\\
14.11775	-3.49050482189371\\
14.13775	-3.49159317636916\\
14.15775	-3.50251066310815\\
14.17775	-3.51342814827903\\
14.19775	-3.51448367009365\\
14.21775	-3.52536892568772\\
14.23775	-3.52461303644915\\
14.25775	-3.53546728853175\\
14.27875	-3.54631726199691\\
14.29775	-3.54733171669943\\
14.31775	-3.5581463052772\\
14.33775	-3.55912258970296\\
14.35775	-3.56993422999991\\
14.37775	-3.5708706614157\\
14.39775	-3.58164094302521\\
14.41775	-3.59060162693585\\
14.43775	-3.59149254652912\\
14.45575	-3.60221846802519\\
14.47775	-3.60310582333846\\
14.49775	-3.61378443907823\\
14.51775	-3.61462823953736\\
14.53775	-3.6154662260371\\
14.55775	-3.62609464552377\\
14.57775	-3.62688845608264\\
14.59775	-3.63569593952689\\
14.61775	-3.63643696892169\\
14.63775	-3.64701237783605\\
14.65775	-3.64774093988069\\
14.67775	-3.64842647809312\\
14.69775	-3.65894616514372\\
14.71775	-3.65961627518546\\
14.73775	-3.67007763634757\\
14.75775	-3.67070507631799\\
14.77775	-3.67933128159756\\
14.79775	-3.67989815460883\\
14.81775	-3.6804650199155\\
14.83775	-3.69084200855945\\
14.85775	-3.69134595433466\\
14.87775	-3.70168329306161\\
14.89775	-3.70216018846987\\
14.91775	-3.70259899373611\\
14.93775	-3.71287092045493\\
14.95775	-3.71327905635374\\
14.97775	-3.721662784228\\
14.99775	-3.72203437718347\\
15.01775	-3.72237004744895\\
15.03775	-3.73250511202743\\
15.05675	-3.7328076033019\\
15.07775	-3.73307035928338\\
15.09775	-3.74313430469604\\
15.11775	-3.74154209976253\\
15.13775	-3.75156052734598\\
15.15775	-3.75171984617338\\
15.17775	-3.75187915355393\\
15.19775	-3.76182104099342\\
15.21775	-3.76190666733022\\
15.23775	-3.76199228172278\\
15.25775	-3.77185632812749\\
15.27775	-3.77003830400559\\
15.29775	-3.77988044903719\\
15.31775	-3.77983255408491\\
15.33775	-3.77976765253154\\
15.35775	-3.78953376707377\\
15.37775	-3.78940648422835\\
15.39775	-3.78926520146602\\
15.41775	-3.78729165042651\\
15.43775	-3.7969134441494\\
15.45775	-3.79669535125638\\
15.47775	-3.79647724509456\\
15.49775	-3.80601856291809\\
15.51775	-3.80572347782249\\
15.53775	-3.80542837931904\\
15.55775	-3.81305182454668\\
15.57775	-3.81267981613263\\
15.59775	-3.81230603249014\\
15.61775	-3.81185743873874\\
15.63775	-3.82123825075958\\
15.65775	-3.82078473936559\\
15.67775	-3.81841925810586\\
15.69775	-3.81789463333032\\
15.71775	-3.82718946238905\\
15.73775	-3.82658967192133\\
15.75775	-3.82598986841646\\
15.77775	-3.82537781192191\\
15.79775	-3.83268818301353\\
15.81575	-3.83201429492203\\
15.83775	-3.83132374051984\\
15.85775	-3.83057712765288\\
15.87875	-3.83965866622396\\
15.89775	-3.83889280569347\\
15.91775	-3.83622759326126\\
15.93775	-3.83540982609757\\
15.95775	-3.83456920034985\\
15.97775	-3.84350971045399\\
15.99775	-3.8426226388061\\
16.01775	-3.84171062353926\\
16.03775	-3.83890580419231\\
16.05775	-3.83795149923933\\
16.07775	-3.83696924852713\\
16.09775	-3.84577702859179\\
16.11775	-3.84475778735532\\
16.13775	-3.84370911681103\\
16.15775	-3.84077407320375\\
16.17775	-3.83969240698468\\
16.19775	-3.8385790535428\\
16.21775	-3.84726417922459\\
16.23775	-3.84612280257511\\
16.25775	-3.84309294743081\\
16.27775	-3.84189477107503\\
16.29775	-3.84069658694136\\
16.31775	-3.83946451090009\\
16.33775	-3.83821260300857\\
16.35775	-3.84492950088756\\
16.37775	-3.8436425367991\\
16.39775	-3.84234013054357\\
16.41575	-3.84103771806388\\
16.43775	-3.83970087538102\\
16.45775	-3.83835134284908\\
16.47775	-3.83514239972558\\
16.49775	-3.84358383571237\\
16.51775	-3.8421906679053\\
16.53775	-3.84079749537922\\
16.55775	-3.83937107929864\\
16.57775	-3.83793786617861\\
16.59775	-3.83464336412208\\
16.61775	-3.83317781029019\\
16.63775	-3.83170821955744\\
16.65775	-3.83023862548866\\
16.67875	-3.82873882269165\\
16.69775	-3.8253737438368\\
16.71775	-3.81404653747192\\
16.73775	-3.81251599197631\\
16.75775	-3.81098485732628\\
16.77775	-3.79776454629563\\
16.79775	-3.79620829982363\\
16.81775	-3.78482732154849\\
16.83775	-3.77158050326777\\
16.85775	-3.77000291797565\\
16.87775	-3.75860069226795\\
16.89775	-3.74533158593049\\
16.91775	-3.73391185131421\\
16.93775	-3.72249211584875\\
16.95775	-3.70920497047884\\
16.97775	-3.69777142381375\\
16.99775	-3.68633787659968\\
17.01775	-3.67303679936506\\
17.03775	-3.65176861972032\\
17.05775	-3.64032490128204\\
17.07775	-3.62701395554731\\
17.09775	-3.61556363199889\\
17.11775	-3.59242326212443\\
17.13775	-3.5809719738683\\
17.15775	-3.55782841695654\\
17.17775	-3.5463749633614\\
17.19775	-3.52509717273239\\
17.21775	-3.51177828104654\\
17.23775	-3.49050064663124\\
17.25575	-3.47718318156569\\
17.27775	-3.45590902309483\\
17.29775	-3.43276926523617\\
17.31775	-3.42132248718651\\
17.33775	-3.39818946091461\\
17.35775	-3.37692188483834\\
17.37775	-3.35379365724051\\
17.39775	-3.33253562962848\\
17.41775	-3.30941236795084\\
17.43775	-3.29798551496037\\
17.45775	-3.27487489836936\\
17.47875	-3.2536292353222\\
17.49775	-3.23052755066455\\
17.51775	-3.20929691131087\\
17.53775	-3.18620165814609\\
17.55775	-3.16498200179861\\
17.57775	-3.14190466277779\\
17.59775	-3.12069153979614\\
17.61775	-3.09762773242324\\
17.63775	-3.07643444846966\\
17.65775	-3.05337739672582\\
17.67775	-3.02051167254572\\
17.69775	-2.99934037118058\\
17.71575	-2.97630579991426\\
17.73475	-2.95515282302964\\
17.75575	-2.93214273979585\\
17.77475	-2.91099538214878\\
17.79575	-2.88800639945098\\
17.81475	-2.8668847643898\\
17.83475	-2.84390098439726\\
17.85475	-2.82094111419403\\
17.87575	-2.7998467908183\\
17.89475	-2.77689094117645\\
17.91575	-2.75582268438932\\
17.93475	-2.73289620057995\\
17.95575	-2.71183058985353\\
17.97475	-2.68893292268848\\
17.99575	-2.66603704506092\\
18.01475	-2.64500136233796\\
18.03575	-2.62213699438963\\
18.05475	-2.6011322676487\\
18.07575	-2.57826964389823\\
18.09475	-2.55729671337684\\
18.11575	-2.53446500664695\\
18.13475	-2.51163671879857\\
18.15575	-2.49069623886722\\
18.17475	-2.46789771795305\\
18.19575	-2.44696177398298\\
18.21475	-2.4241969266921\\
18.23575	-2.4032893710139\\
18.25475	-2.39035708449371\\
18.27675	-2.36948275608285\\
18.29475	-2.34675188554879\\
18.31575	-2.3258852183122\\
18.33475	-2.30318846183999\\
18.35575	-2.29217378281914\\
18.37475	-2.26948666247428\\
18.39575	-2.24867901832617\\
18.41475	-2.22601634309247\\
18.43575	-2.21504594116981\\
18.45475	-2.19241716802745\\
18.47575	-2.17164267165506\\
18.49375	-2.14902641383749\\
18.51575	-2.13811147096564\\
18.53475	-2.11551627306384\\
18.55575	-2.10461500969842\\
18.57475	-2.08205293797691\\
18.59575	-2.06134366523063\\
18.61475	-2.04862379252334\\
18.63575	-2.02794635775732\\
18.65475	-2.01524383623103\\
18.67575	-1.99458258019731\\
18.69475	-1.97208495801481\\
18.71575	-1.96126583365789\\
18.73475	-1.93878582448903\\
18.75575	-1.92799728241791\\
18.77475	-1.9055308001459\\
18.79575	-1.89476063082139\\
18.81475	-1.88215185555593\\
18.83575	-1.86156570639638\\
18.85475	-1.84897661689003\\
18.87575	-1.8284192678435\\
18.89475	-1.81768942387379\\
18.91575	-1.80513045378019\\
18.93475	-1.7846010155803\\
18.95575	-1.77205047619208\\
18.97475	-1.76136934909578\\
18.99575	-1.7390187980904\\
19.01475	-1.72834409984552\\
19.03575	-1.715842940268\\
19.05475	-1.7051944363935\\
19.07675	-1.68287055807586\\
19.09475	-1.67224398780801\\
19.11575	-1.66162076027156\\
19.13475	-1.64915058001333\\
19.15575	-1.63854961236897\\
19.17475	-1.61627629745785\\
19.19475	-1.6056774457141\\
19.21475	-1.59510102723604\\
19.23575	-1.58267980974462\\
19.25475	-1.57210446131091\\
19.27575	-1.55970642593204\\
19.29475	-1.53932549502146\\
19.31575	-1.52877371917271\\
19.33475	-1.51639806779084\\
19.35575	-1.50586728191789\\
19.37475	-1.4934939373553\\
19.39575	-1.48298430940869\\
19.41475	-1.47063029248028\\
19.43575	-1.46012383058725\\
19.45475	-1.44963469141359\\
19.47575	-1.4373016256558\\
19.49475	-1.42681646431929\\
19.51575	-1.41450373381811\\
19.53475	-1.40403448036814\\
19.55575	-1.39356997932516\\
19.57475	-1.38127703851148\\
19.59575	-1.37082713804182\\
19.61475	-1.36038273662561\\
19.63575	-1.34810911351406\\
19.65475	-1.33767810444908\\
19.67575	-1.32541112656272\\
19.69475	-1.31499861941353\\
19.71575	-1.30458611197811\\
19.73475	-1.29233876662496\\
19.75575	-1.28194444189665\\
19.77475	-1.26970833038862\\
19.79475	-1.25932207956579\\
19.81475	-1.24894568650874\\
19.83575	-1.2465569150153\\
19.85475	-1.23618939588184\\
19.87675	-1.22398976787856\\
19.89475	-1.21363112096305\\
19.91575	-1.20328208904344\\
19.93475	-1.1911004810337\\
19.95575	-1.18075945642613\\
19.97475	-1.16858862192142\\
19.99575	-1.16809437885049\\
20.01475	-1.15777090823069\\
20.03575	-1.14561888552947\\
20.05475	-1.13531295033739\\
20.07575	-1.12500701492711\\
20.09475	-1.11287373895195\\
20.11575	-1.11241473792047\\
20.13475	-1.1002869656216\\
20.15575	-1.0900119576574\\
20.17475	-1.07974119652393\\
20.19575	-1.06763143724726\\
20.21475	-1.05737491278425\\
20.23575	-1.05695137257807\\
20.25475	-1.0448597089454\\
20.27575	-1.03462184284555\\
20.29475	-1.02254837153641\\
20.31575	-1.01231309706033\\
20.33475	-1.01192359522418\\
20.35575	-0.99986841513772\\
20.37375	-0.989651028185958\\
20.39575	-0.979450713163986\\
20.41475	-0.967413914049937\\
20.43575	-0.967044253483559\\
20.45475	-0.956862875764455\\
20.47575	-0.944844525546448\\
20.49475	-0.934663971859989\\
20.51575	-0.922664104482245\\
20.53475	-0.922330712855465\\
20.55575	-0.91216913351362\\
20.57475	-0.900187736687549\\
20.59575	-0.899872484106123\\
20.61475	-0.88972985806425\\
20.63575	-0.877766862341288\\
20.65475	-0.867639625153167\\
20.67675	-0.867345920149545\\
20.69475	-0.855401187294713\\
20.71575	-0.845291809890534\\
20.73475	-0.835186673141679\\
20.75575	-0.833090190479965\\
20.77475	-0.822998459944029\\
20.79575	-0.812911762838456\\
20.81475	-0.810833091683942\\
20.83575	-0.800758708067888\\
20.85475	-0.790690088457324\\
20.87575	-0.778798473480899\\
20.89475	-0.778571391100708\\
20.91575	-0.768520374446958\\
20.93475	-0.756645584691825\\
20.95575	-0.756435025460469\\
20.97475	-0.74640101644916\\
20.99575	-0.734542431423252\\
21.01475	-0.734347785974244\\
21.03575	-0.724330061702111\\
21.05475	-0.722317493834273\\
21.07475	-0.712307489421444\\
21.09475	-0.702305194717727\\
21.11575	-0.700307299004594\\
21.13475	-0.690311584099221\\
21.15575	-0.680323729064325\\
21.17475	-0.67833952510015\\
21.19575	-0.668357144874531\\
21.21475	-0.668213321511185\\
21.23575	-0.658243203899167\\
21.25475	-0.646440873730418\\
21.27575	-0.646309153483653\\
21.29475	-0.636350111564835\\
21.31575	-0.624559107093531\\
21.33475	-0.62443809781025\\
21.35575	-0.614488834546192\\
21.37475	-0.612538634208077\\
21.39575	-0.602596016772015\\
21.41475	-0.592655131200852\\
21.43575	-0.592545085579905\\
21.45475	-0.580778449585112\\
21.47675	-0.580675314134878\\
21.49475	-0.570741309746913\\
21.51575	-0.568811532552916\\
21.53475	-0.558882827858227\\
21.55575	-0.548954123150139\\
21.57475	-0.547028485040855\\
21.59575	-0.537103425648608\\
21.61475	-0.537009272777278\\
21.63575	-0.527086096152945\\
21.65475	-0.525162731270038\\
21.67475	-0.515239607873368\\
21.69475	-0.515147569227548\\
21.71575	-0.503393467255724\\
21.73475	-0.493470392938101\\
21.75575	-0.493376734162036\\
21.77475	-0.481620982427601\\
21.79575	-0.481526884276721\\
21.81475	-0.471598903589263\\
21.83575	-0.471501817145121\\
21.85475	-0.459741644099296\\
21.87575	-0.459639325779067\\
21.89475	-0.449706135114753\\
21.91575	-0.447770810957783\\
21.93475	-0.437830633909568\\
21.95575	-0.437721299811368\\
21.97475	-0.427778851879604\\
21.99575	-0.425829021851848\\
22.01475	-0.41588011918138\\
22.03575	-0.415758749345038\\
22.05475	-0.413798229842849\\
22.07575	-0.403838916458506\\
22.09475	-0.403705933071029\\
22.11575	-0.393734594749767\\
22.13475	-0.391761728252909\\
22.15575	-0.381784717225416\\
22.17475	-0.381630481532898\\
22.19575	-0.379643699877434\\
22.21475	-0.369651728705335\\
22.23575	-0.369482452817778\\
22.25475	-0.359482585026059\\
22.27675	-0.357471843653695\\
22.29475	-0.35728621363182\\
22.31475	-0.347270060239794\\
22.33475	-0.345241029634614\\
22.35575	-0.34503786143082\\
22.37475	-0.33500424336885\\
22.39575	-0.332955674732693\\
22.41475	-0.332733925941548\\
22.43575	-0.322681805192136\\
22.45475	-0.322447125332984\\
22.47575	-0.320371404222663\\
22.49475	-0.310299893783001\\
22.51575	-0.310044167319107\\
22.53475	-0.3079477899286\\
22.55575	-0.307686367491885\\
22.57475	-0.297578707858773\\
22.59575	-0.295461097238006\\
22.61475	-0.295178908300119\\
22.63575	-0.294879154136237\\
22.65475	-0.284745776381794\\
22.67575	-0.28260653077026\\
22.69475	-0.282284282231744\\
22.71575	-0.281959555640646\\
22.73475	-0.26996851137618\\
22.75575	-0.269623689142573\\
22.77475	-0.269277539201426\\
22.79575	-0.268931389186354\\
22.81475	-0.266727165937027\\
22.83575	-0.256529959844138\\
22.85475	-0.256161933963092\\
22.87575	-0.255773489467752\\
22.89475	-0.253547235090579\\
22.91575	-0.253157029119124\\
22.93475	-0.252748046359764\\
22.95575	-0.25050079214558\\
22.97475	-0.240259440545129\\
22.99575	-0.239830535641102\\
23.01475	-0.239401630677913\\
23.03575	-0.23713021623724\\
23.05475	-0.236682150484727\\
23.07675	-0.236234084677501\\
23.09475	-0.23578175413768\\
23.11575	-0.235315421801633\\
23.13475	-0.233009529175335\\
23.15575	-0.222708893771217\\
23.17475	-0.222225308642146\\
23.19575	-0.221741723470785\\
23.21475	-0.219412615948401\\
23.23575	-0.218912897115363\\
23.25475	-0.218413178245576\\
23.27575	-0.217907770919627\\
23.29475	-0.215552482060274\\
23.31575	-0.215037839095183\\
23.33475	-0.214517204805209\\
23.35575	-0.213988922817284\\
23.37475	-0.21346064080301\\
23.39575	-0.211085291379539\\
23.41475	-0.210544715349286\\
23.43575	-0.210004139297642\\
23.45475	-0.199628764262677\\
23.47575	-0.197235809817357\\
23.49475	-0.196684329125797\\
23.51575	-0.196127280736329\\
23.53475	-0.195566314020142\\
23.55575	-0.195005347291583\\
23.57475	-0.194439367086998\\
23.59575	-0.192028479632683\\
23.61475	-0.191459460080754\\
23.63575	-0.19088593477021\\
23.65475	-0.190310295746342\\
23.67575	-0.189734656716263\\
23.69475	-0.187313158929246\\
23.71575	-0.186732319374808\\
23.73475	-0.186151479816549\\
23.75575	-0.185567689638091\\
23.77475	-0.184983039984531\\
23.79575	-0.182556171172514\\
23.81475	-0.181969472473277\\
23.83575	-0.181382361578041\\
23.85475	-0.180795250681765\\
23.87675	-0.180207127945918\\
23.89475	-0.177776550886968\\
23.91575	-0.177188274484795\\
23.93475	-0.176600060165991\\
23.95575	-0.176011850789984\\
23.97475	-0.175423641413779\\
23.99575	-0.172994341649802\\
24.01475	-0.172407360246843\\
24.03575	-0.17182042502172\\
24.05475	-0.171235754492189\\
24.07575	-0.168808864337037\\
24.09475	-0.168224452342419\\
24.11575	-0.167643092932245\\
24.13475	-0.167061733521392\\
24.15575	-0.166480854637668\\
24.17475	-0.164061671024415\\
24.19575	-0.163484537657905\\
24.21475	-0.162908272655797\\
24.23575	-0.162336193704771\\
24.25475	-0.161764114751796\\
24.27575	-0.159351431706475\\
24.29475	-0.158785149071815\\
24.31575	-0.15821886643393\\
24.33475	-0.157654297367542\\
24.35575	-0.157094468285609\\
24.37475	-0.154692977819555\\
24.39475	-0.154135423729705\\
24.41475	-0.153582624487941\\
24.43575	-0.153029825241883\\
24.45475	-0.152479731174282\\
24.47575	-0.150093127965597\\
24.49475	-0.149547858931138\\
24.51575	-0.149005912139142\\
24.53475	-0.148468603639335\\
24.55575	-0.14793129513443\\
24.57475	-0.145556790567061\\
24.59575	-0.145027809641686\\
24.61475	-0.144498828710309\\
24.63575	-0.143974262711656\\
24.65475	-0.14161314657792\\
24.67675	-0.141092804442537\\
24.69475	-0.140577528806424\\
24.71575	-0.140066088468443\\
24.73475	-0.139554648123989\\
24.75575	-0.127379163739635\\
24.77475	-0.126876847782723\\
24.79575	-0.126374531819728\\
24.81475	-0.12587835107104\\
24.83575	-0.123545190113769\\
24.85475	-0.123052188727743\\
24.87575	-0.122565982262649\\
24.89475	-0.12208246086227\\
24.91575	-0.121598939455362\\
24.93475	-0.119282931741763\\
24.95575	-0.118809038141957\\
24.97475	-0.118335144535699\\
24.99475	-0.117869165387841\\
25.01475	-0.115565526134096\\
25.03575	-0.115101397585827\\
25.05475	-0.114645866400604\\
25.07575	-0.104362240337098\\
25.09475	-0.102068721635857\\
25.11575	-0.101623745875707\\
25.13475	-0.101179547836467\\
25.15575	-0.100735349791173\\
25.17475	-0.0984620639039369\\
25.19575	-0.0980280383907952\\
25.21475	-0.0975941801297071\\
25.23575	-0.0971704769741129\\
25.25475	-0.0949081716010047\\
25.27575	-0.0944852474629143\\
25.29475	-0.0940720287614116\\
25.31575	-0.0936588100537299\\
25.33475	-0.091408636215109\\
25.35575	-0.0811764649502562\\
25.37475	-0.080773906540113\\
25.39575	-0.0803731257249156\\
25.41475	-0.0781435345592447\\
25.43575	-0.0777518265430044\\
25.45475	-0.0773625716901654\\
25.47675	-0.0769819174987796\\
25.49475	-0.0747636285702553\\
25.51575	-0.074386129810577\\
25.53475	-0.0740167438029373\\
25.55575	-0.0736473577896954\\
25.57475	-0.0714444754385148\\
25.59475	-0.0710865794453386\\
25.61475	-0.0608988829346249\\
25.63575	-0.0605453162097178\\
25.65475	-0.0583622748727342\\
25.67575	-0.0580160935084244\\
25.69475	-0.057675016920276\\
25.71575	-0.057340771961909\\
25.73475	-0.0551699352850195\\
25.75575	-0.054841591828851\\
25.77475	-0.0545194956790169\\
25.79575	-0.0541973995243406\\
25.81475	-0.0520456985259936\\
25.83575	-0.0517359468659291\\
25.85475	-0.0514261952008899\\
25.87575	-0.0511236962827208\\
25.89475	-0.0489907001495524\\
25.91575	-0.0486934639604302\\
25.93475	-0.0385741872183556\\
25.95575	-0.0382896041998011\\
25.97475	-0.0361695456074171\\
25.99575	-0.035893778524128\\
26.01475	-0.0356219441974561\\
26.03575	-0.0353501098673292\\
26.05475	-0.0350875782305722\\
26.07575	-0.0329936357873439\\
26.09475	-0.0327345948465121\\
26.11575	-0.0324855883148762\\
26.13475	-0.0322393259051221\\
26.15575	-0.0301584491871236\\
26.17475	-0.0299228935741684\\
26.19575	-0.0296893265268885\\
26.21475	-0.0294557594767038\\
26.23575	-0.0292332111252684\\
26.25375	-0.0271781324105707\\
26.27675	-0.026957100918338\\
26.29475	-0.0267475874699166\\
26.31575	-0.0265388470853414\\
26.33475	-0.0244963357240136\\
26.35575	-0.0242994924624216\\
26.37475	-0.0241027064278638\\
26.39575	-0.023906275694304\\
26.41475	-0.0237210081769756\\
26.43575	-0.0217024972568796\\
26.45475	-0.0215182063025181\\
26.47575	-0.0213439168330472\\
26.49475	-0.0211696273619051\\
26.51575	-0.0191638602105644\\
26.53475	-0.0189998992248821\\
26.55575	-0.01883593823759\\
26.57475	-0.0186737250654296\\
26.59575	-0.0185193305788252\\
26.61475	-0.0165323865976417\\
26.63575	-0.016380084635558\\
26.65475	-0.0162343795637829\\
26.67575	-0.0160886744912112\\
26.69475	-0.0141129155610868\\
26.71575	-0.0139749058240453\\
26.73475	-0.0138368960862145\\
26.75575	-0.0137010890435283\\
26.77475	-0.0135696631388984\\
26.79575	-0.0116062039917422\\
26.81475	-0.0114768812570367\\
26.83575	-0.0113508113734127\\
26.85475	-0.0112247414893365\\
26.87575	-0.011100391064117\\
26.89475	-0.00914651306804437\\
26.91575	-0.00902445769650351\\
26.93375	-0.00890364923307541\\
26.95575	-0.00878415727779736\\
26.97475	-0.00866466532257082\\
26.99575	-0.00671395738813896\\
27.01475	-0.00659547367930013\\
27.03575	-0.00647698997030899\\
27.05475	-0.00635813262282392\\
27.07675	-0.00623900490450024\\
27.09475	-0.00428812036512349\\
27.11575	-0.00416747031114451\\
27.13475	-0.00404595757205639\\
27.15575	-0.00392444483307264\\
27.17475	-0.0136308725962939\\
27.19575	-0.0116732499512464\\
27.21475	-0.0115475322477678\\
27.23575	-0.0114173288891024\\
27.25475	-0.0112855184665439\\
27.27575	-0.00932166615900565\\
27.29475	-0.0091834724068649\\
27.31575	-0.00904362604324982\\
27.33475	-0.00890377967917466\\
27.35575	-0.00875561566849914\\
27.37475	-0.0067733001321737\\
27.39575	-0.00662343234494411\\
27.41475	-0.00646287613500895\\
27.43575	-0.00630097322129153\\
27.45475	-0.00613907030702565\\
27.47575	-0.00413114299021\\
27.49475	-0.00395517761857445\\
27.51575	-0.00377921224658229\\
27.53475	-0.00358745052075848\\
27.55475	-0.00156200098774661\\
27.57475	-0.0111996445006035\\
27.59575	-0.0109894938979069\\
27.61475	-0.0107793432931516\\
27.63575	-0.0105675907255405\\
27.65475	-0.00850311326729081\\
27.67575	-0.00827288962851336\\
27.69475	-0.00803963161471088\\
27.71575	-0.00778740744225992\\
27.73475	-0.00753518326738645\\
27.75575	-0.00544353262227615\\
27.77475	-0.00516744490390364\\
27.79575	-0.00489135718316547\\
27.81475	-0.00460932780723589\\
27.83575	-0.00430759400181113\\
27.85475	-0.0120000354834313\\
27.87675	-0.0116902661845812\\
27.89475	-0.0113612001229404\\
27.91575	-0.0110321340556627\\
27.93475	-0.0106935394477032\\
27.95575	-0.00849844203363004\\
27.97475	-0.00814046958350367\\
27.99575	-0.00777072769358345\\
28.01475	-0.00738240205556906\\
28.03575	-0.00699407641251115\\
28.05475	-0.0047538282154429\\
28.07575	-0.00433384400583803\\
28.09475	-0.00391385979219949\\
28.11575	-0.0034780776581731\\
28.13475	-0.012854685563171\\
28.15575	-0.0105626358456261\\
28.17375	-0.0100914637328731\\
28.19575	-0.00960487529622478\\
28.21475	-0.00911828685145144\\
28.23575	-0.00677085326554305\\
28.25475	-0.00624966040978858\\
28.27575	-0.00572846754770406\\
28.29475	-0.00518481211157473\\
28.31575	-0.00462838807665644\\
28.33475	-0.00223037917727842\\
28.35575	-0.0114779556917979\\
28.37475	-0.0108858622144403\\
28.39575	-0.010293768726497\\
28.41475	-0.00967525810180536\\
28.43575	-0.00720405580481476\\
28.45475	-0.00657604701188808\\
28.47575	-0.00591949318426277\\
28.49475	-0.00525551761102205\\
28.51575	-0.00459154203101031\\
28.53475	-0.00205305123295174\\
28.55575	-0.00135325156269639\\
28.57475	-0.0104818132887061\\
28.59575	-0.00975037085818675\\
28.61475	-0.00901508191360145\\
28.63575	-0.00643418871762425\\
28.65475	-0.00566569111136772\\
28.67675	-0.00489543544236293\\
28.69475	-0.00412517976691706\\
28.71575	-0.00332116299223806\\
28.73475	-0.00251664451238698\\
28.75475	0.000136484806539716\\
28.77475	-0.00885338944448222\\
28.79575	-0.00801548533765972\\
28.81475	-0.00717488617831075\\
28.83575	-0.00630463682857396\\
28.85475	-0.00358575065616229\\
28.87575	-0.00271109544157033\\
28.89475	-0.00180969245277485\\
28.91575	-0.000908289459435885\\
28.93475	-1.6059165077742e-006\\
28.95575	-0.00889776245009344\\
28.97475	-0.00611652818384778\\
28.99575	-0.0051785088752907\\
29.01475	-0.00421890903573585\\
29.03575	-0.00325930919096873\\
29.05475	-0.00229233489995151\\
29.07575	0.000545326445211547\\
29.09475	0.0015317408794413\\
29.11575	0.00252653923127033\\
29.13475	-0.00628892180656448\\
29.15575	-0.00527733902293548\\
29.17475	-0.00240466957338903\\
29.19575	-0.00136963406813173\\
29.21475	-0.000334598560136179\\
29.23575	0.000709785663090656\\
29.25475	0.00176650827383895\\
29.27575	-0.00515079518277606\\
29.29475	-0.00408425835647908\\
29.31575	-0.00300764493143069\\
29.33475	-0.00193103150404639\\
29.35475	-0.000844851800253732\\
29.37475	0.00210352403499536\\
29.39575	0.00319822009990833\\
29.41475	0.00430247520595595\\
29.43575	-0.00441317035962774\\
29.45475	-0.00330219316658953\\
29.47675	-0.00218212170690446\\
29.49475	0.000797731600945983\\
29.51575	0.00192321346733548\\
29.53475	0.00305744466191005\\
29.55575	0.00419569777232232\\
29.57475	-0.00449255624223044\\
29.59575	-0.00149134989681787\\
29.61475	-0.000341998680694289\\
29.63575	0.000807352535920902\\
29.65475	0.00196414361750064\\
29.67575	0.00312299613955513\\
29.69475	0.00428184866181613\\
29.71575	-0.00252383855781613\\
29.73475	-0.00135699088359909\\
29.75575	-0.000190143209011229\\
29.77475	0.000982511794403784\\
29.79575	0.00215595128215007\\
29.81475	0.00332939077009931\\
29.83575	0.00636325720000519\\
29.85475	-0.00228433773495595\\
29.87575	-0.00110559726923798\\
29.89475	7.71865496793644e-005\\
29.91575	0.00126005733060275\\
29.93475	0.0024429730641593\\
29.95575	0.00548465971209922\\
29.97475	-0.00315568853792225\\
};
\end{axis}

\begin{axis}[%
width=0.95092\figurewidth,
height=0.264706\figureheight,
at={(0\figurewidth,0.367647\figureheight)},
scale only axis,
every outer x axis line/.append style={black},
every x tick label/.append style={font=\color{black}},
xmin=8,
xmax=26,
xmajorgrids,
every outer y axis line/.append style={black},
every y tick label/.append style={font=\color{black}},
ymin=-0.005,
ymax=0.003,
ylabel={$\kappa\text{ [1/m]}$},
ylabel near ticks,
ymajorgrids,
axis x line*=bottom,
axis y line*=left,
legend style={at={(0.618413,0.034139)},anchor=south west,legend cell align=left,align=left,draw=black}
]
\addplot [color=black,solid, line width=1.0]
  table[row sep=crcr]{%
7.97475	0.000229797277775692\\
7.99575	0.000242261326783334\\
8.01475	0.000206719968269\\
8.03575	0.000188121300235039\\
8.05475	0.000180819558099069\\
8.07575	0.000183043938665432\\
8.09475	0.000133398581154166\\
8.11575	0.000152025805996529\\
8.13475	0.000170924502349476\\
8.15575	0.000188922131068519\\
8.17475	0.000196753634621414\\
8.19575	0.00016160337168991\\
8.21475	0.000153779782005094\\
8.23575	0.00015470870996749\\
8.25475	0.000161457260516139\\
8.27575	0.000171705244186627\\
8.29475	0.000184945768717303\\
8.31575	0.000207187934314966\\
8.33475	0.000225495046461849\\
8.35575	0.000231695652649209\\
8.37475	0.00012708290301565\\
8.39575	0.000110541996719143\\
8.41475	0.000109275879955821\\
8.43575	0.000125322436592696\\
8.45475	0.000142710425811825\\
8.47575	0.000160122520623367\\
8.49475	0.000177033346807576\\
8.51575	0.000194396188300731\\
8.53475	0.00020981282254448\\
8.55575	0.000176179986661752\\
8.57375	0.000158271493457264\\
8.59575	0.000151145439788065\\
8.61475	8.63270165936113e-005\\
8.63575	9.71677825014511e-005\\
8.65475	0.000120182943688134\\
8.67675	0.000141579435995375\\
8.69475	0.000152404757019087\\
8.71575	0.000118228349149471\\
8.73475	0.000103070032757305\\
8.75575	0.000101409701573264\\
8.77475	0.000115509972567493\\
8.79575	0.000130094120972577\\
8.81475	0.000144413019334597\\
8.83575	0.000158106628115302\\
8.85475	0.000161937253255748\\
8.87575	0.000122611076243488\\
8.89475	0.000110825602831302\\
8.91575	0.000108267631668128\\
8.93475	0.000111658232538296\\
8.95575	5.36290155744841e-005\\
8.97475	6.92221907095297e-005\\
8.99575	9.50877274953578e-005\\
9.01475	0.00010919697577332\\
9.03575	7.70534077359998e-005\\
9.05475	6.32479232320605e-005\\
9.07575	6.35937173129787e-005\\
9.09475	1.56688541566554e-005\\
9.11575	3.83651566369004e-005\\
9.13475	6.23154453789309e-005\\
9.15575	7.76518254037158e-005\\
9.17375	5.08655497899969e-005\\
9.19575	-1.27301208603531e-005\\
9.21475	2.99028953559011e-006\\
9.23575	2.60523420598734e-005\\
9.25475	-1.134929359809e-005\\
9.27575	2.4661442074366e-005\\
9.29475	-2.46289436336531e-006\\
9.31575	-7.10752632304073e-005\\
9.33475	-0.00011601945789491\\
9.35575	-0.00014346088453176\\
9.37475	-0.000157499849770533\\
9.39575	-0.000159955605757663\\
9.41475	-0.000155384838073169\\
9.43575	-0.000191903624516428\\
9.45475	-0.000206270503735681\\
9.47675	-0.000204159855605604\\
9.49475	-0.00025430432373035\\
9.51575	-0.000292396958948903\\
9.53475	-0.000301936847082801\\
9.55575	-0.000353038272012423\\
9.57475	-0.000384584872305098\\
9.59575	-0.000400870572924014\\
9.61475	-0.000414512906389906\\
9.63575	-0.000395688859998215\\
9.65475	-0.000475130681430397\\
9.67575	-0.000471297880397964\\
9.69475	-0.000464587846986129\\
9.71575	-0.000493203711202852\\
9.73475	-0.000560831479066885\\
9.75575	-0.000549038052346013\\
9.77375	-0.000566061742457251\\
9.79575	-0.00060994339327907\\
9.81475	-0.000643228582273352\\
9.83575	-0.00070424295427186\\
9.85475	-0.00074227786578165\\
9.87575	-0.000798844422538836\\
9.89475	-0.000826996943733055\\
9.91575	-0.00087245030652748\\
9.93475	-0.000888657772645936\\
9.95575	-0.000921702938703169\\
9.97475	-0.0009253853439725\\
9.99575	-0.0010121380525042\\
10.01475	-0.000992468784841672\\
10.03575	-0.00100463795555946\\
10.05475	-0.00109466282509816\\
10.07575	-0.00108178073766633\\
10.09475	-0.00115963340770916\\
10.11575	-0.00124060995353301\\
10.13475	-0.00121796156117119\\
10.15575	-0.00128351498884177\\
10.17475	-0.00122995869103585\\
10.19575	-0.00138471463791672\\
10.21475	-0.00139764566977886\\
10.23575	-0.00143457752282887\\
10.25475	-0.00148518887099058\\
10.27675	-0.00147813364796196\\
10.29475	-0.00154368227084708\\
10.31575	-0.00161087562642624\\
10.33475	-0.00167366198334608\\
10.35575	-0.00173182841011238\\
10.37475	-0.00165550178430073\\
10.39575	-0.0017782310907714\\
10.41475	-0.00183743477870498\\
10.43575	-0.00187055866870042\\
10.45475	-0.00194628036270293\\
10.47575	-0.00193973393259535\\
10.49475	-0.00205579747053742\\
10.51575	-0.00203777738824276\\
10.53475	-0.00212383741998394\\
10.55575	-0.00218536588660328\\
10.57475	-0.00210744609032518\\
10.59575	-0.0022658234823472\\
10.61475	-0.00233669338427767\\
10.63575	-0.00232600543093364\\
10.65475	-0.00241083356687301\\
10.67575	-0.00241081235291236\\
10.69475	-0.00250470532523156\\
10.71475	-0.00250727550428034\\
10.73475	-0.00253255200793413\\
10.75575	-0.00260281483579418\\
10.77475	-0.00264038278035903\\
10.79575	-0.00269745729337661\\
10.81475	-0.00271921018098628\\
10.83575	-0.00276582803595774\\
10.85475	-0.00282138132368821\\
10.87575	-0.00278032547845166\\
10.89475	-0.00283589474093293\\
10.91575	-0.00291539612610488\\
10.93475	-0.00293689744451224\\
10.95575	-0.00296497065828398\\
10.97475	-0.00299210939547966\\
10.99575	-0.00304431846824716\\
11.01475	-0.0030478463327521\\
11.03575	-0.00314217313396534\\
11.05475	-0.00317514856892639\\
11.07675	-0.00321083663963893\\
11.09475	-0.00331284257843781\\
11.11575	-0.00335569632102625\\
11.13475	-0.00339417054838512\\
11.15575	-0.00336844877544648\\
11.17475	-0.00340959087067057\\
11.19575	-0.00345218545049672\\
11.21475	-0.00343974368385144\\
11.23575	-0.00349369479355709\\
11.25475	-0.00348216934276129\\
11.27575	-0.00348925955512633\\
11.29475	-0.00361633365177191\\
11.31575	-0.00372087393750253\\
11.33475	-0.00374471374565614\\
11.35475	-0.00376674157544581\\
11.37475	-0.00378043512279084\\
11.39575	-0.00379263853716579\\
11.41475	-0.00381213444811421\\
11.43575	-0.00382000003073512\\
11.45475	-0.00387386025630964\\
11.47575	-0.0038502442058086\\
11.49475	-0.00390089271162101\\
11.51575	-0.0039343575239591\\
11.53475	-0.00401035489493569\\
11.55575	-0.00400837512649677\\
11.57475	-0.00401955444109901\\
11.59575	-0.00407470925462113\\
11.61475	-0.0041007018325335\\
11.63575	-0.00417538582787244\\
11.65475	-0.0041956770137096\\
11.67575	-0.00425455315551292\\
11.69475	-0.00428506901718965\\
11.71575	-0.0042370021653229\\
11.73475	-0.0042568933441843\\
11.75575	-0.00431272793000686\\
11.77475	-0.00435436221494832\\
11.79575	-0.00431013213849732\\
11.81475	-0.00433279070905878\\
11.83575	-0.00439948286154428\\
11.85475	-0.00438061535486939\\
11.87675	-0.00442258549149359\\
11.89475	-0.00449536098531321\\
11.91575	-0.00448614130360429\\
11.93475	-0.00452391046327512\\
11.95475	-0.00449667653118199\\
11.97475	-0.00458635200731472\\
11.99575	-0.0045799401477967\\
12.01475	-0.00462967409267051\\
12.03575	-0.00460853527394332\\
12.05475	-0.00466136005423064\\
12.07575	-0.00468420182149448\\
12.09475	-0.0046789050697057\\
12.11575	-0.00467465824454515\\
12.13475	-0.00468845452610703\\
12.15575	-0.00466395951355915\\
12.17475	-0.00468353751168328\\
12.19575	-0.00463336983657965\\
12.21475	-0.00465183629783926\\
12.23575	-0.00461174068487255\\
12.25475	-0.00462056193040538\\
12.27575	-0.0046160882834853\\
12.29475	-0.00460030408401524\\
12.31575	-0.00458578042418049\\
12.33475	-0.00458097980670764\\
12.35575	-0.00458726449824993\\
12.37475	-0.00469261476828497\\
12.39575	-0.00470661648346912\\
12.41475	-0.00473083928214376\\
12.43575	-0.00480109982023175\\
12.45475	-0.00479068354600337\\
12.47575	-0.00483899185947334\\
12.49475	-0.00483803421742332\\
12.51575	-0.00484465013571345\\
12.53475	-0.00485286794178348\\
12.55475	-0.00484654454830222\\
12.57475	-0.00484135336777116\\
12.59575	-0.00478084623827289\\
12.61475	-0.00473105497077517\\
12.63575	-0.00474219809468019\\
12.65475	-0.00470549528682677\\
12.67675	-0.00473885841806233\\
12.69475	-0.00470020518084543\\
12.71575	-0.00462593785458557\\
12.73475	-0.0046509036117932\\
12.75575	-0.00463031451248518\\
12.77475	-0.0046821744012986\\
12.79575	-0.00460145353896767\\
12.81475	-0.0045526644348431\\
12.83575	-0.00454370468386578\\
12.85475	-0.00455822512816576\\
12.87575	-0.00462566996109392\\
12.89475	-0.00453122661725447\\
12.91575	-0.00453719162348715\\
12.93475	-0.00455243530757627\\
12.95575	-0.00453533854474012\\
12.97475	-0.00454066398671004\\
12.99575	-0.00448578224806048\\
13.01475	-0.00448187852290335\\
13.03575	-0.00448740489259201\\
13.05475	-0.00446836646150067\\
13.07575	-0.00447857967373081\\
13.09475	-0.00446067783503622\\
13.11575	-0.0044782547411824\\
13.13475	-0.00444806842184426\\
13.15575	-0.004465453292228\\
13.17475	-0.00452064116319978\\
13.19575	-0.00447072309507178\\
13.21775	-0.00442734293532085\\
13.23675	-0.00441445931922815\\
13.25775	-0.00443258373135579\\
13.27775	-0.00444781500233505\\
13.29775	-0.00438613498450983\\
13.31775	-0.0044380256548598\\
13.33775	-0.00438644186823093\\
13.35775	-0.00437698286808431\\
13.37775	-0.00437546759142642\\
13.39775	-0.00435931143553381\\
13.41775	-0.00427353677130675\\
13.43775	-0.0042978757989493\\
13.45775	-0.0043154132076634\\
13.47875	-0.004334931770248\\
13.49775	-0.00434157260904874\\
13.51775	-0.00431499752154489\\
13.53775	-0.00428056551205524\\
13.55775	-0.00425910265700262\\
13.57775	-0.00430577058400783\\
13.59775	-0.0042750638591761\\
13.61775	-0.00423151060984618\\
13.63775	-0.00419182243656769\\
13.65775	-0.00424278080497488\\
13.67775	-0.00422750048418673\\
13.69775	-0.00420371192037714\\
13.71775	-0.00416603844210617\\
13.73775	-0.00409752605864733\\
13.75775	-0.00410579308383766\\
13.77775	-0.00413288735119595\\
13.79775	-0.00409835159157957\\
13.81775	-0.00405781635189004\\
13.83675	-0.0040885592011297\\
13.85775	-0.00408930014736551\\
13.87775	-0.00410645885401581\\
13.89775	-0.00406793696582115\\
13.91775	-0.00409510724314086\\
13.93775	-0.004072858533588\\
13.95775	-0.00406018838334462\\
13.97775	-0.0041291909252624\\
13.99775	-0.00407372239918933\\
14.01775	-0.00404096793710507\\
14.03775	-0.0040404192208136\\
14.05775	-0.00400124717596939\\
14.07775	-0.00402009884035155\\
14.09775	-0.00393654869062081\\
14.11775	-0.00396125228966175\\
14.13775	-0.0039268891335765\\
14.15775	-0.00392264767878621\\
14.17775	-0.00394927257362457\\
14.19775	-0.00390602667868623\\
14.21775	-0.00382809662842638\\
14.23775	-0.00385871661184495\\
14.25775	-0.00386676918568994\\
14.27875	-0.00388392012741953\\
14.29775	-0.00384325342956201\\
14.31775	-0.00385249019221827\\
14.33775	-0.00386077834347348\\
14.35775	-0.00377517864189086\\
14.37775	-0.00385106756298287\\
14.39775	-0.00375198371312436\\
14.41775	-0.00377057166011718\\
14.43775	-0.00386291616437776\\
14.45575	-0.00385213264309378\\
14.47775	-0.00384449892237856\\
14.49775	-0.00381511503403041\\
14.51775	-0.00388755232823171\\
14.53775	-0.00383574954288821\\
14.55775	-0.00384536891499456\\
14.57775	-0.00388081270687816\\
14.59775	-0.0038138419720651\\
14.61775	-0.0037859010345513\\
14.63775	-0.0038075962162804\\
14.65775	-0.00384343269607389\\
14.67775	-0.0037743974093757\\
14.69775	-0.00377633805661148\\
14.71775	-0.00374183460132157\\
14.73775	-0.00375104355732823\\
14.75775	-0.00380227080719151\\
14.77775	-0.003737602564691\\
14.79775	-0.00377242350311045\\
14.81775	-0.00380373690693694\\
14.83775	-0.00373648291735431\\
14.85775	-0.00375130709998497\\
14.87775	-0.00374404142390035\\
14.89775	-0.00378023418403819\\
14.91775	-0.00373137554851885\\
14.93775	-0.00372358286664015\\
14.95775	-0.0036720757365897\\
14.97775	-0.00367418204258813\\
14.99775	-0.00371521969644614\\
15.01775	-0.00366463242930174\\
15.03775	-0.00365852526525367\\
15.05675	-0.00361063212707587\\
15.07775	-0.00361941871698284\\
15.09775	-0.00356850118876161\\
15.11775	-0.00352978639683929\\
15.13775	-0.00352985582186952\\
15.15775	-0.00350036093799457\\
15.17775	-0.00353912562432669\\
15.19775	-0.00349693836592065\\
15.21775	-0.00357132762641205\\
15.23775	-0.00353085462176329\\
15.25775	-0.00355138620343609\\
15.27775	-0.00354879997472609\\
15.29775	-0.00360827858670544\\
15.31775	-0.00359949431841314\\
15.33775	-0.00361577064787155\\
15.35775	-0.00355942157461927\\
15.37775	-0.00362222188884379\\
15.39775	-0.00358736826672603\\
15.41775	-0.0035249933242232\\
15.43775	-0.00351550594659691\\
15.45775	-0.00347680149635341\\
15.47775	-0.00350761140481363\\
15.49775	-0.00345040047924608\\
15.51775	-0.00343093882065385\\
15.53775	-0.00338648796403993\\
15.55775	-0.00341201046487184\\
15.57775	-0.00342393414815714\\
15.59775	-0.00346542832062272\\
15.61775	-0.00340833955921613\\
15.63775	-0.00341711889663159\\
15.65775	-0.00341382395425668\\
15.67775	-0.00348885347872464\\
15.69775	-0.00342910698455701\\
15.71775	-0.00342717444841243\\
15.73775	-0.00340868329483661\\
15.75775	-0.00344875160473843\\
15.77775	-0.00341757808517676\\
15.79775	-0.00343328155107195\\
15.81575	-0.00340310246358035\\
15.83775	-0.00343964840492798\\
15.85775	-0.00339756590663444\\
15.87875	-0.00340574044205142\\
15.89775	-0.00338015379608231\\
15.91775	-0.00339379854328829\\
15.93775	-0.00334825258794578\\
15.95775	-0.00335186009499369\\
15.97775	-0.00328706136882447\\
15.99775	-0.00333408684454448\\
16.01775	-0.00328824294672994\\
16.03775	-0.00331186608808704\\
16.05775	-0.00331143827242066\\
16.07775	-0.00331302451254456\\
16.09775	-0.0032430707152814\\
16.11775	-0.00327889975933987\\
16.13775	-0.00331219901792329\\
16.15775	-0.00333650444910367\\
16.17775	-0.00326440029411889\\
16.19775	-0.00325478929076848\\
16.21775	-0.00327052386351401\\
16.23775	-0.00333499441540837\\
16.25775	-0.00335214242358888\\
16.27775	-0.00333884007416264\\
16.29775	-0.00331446463630285\\
16.31775	-0.00330068083750045\\
16.33775	-0.00328989014694455\\
16.35775	-0.00329154323215481\\
16.37775	-0.00324417948658063\\
16.39775	-0.00325038780853409\\
16.41575	-0.00313851621336981\\
16.43775	-0.00303143498623655\\
16.45775	-0.00280625688139883\\
16.47775	-0.00267957122379637\\
16.49775	-0.00234053677300824\\
16.51775	-0.00218603964744214\\
16.53775	-0.00203824275665495\\
16.55775	-0.00186676822682379\\
16.57775	-0.00187285695271265\\
16.59775	-0.00164905396608242\\
16.61775	-0.00153844186635608\\
16.63775	-0.0015093300693501\\
16.65775	-0.00146095072940854\\
16.67875	-0.00160651422137446\\
16.69775	-0.00142334175110624\\
16.71775	-0.00144878459545515\\
16.73775	-0.00144529806399778\\
16.75775	-0.00151277344135366\\
16.77775	-0.00163345235704574\\
16.79775	-0.00157596027746128\\
16.81775	-0.00158711120632587\\
16.83775	-0.00161798053189399\\
16.85775	-0.00154089742167895\\
16.87775	-0.00167753556010189\\
16.89775	-0.0015758678473946\\
16.91775	-0.00152417299807857\\
16.93775	-0.00146274184525729\\
16.95775	-0.00145016505482399\\
16.97775	-0.00148508844093\\
16.99775	-0.00132188463082665\\
17.01775	-0.00129027004298555\\
17.03775	-0.00125630406613762\\
17.05775	-0.00109583717912677\\
17.07775	-0.00109844469927702\\
17.09775	-0.000979140370634431\\
17.11775	-0.000936926172447469\\
17.13775	-0.000839546113101102\\
17.15775	-0.000751878976922279\\
17.17775	-0.000719137675394394\\
17.19775	-0.000583055891642869\\
17.21775	-0.000549102432817006\\
17.23775	-0.000524975029905414\\
17.25575	-0.000487182720729872\\
17.27775	-0.000525284161572809\\
17.29775	-0.000463950411036412\\
17.31775	-0.000369600503180841\\
17.33775	-0.000429036259279148\\
17.35775	-0.000379235768030975\\
17.37775	-0.000401222368953907\\
17.39775	-0.000366462585704757\\
17.41775	-0.000318808338545041\\
17.43775	-0.000253311469479287\\
17.45775	-0.000292914783150014\\
17.47875	-0.000303928553704049\\
17.49775	-0.00029368832011105\\
17.51775	-0.000241453917665764\\
17.53775	-0.000179039818669185\\
17.55775	-0.000107409226126313\\
17.57775	-0.000116646970746025\\
17.59775	-3.82294007564932e-005\\
17.61775	-2.33894658266679e-005\\
17.63775	-3.52490102402417e-006\\
17.65775	8.82368290953144e-005\\
17.67775	0.000108385773238103\\
17.69775	0.000195640042638932\\
17.71575	0.000263956183747234\\
17.73475	0.000259543191437986\\
17.75575	0.00029304319336834\\
17.77475	0.000329056030879466\\
17.79575	0.000304144373465228\\
17.81475	0.000297325800889925\\
17.83475	0.00035620687756445\\
17.85475	0.000338002688005869\\
17.87575	0.000324534688774083\\
17.89475	0.000338532642767639\\
17.91575	0.000368874955240759\\
17.93475	0.00039642501878099\\
17.95575	0.000395882905164379\\
17.97475	0.00041656813688567\\
17.99575	0.000392459835296586\\
18.01475	0.000446494709778934\\
18.03575	0.000496848631157424\\
18.05475	0.000480967685422556\\
18.07575	0.000466215517878338\\
18.09475	0.000464894603940851\\
18.11575	0.000460562657396176\\
18.13475	0.000462797313517355\\
18.15575	0.000462207502671603\\
18.17475	0.00052190508508141\\
18.19575	0.0004775634493524\\
18.21475	0.000558875733426936\\
18.23575	0.000571526716350685\\
18.25475	0.00056858760151146\\
18.27675	0.000538160786533757\\
18.29475	0.000524892476222542\\
18.31575	0.000522337041069208\\
18.33475	0.00051873783432368\\
18.35575	0.000524970345720525\\
18.37475	0.000556598215040194\\
18.39575	0.000490717115483788\\
18.41475	0.000496350268120046\\
18.43575	0.000573016783787103\\
18.45475	0.000555651712413091\\
18.47575	0.000562685410759731\\
18.49375	0.000576822693132259\\
18.51575	0.000596528398692369\\
18.53475	0.000637009901036176\\
18.55575	0.000564349680293555\\
18.57475	0.00053310686462142\\
18.59575	0.000538236106694272\\
18.61475	0.000547086765346877\\
18.63575	0.000525005114137234\\
18.65475	0.000526563675194649\\
18.67575	0.000511112601812418\\
18.69475	0.000518740205614231\\
18.71575	0.00053647707949895\\
18.73475	0.000520116598930239\\
18.75575	0.000533681455963796\\
18.77475	0.000518728877916149\\
18.79575	0.000526161297199077\\
18.81475	0.000497919647856821\\
18.83575	0.000532315521352934\\
18.85475	0.000461649832209625\\
18.87575	0.000513381972152624\\
18.89475	0.000507513782425473\\
18.91575	0.000471495721525523\\
18.93475	0.000438658211621685\\
18.95575	0.000435160803378911\\
18.97475	0.000481491699354927\\
18.99575	0.000443960452179281\\
19.01475	0.000442062940127608\\
19.03575	0.000487826422854962\\
19.05475	0.000453621798679206\\
19.07675	0.000469714846754791\\
19.09475	0.000458763186604486\\
19.11575	0.000494550118806343\\
19.13475	0.000447008955030763\\
19.15575	0.000464831095557069\\
19.17475	0.000476500189577352\\
19.19475	0.000459384326608515\\
19.21475	0.000418667886054897\\
19.23575	0.000439786127842367\\
19.25475	0.000454968152979841\\
19.27575	0.000448525613279131\\
19.29475	0.000388661762087041\\
19.31575	0.000442207618226509\\
19.33475	0.000412624323829673\\
19.35575	0.000442663619079161\\
19.37475	0.000448712059306636\\
19.39575	0.000450903660205028\\
19.41475	0.000449244811246191\\
19.43575	0.00045369799397564\\
19.45475	0.000451168973912455\\
19.47575	0.000500939742638609\\
19.49475	0.000424496976157937\\
19.51575	0.000480752269195856\\
19.53475	0.000475242996943881\\
19.55575	0.000460785471418934\\
19.57475	0.000445401489600357\\
19.59575	0.000431902591584551\\
19.61475	0.000416626121476263\\
19.63575	0.000399696962504541\\
19.65475	0.000389813991000839\\
19.67575	0.000440393349475887\\
19.69475	0.000365868653781728\\
19.71575	0.000424925531082992\\
19.73475	0.000418191136465812\\
19.75575	0.00041638348314212\\
19.77475	0.000412070248250961\\
19.79475	0.00047975514541677\\
19.81475	0.000473682651443145\\
19.83575	0.00045181471029563\\
19.85475	0.000387579390264515\\
19.87675	0.000408931092878689\\
19.89475	0.00038210095247282\\
19.91575	0.000361279583070766\\
19.93475	0.000347684071618249\\
19.95575	0.000403018186500039\\
19.97475	0.000395389681373797\\
19.99575	0.000387517244524383\\
20.01475	0.000397233665349398\\
20.03575	0.000355222297532118\\
20.05475	0.000397612014087027\\
20.07575	0.000381084190759015\\
20.09475	0.000435727498259773\\
20.11575	0.00048779731957794\\
20.13475	0.000420648579926483\\
20.15575	0.000448510231722211\\
20.17475	0.000484303901602046\\
20.19575	0.000524840946219254\\
20.21475	0.000568707023814668\\
20.23575	0.000598817976334342\\
20.25475	0.00051550439354665\\
20.27575	0.000523195169471352\\
20.29475	0.000542989885344627\\
20.31575	0.000515247394199485\\
20.33475	0.000542481118978056\\
20.35575	0.000466401961359941\\
20.37375	0.000484987456909333\\
20.39575	0.000452074502330084\\
20.41475	0.000431489001000199\\
20.43575	0.000414997852512604\\
20.45475	0.000355176959381727\\
20.47575	0.000389569711897562\\
20.49475	0.00037843471027829\\
20.51575	0.000439106605869617\\
20.53475	0.000434736915117503\\
20.55575	0.000451748750763632\\
20.57475	0.000422678756968931\\
20.59575	0.000471099865839609\\
20.61475	0.000476634169988607\\
20.63575	0.000434221724248595\\
20.65475	0.000482005722999906\\
20.67675	0.000462287848227626\\
20.69475	0.00047212670511952\\
20.71575	0.000441049292231704\\
20.73475	0.000491681338569781\\
20.75575	0.000474395891313867\\
20.77475	0.000491785727354111\\
20.79575	0.000461114928610149\\
20.81475	0.000430614477991643\\
20.83575	0.000443246322669838\\
20.85475	0.000415164600528484\\
20.87575	0.00040373150536052\\
20.89475	0.000461298388411271\\
20.91575	0.000411217933275998\\
20.93475	0.000390352600504384\\
20.95575	0.00045102709061511\\
20.97475	0.000406168699935755\\
20.99575	0.000449129809862829\\
21.01475	0.000437756190194185\\
21.03575	0.000457877175117846\\
21.05475	0.000427164054770136\\
21.07475	0.000442290861743441\\
21.09475	0.000478812960173793\\
21.11575	0.000459612478185344\\
21.13475	0.000480840329248704\\
21.15575	0.000523046774118654\\
21.17475	0.000503527518087627\\
21.19575	0.000522232493907422\\
21.21475	0.00048969294208055\\
21.23575	0.000493313422009844\\
21.25475	0.000460763451118747\\
21.27575	0.000513030133853133\\
21.29475	0.000465854290751874\\
21.31575	0.000511017413205095\\
21.33475	0.000506355113039623\\
21.35575	0.000530135470685748\\
21.37475	0.000563643833157864\\
21.39575	0.000512906370087718\\
21.41475	0.000554020463474453\\
21.43575	0.000537597094597636\\
21.45475	0.000556678426678525\\
21.47675	0.000597022878041712\\
21.49475	0.000541877054791641\\
21.51575	0.000574406048005076\\
21.53475	0.000523124124116502\\
21.55575	0.000565241694780732\\
21.57475	0.000551534687322909\\
21.59575	0.000518623583391395\\
21.61475	0.000568355810447247\\
21.63575	0.000524439869379263\\
21.65475	0.000569651043453889\\
21.67475	0.000531118362384819\\
21.69475	0.000514094079328331\\
21.71575	0.000539234719902519\\
21.73475	0.000536259859384846\\
21.75575	0.000605406130032273\\
21.77475	0.000574824758598365\\
21.79575	0.00057107764505589\\
21.81475	0.000540069258572309\\
21.83575	0.000595997765057559\\
21.85475	0.000558778407086394\\
21.87575	0.000552333147495878\\
21.89475	0.000519894076342094\\
21.91575	0.000514187051065406\\
21.93475	0.00055907756942697\\
21.95575	0.00055421173332043\\
21.97475	0.000524390095606817\\
21.99575	0.00052203533834609\\
22.01475	0.000571560211034082\\
22.03575	0.00057029591962127\\
22.05475	0.000539169506009625\\
22.07575	0.000569632148266397\\
22.09475	0.000559084271096908\\
22.11575	0.000529743213624386\\
22.13475	0.000525843550852524\\
22.15575	0.000511440621632742\\
22.17475	0.000520204252757829\\
22.19575	0.000498845793888052\\
22.21475	0.000473529052491687\\
22.23575	0.000477047407883036\\
22.25475	0.000463834947711898\\
22.27675	0.000479437829842531\\
22.29475	0.000472317095641087\\
22.31475	0.000454475496100527\\
22.33475	0.000467891873377977\\
22.35575	0.00046102979471048\\
22.37475	0.000505908154048668\\
22.39575	0.000513431540301986\\
22.41475	0.000499929907688287\\
22.43575	0.000475104654871754\\
22.45475	0.000481156341926936\\
22.47575	0.0005273347412933\\
22.49475	0.000504146271057105\\
22.51575	0.00050990542685349\\
22.53475	0.000490230435116272\\
22.55575	0.000461901327046237\\
22.57475	0.000428715861980114\\
22.59575	0.00043390197049827\\
22.61475	0.000423198447034854\\
22.63575	0.000459177621243312\\
22.65475	0.000361322772204444\\
22.67575	0.000437023386945609\\
22.69475	0.000428434571070452\\
22.71575	0.000335492510669019\\
22.73475	0.000378047884448105\\
22.75575	0.000398748058842226\\
22.77475	0.000393339251755524\\
22.79575	0.000370002463356567\\
22.81475	0.00033687823074784\\
22.83575	0.000316427555276418\\
22.85475	0.000334017689938674\\
22.87575	0.000392989994142022\\
22.89475	0.000368551568658646\\
22.91575	0.000341093396316964\\
22.93475	0.000369379126414401\\
22.95575	0.000325908101867853\\
22.97475	0.000361627967183562\\
22.99575	0.000365661378121683\\
23.01475	0.000346911339508645\\
23.03575	0.000381240603504058\\
23.05475	0.000345946815435632\\
23.07675	0.000367414480669729\\
23.09475	0.000315437645487247\\
23.11575	0.000261410161358349\\
23.13475	0.000210401018640906\\
23.15575	0.000243964777977044\\
23.17475	0.00025002856878496\\
23.19575	0.000300969774503598\\
23.21475	0.000272063525679003\\
23.23575	0.000307453976318669\\
23.25475	0.000266433273668665\\
23.27575	0.000285256454889523\\
23.29475	0.000234983091272455\\
23.31575	0.000256969406324663\\
23.33475	0.000206555655554324\\
23.35575	0.000219784342599143\\
23.37475	0.000229596114421499\\
23.39575	0.000237472658289332\\
23.41475	0.0001856312572636\\
23.43575	0.00019709668707763\\
23.45475	0.000212370615745254\\
23.47575	0.000270148422712794\\
23.49475	0.000313840888257129\\
23.51575	0.00033748426295806\\
23.53475	0.00034749195808085\\
23.55575	0.000282255526487146\\
23.57475	0.000278156203840148\\
23.59575	0.000272708347142426\\
23.61475	0.000273313321345151\\
23.63575	0.000203619491365301\\
23.65475	0.000199399499484872\\
23.67575	0.000195221308907778\\
23.69475	0.000192921938698026\\
23.71575	0.000198933405790539\\
23.73475	0.000200823904021304\\
23.75575	0.000199537849688371\\
23.77475	0.00019639157471622\\
23.79575	0.000193570050842454\\
23.81475	0.000198091775204371\\
23.83575	0.000197494467188504\\
23.85475	0.000193449941959581\\
23.87675	0.000188110153979905\\
23.89475	0.000183612213978159\\
23.91575	0.000187054280853377\\
23.93475	0.000185775312295801\\
23.95575	0.000182030501977605\\
23.97475	0.000112378529860261\\
23.99575	0.000113139778323921\\
24.01475	0.000122336482100558\\
24.03575	0.000192092352376797\\
24.05475	0.000189803094046475\\
24.07575	0.000186214818036574\\
24.09475	0.000188495129064603\\
24.11575	0.000185632061822423\\
24.13475	0.000243147143479873\\
24.15575	0.000228853103143172\\
24.17475	0.000214538158004379\\
24.19575	0.000207646337522241\\
24.21475	0.000196456326790837\\
24.23575	0.000183220897365236\\
24.25475	0.000169798469048971\\
24.27575	0.000157194722049796\\
24.29475	0.000218699458768439\\
24.31575	0.000207526855787386\\
24.33475	0.000194069638460486\\
24.35575	0.000179523719817616\\
24.37475	0.00016590161655682\\
24.39475	0.000226029141753865\\
24.41475	0.000212674625255766\\
24.43575	0.00019570994943457\\
24.45475	0.000178062916638324\\
24.47575	0.000226993609130129\\
24.49475	0.000215607527794169\\
24.51575	0.000200588610199075\\
24.53475	0.000183003046659906\\
24.55575	0.000230853497260409\\
24.57475	0.000212332794165133\\
24.59575	0.000202414369624207\\
24.61475	0.000190048010286534\\
24.63575	0.000176601344999968\\
24.65475	0.000229522558126177\\
24.67675	0.000222531339881491\\
24.69475	0.000211052717572402\\
24.71575	0.00019721037706992\\
24.73475	0.000182940911509684\\
24.75575	0.000179467826677036\\
24.77475	0.000228440879125003\\
24.79575	0.000253033127041689\\
24.81475	0.000261666665052364\\
24.83575	0.000261263788148821\\
24.85475	0.000262103504900585\\
24.87575	0.000255525817986845\\
24.89475	0.000308875740423934\\
24.91575	0.000290364883549474\\
24.93475	0.000271315134099598\\
24.95575	0.000259705105165728\\
24.97475	0.000245067238406008\\
24.99475	0.000293658299928501\\
25.01475	0.000274345114816083\\
25.03575	0.000262065069629305\\
25.05475	0.000246652211726019\\
25.07575	0.000303458802411001\\
25.09475	0.000334980454675811\\
25.11575	0.000353593492560604\\
25.13475	0.000354605379675347\\
25.15575	0.000344431790202063\\
25.17475	0.000329253558131467\\
25.19575	0.00031924833295768\\
25.21475	0.000304303856991712\\
25.23575	0.00028662345112734\\
25.25475	0.000270185150520641\\
25.27575	0.000262403798101527\\
25.29475	0.000252313892203859\\
25.31575	0.000241677048749031\\
25.33475	0.000232334297608671\\
25.35575	0.000240405228236183\\
25.37475	0.000288488888487711\\
25.39575	0.000314409278769948\\
25.41475	0.000326338838321462\\
25.43575	0.000336137266247881\\
25.45475	0.000335279113276592\\
25.47675	0.000262719142069188\\
25.49475	0.000257075027489224\\
25.51575	0.000258635123003151\\
25.53475	0.000255958397644566\\
25.55575	0.000251101774444222\\
25.57475	0.000246806967284415\\
25.59475	0.000185293039110463\\
25.61475	0.000197934436764085\\
25.63575	0.000252460655725914\\
25.65475	0.000286826046024865\\
25.67575	0.000313929553643922\\
25.69475	0.000326519296812816\\
25.71575	0.000328909893575082\\
25.73475	0.000326460739654104\\
25.75575	0.000328356167796533\\
25.77475	0.000324105819743984\\
25.79575	0.000251166489390567\\
25.81475	0.000247155962683008\\
25.83575	0.000251785343754571\\
25.85475	0.000188139719741081\\
25.87575	0.000192518538087088\\
25.89475	0.000199237850992106\\
25.91575	0.000214315269982051\\
25.93475	0.000234033901161673\\
25.95575	0.000229339321405071\\
25.97475	0.000272729351519302\\
25.99575	0.000309406448434587\\
26.01475	0.000266438443434354\\
26.03575	0.000282557863451987\\
26.05475	0.000292793369776952\\
26.07575	0.00030003876285305\\
26.09475	0.000312210972806625\\
26.11575	0.000253437869324292\\
26.13475	0.000259772229484373\\
26.15575	0.000265936714331089\\
26.17475	0.000279068296438152\\
26.19575	0.000222159247285635\\
26.21475	0.000231090742900181\\
26.23575	0.000238820050544862\\
26.25375	0.000246519458051749\\
26.27675	0.000196191987895801\\
26.29475	0.000210703923874589\\
26.31575	0.000222505186274127\\
26.33475	0.00016873440645512\\
26.35575	0.000191010685194658\\
26.37475	0.000208722413069473\\
26.39575	0.00015788336656969\\
26.41475	0.000173980598897185\\
26.43575	0.000124983910050583\\
26.45475	0.000152087727519353\\
26.47575	0.000175303712454334\\
26.49475	0.00012940317217212\\
26.51575	0.000151221746132348\\
26.53475	0.000179858223936667\\
26.55575	0.00020220524876802\\
26.57475	0.0002191498093636\\
26.59575	0.000231837002725022\\
26.61475	0.000242055449519085\\
26.63575	0.000257366116553721\\
26.65475	0.000202024305811217\\
26.67575	0.000211348699028134\\
26.69475	0.000220601249819123\\
26.71575	0.000236547898051342\\
26.73475	0.000247049254101848\\
26.75575	0.000188962343590243\\
26.77475	0.000197861050295391\\
26.79575	0.000207297698816466\\
26.81475	0.000224023441524893\\
26.83575	0.000236310445513382\\
26.85475	0.000179841248464215\\
26.87575	0.000190286171845956\\
26.89475	0.00020192857755441\\
26.91575	0.000220256233292788\\
26.93375	0.000234014467080335\\
26.95575	0.000244206725739026\\
26.97475	0.000251324741788671\\
26.99575	0.000258357101905986\\
27.01475	0.00027184067481109\\
27.03575	0.000279876703874808\\
27.05475	0.00028434427225573\\
27.07675	0.000286086415872023\\
27.09475	0.000222417302660431\\
27.11575	0.000235631072575654\\
27.13475	0.000245667963748881\\
27.15575	0.000253133313651523\\
27.17475	0.000250627286092907\\
27.19575	0.000140383298472546\\
27.21475	0.000127217263214745\\
27.23575	0.00012573739910155\\
27.25475	0.000131333095881059\\
27.27575	0.000142891925056916\\
27.29475	0.000165032686217346\\
27.31575	0.000184379785294577\\
27.33475	0.000200682963185009\\
27.35575	0.000214695213076982\\
27.37475	0.000227767715787074\\
27.39575	0.000246011051005997\\
27.41475	0.000258892153119065\\
27.43575	0.000202425952803052\\
27.45475	0.000212174417706356\\
27.47575	0.000223133086964504\\
27.49475	0.000240677604881345\\
27.51575	0.0002533269254274\\
27.53475	0.000262748148687123\\
27.55475	0.000271315407369244\\
27.57475	0.0002119571350565\\
27.59575	0.000174004641095998\\
27.61475	0.000154351957583627\\
27.63575	0.000147050176893147\\
27.65475	0.000150120920499197\\
27.67575	0.000166140781295896\\
27.69475	0.00018144163147649\\
27.71575	0.000195888121630192\\
27.73475	0.000209143865006488\\
27.75575	0.000223021159679505\\
27.77475	0.000243498479635013\\
27.79575	0.000194178688976343\\
27.81475	0.000210568439129499\\
27.83575	0.000225399495874731\\
27.85475	0.000232000383753098\\
27.87675	0.000201300517352781\\
27.89475	0.000186193522720912\\
27.91575	0.000181736067018556\\
27.93475	0.000184146907576981\\
27.95575	0.000192829286046785\\
27.97475	0.00021202857877813\\
27.99575	0.000228714978710778\\
28.01475	0.000243262726380706\\
28.03575	0.000256019443391333\\
28.05475	0.000268870476639362\\
28.07575	0.000288023427873876\\
28.09475	0.000237458906344032\\
28.11575	0.000317336215638082\\
28.13475	0.000317549795023555\\
28.15575	0.000273877102329284\\
28.17375	0.000191673380876131\\
28.19575	0.000188430371004293\\
28.21475	0.000192900171400182\\
28.23575	0.000203677672672703\\
28.25475	0.000224948855982563\\
28.27575	0.000243820826208964\\
28.29475	0.00025997083838118\\
28.31575	0.000209279234294757\\
28.33475	0.000228195784088112\\
28.35575	0.00024542742693279\\
28.37475	0.000214801459586904\\
28.39575	0.000201168014925055\\
28.41475	0.000198309179160641\\
28.43575	0.000204931444583575\\
28.45475	0.000159180347724952\\
28.47575	0.000181118758444003\\
28.49475	0.00020307314632564\\
28.51575	0.00022407987573137\\
28.53475	0.000244782345969575\\
28.55575	0.000272038962335509\\
28.57475	0.000285092743542546\\
28.59575	0.000249777290592867\\
28.61475	0.000232109297755002\\
28.63575	0.000228198107035672\\
28.65475	0.000239353251139244\\
28.67675	0.00025210575489427\\
28.69475	0.000265194531353636\\
28.71575	0.000277241127900477\\
28.73475	0.000224231707862931\\
28.75475	0.000241748732667363\\
28.77475	0.000258204717978871\\
28.79575	0.000227536946139564\\
28.81475	0.000213937031332444\\
28.83575	0.000211162464397961\\
28.85475	0.000218140406274152\\
28.87575	0.000237764607501775\\
28.89475	0.00025538918675654\\
28.91575	0.00027193230459183\\
28.93475	0.0002219139652081\\
28.95575	0.000230646487609759\\
28.97475	0.00019795858854829\\
28.99575	0.000193105466985636\\
29.01475	0.000197074975105574\\
29.03575	0.00020721170442053\\
29.05475	0.000220590331316038\\
29.07575	0.000172199118810091\\
29.09475	0.000202376759292115\\
29.11575	0.000229249583400032\\
29.13475	0.000244013186044209\\
29.15575	0.000212822474742184\\
29.17475	0.000201067559329783\\
29.19575	0.000209282333430808\\
29.21475	0.000221454994004446\\
29.23575	0.000170532460480518\\
29.25475	0.000189576314448595\\
29.27575	0.000202875846769818\\
29.29475	0.000180886019423623\\
29.31575	0.000174521255331318\\
29.33475	0.000113746068596\\
29.35475	0.000128749474002352\\
29.37475	0.000150188756824579\\
29.39575	0.000182295927942979\\
29.41475	0.000210747522297148\\
29.43575	0.000227069529669325\\
29.45475	0.000132186592714364\\
29.47675	0.00012450881104883\\
29.49475	0.000131456981159585\\
29.51575	0.000154836905734513\\
29.53475	0.000178660692879934\\
29.55575	0.000201860923765992\\
29.57475	0.000215148629258345\\
29.59575	0.000185061029178408\\
29.61475	0.000181998404372424\\
29.63575	0.000187560635079637\\
29.65475	0.000198413358609764\\
29.67575	0.000147185151015265\\
29.69475	0.000166962115142585\\
29.71575	0.000181147875374718\\
29.73475	0.000160318953646682\\
29.75575	0.000155427627823359\\
29.77475	9.60035614085999e-005\\
29.79575	0.000112337872653785\\
29.81475	0.000133597013226114\\
29.83575	0.000158648444158355\\
29.85475	0.000183542485201615\\
29.87575	0.000160521474182629\\
29.89475	0.000154212087595417\\
29.91575	0.00015826361084125\\
29.93475	0.000168746667581762\\
29.95575	0.000184565381101605\\
29.97475	0.00020105421659939\\
};
\addlegendentry{$\kappa{}_\text{d}'$};

\addplot [color=light-gray,line width=1.0]
  table[row sep=crcr]{%
7.97475	0.00018948043014235\\
7.99575	0.000191578580582164\\
8.01475	0.000193314589630405\\
8.03575	0.000194724807844275\\
8.05475	0.000195890046288378\\
8.07575	0.000196804487537952\\
8.09475	0.000142953079419018\\
8.11575	0.000148089836838658\\
8.13475	0.000153891189325909\\
8.15575	0.000159736854586048\\
8.17475	0.000165333386409014\\
8.19575	0.000170478661987343\\
8.21475	0.000175139281359991\\
8.23575	0.000179219954441969\\
8.25475	0.000182719277169175\\
8.27575	0.000185757412661745\\
8.29475	0.000188250287420407\\
8.31575	0.000190361871510393\\
8.33475	0.000192127239350823\\
8.35575	0.000193539931153176\\
8.37475	0.000140054068222135\\
8.39575	0.000145586589465126\\
8.41475	0.0001516284143641\\
8.43575	0.000157808751895656\\
8.45475	0.000163680506094883\\
8.47575	0.000169018385605706\\
8.49475	0.000173814564053975\\
8.51575	0.000178000659331562\\
8.53475	0.000181575008674364\\
8.55575	0.000184059534273747\\
8.57375	0.000185276444080519\\
8.59575	0.000185660351996148\\
8.61475	0.000130949838481375\\
8.63575	0.000135071888935021\\
8.65475	0.000139820408110324\\
8.67675	0.000145052288704697\\
8.69475	0.000150421766321706\\
8.71575	0.000155810377951397\\
8.73475	0.000161107006402985\\
8.75575	0.000165362507506786\\
8.77475	0.000168509540281844\\
8.79575	0.000170586493048805\\
8.81475	0.000171958124410326\\
8.83575	0.000172946657435048\\
8.85475	0.000173422674634777\\
8.87575	0.000173302041095463\\
8.89475	0.000172806573399926\\
8.91575	0.000172022210228136\\
8.93475	0.000170985978213401\\
8.95575	0.000115145187216841\\
8.97475	0.000118882439753076\\
8.99575	0.000123434473315099\\
9.01475	0.000128507911868343\\
9.03575	0.000133559350893963\\
9.05475	0.000138712999985665\\
9.07575	0.000144042292616871\\
9.09475	9.50392728019138e-005\\
9.11575	0.000105066981130009\\
9.13475	0.000115695291907861\\
9.15575	0.000126481525368072\\
9.17375	0.000137329857688386\\
9.19575	9.37313083078157e-005\\
9.21475	0.000109585497418172\\
9.23575	0.000126274741025763\\
9.25475	8.8975749320149e-005\\
9.27575	0.000111108550541427\\
9.29475	7.97106062121677e-005\\
9.31575	5.37415997282744e-005\\
9.33475	3.34435785090459e-005\\
9.35575	1.81648305966915e-005\\
9.37475	7.87291263737959e-006\\
9.39575	2.06549978929093e-006\\
9.41475	3.00212021610949e-007\\
9.43575	2.61966252877353e-006\\
9.45475	8.11799826206652e-006\\
9.47675	1.72254072868619e-005\\
9.49475	-2.56385462026187e-005\\
9.51575	-6.13604049227011e-005\\
9.53475	-3.52934963265574e-005\\
9.55575	-6.06333548474258e-005\\
9.57475	-8.00251558418613e-005\\
9.59575	-9.328157998856e-005\\
9.61475	-0.000101142021491216\\
9.63575	-4.88616689673462e-005\\
9.65475	-0.000105094562091285\\
9.67575	-9.9541369008053e-005\\
9.69475	-8.99859445754649e-005\\
9.71575	-7.71060071691559e-005\\
9.73475	-0.00011612552250066\\
9.75575	-9.29703392052261e-005\\
9.77375	-6.74514703429493e-005\\
9.79575	-9.36397255048211e-005\\
9.81475	-0.000114729357102457\\
9.83575	-0.000129403595013391\\
9.85475	-0.000138854124853991\\
9.87575	-0.000140913829609257\\
9.89475	-0.000138198844333804\\
9.91575	-0.000131054511627002\\
9.93475	-0.000120376451421681\\
9.95575	-0.000105911934714249\\
9.97475	-8.85014197934532e-005\\
9.99575	-0.000125108295071613\\
10.01475	-9.9818100937148e-005\\
10.03575	-7.38973661869188e-005\\
10.05475	-0.000104354866241544\\
10.07575	-7.40888845336348e-005\\
10.09475	-9.99087878076629e-005\\
10.11575	-0.000122815228680741\\
10.13475	-8.85167053841704e-005\\
10.15575	-0.000108907136678278\\
10.17475	-1.75607822267785e-005\\
10.19575	-9.70360512009615e-005\\
10.21475	-6.07196060207363e-005\\
10.23575	-8.20089725926499e-005\\
10.25475	-9.74471343688086e-005\\
10.27675	-5.61865767854864e-005\\
10.29475	-7.20545903962056e-005\\
10.31575	-8.70738066148772e-005\\
10.33475	-9.69156607521997e-005\\
10.35575	-0.000105077539799112\\
10.37475	-2.78112808286921e-006\\
10.39575	-7.1669605696147e-005\\
10.41475	-8.07629973337452e-005\\
10.43575	-3.46602274114647e-005\\
10.45475	-4.36822569930272e-005\\
10.47575	6.82898399852345e-007\\
10.49475	-6.87262060051171e-005\\
10.51575	-2.36878308335046e-005\\
10.53475	-3.60659898677097e-005\\
10.55575	-4.980157505916e-005\\
10.57475	4.75235577653525e-005\\
10.59575	-2.87429011922611e-005\\
10.61475	-4.67153427015914e-005\\
10.63575	-8.16527640325725e-006\\
10.65475	-2.84822543811943e-005\\
10.67575	3.09792351266285e-006\\
10.69475	-2.30314716816341e-005\\
10.71475	7.0957596715499e-006\\
10.73475	3.17051564137271e-005\\
10.75575	-3.86957904287305e-006\\
10.77475	1.68340755509409e-005\\
10.79575	3.29164042996386e-005\\
10.81475	4.52558699153697e-005\\
10.83575	5.23000145147486e-005\\
10.85475	5.40509732503197e-005\\
10.87575	0.000105925246615745\\
10.89475	9.68705189094429e-005\\
10.91575	8.23113593829285e-005\\
10.93475	0.000120975318797926\\
10.95575	0.000151917504140886\\
10.97475	0.000176407614778694\\
10.99575	0.000142151863202072\\
11.01475	0.000159809236793814\\
11.03575	0.000118395551577648\\
11.05475	0.000132795573489249\\
11.07675	0.000144620457796926\\
11.09475	9.7797141236092e-005\\
11.11575	0.000107356389264261\\
11.13475	0.000113177253581364\\
11.15575	0.000171155289076714\\
11.17475	0.000166930814080759\\
11.19575	0.000158886611877573\\
11.21475	0.000201585329765776\\
11.23575	0.000182403595802681\\
11.25475	0.000213517241553585\\
11.27575	0.000237869327082122\\
11.29475	0.000199718775968722\\
11.31575	0.000159096458998274\\
11.33475	0.000172710644616796\\
11.35475	0.000181448877008452\\
11.37475	0.000185236135518698\\
11.39575	0.000184417921072603\\
11.41475	0.00017918942482453\\
11.43575	0.000225001509114277\\
11.45475	0.000207750800291012\\
11.47575	0.000240685099306611\\
11.49475	0.000211836275834969\\
11.51575	0.000234067619284719\\
11.53475	0.000195277364187294\\
11.55575	0.000209856882175113\\
11.57475	0.000219178356982929\\
11.59575	0.000224532262164255\\
11.61475	0.000225851851879717\\
11.63575	0.00016864748684922\\
11.65475	0.000167288395480522\\
11.67575	0.00016470312972365\\
11.69475	0.000160538732487973\\
11.71575	0.000208769480796493\\
11.73475	0.000196219468917765\\
11.75575	0.000181425506763581\\
11.77475	0.000164978829337575\\
11.79575	0.000202193571507019\\
11.81475	0.000179599351165996\\
11.83575	0.000156482128346542\\
11.85475	0.000187990587863289\\
11.87675	0.000160696273859843\\
11.89475	0.000133702418968025\\
11.91575	0.000161679846953555\\
11.93475	0.000131433988486699\\
11.95475	0.000156524427220357\\
11.97475	0.000123343028647672\\
11.99575	0.000145072573090534\\
12.01475	0.000108436890700878\\
12.03575	0.000127105170455626\\
12.05475	8.70665944911785e-005\\
12.07575	0.000101839507144024\\
12.09475	0.000113518619079695\\
12.11575	0.00012137652352062\\
12.13475	0.000124618106112345\\
12.15575	0.000178390453738304\\
12.17475	0.000168855759336099\\
12.19575	0.00020931453667318\\
12.21475	0.000186542984570169\\
12.23575	0.000214355286685415\\
12.25475	0.000235283384861287\\
12.27575	0.000249985551514152\\
12.29475	0.000259512923095395\\
12.31575	0.000264282947408707\\
12.33475	0.000265419355520834\\
12.35575	0.000262677513400196\\
12.37475	0.000202919999205714\\
12.39575	0.000201336806431039\\
12.41475	0.000198086092122075\\
12.43575	0.000139544427723784\\
12.45475	0.000139480326820171\\
12.47575	8.54992186957386e-005\\
12.49475	8.99182748859697e-005\\
12.51575	9.40012223168104e-005\\
12.53475	9.64934481606716e-005\\
12.55475	9.7894121674507e-005\\
12.57475	9.836765086899e-005\\
12.59575	0.000151622962145861\\
12.61475	0.000197465059673906\\
12.63575	0.000181788686971425\\
12.65475	0.000217554522922316\\
12.67675	0.000192050991954037\\
12.69475	0.00021945268745691\\
12.71575	0.000241947683082185\\
12.73475	0.000203560227515893\\
12.75575	0.000218679542142216\\
12.77475	0.000175284474784279\\
12.79575	0.000241098005049692\\
12.81475	0.000244855086422873\\
12.83575	0.000244420330103919\\
12.85475	0.000239237873844745\\
12.87575	0.000178011382600025\\
12.89475	0.000228881044939847\\
12.91575	0.000219629679416993\\
12.93475	0.000208816549016257\\
12.95575	0.000198163702013599\\
12.97475	0.000187659396775563\\
12.99575	0.000233581717823193\\
13.01475	0.00022021654604395\\
13.03575	0.000207607943990494\\
13.05475	0.000196483186842396\\
13.07575	0.000187478363056451\\
13.09475	0.000181225403454189\\
13.11575	0.000175066099390354\\
13.13475	0.000170918580447255\\
13.15575	0.000114076550522482\\
13.17475	6.23863528081581e-005\\
13.19575	7.37793257456671e-005\\
13.21775	8.64008642844029e-005\\
13.23675	0.00010091446327134\\
13.25775	6.36993410649938e-005\\
13.27775	3.05525579630344e-005\\
13.29775	5.83742889490646e-005\\
13.31775	3.14917621670846e-005\\
13.33775	6.30888380783139e-005\\
13.35775	4.11840424651437e-005\\
13.37775	2.18945108343143e-005\\
13.39775	6.08860046226159e-006\\
13.41775	4.90824829876288e-005\\
13.43775	3.80386078507533e-005\\
13.45775	3.0390233912184e-005\\
13.47875	2.51841486629743e-005\\
13.49775	2.21197004087373e-005\\
13.51775	2.24291264846641e-005\\
13.53775	2.58678268454875e-005\\
13.55775	3.15184631507561e-005\\
13.57775	-1.62591812261026e-005\\
13.59775	-3.84445942657535e-006\\
13.61775	1.06516589103575e-005\\
13.63775	2.61995741765204e-005\\
13.65775	-1.37956325195213e-005\\
13.67775	4.61628049134771e-006\\
13.69775	2.2827682575305e-005\\
13.71775	4.09262956635765e-005\\
13.73775	5.75745376958041e-005\\
13.75775	1.88824420129835e-005\\
13.77775	-1.60570310323662e-005\\
13.79775	8.37571493305689e-006\\
13.81775	3.38649739788904e-005\\
13.83675	4.69488475559057e-006\\
13.85775	-2.08551969623196e-005\\
13.87775	-4.23198411587635e-005\\
13.89775	-5.11887336524168e-006\\
13.91775	-2.18669568600632e-005\\
13.93775	1.79754689277879e-005\\
13.95775	2.21640407698068e-006\\
13.97775	-6.55688390256605e-005\\
13.99775	-1.80232101804405e-005\\
14.01775	-2.660028741845e-005\\
14.03775	-3.50468915324138e-005\\
14.05775	1.07625020508539e-005\\
14.07775	-1.61489428012826e-006\\
14.09775	4.07931555584789e-005\\
14.11775	2.39218248836975e-005\\
14.13775	6.12813131831567e-005\\
14.15775	4.03371048758746e-005\\
14.17775	1.93106900707645e-005\\
14.19775	5.415677117945e-005\\
14.21775	8.69419392989645e-005\\
14.23775	6.17402535649785e-005\\
14.25775	3.71482635009152e-005\\
14.27875	1.54500731103202e-005\\
14.29775	5.17540728151606e-005\\
14.31775	3.18344809416302e-005\\
14.33775	1.51661606926084e-005\\
14.35775	5.69590426864749e-005\\
14.37775	-1.12575256485651e-005\\
14.39775	3.83052928212854e-005\\
14.41775	3.29271230813618e-005\\
14.43775	-2.4423958247397e-005\\
14.45575	-2.02402894786062e-005\\
14.47775	-1.23270727538195e-005\\
14.49775	-1.15971312682952e-006\\
14.51775	-4.34778014493236e-005\\
14.53775	-2.48569873666429e-005\\
14.55775	-5.91120345369619e-005\\
14.57775	-8.87614503007908e-005\\
14.59775	-5.80913408937551e-005\\
14.61775	-2.75281751465856e-005\\
14.63775	-5.29332662888816e-005\\
14.65775	-7.55897892482694e-005\\
14.67775	-3.99614966070439e-005\\
14.69775	-6.01715834085369e-005\\
14.71775	-2.30990285154663e-005\\
14.73775	-4.16399840053945e-005\\
14.75775	-5.88994077328756e-005\\
14.77775	-1.81719620184914e-005\\
14.79775	-3.42980957701595e-005\\
14.81775	-4.98727240243573e-005\\
14.83775	-9.17480726048702e-006\\
14.85775	-2.58276874724515e-005\\
14.87775	-4.15291000183075e-005\\
14.89775	-5.56193885931506e-005\\
14.91775	-1.35757234928079e-005\\
14.93775	-2.93090596190753e-005\\
14.95775	1.072061103274e-005\\
14.97775	-6.48533823770654e-006\\
14.99775	-2.39246246781248e-005\\
15.01775	1.43561528900788e-005\\
15.03775	-4.95937734253177e-006\\
15.05675	3.11980261458078e-005\\
15.07775	9.79680792565839e-006\\
15.09775	4.36399806409715e-005\\
15.11775	7.41408277995003e-005\\
15.13775	4.58375979243167e-005\\
15.15775	7.24434539252435e-005\\
15.17775	4.04574555146902e-005\\
15.19775	6.45642383101608e-005\\
15.21775	3.05349973242344e-005\\
15.23775	5.26000995341608e-005\\
15.25775	1.81305962605013e-005\\
15.27775	4.10263638138986e-005\\
15.29775	6.68676396614057e-006\\
15.31775	3.01676432427165e-005\\
15.33775	-1.73077999555986e-006\\
15.35775	2.39501609917656e-005\\
15.37775	-6.73484524762814e-006\\
15.39775	1.93833251176051e-005\\
15.41775	4.37875816890052e-005\\
15.43775	1.09021756214732e-005\\
15.45775	3.48865575351285e-005\\
15.47775	1.64691159561285e-006\\
15.49775	2.56617812643526e-005\\
15.51775	4.82639166288568e-005\\
15.53775	6.88197548989544e-005\\
15.55775	3.30362226138251e-005\\
15.57775	5.2861485501649e-005\\
15.59775	1.62807774698482e-005\\
15.61775	3.71682693129531e-005\\
15.63775	2.80311913637096e-006\\
15.65775	2.60499794266358e-005\\
15.67775	-7.28380581465604e-006\\
15.69775	1.76549518404945e-005\\
15.71775	-1.25753490036915e-005\\
15.73775	1.45806083633045e-005\\
15.75775	-1.41107380998293e-005\\
15.77775	1.37123813465476e-005\\
15.79775	-1.43952125465334e-005\\
15.81575	1.49589969652872e-005\\
15.83775	-1.14614058407793e-005\\
15.85775	1.91983462249544e-005\\
15.87875	-5.57320872865233e-006\\
15.89775	2.66595118568291e-005\\
15.91775	3.13091365671089e-006\\
15.93775	3.56238924722066e-005\\
15.95775	1.16508325742893e-005\\
15.97775	4.4319656775233e-005\\
15.99775	2.05603093813252e-005\\
16.01775	5.38715790010847e-005\\
16.03775	3.03019553952833e-005\\
16.05775	8.4725791214136e-006\\
16.07775	-1.12055291631799e-005\\
16.09775	2.62163446853074e-005\\
16.11775	7.70825155020744e-006\\
16.13775	-9.01012794271477e-006\\
16.15775	-2.34891510750658e-005\\
16.17775	1.9330091352219e-005\\
16.19775	5.97050354000069e-006\\
16.21775	-5.72183242506033e-006\\
16.23775	-1.53059277600875e-005\\
16.25775	-2.24050950260652e-005\\
16.27775	-2.8235056342017e-005\\
16.29775	-3.21132209462387e-005\\
16.31775	-3.48808723352149e-005\\
16.33775	-3.68772747252911e-005\\
16.35775	-3.73446734045212e-005\\
16.37775	1.65889796305302e-005\\
16.39775	1.10347263297231e-005\\
16.41575	0.000112589585794343e-005\\
16.43775	0.000199747380081449e-005\\
16.45775	0.000374464948538299e-005\\
16.47775	0.000454440229855537e-005\\
16.49775	0.00070479096656914e-005\\
16.51775	0.000833078762709371e-005\\
16.53775	0.000952821849680122e-005\\
16.55775	0.00106854851202335e-005\\
16.57775	0.00102165090511789e-005\\
16.59775	0.00116063558904482e-005\\
16.61775	0.00120139614433534e-005\\
16.63775	0.00116213278586394e-005\\
16.65775	0.00115708047082295e-005\\
16.67875	0.000975539404968574e-005\\
16.69775	0.00107506324101071e-005\\
16.71775	0.00100036636149029e-002\\
16.73775	0.0009275960299992e-002\\
16.75775	0.000810052977908924e-002\\
16.77775	0.000656140439264737e-002\\
16.79775	0.000631773058275627e-002\\
16.81775	0.00056861442628099e-002\\
16.83775	0.00046692048574442e-002\\
16.85775	0.000440097089348674e-002\\
16.87775	0.000266650672622599e-002\\
16.89775	0.00028593817338078e-002\\
16.91775	0.000261830768219097\\
16.93775	0.00024971127176573\\
16.95775	0.000193895497531427\\
16.97775	9.63882109504129e-005\\
16.99775	0.000178843745704161\\
17.01775	0.000154957670191102\\
17.03775	0.000133865796126662\\
17.05775	0.000175962741240849\\
17.07775	0.000109434515517433\\
17.09775	0.00016084262870558\\
17.11775	0.000157192989872737\\
17.13775	0.000157102591467478\\
17.15775	0.00016335040324233\\
17.17775	0.000118461710114429\\
17.19775	0.000190756836247473\\
17.21775	0.000154732255687959\\
17.23775	0.000128644553113159\\
17.25575	0.000111094447365446\\
17.27775	4.91095262327609e-005\\
17.29775	5.57556069457251e-005\\
17.31775	7.3264320998449e-005\\
17.33775	-8.1728354205993e-006\\
17.35775	-1.46884905156926e-005\\
17.37775	-6.48948380160746e-005\\
17.39775	-9.5716866265169e-005\\
17.41775	-0.000110202736016826\\
17.43775	-0.000109150855469357\\
17.45775	-0.000149661272946615\\
17.47875	-0.0001748998209209\\
17.49775	-0.000186956961180409\\
17.51775	-0.00018616489188869\\
17.53775	-0.000174471688424922\\
17.55775	-0.000153294631844891\\
17.57775	-0.000180076735190321\\
17.59775	-0.000141841338551289\\
17.61775	-0.000153677203637505\\
17.63775	-0.00015857072642469\\
17.65775	-0.000102176411318699\\
17.67775	-9.9999309560587e-005\\
17.69775	-9.31392381405224e-005\\
17.71575	-8.24172613638081e-005\\
17.73475	-0.000122890118330508\\
17.75575	-0.000101080414429838\\
17.77475	-7.50319034410172e-005\\
17.79575	-0.000100738516345722\\
17.81475	-0.00011918968011552\\
17.83475	-7.59804979094508e-005\\
17.85475	-8.5097362560119e-005\\
17.87575	-8.75766680921662e-005\\
17.89475	-8.40685158361271e-005\\
17.91575	-7.46660427337455e-005\\
17.93475	-6.0550644326591e-005\\
17.95575	-4.10258861439025e-005\\
17.97475	-1.85353283200535e-005\\
17.99575	-4.81291349875817e-005\\
18.01475	-1.71458887273026e-005\\
18.03575	1.48729606193302e-005\\
18.05475	-6.97642164112522e-006\\
18.07575	-2.4791156159605e-005\\
18.09475	-3.84406308990311e-005\\
18.11575	-4.76827924762378e-005\\
18.13475	-5.2776599353375e-005\\
18.15575	-5.39054067258171e-005\\
18.17475	3.98901859931655e-006\\
18.19575	-4.98119458176933e-005\\
18.21475	1.09167751670862e-005\\
18.23575	1.46720436216462e-005\\
18.25475	1.92130699016513e-005\\
18.27675	2.38230645088447e-005\\
18.29475	2.77161163472308e-005\\
18.31575	3.15481202194355e-005\\
18.33475	3.57128137345933e-005\\
18.35575	3.96160899160269e-005\\
18.37475	9.87164896802628e-005\\
18.39575	4.38003223168112e-005\\
18.41475	4.7561817828721e-005\\
18.43575	0.000108075903726017\\
18.45475	0.000108746708512854\\
18.47575	0.000108744591007075\\
18.49375	0.000108365596490722\\
18.51575	0.000107858985157879\\
18.53475	0.000161889082411358\\
18.55575	0.000102722428704277\\
18.57475	0.000101711307143902\\
18.59575	0.000101174019057643\\
18.61475	0.000100547206678388\\
18.63575	0.000101382926298733\\
18.65475	0.000101957098302975\\
18.67575	0.00010238355779068\\
18.69475	0.000102356860227555\\
18.71575	0.000102736028460239\\
18.73475	0.000103472896097847\\
18.75575	0.00010416844789302\\
18.77475	0.00010413742863865\\
18.79575	0.000103958073452271\\
18.81475	0.000103110000419962\\
18.83575	0.000156045092822953\\
18.85475	9.47996948198489e-005\\
18.87575	0.000145182001605625\\
18.89475	0.000137366592730967\\
18.91575	0.000129498111249743\\
18.93475	0.00012145761872929\\
18.95575	0.000113739481036177\\
18.97475	0.000160489412233797\\
18.99575	0.000148193196450503\\
19.01475	0.000136525842448309\\
19.03575	0.000178766173716407\\
19.05475	0.000161585341446999\\
19.07675	0.000199066981478658\\
19.09475	0.000178983395454304\\
19.11575	0.000212441851695122\\
19.13475	0.000186768425676548\\
19.15575	0.000215498173201873\\
19.17475	0.000239829563440786\\
19.19475	0.000205911709603899\\
19.21475	0.000173626835664599\\
19.23575	0.000196707752317623\\
19.25475	0.000216711778462161\\
19.27575	0.000234744984262439\\
19.29475	0.000195411904403792\\
19.31575	0.000211809456329823\\
19.33475	0.000170850682231891\\
19.35575	0.000185546386560991\\
19.37475	0.000198055708910805\\
19.39575	0.000208460087215088\\
19.41475	0.000216108559000154\\
19.43575	0.000221420403114133\\
19.45475	0.000224258372099736\\
19.47575	0.000280009849200626\\
19.49475	0.000220343761873819\\
19.51575	0.000271787609865366\\
19.53475	0.000263072746544118\\
19.55575	0.000253100473197452\\
19.57475	0.000242195313246038\\
19.59575	0.000231085902890073\\
19.61475	0.000219482171548184\\
19.63575	0.000207828074479311\\
19.65475	0.000196414982370561\\
19.67575	0.000240351855142882\\
19.69475	0.00017155976310913\\
19.71575	0.000216374050791287\\
19.73475	0.000202830170275917\\
19.75575	0.0001896611765715\\
19.77475	0.000176772508482283\\
19.79475	0.000219071909941501\\
19.81475	0.000202789755837216\\
19.83575	0.000186362957199625\\
19.85475	0.000170575165702851\\
19.87675	0.000209609688790494\\
19.89475	0.000190356703443594\\
19.91575	0.000171345638541626\\
19.93475	0.000152614355818254\\
19.95575	0.000190513274290574\\
19.97475	0.000170664059305082\\
19.99575	0.000151778553094401\\
20.01475	0.00018884627678314\\
20.03575	0.000168460591376587\\
20.05475	0.000203824740058068\\
20.07575	0.000181225381844853\\
20.09475	0.00021346458228552\\
20.11575	0.000242092464658585\\
20.13475	0.000212545360428574\\
20.15575	0.000238467057847721\\
20.17475	0.000261668199016451\\
20.19575	0.000281404498590173\\
20.21475	0.000298116364021929\\
20.23575	0.000311775231062629\\
20.25475	0.000268079785173067\\
20.27575	0.000281479304938843\\
20.29475	0.000293739559723031\\
20.31575	0.000249976644582686\\
20.33475	0.000264141318296908\\
20.35575	0.00022287791669829\\
20.37375	0.000240297688786853\\
20.39575	0.000202954340510173\\
20.41475	0.000169100379777368\\
20.43575	0.000139346391885845\\
20.45475	0.000114012608248193\\
20.47575	0.000146760932610937\\
20.49475	0.000123891348687706\\
20.51575	0.000157833986359368\\
20.53475	0.000135234277901246\\
20.55575	0.000168931650207679\\
20.57475	0.000145456012905293\\
20.59575	0.000178034123625524\\
20.61475	0.000207947531581254\\
20.63575	0.000180296796244209\\
20.65475	0.000208725588950877\\
20.67675	0.000180164515865429\\
20.69475	0.000207613150310976\\
20.71575	0.000178826132460596\\
20.73475	0.000206394806739372\\
20.75575	0.000177351840829856\\
20.77475	0.000205045184596794\\
20.79575	0.000175870539139901\\
20.81475	0.000149045128803231\\
20.83575	0.000179677506497503\\
20.85475	0.000153823623496335\\
20.87575	0.000129509197155498\\
20.89475	0.000162650997237538\\
20.91575	0.000139499616324293\\
20.93475	0.000117973332619665\\
20.95575	0.000153207511376281\\
20.97475	0.000131422597183889\\
20.99575	0.000165772966914678\\
21.01475	0.000143159225568226\\
21.03575	0.000176460153428563\\
21.05475	0.000152439214544678\\
21.07475	0.000184277459875875\\
21.09475	0.000213870001440391\\
21.11575	0.000185621054223504\\
21.13475	0.000213399394618341\\
21.15575	0.000238809325241488\\
21.17475	0.000206791240161033\\
21.19575	0.000231322281527504\\
21.21475	0.000199230184385807\\
21.23575	0.000224007310791197\\
21.25475	0.000192583995081866\\
21.27575	0.000218874832882437\\
21.29475	0.000189140749565729\\
21.31575	0.00021656474076186\\
21.33475	0.000188009608997846\\
21.35575	0.00021611271515523\\
21.37475	0.000242693640191033\\
21.39575	0.000212663006357638\\
21.41475	0.000239276397939826\\
21.43575	0.000209413589663218\\
21.45475	0.000236616321719782\\
21.47675	0.00026249628043709\\
21.49475	0.000231343119116752\\
21.51575	0.000257144862923472\\
21.53475	0.000226412810939525\\
21.55575	0.000252444087222107\\
21.57475	0.000223215078374463\\
21.59575	0.000196889244742934\\
21.61475	0.000228592282396056\\
21.63575	0.000205204071818652\\
21.65475	0.000239137387774189\\
21.67475	0.000216897459293267\\
21.69475	0.000197016314301427\\
21.71575	0.000234047493576136\\
21.73475	0.000214394765237465\\
21.75575	0.000251711006411729\\
21.77475	0.000232025308883176\\
21.79575	0.000213854545619217\\
21.81475	0.000197921504591513\\
21.83575	0.00023824740353747\\
21.85475	0.00022141606174591\\
21.87575	0.000205892579278788\\
21.89475	0.000191968463855719\\
21.91575	0.000179503794228235\\
21.93475	0.000223780138030989\\
21.95575	0.000211187234836844\\
21.97475	0.000200127874383302\\
21.99575	0.00019101971103876\\
22.01475	0.000237355766527757\\
22.03575	0.000226523901458445\\
22.05475	0.00021632819588357\\
22.07575	0.000261245269792922\\
22.09475	0.000248670920802785\\
22.11575	0.000236316339638863\\
22.13475	0.000225684513722512\\
22.15575	0.00021630379738153\\
22.17475	0.000208438405239405\\
22.19575	0.000201699915133256\\
22.21475	0.000195959904441023\\
22.23575	0.000192385636814987\\
22.25475	0.000190188462646221\\
22.27675	0.000189167754282874\\
22.29475	0.000189954794220235\\
22.31475	0.000190952136442243\\
22.33475	0.000192588153893189\\
22.35575	0.000194394241098883\\
22.37475	0.000250286533749419\\
22.39575	0.000246365158156713\\
22.41475	0.000241103454821936\\
22.43575	0.000234761619784857\\
22.45475	0.000228183253633264\\
22.47575	0.000276217080281835\\
22.49475	0.000265262439811986\\
22.51575	0.000254152942151582\\
22.53475	0.000243492054893197\\
22.55575	0.000233334971964195\\
22.57475	0.000224167928728239\\
22.59575	0.000215954831652511\\
22.61475	0.000208591211532898\\
22.63575	0.000257229068905332\\
22.65475	0.000192994173871644\\
22.67575	0.000242308718205364\\
22.69475	0.000233416908895425\\
22.71575	0.000170583430487801\\
22.73475	0.000222145167290198\\
22.75575	0.000215971988977255\\
22.77475	0.000209710857995611\\
22.79575	0.000203751601375787\\
22.81475	0.000198502602097025\\
22.83575	0.000193245809975015\\
22.85475	0.00018844196637181\\
22.87575	0.000237598828250038\\
22.89475	0.000226942608501126\\
22.91575	0.000214979332555182\\
22.93475	0.000256744710391676\\
22.95575	0.000239647523004314\\
22.97475	0.000276053655422377\\
22.99575	0.000253509027386032\\
23.01475	0.000231168921507177\\
23.03575	0.000264377773481706\\
23.05475	0.000239921997780373\\
23.07675	0.000270609964706699\\
23.09475	0.000243249822356845\\
23.11575	0.000216452810368938\\
23.13475	0.000191470946380908\\
23.15575	0.000222760642102082\\
23.17475	0.000197213830231887\\
23.19575	0.000228101311408741\\
23.21475	0.000201904376462989\\
23.23575	0.000231582821176221\\
23.25475	0.000203872024055407\\
23.27575	0.00023194804152362\\
23.29475	0.000202114923804683\\
23.31575	0.00022720410584744\\
23.33475	0.000194058706519441\\
23.35575	0.000216107720357749\\
23.37475	0.000235463210048986\\
23.39575	0.000251598525580853\\
23.41475	0.000209841527182191\\
23.43575	0.00022434206180595\\
23.45475	0.000236964690925552\\
23.47575	0.000247440500643898\\
23.49475	0.000255856127444246\\
23.51575	0.00026232001843073\\
23.53475	0.000267154529381522\\
23.55575	0.000215548189723632\\
23.57475	0.000221265524267546\\
23.59575	0.000226280295097209\\
23.61475	0.000230144829053074\\
23.63575	0.00017783508121068\\
23.65475	0.000183937455563007\\
23.67575	0.000190369586535267\\
23.69475	0.000196504411144438\\
23.71575	0.000201891759922235\\
23.73475	0.0002064723695886\\
23.75575	0.000209980108408267\\
23.77475	0.000212684248447769\\
23.79575	0.000214818723683597\\
23.81475	0.000215855266791817\\
23.83575	0.000216108120234552\\
23.85475	0.000215786082636448\\
23.87675	0.000214768454133312\\
23.89475	0.000213483368238976\\
23.91575	0.000212095245807931\\
23.93475	0.000210441312906015\\
23.95575	0.000208879873235987\\
23.97475	0.000152883226538262\\
23.99575	0.000155948940401797\\
24.01475	0.000159661724401404\\
24.03575	0.000218093666323543\\
24.05475	0.000216439900345395\\
24.07575	0.00021354006772121\\
24.09475	0.000209420417135104\\
24.11575	0.000204157464881783\\
24.13475	0.000252935616211249\\
24.15575	0.000242277131793194\\
24.17475	0.000230725390263073\\
24.19575	0.000219332964278798\\
24.21475	0.000208493248651268\\
24.23575	0.000198353098706034\\
24.25475	0.000189298911450919\\
24.27575	0.000180646678792446\\
24.29475	0.000227433796322943\\
24.31575	0.00021615710870209\\
24.33475	0.000205228275291937\\
24.35575	0.000195027731655012\\
24.37475	0.000185258998752854\\
24.39475	0.000230522167347899\\
24.41475	0.000217006234827374\\
24.43575	0.000203277858894543\\
24.45475	0.000190590298008012\\
24.47575	0.000233422417774425\\
24.49475	0.000218159969870092\\
24.51575	0.000203231414165272\\
24.53475	0.000189036217589933\\
24.55575	0.000231268588959185\\
24.57475	0.000216429382752762\\
24.59575	0.00020291544451052\\
24.61475	0.000191133706504771\\
24.63575	0.000181080987497673\\
24.65475	0.000227154807122313\\
24.67675	0.000215763953263466\\
24.69475	0.00020489486163263\\
24.71575	0.000194895620779987\\
24.73475	0.000186001528994615\\
24.75575	0.000178180541625797\\
24.77475	0.00017153333151946\\
24.79575	0.000165887855275912\\
24.81475	0.000161131993591251\\
24.83575	0.00015712377089271\\
24.85475	0.00015386634917836\\
24.87575	0.000151180338528303\\
24.89475	0.000203583309056724\\
24.91575	0.000197258973557821\\
24.93475	0.000190466073123681\\
24.95575	0.00018380332909307\\
24.97475	0.000177566900243765\\
24.99475	0.000226423772564249\\
25.01475	0.000216798579558817\\
25.03575	0.000207073293189764\\
25.05475	0.000197767688897086\\
25.07575	0.000243846495647515\\
25.09475	0.000231549028019551\\
25.11575	0.000219463295766427\\
25.13475	0.000208109102802369\\
25.15575	0.000197792047366132\\
25.17475	0.000188840969634161\\
25.19575	0.0001815219900761\\
25.21475	0.000175687725662253\\
25.23575	0.000171105403834118\\
25.25475	0.000167583592006945\\
25.27575	0.000165636372985633\\
25.29475	0.000165150359978501\\
25.31575	0.000165922597659189\\
25.33475	0.000167588555155158\\
25.35575	0.000169937700229588\\
25.37475	0.000172673943945102\\
25.39575	0.000175572491880776\\
25.41475	0.000178390632206557\\
25.43575	0.000181170027742722\\
25.45475	0.000183771790008534\\
25.47675	0.000131536802156232\\
25.49475	0.000138124614449358\\
25.51575	0.000145302518373199\\
25.53475	0.000152433183958971\\
25.55575	0.00015952128359555\\
25.57475	0.000166336335390712\\
25.59475	0.00011831467281632\\
25.61475	0.000129015829096716\\
25.63575	0.000140013978060848\\
25.65475	0.000150521118827053\\
25.67575	0.00016042788384137\\
25.69475	0.000169324435112878\\
25.71575	0.000177290044396214\\
25.73475	0.000184231505475735\\
25.75575	0.000190270916397728\\
25.77475	0.000195429815130243\\
25.79575	0.000145231202672602\\
25.81475	0.000153383663188921\\
25.83575	0.000162078847861697\\
25.85475	0.000116190415677638\\
25.87575	0.000129942863720484\\
25.89475	0.000144624160735533\\
25.91575	0.000159461208815928\\
25.93475	0.000174159407875379\\
25.95575	0.000133441055837985\\
25.97475	0.000150709277799791\\
25.99575	0.000167680724473871\\
26.01475	0.000128932415157833\\
26.03575	0.000147955844398967\\
26.05475	0.000166138354083032\\
26.07575	0.000182857429547216\\
26.09475	0.000198003222853591\\
26.11575	0.000156653749498957\\
26.13475	0.000172833508721239\\
26.15575	0.00018800520958831\\
26.17475	0.000201950876388199\\
26.19575	0.000159880189233791\\
26.21475	0.000175344345555048\\
26.23575	0.000190010483055595\\
26.25375	0.000203483623536344\\
26.27675	0.00016108094757452\\
26.29475	0.000176243110791472\\
26.31575	0.000190796860756595\\
26.33475	0.000149551985870813\\
26.35575	0.000166086285417512\\
26.37475	0.000182019051168753\\
26.39575	0.000142103736172365\\
26.41475	0.000159883607882614\\
26.43575	0.000122052536458706\\
26.45475	0.00014223286717312\\
26.47575	0.000161561533293064\\
26.49475	0.000124777630046904\\
26.51575	0.000145213926314107\\
26.53475	0.000164746314897575\\
26.55575	0.000182354841789755\\
26.57475	0.000197456089861351\\
26.59575	0.000209918315191429\\
26.61475	0.00021971118836089\\
26.63575	0.000227218352947449\\
26.65475	0.00017836577288939\\
26.67575	0.000186910955014133\\
26.69475	0.000195091276902437\\
26.71575	0.000202619812228375\\
26.73475	0.000209237996207309\\
26.75575	0.000160371942745059\\
26.77475	0.000169871882321588\\
26.79575	0.000179186219998382\\
26.81475	0.000188165947374191\\
26.83575	0.000196462471292231\\
26.85475	0.000149211242505667\\
26.87575	0.000160065953506921\\
26.89475	0.00017099697602886\\
26.91575	0.000181228238418381\\
26.93375	0.000190582920269567\\
26.95575	0.000198965427148567\\
26.97475	0.000206187302148751\\
26.99575	0.000212474152018237\\
27.01475	0.000217917747711562\\
27.03575	0.000222086725177043\\
27.05475	0.000225213378356696\\
27.07675	0.000227367328394788\\
27.09475	0.00017392413376152\\
27.11575	0.000179546172729963\\
27.13475	0.000185564011960927\\
27.15575	0.000191526549076933\\
27.17475	0.000197523439569247\\
27.19575	0.000148723195693874\\
27.21475	0.000158578693315843\\
27.23575	0.000168917704475509\\
27.25475	0.000178821699153526\\
27.27575	0.00018802056837243\\
27.29475	0.00019653361633675\\
27.31575	0.000204033721571887\\
27.33475	0.000210100194100872\\
27.35575	0.000215121445794712\\
27.37475	0.000218624656987141\\
27.39575	0.000220508933584755\\
27.41475	0.000221493924840283\\
27.43575	0.000166872668689337\\
27.45475	0.000170589753919694\\
27.47575	0.00017495513831717\\
27.49475	0.000179062737105417\\
27.51575	0.000182709978570704\\
27.53475	0.000186283040118021\\
27.55475	0.000189340632532495\\
27.57475	0.000137133505954734\\
27.59575	0.000144033066320089\\
27.61475	0.000151135938460161\\
27.63575	0.000157962762378236\\
27.65475	0.000164756496348644\\
27.67575	0.000170657139275123\\
27.69475	0.000175787235986545\\
27.71575	0.000180498529920337\\
27.73475	0.000184391428331574\\
27.75575	0.000187735906187175\\
27.77475	0.000190638644616823\\
27.79575	0.000138291066548205\\
27.81475	0.000144663543924947\\
27.83575	0.000151637199377603\\
27.85475	0.000158512142069702\\
27.87675	0.000164783773482041\\
27.89475	0.000170637410445563\\
27.91575	0.000175699955173041\\
27.93475	0.000180118049994208\\
27.95575	0.000184172796146598\\
27.97475	0.000187325044120962\\
27.99575	0.000189970658856056\\
28.01475	0.000192282767293506\\
28.03575	0.000194051461658344\\
28.05475	0.000195667243675294\\
28.07575	0.000196885119058367\\
28.09475	0.000143234539779053\\
28.11575	0.000203110636180191\\
28.13475	0.00020450914826674\\
28.15575	0.000205253557285947\\
28.17375	0.000150842546151276\\
28.19575	0.000155276085941031\\
28.21475	0.000160197402883541\\
28.23575	0.000165453916625822\\
28.25475	0.000170317307562753\\
28.27575	0.000174722307215387\\
28.29475	0.000178703823764561\\
28.31575	0.000127685037432282\\
28.33475	0.000135384922018203\\
28.35575	0.00014326642854704\\
28.37475	0.000150884482518246\\
28.39575	0.000157950566316765\\
28.41475	0.000164353637915244\\
28.43575	0.000170198070476181\\
28.45475	0.00012045193677022\\
28.47575	0.000129160548354681\\
28.49475	0.000137978908058602\\
28.51575	0.000146404748956431\\
28.53475	0.000154342500076101\\
28.55575	0.00016127593087175\\
28.57475	0.000167380614369936\\
28.59575	0.000172604565019795\\
28.61475	0.000177041385160546\\
28.63575	0.000181000511334822\\
28.65475	0.000184058019554032\\
28.67675	0.000186615246570491\\
28.69475	0.000188750622095997\\
28.71575	0.000190411943000365\\
28.73475	0.000137173975612607\\
28.75475	0.000143054013563938\\
28.77475	0.000149173293637499\\
28.79575	0.000155336226264226\\
28.81475	0.000161225413921467\\
28.83575	0.000166466572145747\\
28.85475	0.00017140984517048\\
28.87575	0.000175625215517395\\
28.89475	0.000179093761647362\\
28.91575	0.000182124635451346\\
28.93475	0.00013014708728564\\
28.95575	0.00013674771125981\\
28.97475	0.000143987868905683\\
28.99575	0.000150954884073254\\
29.01475	0.000157333635874631\\
29.03575	0.000163230787569996\\
29.05475	0.000168495673506716\\
29.07575	0.000118782674588551\\
29.09475	0.00012765776732567\\
29.11575	0.000136650600888105\\
29.13475	0.000145111536624864\\
29.15575	0.000152954633302604\\
29.17475	0.0001600470852776\\
29.19575	0.00016598195615207\\
29.21475	0.000171194651859723\\
29.23575	0.000121028645349101\\
29.25475	0.000129145146063834\\
29.27575	0.000137689085153294\\
29.29475	0.000145680355192586\\
29.31575	0.000152977346250044\\
29.33475	0.000105078912088598\\
29.35475	0.000115419016831818\\
29.37475	0.000125820723104894\\
29.39575	0.000135721768861912\\
29.41475	0.000144683255169394\\
29.43575	0.000152655209642674\\
29.45475	0.000105164627006864\\
29.47675	0.000115754577533827\\
29.49475	0.000126307359567736\\
29.51575	0.000136287665384323\\
29.53475	0.000145227253135042\\
29.55575	0.000153210250685149\\
29.57475	0.000160341777962508\\
29.59575	0.000166394184662964\\
29.61475	0.000171517771272874\\
29.63575	0.000175983799062593\\
29.65475	0.000179565790928621\\
29.67575	0.00012795111154735\\
29.69475	0.000135152400241125\\
29.71575	0.000142533609233218\\
29.73475	0.000149732504580398\\
29.75575	0.000156597371006549\\
29.77475	0.000108015468657101\\
29.79575	0.0001179545191675\\
29.81475	0.000128149629492222\\
29.83575	0.000137680888800094\\
29.85475	0.000146510143418333\\
29.87575	0.000154519957650623\\
29.89475	0.000161364498801439\\
29.91575	0.000167096331183238\\
29.93475	0.000171656683070856\\
29.95575	0.00017492863679831\\
29.97475	0.000177000991160815\\
};
\addlegendentry{$\kappa{}_\mathrm{\text{do}}$};

\end{axis}

\begin{axis}[%
width=0.95092\figurewidth,
height=0.264706\figureheight,
at={(0\figurewidth,0.735294\figureheight)},
scale only axis,
every outer x axis line/.append style={black},
every x tick label/.append style={font=\color{black}},
xmin=8,
xmax=26,
xmajorgrids,
every outer y axis line/.append style={black},
every y tick label/.append style={font=\color{black}},
ymin=-2,
ymax=6,
ylabel={$\tau_{\delta,\mathrm{drv}}\text{ [Nm]}$},
ylabel near ticks,
ymajorgrids,
axis x line*=bottom,
axis y line*=left
]

\addplot [color=black,solid,forget plot, line width=1.0]
  table[row sep=crcr]{%
3.93	0.0199999995529652\\
3.94	0.0199999995529652\\
3.95	0.00499999988824129\\
3.96	-0.0450000017881393\\
3.97	-0.0949999988079071\\
3.98	-0.0900000035762787\\
3.99	-0.0199999995529652\\
4	0.00499999988824129\\
4.01	0.0350000001490116\\
4.02	0.0549999997019768\\
4.03	0.0350000001490116\\
4.04	2.08166817117217e-016\\
4.05	-0.0199999995529652\\
4.06	-0.0500000007450581\\
4.07	-0.0299999993294477\\
4.08	-0.0350000001490116\\
4.09	-0.00499999988824129\\
4.1	0.0399999991059303\\
4.11	0.0850000008940697\\
4.12	0.109999999403954\\
4.13	0.0799999982118607\\
4.14	0.0199999995529652\\
4.15	-0.0149999996647239\\
4.16	-0.0500000007450581\\
4.17	-0.0750000029802322\\
4.18	-0.0500000007450581\\
4.19	-0.0649999976158142\\
4.2	2.08166817117217e-016\\
4.21	0.0649999976158142\\
4.22	0.0500000007450581\\
4.23	0.0399999991059303\\
4.24	0.0199999995529652\\
4.25	-0.0299999993294477\\
4.26	-0.0649999976158142\\
4.27	-0.0500000007450581\\
4.28	0.00999999977648258\\
4.29	0.0199999995529652\\
4.3	0.0850000008940697\\
4.31	0.0850000008940697\\
4.32	0.0649999976158142\\
4.33	0.144999995827675\\
4.34	0.0949999988079071\\
4.35	2.08166817117217e-016\\
4.36	-0.00499999988824129\\
4.37	-0.0799999982118607\\
4.38	-0.115000002086163\\
4.39	0.0199999995529652\\
4.4	0.0850000008940697\\
4.41	0.144999995827675\\
4.42	0.140000000596046\\
4.43	0.170000001788139\\
4.44	0.125\\
4.45	0.0350000001490116\\
4.46	0.00499999988824129\\
4.47	-0.0649999976158142\\
4.48	-0.119999997317791\\
4.49	-0.0900000035762787\\
4.5	-0.0500000007450581\\
4.51	-0.00499999988824129\\
4.52	0.0799999982118607\\
4.53	0.0350000001490116\\
4.54	-0.0450000017881393\\
4.55	-0.0350000001490116\\
4.56	-0.0649999976158142\\
4.57	-0.0149999996647239\\
4.58	-0.0149999996647239\\
4.59	-0.0299999993294477\\
4.6	0.00999999977648258\\
4.61	0.0500000007450581\\
4.62	2.08166817117217e-016\\
4.63	2.08166817117217e-016\\
4.64	-0.0350000001490116\\
4.65	-0.0199999995529652\\
4.66	2.08166817117217e-016\\
4.67	-0.00499999988824129\\
4.68	0.0350000001490116\\
4.69	0.00999999977648258\\
4.7	2.08166817117217e-016\\
4.71	0.00999999977648258\\
4.72	0.00499999988824129\\
4.73	-0.0149999996647239\\
4.74	-0.00499999988824129\\
4.75	-0.0149999996647239\\
4.76	0.00499999988824129\\
4.77	0.00999999977648258\\
4.78	0.00499999988824129\\
4.79	0.00499999988824129\\
4.8	0.00499999988824129\\
4.81	0.00499999988824129\\
4.82	-0.0199999995529652\\
4.83	-0.00499999988824129\\
4.84	-0.00499999988824129\\
4.85	-0.0149999996647239\\
4.86	0.0350000001490116\\
4.87	0.0399999991059303\\
4.88	0.0199999995529652\\
4.89	-0.00499999988824129\\
4.9	-0.00499999988824129\\
4.91	-0.00499999988824129\\
4.92	0.00999999977648258\\
4.93	0.00999999977648258\\
4.94	-0.00499999988824129\\
4.95	0.00499999988824129\\
4.96	-0.0149999996647239\\
4.97	0.00499999988824129\\
4.98	0.0549999997019768\\
4.99	0.0700000002980232\\
5	0.0799999982118607\\
5.01	0.0850000008940697\\
5.02	0.0199999995529652\\
5.03	-0.00499999988824129\\
5.04	0.0199999995529652\\
5.05	-0.0450000017881393\\
5.06	0.025000000372529\\
5.07	0.0649999976158142\\
5.08	-0.0299999993294477\\
5.09	0.00499999988824129\\
5.1	0.0399999991059303\\
5.11	-0.00499999988824129\\
5.12	-0.0149999996647239\\
5.13	2.08166817117217e-016\\
5.14	0.00999999977648258\\
5.15	0.00499999988824129\\
5.16	0.025000000372529\\
5.17	2.08166817117217e-016\\
5.18	-0.0299999993294477\\
5.19	-0.00499999988824129\\
5.2	-0.0199999995529652\\
5.21	-0.0149999996647239\\
5.22	0.0500000007450581\\
5.23	2.08166817117217e-016\\
5.24	2.08166817117217e-016\\
5.25	0.00499999988824129\\
5.26	0.0199999995529652\\
5.27	0.00999999977648258\\
5.28	0.100000001490116\\
5.29	0.0649999976158142\\
5.3	2.08166817117217e-016\\
5.31	-0.0199999995529652\\
5.32	-0.00499999988824129\\
5.33	0.00499999988824129\\
5.34	0.025000000372529\\
5.35	0.0500000007450581\\
5.36	0.0500000007450581\\
5.37	0.0399999991059303\\
5.38	0.00999999977648258\\
5.39	0.00499999988824129\\
5.4	0.00499999988824129\\
5.41	-0.0149999996647239\\
5.42	-0.00499999988824129\\
5.43	-0.0350000001490116\\
5.44	-0.0350000001490116\\
5.45	-0.0149999996647239\\
5.46	-0.00499999988824129\\
5.47	0.0350000001490116\\
5.48	0.025000000372529\\
5.49	-0.0750000029802322\\
5.5	-0.174999997019768\\
5.51	-0.125\\
5.52	-0.104999996721745\\
5.53	-0.0599999986588955\\
5.54	-0.00499999988824129\\
5.55	0.0350000001490116\\
5.56	0.0199999995529652\\
5.57	0.0350000001490116\\
5.58	-0.0350000001490116\\
5.59	-0.115000002086163\\
5.6	-0.125\\
5.61	-0.0949999988079071\\
5.62	-0.0350000001490116\\
5.63	-0.0299999993294477\\
5.64	0.025000000372529\\
5.65	2.08166817117217e-016\\
5.66	0.0199999995529652\\
5.67	-0.00499999988824129\\
5.68	-0.0450000017881393\\
5.69	-0.0649999976158142\\
5.7	-0.0649999976158142\\
5.71	-0.0649999976158142\\
5.72	-0.0599999986588955\\
5.73	-0.0350000001490116\\
5.74	-0.0299999993294477\\
5.75	-0.0199999995529652\\
5.76	-0.0149999996647239\\
5.77	-0.0649999976158142\\
5.78	-0.104999996721745\\
5.79	-0.135000005364418\\
5.8	-0.115000002086163\\
5.81	-0.0949999988079071\\
5.82	-0.0350000001490116\\
5.83	-0.0149999996647239\\
5.84	-0.0199999995529652\\
5.85	0.00499999988824129\\
5.86	-0.0799999982118607\\
5.87	-0.144999995827675\\
5.88	-0.150000005960464\\
5.89	-0.0949999988079071\\
5.9	-0.0750000029802322\\
5.91	-0.0299999993294477\\
5.92	2.08166817117217e-016\\
5.93	-0.0350000001490116\\
5.94	-0.0599999986588955\\
5.95	-0.0599999986588955\\
5.96	-0.0799999982118607\\
5.97	-0.150000005960464\\
5.98	-0.159999996423721\\
5.99	-0.115000002086163\\
6	-0.0450000017881393\\
6.01	-0.0199999995529652\\
6.02	0.00999999977648258\\
6.03	-0.0350000001490116\\
6.04	-0.0900000035762787\\
6.05	-0.104999996721745\\
6.06	-0.144999995827675\\
6.07	-0.209999993443489\\
6.08	-0.174999997019768\\
6.09	-0.144999995827675\\
6.1	-0.0949999988079071\\
6.11	-0.0500000007450581\\
6.12	-0.0199999995529652\\
6.13	-0.0149999996647239\\
6.14	-0.0450000017881393\\
6.15	-0.0799999982118607\\
6.16	-0.115000002086163\\
6.17	-0.135000005364418\\
6.18	-0.104999996721745\\
6.19	-0.115000002086163\\
6.2	-0.104999996721745\\
6.21	-0.0450000017881393\\
6.22	-0.0599999986588955\\
6.23	-0.0649999976158142\\
6.24	-0.0599999986588955\\
6.25	-0.119999997317791\\
6.26	-0.135000005364418\\
6.27	-0.125\\
6.28	-0.125\\
6.29	-0.115000002086163\\
6.3	-0.104999996721745\\
6.31	-0.104999996721745\\
6.32	-0.0900000035762787\\
6.33	-0.0900000035762787\\
6.34	-0.0949999988079071\\
6.35	-0.119999997317791\\
6.36	-0.119999997317791\\
6.37	-0.115000002086163\\
6.38	-0.104999996721745\\
6.39	-0.104999996721745\\
6.4	-0.0799999982118607\\
6.41	-0.144999995827675\\
6.42	-0.150000005960464\\
6.43	-0.119999997317791\\
6.44	-0.125\\
6.45	-0.135000005364418\\
6.46	-0.119999997317791\\
6.47	-0.0949999988079071\\
6.48	-0.0799999982118607\\
6.49	-0.0799999982118607\\
6.5	-0.0900000035762787\\
6.51	-0.0949999988079071\\
6.52	-0.0799999982118607\\
6.53	-0.119999997317791\\
6.54	-0.174999997019768\\
6.55	-0.215000003576279\\
6.56	-0.209999993443489\\
6.57	-0.135000005364418\\
6.58	-0.0949999988079071\\
6.59	-0.0199999995529652\\
6.6	-0.0149999996647239\\
6.61	-0.0599999986588955\\
6.62	-0.0799999982118607\\
6.63	-0.0949999988079071\\
6.64	-0.135000005364418\\
6.65	-0.135000005364418\\
6.66	-0.115000002086163\\
6.67	-0.0799999982118607\\
6.68	-0.0649999976158142\\
6.69	-0.0750000029802322\\
6.7	-0.0649999976158142\\
6.71	-0.0750000029802322\\
6.72	-0.0900000035762787\\
6.73	-0.115000002086163\\
6.74	-0.115000002086163\\
6.75	-0.115000002086163\\
6.76	-0.125\\
6.77	-0.150000005960464\\
6.78	-0.0799999982118607\\
6.79	-0.0450000017881393\\
6.8	-0.0450000017881393\\
6.81	0.00499999988824129\\
6.82	-0.0299999993294477\\
6.83	-0.144999995827675\\
6.84	-0.174999997019768\\
6.85	-0.150000005960464\\
6.86	-0.0900000035762787\\
6.87	-0.0799999982118607\\
6.88	-0.0649999976158142\\
6.89	-0.0649999976158142\\
6.9	-0.0450000017881393\\
6.91	-0.0450000017881393\\
6.92	-0.0900000035762787\\
6.93	-0.119999997317791\\
6.94	-0.115000002086163\\
6.95	-0.104999996721745\\
6.96	-0.0900000035762787\\
6.97	-0.0450000017881393\\
6.98	-0.0500000007450581\\
6.99	-0.0649999976158142\\
7	-0.0799999982118607\\
7.01	-0.0750000029802322\\
7.02	-0.0900000035762787\\
7.03	-0.0649999976158142\\
7.04	-0.0450000017881393\\
7.05	-0.0799999982118607\\
7.06	-0.104999996721745\\
7.07	-0.0799999982118607\\
7.08	-0.115000002086163\\
7.09	-0.119999997317791\\
7.1	-0.0949999988079071\\
7.11	-0.0949999988079071\\
7.12	-0.144999995827675\\
7.13	-0.0949999988079071\\
7.14	-0.0350000001490116\\
7.15	-0.0500000007450581\\
7.16	-0.0949999988079071\\
7.17	-0.104999996721745\\
7.18	-0.159999996423721\\
7.19	-0.0599999986588955\\
7.2	-0.0500000007450581\\
7.21	-0.0649999976158142\\
7.22	-0.0750000029802322\\
7.23	-0.0799999982118607\\
7.24	-0.0900000035762787\\
7.25	-0.0799999982118607\\
7.26	-0.0799999982118607\\
7.27	-0.0799999982118607\\
7.28	-0.0799999982118607\\
7.29	-0.0799999982118607\\
7.3	-0.0949999988079071\\
7.31	-0.0949999988079071\\
7.32	-0.0799999982118607\\
7.33	-0.0949999988079071\\
7.34	-0.104999996721745\\
7.35	-0.0799999982118607\\
7.36	-0.0750000029802322\\
7.37	-0.0799999982118607\\
7.38	-0.0649999976158142\\
7.39	-0.0900000035762787\\
7.4	-0.159999996423721\\
7.41	-0.115000002086163\\
7.42	-0.0649999976158142\\
7.43	-0.0450000017881393\\
7.44	-0.0799999982118607\\
7.45	-0.0900000035762787\\
7.46	-0.0949999988079071\\
7.47	-0.0799999982118607\\
7.48	-0.0599999986588955\\
7.49	-0.0649999976158142\\
7.5	-0.0900000035762787\\
7.51	-0.0949999988079071\\
7.52	-0.0799999982118607\\
7.53	-0.0799999982118607\\
7.54	-0.0649999976158142\\
7.55	-0.0500000007450581\\
7.56	-0.0649999976158142\\
7.57	-0.0649999976158142\\
7.58	-0.0649999976158142\\
7.59	-0.104999996721745\\
7.6	-0.104999996721745\\
7.61	-0.104999996721745\\
7.62	-0.0799999982118607\\
7.63	-0.115000002086163\\
7.64	-0.0949999988079071\\
7.65	-0.0949999988079071\\
7.66	-0.0900000035762787\\
7.67	-0.0900000035762787\\
7.68	-0.0949999988079071\\
7.69	-0.135000005364418\\
7.7	-0.135000005364418\\
7.71	-0.125\\
7.72	-0.115000002086163\\
7.73	-0.125\\
7.74	-0.115000002086163\\
7.75	-0.115000002086163\\
7.76	-0.115000002086163\\
7.77	-0.0949999988079071\\
7.78	-0.115000002086163\\
7.79	-0.135000005364418\\
7.8	-0.0900000035762787\\
7.81	-0.0599999986588955\\
7.82	-0.0949999988079071\\
7.83	-0.0750000029802322\\
7.84	-0.0799999982118607\\
7.85	-0.104999996721745\\
7.86	-0.104999996721745\\
7.87	-0.104999996721745\\
7.88	-0.150000005960464\\
7.89	-0.144999995827675\\
7.9	-0.0799999982118607\\
7.91	2.08166817117217e-016\\
7.92	-0.0199999995529652\\
7.93	-0.0500000007450581\\
7.94	-0.0949999988079071\\
7.95	-0.0799999982118607\\
7.96	-0.0500000007450581\\
7.97	-0.0799999982118607\\
7.98	-0.0799999982118607\\
7.99	-0.0649999976158142\\
8	-0.0649999976158142\\
8.01	-0.0500000007450581\\
8.02	-0.0649999976158142\\
8.03	-0.0649999976158142\\
8.04	-0.0750000029802322\\
8.05	-0.0900000035762787\\
8.06	-0.104999996721745\\
8.07	-0.115000002086163\\
8.08	-0.0799999982118607\\
8.09	-0.0799999982118607\\
8.1	-0.0500000007450581\\
8.11	-0.0649999976158142\\
8.12	-0.0299999993294477\\
8.13	-0.0750000029802322\\
8.14	-0.115000002086163\\
8.15	-0.0500000007450581\\
8.16	-0.189999997615814\\
8.17	-0.150000005960464\\
8.18	-0.00499999988824129\\
8.19	0.0500000007450581\\
8.2	0.00999999977648258\\
8.21	-0.0350000001490116\\
8.22	-0.239999994635582\\
8.23	-0.119999997317791\\
8.24	0.0500000007450581\\
8.25	0.00499999988824129\\
8.26	-0.104999996721745\\
8.27	-0.115000002086163\\
8.28	-0.0799999982118607\\
8.29	-0.0799999982118607\\
8.3	-0.0450000017881393\\
8.31	0.0649999976158142\\
8.32	-0.0199999995529652\\
8.33	-0.115000002086163\\
8.34	-0.0799999982118607\\
8.35	-0.0799999982118607\\
8.36	-0.165000006556511\\
8.37	-0.0949999988079071\\
8.38	-0.0599999986588955\\
8.39	-0.0450000017881393\\
8.4	-0.0299999993294477\\
8.41	-0.0450000017881393\\
8.42	-0.0450000017881393\\
8.43	-0.0199999995529652\\
8.44	-0.0450000017881393\\
8.45	-0.0799999982118607\\
8.46	-0.0750000029802322\\
8.47	-0.0799999982118607\\
8.48	-0.0450000017881393\\
8.49	-0.0450000017881393\\
8.5	2.08166817117217e-016\\
8.51	-0.00499999988824129\\
8.52	-0.0299999993294477\\
8.53	-0.0500000007450581\\
8.54	-0.104999996721745\\
8.55	-0.115000002086163\\
8.56	-0.115000002086163\\
8.57	-0.0750000029802322\\
8.58	-0.0799999982118607\\
8.59	-0.0649999976158142\\
8.6	-0.0149999996647239\\
8.61	-0.0199999995529652\\
8.62	-0.0450000017881393\\
8.63	-0.0149999996647239\\
8.64	-0.0350000001490116\\
8.65	-0.0350000001490116\\
8.66	-0.0350000001490116\\
8.67	2.08166817117217e-016\\
8.68	0.0350000001490116\\
8.69	0.0399999991059303\\
8.7	0.0350000001490116\\
8.71	-0.00499999988824129\\
8.72	-0.0450000017881393\\
8.73	-0.0649999976158142\\
8.74	-0.0799999982118607\\
8.75	-0.0450000017881393\\
8.76	0.0549999997019768\\
8.77	0.115000002086163\\
8.78	0.140000000596046\\
8.79	0.165000006556511\\
8.8	0.115000002086163\\
8.81	0.0649999976158142\\
8.82	0.0549999997019768\\
8.83	0.0500000007450581\\
8.84	0.109999999403954\\
8.85	0.215000003576279\\
8.86	0.319999992847443\\
8.87	0.365000009536743\\
8.88	0.425000011920929\\
8.89	0.455000013113022\\
8.9	0.485000014305115\\
8.91	0.529999971389771\\
8.92	0.524999976158142\\
8.93	0.514999985694885\\
8.94	0.584999978542328\\
8.95	0.689999997615814\\
8.96	0.764999985694885\\
8.97	0.850000023841858\\
8.98	0.925000011920929\\
8.99	1.02999997138977\\
9	1.17999994754791\\
9.01	1.29499995708466\\
9.02	1.25\\
9.03	1.29499995708466\\
9.04	1.35500001907349\\
9.05	1.42999994754791\\
9.06	1.51499998569489\\
9.07	1.61000001430511\\
9.08	1.72000002861023\\
9.09	1.76499998569489\\
9.1	1.8400000333786\\
9.11	1.88999998569489\\
9.12	1.8400000333786\\
9.13	1.90499997138977\\
9.14	1.94500005245209\\
9.15	1.9650000333786\\
9.16	1.96000003814697\\
9.17	1.98000001907349\\
9.18	1.97500002384186\\
9.19	1.97500002384186\\
9.2	2.0550000667572\\
9.21	2.03500008583069\\
9.22	1.9650000333786\\
9.23	2.01999998092651\\
9.24	2.10500001907349\\
9.25	2.1800000667572\\
9.26	2.28999996185303\\
9.27	2.35500001907349\\
9.28	2.26999998092651\\
9.29	2.28999996185303\\
9.3	2.24499988555908\\
9.31	2.18499994277954\\
9.32	2.40000009536743\\
9.33	2.5550000667572\\
9.34	2.56500005722046\\
9.35	2.50500011444092\\
9.36	2.60999989509583\\
9.37	2.70499992370605\\
9.38	2.8199999332428\\
9.39	2.93499994277954\\
9.4	2.85999989509583\\
9.41	2.75\\
9.42	2.8050000667572\\
9.43	2.875\\
9.44	2.92000007629395\\
9.45	2.96499991416931\\
9.46	2.95499992370605\\
9.47	2.96499991416931\\
9.48	2.92000007629395\\
9.49	2.91000008583069\\
9.5	2.85999989509583\\
9.51	2.98499989509583\\
9.52	3.0699999332428\\
9.53	2.98000001907349\\
9.54	3.00999999046326\\
9.55	3.08999991416931\\
9.56	3.14499998092651\\
9.57	3.26500010490417\\
9.58	3.25\\
9.59	3.08999991416931\\
9.6	3.13499999046326\\
9.61	3.24499988555908\\
9.62	3.36500000953674\\
9.63	3.4300000667572\\
9.64	3.38499999046326\\
9.65	3.35500001907349\\
9.66	3.27999997138977\\
9.67	3.3199999332428\\
9.68	3.42499995231628\\
9.69	3.41499996185303\\
9.7	3.45499992370605\\
9.71	3.51500010490417\\
9.72	3.52999997138977\\
9.73	3.39499998092651\\
9.74	3.38000011444092\\
9.75	3.41499996185303\\
9.76	3.54500007629395\\
9.77	3.60500001907349\\
9.78	3.55999994277954\\
9.79	3.51999998092651\\
9.8	3.55999994277954\\
9.81	3.61500000953674\\
9.82	3.57999992370605\\
9.83	3.67499995231628\\
9.84	3.65000009536743\\
9.85	3.60500001907349\\
9.86	3.69000005722046\\
9.87	3.75500011444092\\
9.88	3.75500011444092\\
9.89	3.65000009536743\\
9.9	3.63000011444092\\
9.91	3.63499999046326\\
9.92	3.66499996185303\\
9.93	3.73499989509583\\
9.94	3.75\\
9.95	3.73499989509583\\
9.96	3.73499989509583\\
9.97	3.65000009536743\\
9.98	3.72000002861023\\
9.99	3.78999996185303\\
10	3.78999996185303\\
10.01	3.76999998092651\\
10.02	3.75\\
10.03	3.79999995231628\\
10.04	3.83999991416931\\
10.05	3.94000005722046\\
10.06	4\\
10.07	3.94000005722046\\
10.08	3.89499998092651\\
10.09	3.94000005722046\\
10.1	4\\
10.11	4.03499984741211\\
10.12	4.09499979019165\\
10.13	4.125\\
10.14	4\\
10.15	4.01999998092651\\
10.16	4.07499980926514\\
10.17	4.1100001335144\\
10.18	4.15500020980835\\
10.19	4.17999982833862\\
10.2	4.15500020980835\\
10.21	4.1399998664856\\
10.22	4.17999982833862\\
10.23	4.23999977111816\\
10.24	4.33500003814697\\
10.25	4.3600001335144\\
10.26	4.28499984741211\\
10.27	4.19999980926514\\
10.28	4.18499994277954\\
10.29	4.17999982833862\\
10.3	4.22499990463257\\
10.31	4.28499984741211\\
10.32	4.24499988555908\\
10.33	4.06500005722046\\
10.34	4.15500020980835\\
10.35	4.24499988555908\\
10.36	4.26000022888184\\
10.37	4.28499984741211\\
10.38	4.30000019073486\\
10.39	4.25500011444092\\
10.4	4.26000022888184\\
10.41	4.30499982833862\\
10.42	4.28499984741211\\
10.43	4.26999998092651\\
10.44	4.28499984741211\\
10.45	4.26999998092651\\
10.46	4.24499988555908\\
10.47	4.33500003814697\\
10.48	4.40500020980835\\
10.49	4.3899998664856\\
10.5	4.40000009536743\\
10.51	4.40000009536743\\
10.52	4.32499980926514\\
10.53	4.42000007629395\\
10.54	4.52500009536743\\
10.55	4.46000003814697\\
10.56	4.42000007629395\\
10.57	4.40500020980835\\
10.58	4.35500001907349\\
10.59	4.42000007629395\\
10.6	4.52500009536743\\
10.61	4.49499988555908\\
10.62	4.40500020980835\\
10.63	4.43499994277954\\
10.64	4.40000009536743\\
10.65	4.36999988555908\\
10.66	4.46500015258789\\
10.67	4.55499982833862\\
10.68	4.52500009536743\\
10.69	4.51999998092651\\
10.7	4.46000003814697\\
10.71	4.48999977111816\\
10.72	4.53999996185303\\
10.73	4.56500005722046\\
10.74	4.53999996185303\\
10.75	4.49499988555908\\
10.76	4.47499990463257\\
10.77	4.55000019073486\\
10.78	4.59499979019165\\
10.79	4.6100001335144\\
10.8	4.57000017166138\\
10.81	4.57000017166138\\
10.82	4.57000017166138\\
10.83	4.6100001335144\\
10.84	4.6399998664856\\
10.85	4.69500017166138\\
10.86	4.74499988555908\\
10.87	4.67000007629395\\
10.88	4.67000007629395\\
10.89	4.65000009536743\\
10.9	4.69500017166138\\
10.91	4.73000001907349\\
10.92	4.69500017166138\\
10.93	4.625\\
10.94	4.51999998092651\\
10.95	4.52500009536743\\
10.96	4.93499994277954\\
10.97	4.83500003814697\\
10.98	4.73999977111816\\
10.99	4.67999982833862\\
11	4.6100001335144\\
11.01	4.67000007629395\\
11.02	4.71000003814697\\
11.03	4.75500011444092\\
11.04	4.77500009536743\\
11.05	4.71000003814697\\
11.06	4.65000009536743\\
11.07	4.625\\
11.08	4.67999982833862\\
11.09	4.76000022888184\\
11.1	4.77500009536743\\
11.11	4.73999977111816\\
11.12	4.63000011444092\\
11.13	4.57000017166138\\
11.14	4.61999988555908\\
11.15	4.73999977111816\\
11.16	4.76999998092651\\
11.17	4.78499984741211\\
11.18	4.84999990463257\\
11.19	4.81500005722046\\
11.2	4.78999996185303\\
11.21	4.76999998092651\\
11.22	4.74499988555908\\
11.23	4.73999977111816\\
11.24	4.73999977111816\\
11.25	4.68499994277954\\
11.26	4.68499994277954\\
11.27	4.78499984741211\\
11.28	4.78499984741211\\
11.29	4.71500015258789\\
11.3	4.67000007629395\\
11.31	4.625\\
11.32	4.71500015258789\\
11.33	4.81500005722046\\
11.34	4.88000011444092\\
11.35	4.84999990463257\\
11.36	4.88000011444092\\
11.37	4.89499998092651\\
11.38	4.84499979019165\\
11.39	4.83500003814697\\
11.4	4.82999992370605\\
11.41	4.77500009536743\\
11.42	4.76999998092651\\
11.43	4.75500011444092\\
11.44	4.80499982833862\\
11.45	4.88000011444092\\
11.46	4.8899998664856\\
11.47	4.88000011444092\\
11.48	4.90500020980835\\
11.49	4.875\\
11.5	4.82000017166138\\
11.51	4.98000001907349\\
11.52	5.03999996185303\\
11.53	4.94999980926514\\
11.54	4.88000011444092\\
11.55	4.82999992370605\\
11.56	4.86499977111816\\
11.57	5.01000022888184\\
11.58	5.08500003814697\\
11.59	5.07999992370605\\
11.6	5.04500007629395\\
11.61	5.0149998664856\\
11.62	5.02500009536743\\
11.63	5.07999992370605\\
11.64	5.07999992370605\\
11.65	5.1399998664856\\
11.66	5.13000011444092\\
11.67	5.09499979019165\\
11.68	5.08500003814697\\
11.69	5.07000017166138\\
11.7	5.09499979019165\\
11.71	5.19000005722046\\
11.72	5.19000005722046\\
11.73	5.15999984741211\\
11.74	5.11499977111816\\
11.75	5.18499994277954\\
11.76	5.14499998092651\\
11.77	5.11499977111816\\
11.78	5.07999992370605\\
11.79	5.02500009536743\\
11.8	5\\
11.81	5.1100001335144\\
11.82	5.11499977111816\\
11.83	5.07999992370605\\
11.84	5.05499982833862\\
11.85	5.02500009536743\\
11.86	5\\
11.87	5.125\\
11.88	5.125\\
11.89	5.07999992370605\\
11.9	5.06500005722046\\
11.91	5.0149998664856\\
11.92	4.94000005722046\\
11.93	5.04500007629395\\
11.94	5.15999984741211\\
11.95	5.13000011444092\\
11.96	5.07000017166138\\
11.97	5.0149998664856\\
11.98	4.92000007629395\\
11.99	4.93499994277954\\
12	4.98000001907349\\
12.01	4.96999979019165\\
12.02	5.02500009536743\\
12.03	5.09999990463257\\
12.04	4.9850001335144\\
12.05	4.96999979019165\\
12.06	5.02500009536743\\
12.07	4.9850001335144\\
12.08	4.93499994277954\\
12.09	4.92000007629395\\
12.1	4.92500019073486\\
12.11	4.9850001335144\\
12.12	5.04500007629395\\
12.13	5.09499979019165\\
12.14	5.09999990463257\\
12.15	5.07999992370605\\
12.16	5.05499982833862\\
12.17	5.03999996185303\\
12.18	5.11499977111816\\
12.19	5.15500020980835\\
12.2	5.13000011444092\\
12.21	5.09999990463257\\
12.22	5.07999992370605\\
12.23	5.09499979019165\\
12.24	5.11499977111816\\
12.25	5.08500003814697\\
12.26	5.02500009536743\\
12.27	4.96500015258789\\
12.28	4.96999979019165\\
12.29	4.94999980926514\\
12.3	5.01000022888184\\
12.31	4.96999979019165\\
12.32	4.94000005722046\\
12.33	4.93499994277954\\
12.34	4.92000007629395\\
12.35	4.8600001335144\\
12.36	4.8600001335144\\
12.37	4.8600001335144\\
12.38	4.93499994277954\\
12.39	4.93499994277954\\
12.4	4.88000011444092\\
12.41	4.92000007629395\\
12.42	5.05499982833862\\
12.43	5.07999992370605\\
12.44	5.03999996185303\\
12.45	4.9850001335144\\
12.46	4.96500015258789\\
12.47	4.96999979019165\\
12.48	5.02500009536743\\
12.49	5.07999992370605\\
12.5	5.06500005722046\\
12.51	4.99499988555908\\
12.52	4.95499992370605\\
12.53	4.90500020980835\\
12.54	4.89499998092651\\
12.55	4.90500020980835\\
12.56	4.8600001335144\\
12.57	4.80499982833862\\
12.58	4.82000017166138\\
12.59	4.74499988555908\\
12.6	4.77500009536743\\
12.61	4.84499979019165\\
12.62	4.84499979019165\\
12.63	4.80499982833862\\
12.64	4.71000003814697\\
12.65	4.66499996185303\\
12.66	4.68499994277954\\
12.67	4.67999982833862\\
12.68	4.67999982833862\\
12.69	4.65000009536743\\
12.7	4.6100001335144\\
12.71	4.57000017166138\\
12.72	4.66499996185303\\
12.73	4.75500011444092\\
12.74	4.78499984741211\\
12.75	4.77500009536743\\
12.76	4.78999996185303\\
12.77	4.80499982833862\\
12.78	4.80499982833862\\
12.79	4.88000011444092\\
12.8	4.88000011444092\\
12.81	4.8899998664856\\
12.82	4.99499988555908\\
12.83	5.01000022888184\\
12.84	4.93499994277954\\
12.85	4.95499992370605\\
12.86	5.05499982833862\\
12.87	4.99499988555908\\
12.88	4.95499992370605\\
12.89	5.01000022888184\\
12.9	5.05499982833862\\
12.91	5.07000017166138\\
12.92	5.13000011444092\\
12.93	5.11499977111816\\
12.94	5.04500007629395\\
12.95	5.1100001335144\\
12.96	5.125\\
12.97	5.14499998092651\\
12.98	5.17000007629395\\
12.99	5.2350001335144\\
13	5.20499992370605\\
13.01	5.08500003814697\\
13.02	5.11499977111816\\
13.03	5.11499977111816\\
13.04	5.14499998092651\\
13.05	5.18499994277954\\
13.06	5.15500020980835\\
13.07	5.07999992370605\\
13.08	5.1100001335144\\
13.09	5.1399998664856\\
13.1	5.21999979019165\\
13.11	5.28000020980835\\
13.12	5.2649998664856\\
13.13	5.26000022888184\\
13.14	5.25\\
13.15	5.2649998664856\\
13.16	5.26000022888184\\
13.17	5.2649998664856\\
13.18	5.32000017166138\\
13.19	5.28000020980835\\
13.2	5.24499988555908\\
13.21	5.29500007629395\\
13.22	5.28999996185303\\
13.23	5.32000017166138\\
13.24	5.32999992370605\\
13.25	5.25\\
13.26	5.28999996185303\\
13.27	5.34999990463257\\
13.28	5.375\\
13.29	5.42999982833862\\
13.3	5.41499996185303\\
13.31	5.3600001335144\\
13.32	5.34999990463257\\
13.33	5.40500020980835\\
13.34	5.44000005722046\\
13.35	5.44000005722046\\
13.36	5.46000003814697\\
13.37	5.46000003814697\\
13.38	5.3600001335144\\
13.39	5.32000017166138\\
13.4	5.34999990463257\\
13.41	5.2649998664856\\
13.42	5.19000005722046\\
13.43	5.17000007629395\\
13.44	5.07000017166138\\
13.45	5\\
13.46	5.03999996185303\\
13.47	5.09999990463257\\
13.48	5.09499979019165\\
13.49	5.07999992370605\\
13.5	5.07999992370605\\
13.51	5.08500003814697\\
13.52	5.15500020980835\\
13.53	5.23000001907349\\
13.54	5.26000022888184\\
13.55	5.19000005722046\\
13.56	5.21999979019165\\
13.57	5.25\\
13.58	5.30499982833862\\
13.59	5.30999994277954\\
13.6	5.2649998664856\\
13.61	5.19999980926514\\
13.62	5.21999979019165\\
13.63	5.2350001335144\\
13.64	5.28000020980835\\
13.65	5.32000017166138\\
13.66	5.34000015258789\\
13.67	5.24499988555908\\
13.68	5.19000005722046\\
13.69	5.19999980926514\\
13.7	5.24499988555908\\
13.71	5.28999996185303\\
13.72	5.34999990463257\\
13.73	5.26000022888184\\
13.74	5.2350001335144\\
13.75	5.24499988555908\\
13.76	5.28999996185303\\
13.77	5.33500003814697\\
13.78	5.34000015258789\\
13.79	5.14499998092651\\
13.8	5.14499998092651\\
13.81	5.13000011444092\\
13.82	5.11499977111816\\
13.83	5.07999992370605\\
13.84	5.05499982833862\\
13.85	5.03499984741211\\
13.86	4.92000007629395\\
13.87	4.8600001335144\\
13.88	4.84999990463257\\
13.89	4.8600001335144\\
13.9	4.93499994277954\\
13.91	4.8899998664856\\
13.92	4.83500003814697\\
13.93	4.83500003814697\\
13.94	4.80499982833862\\
13.95	4.88000011444092\\
13.96	4.92000007629395\\
13.97	4.95499992370605\\
13.98	4.92000007629395\\
13.99	4.82000017166138\\
14	4.83500003814697\\
14.01	4.88000011444092\\
14.02	4.90999984741211\\
14.03	4.94000005722046\\
14.04	4.93499994277954\\
14.05	4.89499998092651\\
14.06	4.89499998092651\\
14.07	4.92000007629395\\
14.08	5.0149998664856\\
14.09	5.03499984741211\\
14.1	4.95499992370605\\
14.11	4.96500015258789\\
14.12	4.99499988555908\\
14.13	5.03499984741211\\
14.14	5.07999992370605\\
14.15	5.11499977111816\\
14.16	5.09499979019165\\
14.17	5.04500007629395\\
14.18	5.03499984741211\\
14.19	5.125\\
14.2	5.18499994277954\\
14.21	5.21999979019165\\
14.22	5.17000007629395\\
14.23	5.15500020980835\\
14.24	5.15999984741211\\
14.25	5.21999979019165\\
14.26	5.24499988555908\\
14.27	5.20499992370605\\
14.28	5.17500019073486\\
14.29	5.19000005722046\\
14.3	5.21999979019165\\
14.31	5.25\\
14.32	5.33500003814697\\
14.33	5.38000011444092\\
14.34	5.28000020980835\\
14.35	5.19999980926514\\
14.36	5.13000011444092\\
14.37	5.14499998092651\\
14.38	5.15500020980835\\
14.39	5.18499994277954\\
14.4	5.1100001335144\\
14.41	5\\
14.42	4.99499988555908\\
14.43	4.95499992370605\\
14.44	4.9850001335144\\
14.45	4.99499988555908\\
14.46	4.90999984741211\\
14.47	4.90999984741211\\
14.48	4.94999980926514\\
14.49	5.0149998664856\\
14.5	5.09499979019165\\
14.51	5.15500020980835\\
14.52	5.125\\
14.53	5.04500007629395\\
14.54	5.09999990463257\\
14.55	5.15500020980835\\
14.56	5.15999984741211\\
14.57	5.19999980926514\\
14.58	5.19000005722046\\
14.59	5.125\\
14.6	5.09499979019165\\
14.61	5.09999990463257\\
14.62	5.09499979019165\\
14.63	5.09999990463257\\
14.64	5.125\\
14.65	4.96999979019165\\
14.66	4.92500019073486\\
14.67	4.92000007629395\\
14.68	4.95499992370605\\
14.69	4.93499994277954\\
14.7	4.96500015258789\\
14.71	4.95499992370605\\
14.72	4.94000005722046\\
14.73	4.95499992370605\\
14.74	4.96999979019165\\
14.75	4.96999979019165\\
14.76	4.96999979019165\\
14.77	4.94000005722046\\
14.78	4.90500020980835\\
14.79	4.875\\
14.8	4.9850001335144\\
14.81	5.09499979019165\\
14.82	5.0149998664856\\
14.83	5.0149998664856\\
14.84	4.96999979019165\\
14.85	4.89499998092651\\
14.86	4.93499994277954\\
14.87	4.96500015258789\\
14.88	4.92000007629395\\
14.89	4.8899998664856\\
14.9	4.92500019073486\\
14.91	4.92500019073486\\
14.92	4.90500020980835\\
14.93	4.92000007629395\\
14.94	4.89499998092651\\
14.95	4.88000011444092\\
14.96	4.82999992370605\\
14.97	4.78499984741211\\
14.98	4.74499988555908\\
14.99	4.72499990463257\\
15	4.69999980926514\\
15.01	4.67999982833862\\
15.02	4.63000011444092\\
15.03	4.47499990463257\\
15.04	4.53499984741211\\
15.05	4.57000017166138\\
15.06	4.53499984741211\\
15.07	4.48999977111816\\
15.08	4.46000003814697\\
15.09	4.41499996185303\\
15.1	4.48999977111816\\
15.11	4.56500005722046\\
15.12	4.58500003814697\\
15.13	4.57999992370605\\
15.14	4.55000019073486\\
15.15	4.48000001907349\\
15.16	4.53499984741211\\
15.17	4.55499982833862\\
15.18	4.63000011444092\\
15.19	4.65000009536743\\
15.2	4.59999990463257\\
15.21	4.56500005722046\\
15.22	4.52500009536743\\
15.23	4.56500005722046\\
15.24	4.55499982833862\\
15.25	4.58500003814697\\
15.26	4.65000009536743\\
15.27	4.6100001335144\\
15.28	4.59499979019165\\
15.29	4.66499996185303\\
15.3	4.69500017166138\\
15.31	4.73000001907349\\
15.32	4.68499994277954\\
15.33	4.67000007629395\\
15.34	4.75500011444092\\
15.35	4.73000001907349\\
15.36	4.76999998092651\\
15.37	4.76999998092651\\
15.38	4.76999998092651\\
15.39	4.69500017166138\\
15.4	4.61999988555908\\
15.41	4.65500020980835\\
15.42	4.6399998664856\\
15.43	4.61999988555908\\
15.44	4.65500020980835\\
15.45	4.65500020980835\\
15.46	4.57999992370605\\
15.47	4.6399998664856\\
15.48	4.67999982833862\\
15.49	4.69500017166138\\
15.5	4.67000007629395\\
15.51	4.57000017166138\\
15.52	4.61999988555908\\
15.53	4.67999982833862\\
15.54	4.68499994277954\\
15.55	4.65000009536743\\
15.56	4.65500020980835\\
15.57	4.59499979019165\\
15.58	4.53499984741211\\
15.59	4.56500005722046\\
15.6	4.65000009536743\\
15.61	4.61999988555908\\
15.62	4.6100001335144\\
15.63	4.72499990463257\\
15.64	4.71500015258789\\
15.65	4.69999980926514\\
15.66	4.76999998092651\\
15.67	4.82000017166138\\
15.68	4.80499982833862\\
15.69	4.80499982833862\\
15.7	4.67999982833862\\
15.71	4.73000001907349\\
15.72	4.76000022888184\\
15.73	4.76000022888184\\
15.74	4.78999996185303\\
15.75	4.80000019073486\\
15.76	4.80499982833862\\
15.77	4.80499982833862\\
15.78	4.78499984741211\\
15.79	4.82999992370605\\
15.8	4.80000019073486\\
15.81	4.73999977111816\\
15.82	4.77500009536743\\
15.83	4.77500009536743\\
15.84	4.75500011444092\\
15.85	4.81500005722046\\
15.86	4.78499984741211\\
15.87	4.73000001907349\\
15.88	4.73999977111816\\
15.89	4.71000003814697\\
15.9	4.67000007629395\\
15.91	4.76999998092651\\
15.92	4.80499982833862\\
15.93	4.69500017166138\\
15.94	4.63000011444092\\
15.95	4.69999980926514\\
15.96	4.72499990463257\\
15.97	4.74499988555908\\
15.98	4.78499984741211\\
15.99	4.63000011444092\\
16	4.65500020980835\\
16.01	4.69999980926514\\
16.02	4.69999980926514\\
16.03	4.71500015258789\\
16.04	4.75500011444092\\
16.05	4.71500015258789\\
16.06	4.66499996185303\\
16.07	4.67999982833862\\
16.08	4.67999982833862\\
16.09	4.72499990463257\\
16.1	4.78499984741211\\
16.11	4.77500009536743\\
16.12	4.67999982833862\\
16.13	4.71000003814697\\
16.14	4.73999977111816\\
16.15	4.74499988555908\\
16.16	4.73999977111816\\
16.17	4.69500017166138\\
16.18	4.56500005722046\\
16.19	4.55000019073486\\
16.2	4.68499994277954\\
16.21	4.59999990463257\\
16.22	4.50500011444092\\
16.23	4.44500017166138\\
16.24	4.375\\
16.25	4.30499982833862\\
16.26	4.34499979019165\\
16.27	4.22499990463257\\
16.28	4.1100001335144\\
16.29	4.09000015258789\\
16.3	3.74000000953674\\
16.31	3.04500007629395\\
16.32	2.18499994277954\\
16.33	1.17999994754791\\
16.34	0.435000002384186\\
16.35	-0.215000003576279\\
16.36	-0.610000014305115\\
16.37	-0.649999976158142\\
16.38	-0.790000021457672\\
16.39	-0.894999980926514\\
16.4	-1.03999996185303\\
16.41	-1.46000003814697\\
16.42	-1.73000001907349\\
16.43	-1.64499998092651\\
16.44	-1.125\\
16.45	-0.270000010728836\\
16.46	0.589999973773956\\
16.47	1.10000002384186\\
16.48	1.26999998092651\\
16.49	1.0900000333786\\
16.5	0.654999971389771\\
16.51	0.174999997019768\\
16.52	-0.174999997019768\\
16.53	-0.435000002384186\\
16.54	-0.625\\
16.55	-0.745000004768372\\
16.56	-0.745000004768372\\
16.57	-0.595000028610229\\
16.58	-0.25\\
16.59	0.109999999403954\\
16.6	0.140000000596046\\
16.61	0.140000000596046\\
16.62	0.194999992847443\\
16.63	0.115000002086163\\
16.64	-0.0199999995529652\\
16.65	-0.25\\
16.66	-0.324999988079071\\
16.67	-0.310000002384186\\
16.68	-0.354999989271164\\
16.69	-0.360000014305115\\
16.7	-0.324999988079071\\
16.71	-0.209999993443489\\
16.72	-0.0949999988079071\\
16.73	-0.0299999993294477\\
16.74	-0.0149999996647239\\
16.75	-0.0450000017881393\\
16.76	2.08166817117217e-016\\
16.77	-0.0299999993294477\\
16.78	-0.0750000029802322\\
16.79	-0.189999997615814\\
16.8	-0.119999997317791\\
16.81	0.00499999988824129\\
16.82	0.0549999997019768\\
16.83	0.00499999988824129\\
16.84	-0.135000005364418\\
16.85	-0.314999997615814\\
16.86	-0.284999996423721\\
16.87	-0.215000003576279\\
16.88	-0.174999997019768\\
16.89	-0.144999995827675\\
16.9	-0.0799999982118607\\
16.91	-0.0750000029802322\\
16.92	-0.0149999996647239\\
16.93	0.00999999977648258\\
16.94	-0.0949999988079071\\
16.95	-0.115000002086163\\
16.96	-0.159999996423721\\
16.97	-0.254999995231628\\
16.98	-0.324999988079071\\
16.99	-0.340000003576279\\
17	-0.264999985694885\\
17.01	-0.180000007152557\\
17.02	-0.0750000029802322\\
17.03	-0.0599999986588955\\
17.04	-0.159999996423721\\
17.05	-0.204999998211861\\
17.06	-0.159999996423721\\
17.07	-0.189999997615814\\
17.08	-0.165000006556511\\
17.09	-0.119999997317791\\
17.1	-0.104999996721745\\
17.11	-0.135000005364418\\
17.12	-0.150000005960464\\
17.13	-0.264999985694885\\
17.14	-0.375\\
17.15	-0.370000004768372\\
17.16	-0.144999995827675\\
17.17	-0.0450000017881393\\
17.18	-0.125\\
17.19	-0.204999998211861\\
17.2	-0.360000014305115\\
17.21	-0.280000001192093\\
17.22	-0.25\\
17.23	-0.215000003576279\\
17.24	-0.119999997317791\\
17.25	-0.125\\
17.26	-0.135000005364418\\
17.27	-0.189999997615814\\
17.28	-0.0500000007450581\\
17.29	0.0949999988079071\\
17.3	-0.00499999988824129\\
17.31	-0.0599999986588955\\
17.32	-0.159999996423721\\
17.33	-0.239999994635582\\
17.34	-0.209999993443489\\
17.35	-0.209999993443489\\
17.36	-0.264999985694885\\
17.37	-0.284999996423721\\
17.38	-0.189999997615814\\
17.39	-0.119999997317791\\
17.4	-0.0900000035762787\\
17.41	-0.0900000035762787\\
17.42	-0.150000005960464\\
17.43	-0.119999997317791\\
17.44	-0.0900000035762787\\
17.45	-0.0500000007450581\\
17.46	-0.159999996423721\\
17.47	-0.144999995827675\\
17.48	-0.215000003576279\\
17.49	-0.189999997615814\\
17.5	-0.119999997317791\\
17.51	-0.0900000035762787\\
17.52	-0.119999997317791\\
17.53	-0.0799999982118607\\
17.54	-0.0450000017881393\\
17.55	-0.194999992847443\\
17.56	-0.194999992847443\\
17.57	-0.159999996423721\\
17.58	-0.204999998211861\\
17.59	-0.189999997615814\\
17.6	-0.284999996423721\\
17.61	-0.344999998807907\\
17.62	-0.234999999403954\\
17.63	-0.150000005960464\\
17.64	-0.135000005364418\\
17.65	-0.0799999982118607\\
17.66	-0.0649999976158142\\
17.67	-0.0949999988079071\\
17.68	-0.0149999996647239\\
17.69	-0.0299999993294477\\
17.7	-0.119999997317791\\
17.71	-0.234999999403954\\
17.72	-0.324999988079071\\
17.73	-0.254999995231628\\
17.74	-0.25\\
17.75	-0.239999994635582\\
17.76	-0.209999993443489\\
17.77	-0.180000007152557\\
17.78	-0.125\\
17.79	-0.0900000035762787\\
17.8	-0.0949999988079071\\
17.81	-0.0750000029802322\\
17.82	-0.0599999986588955\\
17.83	-0.0500000007450581\\
17.84	-0.0750000029802322\\
17.85	-0.0450000017881393\\
17.86	-0.0500000007450581\\
17.87	-0.119999997317791\\
17.88	-0.159999996423721\\
17.89	-0.215000003576279\\
17.9	-0.239999994635582\\
17.91	-0.215000003576279\\
17.92	-0.174999997019768\\
17.93	-0.104999996721745\\
17.94	-0.0799999982118607\\
17.95	-0.0500000007450581\\
17.96	-0.0799999982118607\\
17.97	-0.150000005960464\\
17.98	-0.189999997615814\\
17.99	-0.234999999403954\\
18	-0.264999985694885\\
18.01	-0.165000006556511\\
18.02	-0.119999997317791\\
18.03	-0.0799999982118607\\
18.04	-0.0299999993294477\\
18.05	-0.0500000007450581\\
18.06	-0.0599999986588955\\
18.07	-0.0599999986588955\\
18.08	-0.159999996423721\\
18.09	-0.234999999403954\\
18.1	-0.209999993443489\\
18.11	-0.159999996423721\\
18.12	-0.135000005364418\\
18.13	-0.135000005364418\\
18.14	-0.0799999982118607\\
18.15	-0.0500000007450581\\
18.16	-0.0799999982118607\\
18.17	-0.0599999986588955\\
18.18	-0.119999997317791\\
18.19	-0.194999992847443\\
18.2	-0.159999996423721\\
18.21	-0.125\\
18.22	-0.115000002086163\\
18.23	-0.0750000029802322\\
18.24	-0.0599999986588955\\
18.25	-0.0900000035762787\\
18.26	-0.115000002086163\\
18.27	-0.135000005364418\\
18.28	-0.189999997615814\\
18.29	-0.215000003576279\\
18.3	-0.174999997019768\\
18.31	-0.0949999988079071\\
18.32	-0.104999996721745\\
18.33	-0.0450000017881393\\
18.34	-0.0149999996647239\\
18.35	-0.0149999996647239\\
18.36	-0.0149999996647239\\
18.37	-0.0799999982118607\\
18.38	-0.104999996721745\\
18.39	-0.115000002086163\\
18.4	-0.0750000029802322\\
18.41	-0.0149999996647239\\
18.42	-0.0450000017881393\\
18.43	-0.0750000029802322\\
18.44	-0.104999996721745\\
18.45	-0.0799999982118607\\
18.46	-0.0799999982118607\\
18.47	-0.0649999976158142\\
18.48	-0.0900000035762787\\
18.49	-0.0149999996647239\\
18.5	-0.0649999976158142\\
18.51	-0.135000005364418\\
18.52	-0.0799999982118607\\
18.53	-0.115000002086163\\
18.54	-0.119999997317791\\
18.55	-0.0350000001490116\\
18.56	-0.0599999986588955\\
18.57	-0.115000002086163\\
18.58	-0.115000002086163\\
18.59	-0.0649999976158142\\
18.6	-0.0450000017881393\\
18.61	-0.00499999988824129\\
18.62	0.00499999988824129\\
18.63	-0.0299999993294477\\
18.64	-0.0450000017881393\\
18.65	-0.0599999986588955\\
18.66	-0.115000002086163\\
18.67	-0.0949999988079071\\
18.68	-0.0900000035762787\\
18.69	-0.0799999982118607\\
18.7	-0.0799999982118607\\
18.71	-0.0799999982118607\\
18.72	-0.0350000001490116\\
18.73	0.0199999995529652\\
18.74	0.0799999982118607\\
18.75	-0.0149999996647239\\
18.76	-0.0750000029802322\\
18.77	-0.0949999988079071\\
18.78	-0.0799999982118607\\
18.79	-0.0500000007450581\\
18.8	-0.0299999993294477\\
18.81	0.0199999995529652\\
18.82	-0.00499999988824129\\
18.83	0.00999999977648258\\
18.84	0.0199999995529652\\
18.85	-0.0450000017881393\\
18.86	-0.0500000007450581\\
18.87	0.00499999988824129\\
18.88	0.0199999995529652\\
18.89	0.0549999997019768\\
18.9	0.0350000001490116\\
18.91	0.0649999976158142\\
18.92	0.0350000001490116\\
18.93	2.08166817117217e-016\\
18.94	-0.0199999995529652\\
18.95	-0.0500000007450581\\
18.96	-0.0450000017881393\\
18.97	-0.0149999996647239\\
18.98	0.0549999997019768\\
18.99	0.0700000002980232\\
19	0.0949999988079071\\
19.01	0.0949999988079071\\
19.02	0.0649999976158142\\
19.03	0.0549999997019768\\
19.04	0.00999999977648258\\
19.05	2.08166817117217e-016\\
19.06	-0.0299999993294477\\
19.07	-0.0450000017881393\\
19.08	-0.0799999982118607\\
19.09	-0.0149999996647239\\
19.1	0.0549999997019768\\
19.11	0.115000002086163\\
19.12	0.140000000596046\\
19.13	0.100000001490116\\
19.14	0.0500000007450581\\
19.15	2.08166817117217e-016\\
19.16	0.0199999995529652\\
19.17	-0.0500000007450581\\
19.18	-0.0649999976158142\\
19.19	0.0649999976158142\\
19.2	0.0949999988079071\\
19.21	0.100000001490116\\
19.22	0.0649999976158142\\
19.23	-0.0299999993294477\\
19.24	-0.0500000007450581\\
19.25	0.0199999995529652\\
19.26	0.0700000002980232\\
19.27	0.129999995231628\\
19.28	0.165000006556511\\
19.29	0.0949999988079071\\
19.3	0.0500000007450581\\
19.31	0.0199999995529652\\
19.32	-0.0199999995529652\\
19.33	-0.0500000007450581\\
19.34	0.00999999977648258\\
19.35	0.0700000002980232\\
19.36	0.109999999403954\\
19.37	0.115000002086163\\
19.38	0.185000002384186\\
19.39	0.129999995231628\\
19.4	0.0799999982118607\\
19.41	0.0199999995529652\\
19.42	-0.0599999986588955\\
19.43	-0.104999996721745\\
19.44	-0.0949999988079071\\
19.45	-0.0500000007450581\\
19.46	-0.0149999996647239\\
19.47	0.0350000001490116\\
19.48	0.0850000008940697\\
19.49	0.109999999403954\\
19.5	0.0549999997019768\\
19.51	0.0199999995529652\\
19.52	-0.0149999996647239\\
19.53	-0.119999997317791\\
19.54	-0.0949999988079071\\
19.55	-0.0900000035762787\\
19.56	0.0199999995529652\\
19.57	0.185000002384186\\
19.58	0.170000001788139\\
19.59	0.129999995231628\\
19.6	0.0549999997019768\\
19.61	-0.0599999986588955\\
19.62	-0.174999997019768\\
19.63	-0.159999996423721\\
19.64	-0.0949999988079071\\
19.65	-0.0199999995529652\\
19.66	0.0500000007450581\\
19.67	0.0700000002980232\\
19.68	0.0850000008940697\\
19.69	0.0549999997019768\\
19.7	0.0199999995529652\\
19.71	-0.0199999995529652\\
19.72	-0.0350000001490116\\
19.73	-0.0199999995529652\\
19.74	-0.0149999996647239\\
19.75	-0.00499999988824129\\
19.76	0.0549999997019768\\
19.77	0.0549999997019768\\
19.78	0.0649999976158142\\
19.79	0.0850000008940697\\
19.8	2.08166817117217e-016\\
19.81	-0.0350000001490116\\
19.82	-0.0350000001490116\\
19.83	-0.0199999995529652\\
19.84	0.00499999988824129\\
19.85	0.00499999988824129\\
19.86	0.00999999977648258\\
19.87	0.0199999995529652\\
19.88	0.0500000007450581\\
19.89	0.0199999995529652\\
19.9	0.00499999988824129\\
19.91	2.08166817117217e-016\\
19.92	2.08166817117217e-016\\
19.93	0.025000000372529\\
19.94	0.00999999977648258\\
19.95	0.0399999991059303\\
19.96	0.0949999988079071\\
19.97	0.100000001490116\\
19.98	0.100000001490116\\
19.99	0.0649999976158142\\
20	-0.0149999996647239\\
20.01	-0.0299999993294477\\
20.02	0.00999999977648258\\
20.03	0.0199999995529652\\
20.04	0.025000000372529\\
20.05	0.0350000001490116\\
20.06	0.00999999977648258\\
20.07	0.0500000007450581\\
20.08	0.109999999403954\\
20.09	0.100000001490116\\
20.1	0.0700000002980232\\
20.11	0.0850000008940697\\
20.12	0.0549999997019768\\
20.13	0.00499999988824129\\
20.14	0.0199999995529652\\
20.15	-0.0299999993294477\\
20.16	-0.0199999995529652\\
20.17	-0.0500000007450581\\
20.18	-0.0299999993294477\\
20.19	0.0500000007450581\\
20.2	0.0350000001490116\\
20.21	0.0199999995529652\\
20.22	-0.00499999988824129\\
20.23	-0.0649999976158142\\
20.24	-0.0750000029802322\\
20.25	-0.0799999982118607\\
20.26	-0.0750000029802322\\
20.27	-0.150000005960464\\
20.28	-0.115000002086163\\
20.29	-0.0649999976158142\\
20.3	-0.0199999995529652\\
20.31	2.08166817117217e-016\\
20.32	-0.0649999976158142\\
20.33	-0.119999997317791\\
20.34	-0.0799999982118607\\
20.35	-0.0799999982118607\\
20.36	-0.0450000017881393\\
20.37	-0.0500000007450581\\
20.38	-0.0799999982118607\\
20.39	-0.0500000007450581\\
20.4	0.00499999988824129\\
20.41	-0.0199999995529652\\
20.42	-0.0500000007450581\\
20.43	-0.0350000001490116\\
20.44	-0.0750000029802322\\
20.45	-0.0649999976158142\\
20.46	2.08166817117217e-016\\
20.47	-0.0450000017881393\\
20.48	-0.0450000017881393\\
20.49	-0.00499999988824129\\
20.5	-0.0299999993294477\\
20.51	-0.0149999996647239\\
20.52	-0.0199999995529652\\
20.53	-0.0500000007450581\\
20.54	-0.0500000007450581\\
20.55	-0.00499999988824129\\
20.56	-0.0199999995529652\\
20.57	-0.0299999993294477\\
20.58	-0.0199999995529652\\
20.59	-0.0199999995529652\\
20.6	2.08166817117217e-016\\
20.61	-0.0350000001490116\\
20.62	-0.0299999993294477\\
20.63	-0.0350000001490116\\
20.64	-0.0450000017881393\\
20.65	-0.0500000007450581\\
20.66	-0.0450000017881393\\
20.67	-0.0500000007450581\\
20.68	-0.0299999993294477\\
20.69	-0.0299999993294477\\
20.7	-0.0450000017881393\\
20.71	-0.0500000007450581\\
20.72	0.00499999988824129\\
20.73	-0.0199999995529652\\
20.74	-0.0199999995529652\\
20.75	-0.0450000017881393\\
20.76	-0.0949999988079071\\
20.77	-0.0949999988079071\\
20.78	-0.0799999982118607\\
20.79	-0.0299999993294477\\
20.8	-0.0199999995529652\\
20.81	-0.0149999996647239\\
20.82	-0.0350000001490116\\
20.83	-0.0299999993294477\\
20.84	-0.0599999986588955\\
20.85	-0.0949999988079071\\
20.86	-0.0900000035762787\\
20.87	-0.0599999986588955\\
20.88	-0.0750000029802322\\
20.89	-0.0350000001490116\\
20.9	-0.0299999993294477\\
20.91	-0.0199999995529652\\
20.92	-0.0450000017881393\\
20.93	-0.0450000017881393\\
20.94	-0.0649999976158142\\
20.95	-0.0799999982118607\\
20.96	-0.0500000007450581\\
20.97	-0.0199999995529652\\
20.98	-0.00499999988824129\\
20.99	-0.0450000017881393\\
21	-0.0500000007450581\\
21.01	-0.0500000007450581\\
21.02	-0.0450000017881393\\
21.03	-0.0500000007450581\\
21.04	-0.0599999986588955\\
21.05	-0.0500000007450581\\
21.06	-0.0799999982118607\\
21.07	-0.0149999996647239\\
21.08	0.00499999988824129\\
21.09	-0.0450000017881393\\
21.1	-0.0500000007450581\\
21.11	-0.0799999982118607\\
21.12	-0.0799999982118607\\
21.13	-0.104999996721745\\
21.14	-0.0900000035762787\\
21.15	-0.0750000029802322\\
21.16	-0.0299999993294477\\
21.17	-0.0450000017881393\\
21.18	-0.0500000007450581\\
21.19	-0.0799999982118607\\
21.2	-0.0799999982118607\\
21.21	-0.115000002086163\\
21.22	-0.144999995827675\\
21.23	-0.159999996423721\\
21.24	-0.0949999988079071\\
21.25	-0.0949999988079071\\
21.26	-0.0500000007450581\\
21.27	-0.0350000001490116\\
21.28	-0.0500000007450581\\
21.29	-0.0500000007450581\\
21.3	-0.0450000017881393\\
21.31	-0.0500000007450581\\
21.32	-0.0900000035762787\\
21.33	-0.119999997317791\\
21.34	-0.0949999988079071\\
21.35	-0.0799999982118607\\
21.36	-0.0799999982118607\\
21.37	-0.0799999982118607\\
21.38	-0.0599999986588955\\
21.39	-0.0450000017881393\\
21.4	-0.0900000035762787\\
21.41	-0.0799999982118607\\
21.42	-0.165000006556511\\
21.43	-0.174999997019768\\
21.44	-0.125\\
21.45	-0.104999996721745\\
21.46	-0.0949999988079071\\
21.47	-0.0949999988079071\\
21.48	-0.0799999982118607\\
21.49	-0.0900000035762787\\
21.5	-0.135000005364418\\
21.51	-0.174999997019768\\
21.52	-0.194999992847443\\
21.53	-0.194999992847443\\
21.54	-0.150000005960464\\
21.55	-0.104999996721745\\
21.56	-0.0799999982118607\\
21.57	-0.0500000007450581\\
21.58	-0.0450000017881393\\
21.59	-0.0799999982118607\\
21.6	-0.0799999982118607\\
21.61	-0.0949999988079071\\
21.62	-0.144999995827675\\
21.63	-0.125\\
21.64	-0.119999997317791\\
21.65	-0.119999997317791\\
21.66	-0.0949999988079071\\
21.67	-0.0450000017881393\\
21.68	-0.0299999993294477\\
21.69	-0.0500000007450581\\
21.7	-0.0649999976158142\\
21.71	-0.194999992847443\\
21.72	-0.25\\
21.73	-0.115000002086163\\
21.74	-0.0450000017881393\\
21.75	-0.119999997317791\\
21.76	-0.0649999976158142\\
21.77	-0.0299999993294477\\
21.78	2.08166817117217e-016\\
21.79	-0.0199999995529652\\
21.8	-0.115000002086163\\
21.81	-0.204999998211861\\
21.82	-0.209999993443489\\
21.83	-0.144999995827675\\
21.84	-0.0949999988079071\\
21.85	-0.125\\
21.86	-0.0500000007450581\\
21.87	0.0350000001490116\\
21.88	-0.0299999993294477\\
21.89	0.0700000002980232\\
21.9	-0.0799999982118607\\
21.91	-0.294999986886978\\
21.92	-0.264999985694885\\
21.93	-0.104999996721745\\
21.94	-0.0500000007450581\\
21.95	-0.0450000017881393\\
21.96	-0.0199999995529652\\
21.97	-0.0799999982118607\\
21.98	-0.0649999976158142\\
21.99	-0.0799999982118607\\
22	-0.115000002086163\\
22.01	-0.204999998211861\\
22.02	-0.104999996721745\\
22.03	-0.0750000029802322\\
22.04	-0.0799999982118607\\
22.05	-0.0900000035762787\\
22.06	-0.115000002086163\\
22.07	-0.159999996423721\\
22.08	-0.125\\
22.09	-0.174999997019768\\
22.1	-0.135000005364418\\
22.11	-0.0900000035762787\\
22.12	-0.104999996721745\\
22.13	-0.0949999988079071\\
22.14	-0.0949999988079071\\
22.15	-0.119999997317791\\
22.16	-0.165000006556511\\
22.17	-0.104999996721745\\
22.18	-0.0450000017881393\\
22.19	-0.119999997317791\\
22.2	-0.115000002086163\\
22.21	-0.0500000007450581\\
22.22	-0.0649999976158142\\
22.23	-0.119999997317791\\
22.24	-0.115000002086163\\
22.25	-0.0299999993294477\\
22.26	-0.0900000035762787\\
22.27	-0.119999997317791\\
22.28	-0.0450000017881393\\
22.29	-0.0450000017881393\\
22.3	-0.0350000001490116\\
22.31	0.00499999988824129\\
22.32	-0.0799999982118607\\
22.33	-0.115000002086163\\
22.34	-0.0799999982118607\\
22.35	-0.0799999982118607\\
22.36	-0.0500000007450581\\
22.37	-0.0500000007450581\\
22.38	-0.0500000007450581\\
22.39	-0.0500000007450581\\
22.4	-0.0199999995529652\\
22.41	-0.0350000001490116\\
22.42	-0.0599999986588955\\
22.43	-0.0900000035762787\\
22.44	-0.0500000007450581\\
22.45	-0.0500000007450581\\
22.46	-0.0599999986588955\\
22.47	-0.0500000007450581\\
22.48	-0.0500000007450581\\
22.49	-0.0450000017881393\\
22.5	-0.0450000017881393\\
22.51	-0.0450000017881393\\
22.52	-0.0799999982118607\\
22.53	-0.0750000029802322\\
22.54	-0.0599999986588955\\
22.55	-0.0599999986588955\\
22.56	-0.0599999986588955\\
22.57	-0.0750000029802322\\
22.58	-0.0949999988079071\\
22.59	-0.0450000017881393\\
22.6	-0.00499999988824129\\
22.61	-0.0199999995529652\\
22.62	-0.0199999995529652\\
22.63	-0.0350000001490116\\
22.64	-0.0599999986588955\\
22.65	-0.0500000007450581\\
22.66	-0.0649999976158142\\
22.67	-0.104999996721745\\
22.68	-0.0599999986588955\\
22.69	2.08166817117217e-016\\
22.7	0.0399999991059303\\
22.71	0.0199999995529652\\
22.72	0.0549999997019768\\
22.73	0.025000000372529\\
22.74	-0.0500000007450581\\
22.75	-0.0599999986588955\\
22.76	-0.0199999995529652\\
22.77	0.025000000372529\\
22.78	0.0949999988079071\\
22.79	0.0799999982118607\\
22.8	0.0500000007450581\\
22.81	-0.0500000007450581\\
22.82	-0.0199999995529652\\
22.83	0.0350000001490116\\
22.84	0.0549999997019768\\
22.85	0.0649999976158142\\
22.86	2.08166817117217e-016\\
22.87	0.0199999995529652\\
22.88	0.0549999997019768\\
22.89	0.0500000007450581\\
22.9	0.025000000372529\\
22.91	0.0199999995529652\\
22.92	0.0549999997019768\\
22.93	0.0199999995529652\\
22.94	0.00499999988824129\\
22.95	0.00499999988824129\\
22.96	-0.0199999995529652\\
22.97	0.0199999995529652\\
22.98	0.0850000008940697\\
22.99	0.0549999997019768\\
23	0.0799999982118607\\
23.01	0.0549999997019768\\
23.02	0.00499999988824129\\
23.03	2.08166817117217e-016\\
23.04	-0.0350000001490116\\
23.05	-0.0799999982118607\\
23.06	0.00499999988824129\\
23.07	0.0500000007450581\\
23.08	0.00499999988824129\\
23.09	0.00499999988824129\\
23.1	0.0500000007450581\\
23.11	0.0199999995529652\\
23.12	0.0350000001490116\\
23.13	0.0199999995529652\\
23.14	-0.00499999988824129\\
23.15	0.0700000002980232\\
23.16	0.115000002086163\\
23.17	0.100000001490116\\
23.18	0.129999995231628\\
23.19	0.100000001490116\\
23.2	0.0799999982118607\\
23.21	-0.00499999988824129\\
23.22	-0.0649999976158142\\
23.23	-0.0649999976158142\\
23.24	-0.0900000035762787\\
23.25	-0.0149999996647239\\
23.26	0.115000002086163\\
23.27	0.0700000002980232\\
23.28	0.00499999988824129\\
23.29	0.0700000002980232\\
23.3	0.115000002086163\\
23.31	0.0500000007450581\\
23.32	0.00499999988824129\\
23.33	0.0199999995529652\\
23.34	2.08166817117217e-016\\
23.35	2.08166817117217e-016\\
23.36	0.00499999988824129\\
23.37	0.0549999997019768\\
23.38	0.0949999988079071\\
23.39	0.0949999988079071\\
23.4	0.100000001490116\\
23.41	0.0850000008940697\\
23.42	0.0199999995529652\\
23.43	2.08166817117217e-016\\
23.44	2.08166817117217e-016\\
23.45	0.00499999988824129\\
23.46	0.00499999988824129\\
23.47	0.0500000007450581\\
23.48	0.0549999997019768\\
23.49	0.109999999403954\\
23.5	0.115000002086163\\
23.51	0.0399999991059303\\
23.52	0.00499999988824129\\
23.53	-0.0350000001490116\\
23.54	-0.0199999995529652\\
23.55	-0.0199999995529652\\
23.56	-0.00499999988824129\\
23.57	0.00499999988824129\\
23.58	0.0199999995529652\\
23.59	0.00499999988824129\\
23.6	0.00499999988824129\\
23.61	-0.00499999988824129\\
23.62	-0.0299999993294477\\
23.63	-0.0199999995529652\\
23.64	2.08166817117217e-016\\
23.65	2.08166817117217e-016\\
23.66	0.0199999995529652\\
23.67	0.0199999995529652\\
23.68	0.0500000007450581\\
23.69	0.025000000372529\\
23.7	0.0549999997019768\\
23.71	0.025000000372529\\
23.72	0.0199999995529652\\
23.73	0.0399999991059303\\
23.74	0.0549999997019768\\
23.75	0.0549999997019768\\
23.76	0.0399999991059303\\
23.77	0.025000000372529\\
23.78	0.00499999988824129\\
23.79	0.0199999995529652\\
23.8	2.08166817117217e-016\\
23.81	0.00499999988824129\\
23.82	0.0199999995529652\\
23.83	0.0399999991059303\\
23.84	0.0799999982118607\\
23.85	0.0549999997019768\\
23.86	0.0500000007450581\\
23.87	0.0500000007450581\\
23.88	0.0199999995529652\\
23.89	-0.0199999995529652\\
23.9	-0.0350000001490116\\
23.91	2.08166817117217e-016\\
23.92	0.0399999991059303\\
23.93	0.0199999995529652\\
23.94	0.0549999997019768\\
23.95	0.0700000002980232\\
23.96	0.144999995827675\\
23.97	0.125\\
23.98	0.0350000001490116\\
23.99	-0.0450000017881393\\
24	-0.125\\
24.01	-0.0649999976158142\\
24.02	0.109999999403954\\
24.03	0.125\\
24.04	0.0700000002980232\\
24.05	0.0199999995529652\\
24.06	0.0649999976158142\\
24.07	0.0199999995529652\\
24.08	0.0199999995529652\\
24.09	0.0199999995529652\\
24.1	-0.00499999988824129\\
24.11	2.08166817117217e-016\\
24.12	0.025000000372529\\
24.13	0.00499999988824129\\
24.14	-0.0199999995529652\\
24.15	0.0949999988079071\\
24.16	0.0700000002980232\\
24.17	-0.0199999995529652\\
24.18	-0.0299999993294477\\
24.19	-0.0199999995529652\\
24.2	-0.0199999995529652\\
24.21	-0.00499999988824129\\
24.22	0.00499999988824129\\
24.23	-0.0149999996647239\\
24.24	-0.00499999988824129\\
24.25	0.0649999976158142\\
24.26	0.0500000007450581\\
24.27	0.00499999988824129\\
24.28	-0.0199999995529652\\
24.29	-0.0599999986588955\\
24.3	-0.0199999995529652\\
24.31	0.00499999988824129\\
24.32	-0.00499999988824129\\
24.33	0.0199999995529652\\
24.34	0.0850000008940697\\
24.35	0.0649999976158142\\
24.36	0.0199999995529652\\
24.37	-0.0149999996647239\\
24.38	-0.0500000007450581\\
24.39	-0.0450000017881393\\
24.4	0.00499999988824129\\
24.41	0.00499999988824129\\
24.42	-0.0350000001490116\\
24.43	-0.0299999993294477\\
24.44	0.0199999995529652\\
24.45	-0.0149999996647239\\
24.46	-0.0199999995529652\\
24.47	-0.0450000017881393\\
24.48	-0.0500000007450581\\
24.49	-0.0149999996647239\\
24.5	-0.0149999996647239\\
24.51	0.00499999988824129\\
24.52	-0.0299999993294477\\
24.53	-0.0649999976158142\\
24.54	2.08166817117217e-016\\
24.55	0.0500000007450581\\
24.56	0.0199999995529652\\
24.57	-0.0199999995529652\\
24.58	-0.0350000001490116\\
24.59	-0.0599999986588955\\
24.6	-0.0199999995529652\\
24.61	-0.0350000001490116\\
24.62	-0.0149999996647239\\
24.63	0.0549999997019768\\
24.64	0.0549999997019768\\
24.65	-0.0199999995529652\\
24.66	-0.0900000035762787\\
24.67	-0.0949999988079071\\
24.68	-0.0500000007450581\\
24.69	0.0350000001490116\\
24.7	0.0399999991059303\\
24.71	0.025000000372529\\
24.72	0.00499999988824129\\
24.73	-0.0149999996647239\\
24.74	-0.0299999993294477\\
24.75	-0.0149999996647239\\
24.76	-0.0199999995529652\\
24.77	-0.0199999995529652\\
24.78	0.0199999995529652\\
24.79	0.0199999995529652\\
24.8	-0.0199999995529652\\
24.81	-0.0299999993294477\\
24.82	-0.0199999995529652\\
24.83	-0.0149999996647239\\
24.84	-0.00499999988824129\\
24.85	-0.0199999995529652\\
24.86	-0.0599999986588955\\
24.87	-0.0450000017881393\\
24.88	-0.0149999996647239\\
24.89	-0.00499999988824129\\
24.9	-0.00499999988824129\\
24.91	0.0199999995529652\\
24.92	2.08166817117217e-016\\
24.93	-0.0450000017881393\\
24.94	-0.0299999993294477\\
24.95	-0.0450000017881393\\
24.96	-0.0599999986588955\\
24.97	-0.0350000001490116\\
24.98	-0.0149999996647239\\
24.99	-0.0500000007450581\\
25	-0.0599999986588955\\
25.01	-0.0299999993294477\\
25.02	-0.0299999993294477\\
25.03	-0.0199999995529652\\
25.04	-0.0649999976158142\\
25.05	-0.119999997317791\\
25.06	-0.0949999988079071\\
25.07	-0.0750000029802322\\
25.08	-0.0750000029802322\\
25.09	2.08166817117217e-016\\
25.1	-0.0500000007450581\\
25.11	-0.0799999982118607\\
25.12	-0.0799999982118607\\
25.13	-0.104999996721745\\
25.14	-0.125\\
25.15	-0.135000005364418\\
25.16	-0.119999997317791\\
25.17	-0.0949999988079071\\
25.18	-0.0900000035762787\\
25.19	-0.115000002086163\\
25.2	-0.0900000035762787\\
25.21	-0.0799999982118607\\
25.22	-0.104999996721745\\
25.23	-0.0900000035762787\\
25.24	-0.0900000035762787\\
25.25	-0.0750000029802322\\
25.26	-0.0750000029802322\\
25.27	-0.0750000029802322\\
25.28	-0.0900000035762787\\
25.29	-0.104999996721745\\
25.3	-0.0900000035762787\\
25.31	-0.0949999988079071\\
25.32	-0.119999997317791\\
25.33	-0.119999997317791\\
25.34	-0.0900000035762787\\
25.35	-0.0350000001490116\\
25.36	-0.0149999996647239\\
25.37	-0.0500000007450581\\
25.38	-0.104999996721745\\
25.39	-0.0949999988079071\\
25.4	-0.119999997317791\\
25.41	-0.135000005364418\\
25.42	-0.104999996721745\\
25.43	-0.0799999982118607\\
25.44	-0.0649999976158142\\
25.45	-0.0900000035762787\\
25.46	-0.0799999982118607\\
25.47	-0.115000002086163\\
25.48	-0.174999997019768\\
25.49	-0.189999997615814\\
25.5	-0.150000005960464\\
25.51	-0.119999997317791\\
25.52	-0.174999997019768\\
25.53	-0.0750000029802322\\
25.54	0.115000002086163\\
25.55	0.0500000007450581\\
25.56	-0.0649999976158142\\
25.57	-0.174999997019768\\
25.58	-0.180000007152557\\
25.59	-0.125\\
25.6	-0.0750000029802322\\
25.61	-0.0199999995529652\\
25.62	-0.0149999996647239\\
25.63	-0.0599999986588955\\
25.64	-0.0350000001490116\\
25.65	-0.0599999986588955\\
25.66	-0.0900000035762787\\
25.67	-0.104999996721745\\
25.68	-0.135000005364418\\
25.69	-0.0799999982118607\\
25.7	-0.0599999986588955\\
25.71	-0.204999998211861\\
25.72	-0.150000005960464\\
25.73	-0.119999997317791\\
25.74	-0.125\\
25.75	-0.0350000001490116\\
25.76	-0.0799999982118607\\
25.77	-0.0799999982118607\\
25.78	-0.0649999976158142\\
25.79	-0.0649999976158142\\
25.8	-0.119999997317791\\
25.81	-0.159999996423721\\
25.82	-0.144999995827675\\
25.83	-0.0949999988079071\\
25.84	-0.00499999988824129\\
25.85	0.0199999995529652\\
25.86	-0.0350000001490116\\
25.87	-0.0649999976158142\\
25.88	-0.0949999988079071\\
25.89	-0.115000002086163\\
25.9	-0.0949999988079071\\
25.91	-0.115000002086163\\
25.92	-0.0799999982118607\\
25.93	-0.119999997317791\\
};
\end{axis}
\end{tikzpicture}%
			\caption{Verhalten des Krümmungs-Störgrößenbeobachters bei aktivem Fahrereingriff}
			\label{abb_messung_fahrer}
		\end{minipage}
	 \hfill
		 \begin{minipage}[t]{0.45\linewidth} 
			\centering
			\setlength\figureheight{6cm} 
			\setlength\figurewidth{4cm}
			% This file was created by matlab2tikz v0.5.0 running on MATLAB 7.11.1.
%Copyright (c) 2008--2014, Nico Schlömer <nico.schloemer@gmail.com>
%All rights reserved.
%Minimal pgfplots version: 1.3
%
%The latest updates can be retrieved from
%  http://www.mathworks.com/matlabcentral/fileexchange/22022-matlab2tikz
%where you can also make suggestions and rate matlab2tikz.
%
\begin{tikzpicture}

\begin{axis}[%
 /pgf/number format/.cd,
        use comma,
        1000 sep={},
width=0.95092\figurewidth,
height=0.264706\figureheight,
at={(0\figurewidth,0\figureheight)},
scale only axis,
every outer x axis line/.append style={black},
every x tick label/.append style={font=\color{black}},
xmin=18,
xmax=26,
xlabel={$t$ [s]},
xlabel near ticks,
xmajorgrids,
every outer y axis line/.append style={black},
every y tick label/.append style={font=\color{black}},
ymin=-0.2,
ymax=0.2,
ylabel={$\Delta d\text{ [m]}$},
ylabel near ticks,
ymajorgrids,
axis x line*=bottom,
axis y line*=left
]
\addplot [color=black,solid,forget plot, line width=1.0]
  table[row sep=crcr]{%
17.97375	0.00420802084173699\\
17.99375	0.00161599053053685\\
18.01375	-0.00096614106169346\\
18.03375	0.00629452986805701\\
18.05375	0.00191130067663581\\
18.07375	-0.000641930319537742\\
18.09375	-0.00318531310096226\\
18.11375	-0.00571914930865969\\
18.13375	0.00158919981364258\\
18.15375	-0.000925437397504947\\
18.17375	-0.00343066310150153\\
18.19375	-0.00592649052508554\\
18.21275	0.00141968150186633\\
18.23375	-0.00105735099733373\\
18.25375	-0.00352513564015355\\
18.27375	0.00384878148078593\\
18.29375	0.0013995121263255\\
18.31375	-0.00104066364470556\\
18.33375	-0.00347154597796084\\
18.35375	0.00393861406876184\\
18.37375	0.001525601092637\\
18.39375	-0.000878264065052647\\
18.41375	0.00655860661505203\\
18.43375	0.00417230357945853\\
18.45375	0.00179498863803884\\
18.47375	0.00925810684766493\\
18.49375	0.00689804818621065\\
18.51375	0.0045468067674248\\
18.53375	0.00220409408866606\\
18.55475	0.00970140389891805\\
18.57375	0.00737575244946109\\
18.59375	0.00505847566569351\\
18.61375	0.0125809119154026\\
18.63375	0.010280356617741\\
18.65375	0.00798801828011486\\
18.67375	0.00570381251178231\\
18.69375	0.0132591203314165\\
18.71375	0.0109912133030106\\
18.73375	0.00873126914679956\\
18.75375	0.00647944115971955\\
18.77375	0.0140665239901598\\
18.79375	0.0118303221003906\\
18.81275	0.00960205042599416\\
18.83275	0.00738148832112895\\
18.85375	0.0149993991681288\\
18.87375	0.0127941533101432\\
18.89375	0.0105964464561792\\
18.91375	0.00840617591874526\\
18.93375	0.016054144501771\\
18.95375	0.0138787513965157\\
18.97375	0.0117106139346177\\
18.99375	0.00954984678895521\\
19.01375	0.0172267670570156\\
19.03375	0.0150802113004977\\
19.05375	0.0129408305685472\\
19.07375	0.0108084474338375\\
19.09375	0.0086829513099298\\
19.11375	0.016394749027512\\
19.13375	0.0142830109816918\\
19.15375	0.0121779725643498\\
19.17375	0.0100797244182682\\
19.19375	0.00798809715153448\\
19.21375	0.00590297988635768\\
19.23375	0.00382446306821116\\
19.25375	0.00175237000382422\\
19.27375	0.00951660191001036\\
19.29375	0.00745719210645035\\
19.31375	0.00540401880677699\\
19.33375	0.00335697127836054\\
19.35475	0.00131612816299898\\
19.37375	-0.000718678634872294\\
19.39375	-0.00274755427715379\\
19.41375	-0.00477042081758539\\
19.43275	-0.0067874523685294\\
19.45375	0.0010309420692951\\
19.47375	-0.000974581930611063\\
19.49375	-0.00297446295972614\\
19.51375	-0.00496880573031255\\
19.53375	-0.00695754406029048\\
19.55375	-0.00894084292314856\\
19.57375	-0.0109188018268354\\
19.59375	-0.0128913557669694\\
19.61375	-0.0148586744368751\\
19.63375	-0.0168208523821169\\
19.65375	-0.00894845544504674\\
19.67375	-0.0109004248989724\\
19.69375	-0.0128474533347842\\
19.71375	-0.014789487412203\\
19.73375	-0.016726687172135\\
19.75375	-0.0186591465741932\\
19.77375	-0.0205868139154894\\
19.79375	-0.0225098526570795\\
19.81375	-0.014599199157812\\
19.83375	-0.0165131410373807\\
19.85375	-0.0184226519110373\\
19.87375	-0.0203278247454914\\
19.89375	-0.022228619335027\\
19.91375	-0.0241251890095335\\
19.93375	-0.0161886203997366\\
19.95375	-0.018076907726702\\
19.97375	-0.019961168506013\\
19.99375	-0.0218414943084873\\
20.01375	-0.0237178560385831\\
20.03375	-0.0157615124534005\\
20.05375	-0.0176303397338167\\
20.07375	-0.0194954127944773\\
20.09375	-0.0213568636849431\\
20.11375	-0.013385980353434\\
20.13375	-0.0152403723214452\\
20.15475	-0.0170913406707429\\
20.17275	-0.0189389771944017\\
20.19375	-0.0207832714951834\\
20.21375	-0.0127956440969959\\
20.23375	-0.0146336042158413\\
20.25375	-0.016468429637333\\
20.27375	-0.0201435078525685\\
20.29375	-0.0121438282148798\\
20.31375	-0.0139698939132944\\
20.33375	-0.0157931069247819\\
20.35375	-0.0176135843375165\\
20.37375	-0.00960279106662432\\
20.39375	-0.0114179282181053\\
20.41375	-0.0150743878226072\\
20.43375	-0.00705605410665111\\
20.45375	-0.00886377814175177\\
20.47375	-0.0125132434870214\\
20.49375	-0.0143164076115201\\
20.51375	-0.00628883576354911\\
20.53375	-0.00993186800731971\\
20.55375	-0.0135729295560894\\
20.57375	-0.00553923418649838\\
20.59375	-0.00917644477360779\\
20.61375	-0.0109673316506664\\
20.63375	-0.0146009648692336\\
20.65375	-0.00840452528772229\\
20.67375	-0.012034840859751\\
20.69375	-0.0156635988379388\\
20.71375	-0.0192908564592642\\
20.73375	-0.0130883134143227\\
20.75375	-0.0167127275625707\\
20.77375	-0.0203358053164591\\
20.79375	-0.0239576118068259\\
20.81375	-0.0177498770959161\\
20.83275	-0.023214344531679\\
20.85375	-0.0268326917788246\\
20.87375	-0.0304499890704473\\
20.89375	-0.0260831250952509\\
20.91375	-0.0296985266416274\\
20.93375	-0.0351582289159653\\
20.95475	-0.0387719321006981\\
20.97375	-0.0344018124729533\\
20.99375	-0.0398592992096147\\
21.01375	-0.0434708786836611\\
21.03375	-0.0390988698351404\\
21.05375	-0.0445546044414828\\
21.07375	-0.0500098583702226\\
21.09375	-0.0554646760329049\\
21.11375	-0.0510908515180457\\
21.13375	-0.0565449524290216\\
21.15375	-0.0521705111367234\\
21.17375	-0.0576240862668715\\
21.19375	-0.0630774685670845\\
21.21375	-0.0587024531738809\\
21.23375	-0.064155590973717\\
21.25375	-0.0597804246476401\\
21.27375	-0.0670788969374723\\
21.29375	-0.0627037619602717\\
21.31375	-0.0681569622366034\\
21.33375	-0.0656274216506754\\
21.35375	-0.07108091825986\\
21.37375	-0.0667063766164859\\
21.39375	-0.0740057113126005\\
21.41375	-0.0696317221446194\\
21.43375	-0.0671034152530101\\
21.45375	-0.0744037335372325\\
21.47375	-0.070030879419932\\
21.49275	-0.0675037898681428\\
21.51275	-0.0748054509043659\\
21.53375	-0.0722793578597218\\
21.55375	-0.0679085779057944\\
21.57375	-0.0752119333596162\\
21.59375	-0.0726876379222152\\
21.61375	-0.0701640083195745\\
21.63375	-0.067641080808087\\
21.65375	-0.0749471951807816\\
21.67375	-0.072425785104631\\
21.69375	-0.0699051848616263\\
21.71375	-0.0673854297089722\\
21.73375	-0.0667115562869163\\
21.75475	-0.0641935632264388\\
21.77375	-0.0715048539678129\\
21.79375	-0.0689887971703631\\
21.81375	-0.0664737563208053\\
21.83375	-0.0658046011021116\\
21.85375	-0.0632916545215378\\
21.87375	-0.0607798222422984\\
21.89375	-0.0601138564765376\\
21.91375	-0.0576043090551064\\
21.93375	-0.0569406079545773\\
21.95375	-0.0544334686990688\\
21.97375	-0.0519276011733774\\
21.99375	-0.0512675397889333\\
22.01375	-0.0506087188732733\\
22.03375	-0.0481067561167223\\
22.05375	-0.0474505459709089\\
22.07375	-0.0449513465862346\\
22.09375	-0.0442978577758431\\
22.11375	-0.0418015351490206\\
22.13275	-0.0411508757862338\\
22.15375	-0.0405016441757371\\
22.17375	-0.0380098164544469\\
22.19375	-0.0373635708402005\\
22.21375	-0.0367188368359406\\
22.23375	-0.0360756402253175\\
22.25375	-0.0335901541408883\\
22.27375	-0.0329501437344746\\
22.29375	-0.0323117330522464\\
22.31375	-0.0316749450764049\\
22.33375	-0.0310398024801035\\
22.35375	-0.0304063276254771\\
22.37375	-0.027931066757489\\
22.39375	-0.0174724339970855\\
22.41375	-0.0168441459869104\\
22.43375	-0.0162176126785667\\
22.45375	-0.01743610158689\\
22.47375	-0.0168130943056908\\
22.49375	-0.0161919027959119\\
22.51375	-0.0155725468662151\\
22.53375	-0.0149550460003751\\
22.55475	-0.0143394193559039\\
22.57375	-0.0137256857624166\\
22.59375	-0.0149566177837266\\
22.61375	-0.01434666295888\\
22.63375	-0.0137386687607748\\
22.65375	-0.0131326393794966\\
22.67375	-0.0045422824730057\\
22.69375	-0.00394014561342404\\
22.71375	-0.00334002332768124\\
22.73375	-0.00458419390438713\\
22.75275	-0.0039880727574122\\
22.77375	-0.0033940273920523\\
22.79375	-0.00464412670742576\\
22.81375	-0.00405418248001066\\
22.83375	-0.0053082456227993\\
22.85375	-0.00472244818129131\\
22.87375	0.00569015203574086\\
22.89375	0.00443002109786628\\
22.91375	0.00500945386981044\\
22.93275	0.00374518022132264\\
22.95375	0.0024788842869623\\
22.97375	0.00305192214852212\\
22.99375	0.00178146788383948\\
23.01375	0.000508959773844797\\
23.03375	0.00107548400206792\\
23.05375	-0.00020129842036809\\
23.07375	0.00834891489897371\\
23.09375	0.00890886960176474\\
23.11375	0.00762574317276599\\
23.13375	0.00634050328386282\\
23.15375	0.00689376668061792\\
23.17375	0.00560414907815909\\
23.19375	0.0043123990454732\\
23.21375	0.0048589241913124\\
23.23375	0.0035628027279726\\
23.25375	0.00226452188579396\\
23.27375	0.00280421734395508\\
23.29375	0.00150147698133463\\
23.31375	0.00203657723843031\\
23.33175	0.000729404300361391\\
23.35475	0.00125988749266615\\
23.37375	-5.17376796089764e-005\\
23.39375	-0.00136558857939484\\
23.41375	-0.000842049895832364\\
23.43375	-0.00216039780413801\\
23.45375	-0.00164151873170715\\
23.47375	-0.00296435646252835\\
23.49375	-0.00245014906718888\\
23.51375	-0.00193832535245742\\
23.53375	-0.00326799842703007\\
23.55375	-0.00276088737285507\\
23.57375	-0.0022561649583972\\
23.59375	-0.00359266905649003\\
23.61375	-0.00309263771288038\\
23.63375	-0.00259499663094509\\
23.65375	-0.00393831685899082\\
23.67375	-0.0034453761405997\\
23.69375	-0.00295484273568203\\
23.71375	-0.00246669659096765\\
23.73375	-0.0038191829070322\\
23.75375	-0.00333571754657402\\
23.77375	-0.00285463335228808\\
23.79375	-0.0023759275256654\\
23.81375	-0.00189959689388619\\
23.83375	-0.00142563791023109\\
23.85375	-0.00095405379404756\\
23.87375	-0.00232251427672603\\
23.89375	-0.00185558544137177\\
23.91375	-0.00139101039241218\\
23.93275	-0.000928783717829784\\
23.95375	-0.000468899633440323\\
23.97375	-0.00984113328234004\\
23.99375	-0.0075487917789232\\
24.01375	-0.00709599902158464\\
24.03375	-0.00664552166151244\\
24.05375	-0.00619735204339822\\
24.07375	-0.00391468416411778\\
24.09375	-0.0133010857605402\\
24.11375	-0.0110232594096171\\
24.13375	-0.00874787401334798\\
24.15475	-0.0083114081628346\\
24.17375	-0.00604078900288219\\
24.19375	-0.013602528062679\\
24.21375	-0.011336728767084\\
24.23375	-0.00907331335847728\\
24.25375	-0.00681226910430066\\
24.27375	-0.00455358285388785\\
24.29375	-0.00229724104042894\\
24.31375	0.00179260285944727\\
24.33375	-0.00578584395938941\\
24.35375	-0.00353654515438118\\
24.37375	0.000546039120619568\\
24.39375	0.00279070484826294\\
24.41375	0.00686853295936096\\
24.43375	0.00910863209740764\\
24.45375	0.0131817745160938\\
24.47375	0.0154173769611425\\
24.49375	0.0194859076640679\\
24.51375	0.0235521241497798\\
24.53275	0.0159508792611209\\
24.55375	0.0200125885124356\\
24.57375	0.0240720456648678\\
24.59375	0.0281292724887954\\
24.61375	0.0321842780075716\\
24.63375	0.0362370869656772\\
24.65375	0.0304574459992448\\
24.67375	0.0345059419836398\\
24.69375	0.0403866999381375\\
24.71375	0.04443092485129\\
24.73375	0.0386427471668052\\
24.75375	0.0445170199784064\\
24.77375	0.0485550295480657\\
24.79375	0.0544250812166207\\
24.81375	0.0486286931949138\\
24.83375	0.0544946556904335\\
24.85375	0.058524750493731\\
24.87375	0.0545563139247487\\
24.89375	0.0604162992446038\\
24.91375	0.0662743519356321\\
24.93375	0.0604664554746388\\
24.95475	0.0663207844349065\\
24.97375	0.0623427827407639\\
24.99375	0.0681934246689226\\
25.01375	0.0623784423737725\\
25.03375	0.0682256233392895\\
25.05375	0.0740711031851022\\
25.07375	0.0700843721398998\\
25.09375	0.0759265125065984\\
25.11375	0.0719365044558207\\
25.13375	0.0661118834189303\\
25.15375	0.0719493791618957\\
25.17375	0.0679547970919243\\
25.19375	0.0737893224280408\\
25.21375	0.0697918403272184\\
25.23375	0.0756235784926562\\
25.25275	0.0697906042209926\\
25.27375	0.0657891143085978\\
25.29375	0.071616971442598\\
25.31375	0.0676129775936491\\
25.33375	0.061775190138313\\
25.35375	0.067599546718013\\
25.37375	0.0635921404131836\\
25.39375	0.0595836597033768\\
25.41375	0.0635723591193438\\
25.43375	0.0595619125090328\\
25.45375	0.0555505281904805\\
25.47375	0.0613689220704634\\
25.49375	0.0555234264541871\\
25.51375	0.0515095001072763\\
25.53375	0.0474948100760431\\
25.55375	0.0533101040315294\\
25.57375	0.0492940121649199\\
25.59375	0.0452772885302304\\
25.61375	0.0394277572279895\\
25.63375	0.0452406454650385\\
25.65375	0.0412223250136798\\
25.67375	0.0372035481245403\\
25.69375	0.0331843709715556\\
25.71375	0.0291648405827929\\
25.73375	0.0349757328205498\\
25.75475	0.0309556351889095\\
25.77375	0.0269353245409336\\
25.79375	0.0229148509352912\\
25.81375	0.0305571211371967\\
25.83375	0.0265364636365675\\
25.85375	0.0225157896825992\\
25.87375	0.0283258813713774\\
25.89375	0.0243053205561194\\
25.91375	0.0202848904168391\\
25.93275	0.0180967853085376\\
25.95375	0.0239075049479811\\
25.97375	0.0198877751002047\\
};
\end{axis}

\begin{axis}[%
width=0.95092\figurewidth,
height=0.264706\figureheight,
at={(0\figurewidth,0.367647\figureheight)},
scale only axis,
every outer x axis line/.append style={black},
every x tick label/.append style={font=\color{black}},
xmin=18,
xmax=26,
xmajorgrids,
every outer y axis line/.append style={black},
every y tick label/.append style={font=\color{black}},
ymin=-0.005,
ymax=0.003,
ylabel={$\kappa\text{ [1/m]}$},
ylabel near ticks,
ymajorgrids,
axis x line*=bottom,
axis y line*=left,
legend style={at={(0.018413,0.734139)},anchor=south west,legend columns=2,legend cell align=left,align=left,draw=black}
]
\addplot [color=black,solid, line width=1.0]
  table[row sep=crcr]{%
17.97375	0.000323821473980823\\
17.99375	0.00045161412906528\\
18.01375	0.000399209081593525\\
18.03375	0.000434593753106015\\
18.05375	0.000445119855728241\\
18.07375	0.00045756984332177\\
18.09375	0.000429477955209978\\
18.11375	0.000180056749310244\\
18.13375	0.000162682145702333\\
18.15375	0.000158315521384546\\
18.17375	0.000100027752142495\\
18.19375	9.75916267243263e-006\\
18.21275	-5.94942101145908e-005\\
18.23375	-2.02311436937398e-005\\
18.25375	-0.000139222986038058\\
18.27375	-0.000152647108222784\\
18.29375	-7.28141781894108e-005\\
18.31375	-0.000163327018777421\\
18.33375	-0.000195235817850069\\
18.35375	-0.000216272828693524\\
18.37375	-0.000139692987940828\\
18.39375	-0.000131617058448218\\
18.41375	-3.36206007454451e-005\\
18.43375	4.47913983356136e-005\\
18.45375	0.000140963477348756\\
18.47375	0.00011884394050142\\
18.49375	0.000273287255974575\\
18.51375	0.000340368666405241\\
18.53375	0.000346826265890152\\
18.55475	0.000350545902230609\\
18.57375	0.000441789863853653\\
18.59375	0.00045727860726829\\
18.61375	0.000362771116426865\\
18.63375	0.000456023119103066\\
18.65375	0.000472649775070843\\
18.67375	0.000439883336101985\\
18.69375	0.000414973701032743\\
18.71375	0.000389294512630722\\
18.73375	0.000399832083636455\\
18.75375	0.00036870166207992\\
18.77375	0.000447529635901847\\
18.79375	0.000423521561929034\\
18.81275	0.000434959545408876\\
18.83275	0.000306671407049181\\
18.85375	0.000294851605316549\\
18.87375	0.000286700364880953\\
18.89375	0.000218514111576058\\
18.91375	0.000212822214341372\\
18.93375	0.000122140998494146\\
18.95375	0.000136761433785597\\
18.97375	0.000189285978879197\\
18.99375	0.000198382119548506\\
19.01375	0.000118604128106723\\
19.03375	0.000140437608716426\\
19.05375	9.89756761014846e-005\\
19.07375	0.000117324041604288\\
19.09375	8.70774333736118e-006\\
19.11375	-7.57457174859962e-005\\
19.13375	4.88013889473028e-005\\
19.15375	1.00284271825252e-005\\
19.17375	3.05019906929371e-005\\
19.19375	2.24477450129483e-005\\
19.21375	-0.000100764132413352\\
19.23375	-3.31758302723329e-005\\
19.25375	-0.000172711951899935\\
19.27375	-0.000175637436733794\\
19.29375	3.36527251024043e-005\\
19.31375	7.83675025637052e-005\\
19.33375	7.97356307252351e-005\\
19.35475	5.29302991310901e-005\\
19.37375	-8.90579001756781e-005\\
19.39375	-0.000136495235443338\\
19.41375	-0.000186631073634458\\
19.43275	-0.000335077311152974\\
19.45375	-0.00034018442078567\\
19.47375	-0.000227481230779736\\
19.49375	-0.000171394983312633\\
19.51375	-0.000154176777073436\\
19.53375	-0.000261048893083489\\
19.55375	-0.000280561860464454\\
19.57375	-0.000308955209667207\\
19.59375	-0.00034408002722953\\
19.61375	-0.000383770584746879\\
19.63375	-0.0004258480491716\\
19.65375	-0.000430972843209058\\
19.67375	-0.00022399409233144\\
19.69375	-0.000180957181531054\\
19.71375	-0.000180304258903954\\
19.73375	-0.000207928175852328\\
19.75375	-0.00015634678906037\\
19.77375	-0.00012276926888089\\
19.79375	-0.000201692149281296\\
19.81375	-0.000150530644718124\\
19.83375	6.76014071551497e-006\\
19.85375	9.6970207476391e-005\\
19.87375	4.30820787931635e-005\\
19.89375	-3.85432456040216e-005\\
19.91375	-0.000134545928354282\\
19.93375	-0.000100525061895634\\
19.95375	-5.68605793210454e-005\\
19.97375	-7.17471632411205e-005\\
19.99375	-0.000125007907298938\\
20.01375	-0.000199176453987034\\
20.03375	-0.000147694619098311\\
20.05375	-8.95386096327253e-005\\
20.07375	4.94573424291097e-006\\
20.09375	5.43784453360826e-005\\
20.11375	0.000109832986858582\\
20.13375	0.000158836452253588\\
20.15475	0.000236925732392618\\
20.17275	0.000265600719654958\\
20.19375	0.000162315141543912\\
20.21375	-1.71093094290202e-005\\
20.23375	1.59704752620531e-005\\
20.25375	-0.000105690192223767\\
20.27375	-0.000262969819497784\\
20.29375	-0.000317286016729836\\
20.31375	-0.000349947319290289\\
20.33375	-0.000429817246776707\\
20.35375	-0.000536366249997246\\
20.37375	-0.000715467866698612\\
20.39375	-0.000678601356612284\\
20.41375	-0.000902696239761023\\
20.43375	-0.00093786257901036\\
20.45375	-0.0010615598354044\\
20.47375	-0.00134183385598381\\
20.49375	-0.00156488980526758\\
20.51375	-0.00174737846784415\\
20.53375	-0.00180937362651304\\
20.55375	-0.00194920543529313\\
20.57375	-0.00199023444176454\\
20.59375	-0.00202308634283124\\
20.61375	-0.00213516698950846\\
20.63375	-0.00217844223409183\\
20.65375	-0.00223805821487355\\
20.67375	-0.00231180197161561\\
20.69375	-0.00236680071906846\\
20.71375	-0.00247894455603672\\
20.73375	-0.00258951184716443\\
20.75375	-0.00260509301863177\\
20.77375	-0.00260136061502081\\
20.79375	-0.0027543649585232\\
20.81375	-0.002801278595139\\
20.83275	-0.00276282235567111\\
20.85375	-0.00272888390044452\\
20.87375	-0.00284587730514393\\
20.89375	-0.00286121847512024\\
20.91375	-0.00280378535957563\\
20.93375	-0.00282466917410288\\
20.95475	-0.00291899050677228\\
20.97375	-0.00290210031655685\\
20.99375	-0.0029166366546577\\
21.01375	-0.00281388964436419\\
21.03375	-0.00282409989934914\\
21.05375	-0.00277079537991434\\
21.07375	-0.00282127212504977\\
21.09375	-0.00283725841705991\\
21.11375	-0.00286444297304743\\
21.13375	-0.00290390358148723\\
21.15375	-0.00288081113152738\\
21.17375	-0.00289466470816514\\
21.19375	-0.00290405546436051\\
21.21375	-0.00293796361147142\\
21.23375	-0.00288528576330111\\
21.25375	-0.00298850959545578\\
21.27375	-0.00302517510169915\\
21.29375	-0.00303931396188964\\
21.31375	-0.00318372354083798\\
21.33375	-0.00316660945537203\\
21.35375	-0.00310890329871395\\
21.37375	-0.00320467456586801\\
21.39375	-0.0030176066587292\\
21.41375	-0.00303116452045752\\
21.43375	-0.00282692109798413\\
21.45375	-0.00263186174655712\\
21.47375	-0.0027687962596189\\
21.49275	-0.00269166200713433\\
21.51275	-0.00240923710455829\\
21.53375	-0.002468978924594\\
21.55375	-0.00244006572911617\\
21.57375	-0.00239276058360241\\
21.59375	-0.00246221988442664\\
21.61375	-0.00245055592318001\\
21.63375	-0.00238661885103895\\
21.65375	-0.00243921857089237\\
21.67375	-0.00249369763786234\\
21.69375	-0.00256672981253389\\
21.71375	-0.00258418916909743\\
21.73375	-0.00256801054152027\\
21.75475	-0.00243950020031596\\
21.77375	-0.00232797899892127\\
21.79375	-0.00255974997561125\\
21.81375	-0.00248385976502981\\
21.83375	-0.00247319984559621\\
21.85375	-0.00245321153214046\\
21.87375	-0.00239862947682175\\
21.89375	-0.0023278630578567\\
21.91375	-0.00237760489533707\\
21.93375	-0.00229904326457157\\
21.95375	-0.00233949604509737\\
21.97375	-0.00224448388315604\\
21.99375	-0.0022509254867907\\
22.01375	-0.00217266777843822\\
22.03375	-0.0022125335107947\\
22.05375	-0.00222905196239842\\
22.07375	-0.00225043335292683\\
22.09375	-0.00225039873950624\\
22.11375	-0.00225529349722467\\
22.13275	-0.00223854542775768\\
22.15375	-0.00235002303364477\\
22.17375	-0.00234183447985816\\
22.19375	-0.00241668158946598\\
22.21375	-0.00249756625566407\\
22.23375	-0.00246584030388298\\
22.25375	-0.00253695010054124\\
22.27375	-0.00246401411199097\\
22.29375	-0.00251512184032746\\
22.31375	-0.00245954704571414\\
22.33375	-0.00252093420268292\\
22.35375	-0.00246924971489825\\
22.37375	-0.00252237976003563\\
22.39375	-0.00238459511940376\\
22.41375	-0.00222780060816951\\
22.43375	-0.00215565833108032\\
22.45375	-0.00203607813739035\\
22.47375	-0.00198647893802207\\
22.49375	-0.00206482824057716\\
22.51375	-0.00203597040155152\\
22.53375	-0.00212503431577072\\
22.55475	-0.0022114791067897\\
22.57375	-0.00229231994322249\\
22.59375	-0.00237406370277061\\
22.61375	-0.00236355265670791\\
22.63375	-0.0025592481012526\\
22.65375	-0.00261576835105801\\
22.67375	-0.00261435396732459\\
22.69375	-0.00247645359180749\\
22.71375	-0.00239331846780674\\
22.73375	-0.00224454453566857\\
22.75275	-0.00227326053692674\\
22.77375	-0.00230453891415206\\
22.79375	-0.0023465800952959\\
22.81375	-0.00241829992239668\\
22.83375	-0.00236982351231996\\
22.85375	-0.00245905349557895\\
22.87375	-0.00247574158799415\\
22.89375	-0.00221917446168961\\
22.91375	-0.00206514520708224\\
22.93275	-0.00207451541645559\\
22.95375	-0.00203220827819631\\
22.97375	-0.00213953794457727\\
22.99375	-0.00223980912063164\\
23.01375	-0.00225029407087832\\
23.03375	-0.00238375298826537\\
23.05375	-0.00249182205274264\\
23.07375	-0.00255976256561542\\
23.09375	-0.00246543196318904\\
23.11375	-0.0024206999871339\\
23.13375	-0.00244604853193489\\
23.15375	-0.0023853714834159\\
23.17375	-0.00232938870699695\\
23.19375	-0.00242660153882729\\
23.21375	-0.00241725990435068\\
23.23375	-0.00239566136541274\\
23.25375	-0.00251416590522366\\
23.27375	-0.00252122925673683\\
23.29375	-0.00262443806374023\\
23.31375	-0.00273418107606645\\
23.33175	-0.00270251327788349\\
23.35475	-0.00268485937600365\\
23.37375	-0.00265382232815122\\
23.39375	-0.00276207655252082\\
23.41375	-0.00264190472995799\\
23.43375	-0.00273994372032036\\
23.45375	-0.00272997741134997\\
23.47375	-0.00270447022082425\\
23.49375	-0.00280923290953783\\
23.51375	-0.00265529711504313\\
23.53375	-0.00261892354341589\\
23.55375	-0.00260842981766129\\
23.57375	-0.00258301217911086\\
23.59375	-0.00244509571336911\\
23.61375	-0.00245513439195081\\
23.63375	-0.00245310693349064\\
23.65375	-0.0024540645415044\\
23.67375	-0.00248056829685377\\
23.69375	-0.00249088858757041\\
23.71375	-0.00249127728733057\\
23.73375	-0.0024948787812187\\
23.75375	-0.00252373770645726\\
23.77375	-0.00242124296452652\\
23.79375	-0.00242792209739422\\
23.81375	-0.00242957048099189\\
23.83375	-0.0023141483695082\\
23.85375	-0.00231754531194331\\
23.87375	-0.00233167399900701\\
23.89375	-0.00226107134148708\\
23.91375	-0.00229598202246269\\
23.93275	-0.00209735183925201\\
23.95375	-0.00201785056091119\\
23.97375	-0.00177316510202208\\
23.99375	-0.00169045879731229\\
24.01375	-0.00141199124367529\\
24.03375	-0.0012392586036165\\
24.05375	-0.00107084087573005\\
24.07375	-0.000789286685981336\\
24.09375	-0.000660338661701205\\
24.11375	-0.000587988623384013\\
24.13375	-0.000429579520122222\\
24.15475	-0.000236627942679273\\
24.17375	-0.000151460220940395\\
24.19375	-7.76600936107245e-005\\
24.21375	-0.000139858313938592\\
24.23375	-0.000218477477458518\\
24.25375	-0.000116699515724949\\
24.27375	-7.62090646837579e-005\\
24.29375	-2.80676250000189e-007\\
24.31375	0.000103498989968883\\
24.33375	0.000101417923033581\\
24.35375	7.76706962365598e-005\\
24.37375	2.90870227995424e-005\\
24.39375	6.60704535209348e-005\\
24.41375	0.000141173164562945\\
24.43375	0.00015559446147936\\
24.45375	0.000184229688060798\\
24.47375	0.000247003240572216\\
24.49375	0.000313403026644311\\
24.51375	0.000413080616349436\\
24.53275	0.000478896789640639\\
24.55375	0.000293634934129457\\
24.57375	0.000315586487662916\\
24.59375	0.00029686351924079\\
24.61375	0.000431344751727054\\
24.63375	0.000487094552950158\\
24.65375	0.000518675534878985\\
24.67375	0.000434359424619038\\
24.69375	0.000440095827122616\\
24.71375	0.000524060544901119\\
24.73375	0.000484569026725949\\
24.75375	0.000343609679827388\\
24.77375	0.000414669226143357\\
24.79375	0.000432453184698093\\
24.81375	0.000353369866336333\\
24.83375	0.000380518342477067\\
24.85375	0.000395182394451402\\
24.87375	0.000409226747948122\\
24.89375	0.000350522808301341\\
24.91375	0.000390184059280244\\
24.93375	0.000339498956451529\\
24.95475	0.000392026488954283\\
24.97375	0.000397001458849489\\
24.99375	0.000339500674580071\\
25.01375	0.000534750819528195\\
25.03375	0.000427594539442169\\
25.05375	0.000442602861748597\\
25.07375	0.000484451050742458\\
25.09375	0.000444693290986896\\
25.11375	0.00044653166395918\\
25.13375	0.000442156742684429\\
25.15375	0.000387699440962677\\
25.17375	0.000422872758135832\\
25.19375	0.000308142672720352\\
25.21375	0.000382530078582161\\
25.23375	0.000302618872766323\\
25.25275	0.000279494290951488\\
25.27375	0.000241568192815934\\
25.29375	0.000102915984848561\\
25.31375	6.01369907256662e-005\\
25.33375	-7.18440608481656e-005\\
25.35375	-0.000320119865237572\\
25.37375	-0.000350538820761782\\
25.39375	-0.00034861432921719\\
25.41375	-0.000555754136001189\\
25.43375	-0.00047345524448526\\
25.45375	-0.000463807534672573\\
25.47375	-0.000557440971237212\\
25.49375	-0.000362399435651973\\
25.51375	-0.000384305442581676\\
25.53375	-0.000449069544550439\\
25.55375	-0.000406441157596343\\
25.57375	-0.000269035019867403\\
25.59375	-0.000219391118246141\\
25.61375	-0.000239995342265097\\
25.63375	-0.000285626342997826\\
25.65375	-0.000124005251971827\\
25.67375	-0.000145003911010797\\
25.69375	-0.00011620908858934\\
25.71375	-0.000127241605756966\\
25.73375	-0.000226173188475251\\
25.75475	-0.000112533452632019\\
25.77375	-7.43371729986891e-005\\
25.79375	-0.000189126594918143\\
25.81375	-0.000180916328910701\\
25.83375	-4.71907400696992e-005\\
25.85375	-0.000103487928654826\\
25.87375	-7.1279707641639e-005\\
25.89375	-5.60636309245096e-005\\
25.91375	-0.000223286890643139\\
25.93275	-0.000325539709302636\\
25.95375	-0.000290537786726986\\
25.97375	-0.000266694459689462\\
};
\addlegendentry{$\kappa{}_\mathrm{d}'$};



\addplot [color=black,solid,forget plot, line width=1.0]
  table[row sep=crcr]{%
17.97375	0.000323821473980823\\
17.99375	0.00045161412906528\\
18.01375	0.000399209081593525\\
18.03375	0.000434593753106015\\
18.05375	0.000445119855728241\\
18.07375	0.00045756984332177\\
18.09375	0.000429477955209978\\
18.11375	0.000180056749310244\\
18.13375	0.000162682145702333\\
18.15375	0.000158315521384546\\
18.17375	0.000100027752142495\\
18.19375	9.75916267243263e-006\\
18.21275	-5.94942101145908e-005\\
18.23375	-2.02311436937398e-005\\
18.25375	-0.000139222986038058\\
18.27375	-0.000152647108222784\\
18.29375	-7.28141781894108e-005\\
18.31375	-0.000163327018777421\\
18.33375	-0.000195235817850069\\
18.35375	-0.000216272828693524\\
18.37375	-0.000139692987940828\\
18.39375	-0.000131617058448218\\
18.41375	-3.36206007454451e-005\\
18.43375	4.47913983356136e-005\\
18.45375	0.000140963477348756\\
18.47375	0.00011884394050142\\
18.49375	0.000273287255974575\\
18.51375	0.000340368666405241\\
18.53375	0.000346826265890152\\
18.55475	0.000350545902230609\\
18.57375	0.000441789863853653\\
18.59375	0.00045727860726829\\
18.61375	0.000362771116426865\\
18.63375	0.000456023119103066\\
18.65375	0.000472649775070843\\
18.67375	0.000439883336101985\\
18.69375	0.000414973701032743\\
18.71375	0.000389294512630722\\
18.73375	0.000399832083636455\\
18.75375	0.00036870166207992\\
18.77375	0.000447529635901847\\
18.79375	0.000423521561929034\\
18.81275	0.000434959545408876\\
18.83275	0.000306671407049181\\
18.85375	0.000294851605316549\\
18.87375	0.000286700364880953\\
18.89375	0.000218514111576058\\
18.91375	0.000212822214341372\\
18.93375	0.000122140998494146\\
18.95375	0.000136761433785597\\
18.97375	0.000189285978879197\\
18.99375	0.000198382119548506\\
19.01375	0.000118604128106723\\
19.03375	0.000140437608716426\\
19.05375	9.89756761014846e-005\\
19.07375	0.000117324041604288\\
19.09375	8.70774333736118e-006\\
19.11375	-7.57457174859962e-005\\
19.13375	4.88013889473028e-005\\
19.15375	1.00284271825252e-005\\
19.17375	3.05019906929371e-005\\
19.19375	2.24477450129483e-005\\
19.21375	-0.000100764132413352\\
19.23375	-3.31758302723329e-005\\
19.25375	-0.000172711951899935\\
19.27375	-0.000175637436733794\\
19.29375	3.36527251024043e-005\\
19.31375	7.83675025637052e-005\\
19.33375	7.97356307252351e-005\\
19.35475	5.29302991310901e-005\\
19.37375	-8.90579001756781e-005\\
19.39375	-0.000136495235443338\\
19.41375	-0.000186631073634458\\
19.43275	-0.000335077311152974\\
19.45375	-0.00034018442078567\\
19.47375	-0.000227481230779736\\
19.49375	-0.000171394983312633\\
19.51375	-0.000154176777073436\\
19.53375	-0.000261048893083489\\
19.55375	-0.000280561860464454\\
19.57375	-0.000308955209667207\\
19.59375	-0.00034408002722953\\
19.61375	-0.000383770584746879\\
19.63375	-0.0004258480491716\\
19.65375	-0.000430972843209058\\
19.67375	-0.00022399409233144\\
19.69375	-0.000180957181531054\\
19.71375	-0.000180304258903954\\
19.73375	-0.000207928175852328\\
19.75375	-0.00015634678906037\\
19.77375	-0.00012276926888089\\
19.79375	-0.000201692149281296\\
19.81375	-0.000150530644718124\\
19.83375	6.76014071551497e-006\\
19.85375	9.6970207476391e-005\\
19.87375	4.30820787931635e-005\\
19.89375	-3.85432456040216e-005\\
19.91375	-0.000134545928354282\\
19.93375	-0.000100525061895634\\
19.95375	-5.68605793210454e-005\\
19.97375	-7.17471632411205e-005\\
19.99375	-0.000125007907298938\\
20.01375	-0.000199176453987034\\
20.03375	-0.000147694619098311\\
20.05375	-8.95386096327253e-005\\
20.07375	4.94573424291097e-006\\
20.09375	5.43784453360826e-005\\
20.11375	0.000109832986858582\\
20.13375	0.000158836452253588\\
20.15475	0.000236925732392618\\
20.17275	0.000265600719654958\\
20.19375	0.000162315141543912\\
20.21375	-1.71093094290202e-005\\
20.23375	1.59704752620531e-005\\
20.25375	-0.000105690192223767\\
20.27375	-0.000262969819497784\\
20.29375	-0.000317286016729836\\
20.31375	-0.000349947319290289\\
20.33375	-0.000429817246776707\\
20.35375	-0.000536366249997246\\
20.37375	-0.000715467866698612\\
20.39375	-0.000678601356612284\\
20.41375	-0.000902696239761023\\
20.43375	-0.00093786257901036\\
20.45375	-0.0010615598354044\\
20.47375	-0.00134183385598381\\
20.49375	-0.00156488980526758\\
20.51375	-0.00174737846784415\\
20.53375	-0.00180937362651304\\
20.55375	-0.00194920543529313\\
20.57375	-0.00199023444176454\\
20.59375	-0.00202308634283124\\
20.61375	-0.00213516698950846\\
20.63375	-0.00217844223409183\\
20.65375	-0.00223805821487355\\
20.67375	-0.00231180197161561\\
20.69375	-0.00236680071906846\\
20.71375	-0.00247894455603672\\
20.73375	-0.00258951184716443\\
20.75375	-0.00260509301863177\\
20.77375	-0.00260136061502081\\
20.79375	-0.0027543649585232\\
20.81375	-0.002801278595139\\
20.83275	-0.00276282235567111\\
20.85375	-0.00272888390044452\\
20.87375	-0.00284587730514393\\
20.89375	-0.00286121847512024\\
20.91375	-0.00280378535957563\\
20.93375	-0.00282466917410288\\
20.95475	-0.00291899050677228\\
20.97375	-0.00290210031655685\\
20.99375	-0.0029166366546577\\
21.01375	-0.00281388964436419\\
21.03375	-0.00282409989934914\\
21.05375	-0.00277079537991434\\
21.07375	-0.00282127212504977\\
21.09375	-0.00283725841705991\\
21.11375	-0.00286444297304743\\
21.13375	-0.00290390358148723\\
21.15375	-0.00288081113152738\\
21.17375	-0.00289466470816514\\
21.19375	-0.00290405546436051\\
21.21375	-0.00293796361147142\\
21.23375	-0.00288528576330111\\
21.25375	-0.00298850959545578\\
21.27375	-0.00302517510169915\\
21.29375	-0.00303931396188964\\
21.31375	-0.00318372354083798\\
21.33375	-0.00316660945537203\\
21.35375	-0.00310890329871395\\
21.37375	-0.00320467456586801\\
21.39375	-0.0030176066587292\\
21.41375	-0.00303116452045752\\
21.43375	-0.00282692109798413\\
21.45375	-0.00263186174655712\\
21.47375	-0.0027687962596189\\
21.49275	-0.00269166200713433\\
21.51275	-0.00240923710455829\\
21.53375	-0.002468978924594\\
21.55375	-0.00244006572911617\\
21.57375	-0.00239276058360241\\
21.59375	-0.00246221988442664\\
21.61375	-0.00245055592318001\\
21.63375	-0.00238661885103895\\
21.65375	-0.00243921857089237\\
21.67375	-0.00249369763786234\\
21.69375	-0.00256672981253389\\
21.71375	-0.00258418916909743\\
21.73375	-0.00256801054152027\\
21.75475	-0.00243950020031596\\
21.77375	-0.00232797899892127\\
21.79375	-0.00255974997561125\\
21.81375	-0.00248385976502981\\
21.83375	-0.00247319984559621\\
21.85375	-0.00245321153214046\\
21.87375	-0.00239862947682175\\
21.89375	-0.0023278630578567\\
21.91375	-0.00237760489533707\\
21.93375	-0.00229904326457157\\
21.95375	-0.00233949604509737\\
21.97375	-0.00224448388315604\\
21.99375	-0.0022509254867907\\
22.01375	-0.00217266777843822\\
22.03375	-0.0022125335107947\\
22.05375	-0.00222905196239842\\
22.07375	-0.00225043335292683\\
22.09375	-0.00225039873950624\\
22.11375	-0.00225529349722467\\
22.13275	-0.00223854542775768\\
22.15375	-0.00235002303364477\\
22.17375	-0.00234183447985816\\
22.19375	-0.00241668158946598\\
22.21375	-0.00249756625566407\\
22.23375	-0.00246584030388298\\
22.25375	-0.00253695010054124\\
22.27375	-0.00246401411199097\\
22.29375	-0.00251512184032746\\
22.31375	-0.00245954704571414\\
22.33375	-0.00252093420268292\\
22.35375	-0.00246924971489825\\
22.37375	-0.00252237976003563\\
22.39375	-0.00238459511940376\\
22.41375	-0.00222780060816951\\
22.43375	-0.00215565833108032\\
22.45375	-0.00203607813739035\\
22.47375	-0.00198647893802207\\
22.49375	-0.00206482824057716\\
22.51375	-0.00203597040155152\\
22.53375	-0.00212503431577072\\
22.55475	-0.0022114791067897\\
22.57375	-0.00229231994322249\\
22.59375	-0.00237406370277061\\
22.61375	-0.00236355265670791\\
22.63375	-0.0025592481012526\\
22.65375	-0.00261576835105801\\
22.67375	-0.00261435396732459\\
22.69375	-0.00247645359180749\\
22.71375	-0.00239331846780674\\
22.73375	-0.00224454453566857\\
22.75275	-0.00227326053692674\\
22.77375	-0.00230453891415206\\
22.79375	-0.0023465800952959\\
22.81375	-0.00241829992239668\\
22.83375	-0.00236982351231996\\
22.85375	-0.00245905349557895\\
22.87375	-0.00247574158799415\\
22.89375	-0.00221917446168961\\
22.91375	-0.00206514520708224\\
22.93275	-0.00207451541645559\\
22.95375	-0.00203220827819631\\
22.97375	-0.00213953794457727\\
22.99375	-0.00223980912063164\\
23.01375	-0.00225029407087832\\
23.03375	-0.00238375298826537\\
23.05375	-0.00249182205274264\\
23.07375	-0.00255976256561542\\
23.09375	-0.00246543196318904\\
23.11375	-0.0024206999871339\\
23.13375	-0.00244604853193489\\
23.15375	-0.0023853714834159\\
23.17375	-0.00232938870699695\\
23.19375	-0.00242660153882729\\
23.21375	-0.00241725990435068\\
23.23375	-0.00239566136541274\\
23.25375	-0.00251416590522366\\
23.27375	-0.00252122925673683\\
23.29375	-0.00262443806374023\\
23.31375	-0.00273418107606645\\
23.33175	-0.00270251327788349\\
23.35475	-0.00268485937600365\\
23.37375	-0.00265382232815122\\
23.39375	-0.00276207655252082\\
23.41375	-0.00264190472995799\\
23.43375	-0.00273994372032036\\
23.45375	-0.00272997741134997\\
23.47375	-0.00270447022082425\\
23.49375	-0.00280923290953783\\
23.51375	-0.00265529711504313\\
23.53375	-0.00261892354341589\\
23.55375	-0.00260842981766129\\
23.57375	-0.00258301217911086\\
23.59375	-0.00244509571336911\\
23.61375	-0.00245513439195081\\
23.63375	-0.00245310693349064\\
23.65375	-0.0024540645415044\\
23.67375	-0.00248056829685377\\
23.69375	-0.00249088858757041\\
23.71375	-0.00249127728733057\\
23.73375	-0.0024948787812187\\
23.75375	-0.00252373770645726\\
23.77375	-0.00242124296452652\\
23.79375	-0.00242792209739422\\
23.81375	-0.00242957048099189\\
23.83375	-0.0023141483695082\\
23.85375	-0.00231754531194331\\
23.87375	-0.00233167399900701\\
23.89375	-0.00226107134148708\\
23.91375	-0.00229598202246269\\
23.93275	-0.00209735183925201\\
23.95375	-0.00201785056091119\\
23.97375	-0.00177316510202208\\
23.99375	-0.00169045879731229\\
24.01375	-0.00141199124367529\\
24.03375	-0.0012392586036165\\
24.05375	-0.00107084087573005\\
24.07375	-0.000789286685981336\\
24.09375	-0.000660338661701205\\
24.11375	-0.000587988623384013\\
24.13375	-0.000429579520122222\\
24.15475	-0.000236627942679273\\
24.17375	-0.000151460220940395\\
24.19375	-7.76600936107245e-005\\
24.21375	-0.000139858313938592\\
24.23375	-0.000218477477458518\\
24.25375	-0.000116699515724949\\
24.27375	-7.62090646837579e-005\\
24.29375	-2.80676250000189e-007\\
24.31375	0.000103498989968883\\
24.33375	0.000101417923033581\\
24.35375	7.76706962365598e-005\\
24.37375	2.90870227995424e-005\\
24.39375	6.60704535209348e-005\\
24.41375	0.000141173164562945\\
24.43375	0.00015559446147936\\
24.45375	0.000184229688060798\\
24.47375	0.000247003240572216\\
24.49375	0.000313403026644311\\
24.51375	0.000413080616349436\\
24.53275	0.000478896789640639\\
24.55375	0.000293634934129457\\
24.57375	0.000315586487662916\\
24.59375	0.00029686351924079\\
24.61375	0.000431344751727054\\
24.63375	0.000487094552950158\\
24.65375	0.000518675534878985\\
24.67375	0.000434359424619038\\
24.69375	0.000440095827122616\\
24.71375	0.000524060544901119\\
24.73375	0.000484569026725949\\
24.75375	0.000343609679827388\\
24.77375	0.000414669226143357\\
24.79375	0.000432453184698093\\
24.81375	0.000353369866336333\\
24.83375	0.000380518342477067\\
24.85375	0.000395182394451402\\
24.87375	0.000409226747948122\\
24.89375	0.000350522808301341\\
24.91375	0.000390184059280244\\
24.93375	0.000339498956451529\\
24.95475	0.000392026488954283\\
24.97375	0.000397001458849489\\
24.99375	0.000339500674580071\\
25.01375	0.000534750819528195\\
25.03375	0.000427594539442169\\
25.05375	0.000442602861748597\\
25.07375	0.000484451050742458\\
25.09375	0.000444693290986896\\
25.11375	0.00044653166395918\\
25.13375	0.000442156742684429\\
25.15375	0.000387699440962677\\
25.17375	0.000422872758135832\\
25.19375	0.000308142672720352\\
25.21375	0.000382530078582161\\
25.23375	0.000302618872766323\\
25.25275	0.000279494290951488\\
25.27375	0.000241568192815934\\
25.29375	0.000102915984848561\\
25.31375	6.01369907256662e-005\\
25.33375	-7.18440608481656e-005\\
25.35375	-0.000320119865237572\\
25.37375	-0.000350538820761782\\
25.39375	-0.00034861432921719\\
25.41375	-0.000555754136001189\\
25.43375	-0.00047345524448526\\
25.45375	-0.000463807534672573\\
25.47375	-0.000557440971237212\\
25.49375	-0.000362399435651973\\
25.51375	-0.000384305442581676\\
25.53375	-0.000449069544550439\\
25.55375	-0.000406441157596343\\
25.57375	-0.000269035019867403\\
25.59375	-0.000219391118246141\\
25.61375	-0.000239995342265097\\
25.63375	-0.000285626342997826\\
25.65375	-0.000124005251971827\\
25.67375	-0.000145003911010797\\
25.69375	-0.00011620908858934\\
25.71375	-0.000127241605756966\\
25.73375	-0.000226173188475251\\
25.75475	-0.000112533452632019\\
25.77375	-7.43371729986891e-005\\
25.79375	-0.000189126594918143\\
25.81375	-0.000180916328910701\\
25.83375	-4.71907400696992e-005\\
25.85375	-0.000103487928654826\\
25.87375	-7.1279707641639e-005\\
25.89375	-5.60636309245096e-005\\
25.91375	-0.000223286890643139\\
25.93275	-0.000325539709302636\\
25.95375	-0.000290537786726986\\
25.97375	-0.000266694459689462\\
};


\addplot [color=light-gray,line width=1.0]
  table[row sep=crcr]{%
17.97375	-0.000109741569093609\\
17.99375	-2.84743639141675e-005\\
18.01375	-3.15651120133146e-005\\
18.03375	3.99107703074982e-005\\
18.05375	2.74871274458214e-005\\
18.07375	9.0354999323499e-005\\
18.09375	0.000147150153693112\\
18.11375	4.50011507948981e-005\\
18.13375	0.000101559505449994\\
18.15375	7.82077757219143e-005\\
18.17375	5.71216365293629e-005\\
18.19375	3.88309999909723e-005\\
18.21275	2.30312874404388e-005\\
18.23375	1.08354668831671e-005\\
18.25375	-7.4919894708286e-005\\
18.27375	-7.47269845496809e-005\\
18.29375	-7.25711719083275e-005\\
18.31375	-0.00014616789187268\\
18.33375	-0.000135741017031749\\
18.35375	-0.000124682134092586\\
18.37375	-0.00011494709951226\\
18.39375	-0.00010624174848388\\
18.41375	-2.40018287883962e-005\\
18.43375	-2.73379825564923e-005\\
18.45375	4.12651085055649e-005\\
18.47375	2.42797107902449e-005\\
18.49375	7.88169775189501e-005\\
18.51375	0.000123868249322105\\
18.53375	0.000159400305807694\\
18.55475	0.000185885543158834\\
18.57375	0.000203888381936245\\
18.59375	0.000215539571821987\\
18.61375	0.000145879412827945\\
18.63375	0.000155507650141429\\
18.65375	0.00016367389989113\\
18.67375	0.000171048477260793\\
18.69375	0.000177883129107371\\
18.71375	0.000108097697500276\\
18.73375	0.000120852574515586\\
18.75375	0.000134736946789581\\
18.77375	0.000224947212890107\\
18.79375	0.000155842308968326\\
18.81275	0.000168056235626272\\
18.83275	0.000104493183222035\\
18.85375	0.000123463093079395\\
18.87375	6.6757687727669e-005\\
18.89375	1.60347087788996e-005\\
18.91375	4.81712728210701e-005\\
18.93375	3.94071837078038e-006\\
18.95375	-3.54230925045262e-005\\
18.97375	6.97276431503984e-006\\
18.99375	4.79037731982114e-005\\
19.01375	9.63217838025023e-006\\
19.03375	-2.63162808491506e-005\\
19.05375	-5.92989109735701e-005\\
19.07375	-1.20177885182898e-005\\
19.09375	-4.43942843186126e-005\\
19.11375	-7.62293950702032e-005\\
19.13375	-2.91975061827887e-005\\
19.15375	-6.25987634403478e-005\\
19.17375	-1.85147220793589e-005\\
19.19375	2.09311636321359e-005\\
19.21375	-1.98435821703872e-005\\
19.23375	9.35106150363971e-005\\
19.25375	4.40665099031754e-005\\
19.27375	7.05073737507591e-005\\
19.29375	0.000169604948716613\\
19.31375	0.000181791805524014\\
19.33375	0.000188884138633014\\
19.35475	0.00019218396099412\\
19.37375	0.000116629104436978\\
19.39375	0.000121778595541663\\
19.41375	0.000126497990276262\\
19.43275	5.40755291998419e-005\\
19.45375	6.30034601540867e-005\\
19.47375	7.22718543665006e-005\\
19.49375	8.11537833344531e-005\\
19.51375	8.93456501271996e-005\\
19.53375	2.02449548734168e-005\\
19.55375	3.23479863444908e-005\\
19.57375	4.3603782098842e-005\\
19.59375	5.19462236961943e-005\\
19.61375	5.69074041081535e-005\\
19.63375	5.86988728109314e-005\\
19.65375	5.75190593196574e-005\\
19.67375	0.000130264792392676\\
19.69375	0.000118362878151158\\
19.71375	0.000104021072395756\\
19.73375	8.79730347767488e-005\\
19.75375	0.000146489953464269\\
19.77375	0.000196730867171782\\
19.79375	0.000161937416520216\\
19.81375	0.000200711783123687\\
19.83375	0.000232443905685574\\
19.85375	0.000257163212971509\\
19.87375	0.000200456191070416\\
19.89375	0.000144224630901529\\
19.91375	9.0713419041609e-005\\
19.93375	0.000117203274041898\\
19.95375	6.52267588308302e-005\\
19.97375	1.68811938589031e-005\\
19.99375	-2.76441867857979e-005\\
20.01375	-6.7035808287525e-005\\
20.03375	-2.49951458624193e-005\\
20.05375	-6.04439658392117e-005\\
20.07375	-1.63403855948473e-005\\
20.09375	2.54912844954199e-005\\
20.11375	6.37580671291979e-005\\
20.13375	2.11337951287616e-005\\
20.15475	5.62116557933571e-005\\
20.17275	8.79513503692736e-005\\
20.19375	3.93835159652965e-005\\
20.21375	-8.33291841499223e-005\\
20.23375	-0.000120000468041072\\
20.25375	-0.000227226752780764\\
20.27375	-0.000322629198783447\\
20.29375	-0.000328890284601651\\
20.31375	-0.000405195856437667\\
20.33375	-0.000470592711917961\\
20.35375	-0.000525133889210382\\
20.37375	-0.000646401942280724\\
20.39375	-0.000676482908079096\\
20.41375	-0.000852676235797337\\
20.43375	-0.000858057353818762\\
20.45375	-0.00101150591854821\\
20.47375	-0.00122666115628967\\
20.49375	-0.00134319151603453\\
20.51375	-0.00144607076099854\\
20.53375	-0.00153433518198729\\
20.55375	-0.00161247594529854\\
20.57375	-0.00160442259899849\\
20.59375	-0.00167427985641082\\
20.61375	-0.00174123620518247\\
20.63375	-0.00172756614604814\\
20.65375	-0.00172518875918413\\
20.67375	-0.00180756638274606\\
20.69375	-0.0018191677935627\\
20.71375	-0.00184186573824869\\
20.73375	-0.001875020766939\\
20.75375	-0.00191972482283002\\
20.77375	-0.00190036579649353\\
20.79375	-0.00197265621132384\\
20.81375	-0.00197473831465509\\
20.83275	-0.00199200276388881\\
20.85375	-0.00194372794641979\\
20.87375	-0.00199069113067647\\
20.89375	-0.00197103936287993\\
20.91375	-0.00196747431062846\\
20.93375	-0.00197835355648575\\
20.95475	-0.00200320843659702\\
20.97375	-0.00196131255259917\\
20.99375	-0.00201586402669951\\
21.01375	-0.0019193095983295\\
21.03375	-0.00192329200130751\\
21.05375	-0.00194179124430357\\
21.07375	-0.00197201188628636\\
21.09375	-0.00193061820657329\\
21.11375	-0.00190431631448941\\
21.13375	-0.00197279783424206\\
21.15375	-0.00196552270525704\\
21.17375	-0.00204991585644372\\
21.19375	-0.00205634943588391\\
21.21375	-0.00206792224341993\\
21.23375	-0.00208282156528569\\
21.25375	-0.00218235189636223\\
21.27375	-0.00227725775155933\\
21.29375	-0.00228606317733597\\
21.31375	-0.00245850621294805\\
21.33375	-0.00245099313611578\\
21.35375	-0.0024439185768371\\
21.37375	-0.0025185091440333\\
21.39375	-0.00242065455774226\\
21.41375	-0.00241330322450381\\
21.43375	-0.00232442796748193\\
21.45375	-0.00224255819316425\\
21.47375	-0.00234034007818853\\
21.49275	-0.00235117295706589\\
21.51275	-0.00219602109834551\\
21.53375	-0.00222193398758112\\
21.55375	-0.00224351782400061\\
21.57375	-0.00227024674587243\\
21.59375	-0.00229459804372184\\
21.61375	-0.00231483918246905\\
21.63375	-0.0023311513743558\\
21.65375	-0.00242551986109079\\
21.67375	-0.00242313782624892\\
21.69375	-0.00249391766530865\\
21.71375	-0.002554546995565\\
21.73375	-0.00260374149068091\\
21.75475	-0.00255681856379553\\
21.77375	-0.00250530862942369\\
21.79375	-0.00262189876541721\\
21.81375	-0.00255406028946823\\
21.83375	-0.0025691005638896\\
21.85375	-0.00257826957199012\\
21.87375	-0.00258311715849224\\
21.89375	-0.00258276092422837\\
21.91375	-0.00266246956037684\\
21.93375	-0.00264576321395388\\
21.95375	-0.0027079353242281\\
21.97375	-0.00267483554671362\\
21.99375	-0.0027218109051138\\
22.01375	-0.00267475146329291\\
22.03375	-0.00271136230641902\\
22.05375	-0.00273941813957171\\
22.07375	-0.0027592621075401\\
22.09375	-0.00277154033073879\\
22.11375	-0.00277608218553397\\
22.13275	-0.00277255574126956\\
22.15375	-0.00284935529262826\\
22.17375	-0.00282723931247666\\
22.19375	-0.00288470004323277\\
22.21375	-0.00293047083709389\\
22.23375	-0.00287885419378239\\
22.25375	-0.00291072946760182\\
22.27375	-0.00284822955751797\\
22.29375	-0.00287233436204618\\
22.31375	-0.00280427741669985\\
22.33375	-0.00282477304760921\\
22.35375	-0.0027553975265739\\
22.37375	-0.00277586726842545\\
22.39375	-0.00270693358874179\\
22.41375	-0.00272973838043288\\
22.43375	-0.00275125004087521\\
22.45375	-0.00268178432741756\\
22.47375	-0.0026161353017029\\
22.49375	-0.00264352638881824\\
22.51375	-0.0025820917781554\\
22.53375	-0.00261292546542493\\
22.55475	-0.00264189350250834\\
22.57375	-0.00266792212225169\\
22.59375	-0.00268941458559511\\
22.61375	-0.00261995565558869\\
22.63375	-0.00272610698014335\\
22.65375	-0.0027326150388004\\
22.67375	-0.00273390690722628\\
22.69375	-0.00273048773161369\\
22.71375	-0.00272456660233906\\
22.73375	-0.00262879526416457\\
22.75275	-0.00263038135032359\\
22.77375	-0.00263067558528947\\
22.79375	-0.00263166716210099\\
22.81375	-0.00263445677416206\\
22.83375	-0.00254953289124413\\
22.85375	-0.00256172029151297\\
22.87375	-0.00257567456154296\\
22.89375	-0.00250262839505231\\
22.91375	-0.00243555340130228\\
22.93275	-0.00246430074044818\\
22.95375	-0.00240433493714289\\
22.97375	-0.00243999370270037\\
22.99375	-0.00247493428901987\\
23.01375	-0.00241895986666476\\
23.03375	-0.00245683754213315\\
23.05375	-0.00249263369229754\\
23.07375	-0.00252427957902123\\
23.09375	-0.00255173386353307\\
23.11375	-0.00257481844804245\\
23.13375	-0.00259781977239095\\
23.15375	-0.00252519748157224\\
23.17375	-0.00245881101579811\\
23.19375	-0.00248910502552008\\
23.21375	-0.00242859332631884\\
23.23375	-0.00237578449754161\\
23.25375	-0.00241893505145775\\
23.27375	-0.00237483735541882\\
23.29375	-0.00242736352038352\\
23.31375	-0.00248064664822007\\
23.33175	-0.00243958755467481\\
23.35475	-0.00240421371701911\\
23.37375	-0.00237411044862541\\
23.39375	-0.00243671945528162\\
23.41375	-0.00231821745796042\\
23.43375	-0.00239030539652354\\
23.45375	-0.00236990753556645\\
23.47375	-0.00235240766388942\\
23.49375	-0.00242811435079376\\
23.51375	-0.00232069749380717\\
23.53375	-0.00231234184865062\\
23.55375	-0.00230867039461486\\
23.57375	-0.00230892792052496\\
23.59375	-0.00222350771164694\\
23.61375	-0.00223741761477939\\
23.63375	-0.00225563560042818\\
23.65375	-0.00227683690861526\\
23.67375	-0.00229995699070651\\
23.69375	-0.00232424424250684\\
23.71375	-0.00234892368156437\\
23.73375	-0.00237303785610616\\
23.75375	-0.00239624948087547\\
23.77375	-0.00232918937615965\\
23.79375	-0.0023556304187233\\
23.81375	-0.00238141266172456\\
23.83375	-0.00231722098242184\\
23.85375	-0.00234682342767296\\
23.87375	-0.00237575194543084\\
23.89375	-0.00231462264750459\\
23.91375	-0.00234674950410222\\
23.93275	-0.00220311107326323\\
23.95375	-0.00215793483510624\\
23.97375	-0.00194741619689087\\
23.99375	-0.00175814706065831\\
24.01375	-0.00150250903850007\\
24.03375	-0.00136272442828178\\
24.05375	-0.00124741340213869\\
24.07375	-0.00106424910706376\\
24.09375	-0.000990150790575757\\
24.11375	-0.000845410722385304\\
24.13375	-0.000716126329308361\\
24.15475	-0.000606496641840118\\
24.17375	-0.000590276018151051\\
24.19375	-0.000569127606657594\\
24.21375	-0.000557360074255474\\
24.23375	-0.000622469082517049\\
24.25375	-0.000585441665642597\\
24.27375	-0.000617392333059451\\
24.29375	-0.00063047340756718\\
24.31375	-0.000631989668725457\\
24.33375	-0.000693666055728772\\
24.35375	-0.000644890826270148\\
24.37375	-0.000666661250714953\\
24.39375	-0.000668453610029677\\
24.41375	-0.000652142080404828\\
24.43375	-0.000703065126494753\\
24.45375	-0.000736199402611948\\
24.47375	-0.000750272586901706\\
24.49375	-0.000747459287342864\\
24.51375	-0.000728922677998657\\
24.53275	-0.00069709336477288\\
24.55375	-0.000735373153440438\\
24.57375	-0.000675770958790599\\
24.59375	-0.000689652770473316\\
24.61375	-0.000609394162452138\\
24.63375	-0.000605185429603344\\
24.65375	-0.000590486015653252\\
24.67375	-0.000565676219286439\\
24.69375	-0.000530977985302664\\
24.71375	-0.00048921010084467\\
24.73375	-0.000521268494307626\\
24.75375	-0.000542909691721667\\
24.77375	-0.000474574145685203\\
24.79375	-0.000482641953403361\\
24.81375	-0.000565569263243413\\
24.83375	-0.000473041851352626\\
24.85375	-0.000459565113801803\\
24.87375	-0.000440825522471358\\
24.89375	-0.000418092294481824\\
24.91375	-0.00039388947154346\\
24.93375	-0.000448987452389538\\
24.95475	-0.000338661774572192\\
24.97375	-0.000312456843048533\\
24.99375	-0.000286890411644974\\
25.01375	-0.000107103349149291\\
25.03375	-0.000104384378836247\\
25.05375	-0.000101636250839165\\
25.07375	-0.000102149128411082\\
25.09375	-0.000104400388246354\\
25.11375	-0.000113052939925771\\
25.13375	-4.08331691004107e-005\\
25.15375	2.72435120992144e-005\\
25.17375	8.88114706209962e-005\\
25.19375	5.92151852296954e-005\\
25.21375	0.000111979481574983\\
25.23375	8.17541023442627e-005\\
25.25275	4.99812340240348e-005\\
25.27375	9.71301605627778e-005\\
25.29375	6.68635321657362e-005\\
25.31375	3.86207333152513e-005\\
25.33375	7.77893399821219e-006\\
25.35375	-9.42410011505055e-005\\
25.37375	-0.000109032975457789\\
25.39375	-4.62716480634187e-005\\
25.41375	-0.000144338164058734\\
25.43375	-7.62732389128674e-005\\
25.45375	-1.70990455003279e-005\\
25.47375	-4.27880399263117e-005\\
25.49375	8.31984228906131e-005\\
25.51375	0.000110671793847961\\
25.53375	0.000128401305937809\\
25.55375	0.000211997351431151\\
25.57375	0.000277088460863924\\
25.59375	0.000323582822743349\\
25.61375	0.000352592821469353\\
25.63375	0.000366188366984692\\
25.65375	0.0004454748670621\\
25.67375	0.000428492256036519\\
25.69375	0.000480954430830054\\
25.71375	0.000519869320935769\\
25.73375	0.00046803249182517\\
25.75475	0.000491535479582812\\
25.77375	0.000506788077613347\\
25.79375	0.000436211927373805\\
25.81375	0.000445361751386333\\
25.83375	0.000450976483341356\\
25.85375	0.00037453255460056\\
25.87375	0.000380225671535893\\
25.89375	0.000305777369749748\\
25.91375	0.000157332471156176\\
25.93275	9.91507210783348e-005\\
25.95375	0.000125723211162489\\
25.97375	7.37304282153718e-005\\
};
\addlegendentry{$\kappa{}_\text{do}$};

\end{axis}

\begin{axis}[%
width=0.95092\figurewidth,
height=0.264706\figureheight,
at={(0\figurewidth,0.735294\figureheight)},
scale only axis,
every outer x axis line/.append style={black},
every x tick label/.append style={font=\color{black}},
xmin=18,
xmax=26,
xmajorgrids,
every outer y axis line/.append style={black},
every y tick label/.append style={font=\color{black}},
ymin=-50,
ymax=1500,
ylabel={St\"orung [Nm]},ymajorgrids,
ylabel near ticks,
axis x line*=bottom,
axis y line*=left
]
\addplot [color=black,solid,forget plot, line width=1.0]
  table[row sep=crcr]{%
17.97375	0\\
17.99375	0\\
19.81375	0\\
19.83375	0\\
19.85375	0\\
19.8538 	1000\\
19.89375	1000\\
23.73375	1000\\
23.75375	1000\\
23.75475	0\\
23.79375	0\\
25.97375	0\\
};

\end{axis}
\end{tikzpicture}%	
			\caption{Unterdrückung von Seitenkraftstörungen}
			\label{abb_stoerunterdrueckung_sgs}
		\end{minipage}
\end{figure}
Das Fahrzeug fährt entlang einer geraden Referenz mit einer Geschwindigkeit von 50 km/h. Der Fahrer greift dabei in die Querführung ein (gut sichtbar am Fahrerhandmoment $\tau_{\delta,\mathrm{drv}}$) und lenkt das Fahrzeug einige Meter neben die geplante Trajektorie.  Gewöhnliche Folgeregler mit integrierendem Verhalten würden in diesem Fall stetig die Stellgröße erhöhen und gegen den Fahrer arbeiten.  Durch die Verwendung der präsentierten Regelungsstruktur mit Störgrößenbeobachter lässt sich dagegen kooperatives Fahrerverhalten, wie es beispielsweise für Spurhalteassistenten erforderlich ist, darstellen.  Der Ausgang des Störgrößenbeobachters $\kappa_\mathrm{do}$ ist über die ganze Dauer des Fahrereingriffs nahezu 0 1/m da keine Störung außer dem Fahrer auf das Fahrzeug wirkt.  Lediglich das Stellgesetz der Trajektorienfolgeregelung arbeitet gegen den Fahrer. Dies ist wie bereits zuvor erwähnt gewünscht, um dem Fahrer Rückmeldung über das Ziel des umgesetzten Fahrerassistenzsystems zu geben. Nach dem Handmomentenabfall führt der Trajektoriefolgeregler das Fahrzeug zurück zur Referenz.  

Störungen die von außen auf das Fahrzeug wirken (wie z.B. Seitenwinde oder hängende Fahrbahnen) werden im Gegensatz zu den Fahrereingriffen robust unterdrückt. Diese wird in
Abb.~\ref{abb_stoerunterdrueckung_sgs} demonstriert.  Wieder hat das Fahrzeug das Ziel, der geraden Fahrspur mit einer Geschwindigkeit von 50 km/h zu folgen. Zur Nachstellung von reproduzierbaren Seitenkraftstörungen werden einseitige Bremseneingriffe eingesetzt. Dazu werden die rechten Räder der Vorder- und Hinterachse mit jeweils 500 Nm Bremsmoment beaufschlagt. Dies führt zu einer Seitenkraftstörung. Der Störgrößenbeobachter kompensiert die Störung, ohne dass sich ein nennenswerter Querablagefehler $\Delta d$ aufbaut.

\FloatBarrier
