\section{Introduction}\label{S:Intro}

Advanced collision avoidance systems, lane keeping support, traffic jam assistance, and remote valet parking; they all operate on the actuators to relieve the drivers of the lateral and/or longitudinal vehicle control or make it safer for them. Well-defined tasks, such as staying in the middle of the marked lane while following the vehicle ahead, can be handled by a set-point controller (Gayko 2012). And yet standard automated parking maneuvers already require a calculated trajectory\footnote{More precisely: a path, which does not have any time dependency} that must be adapted to the available parking space (Katzwinkel et al. 2012). As systems need to cover more and more situations, the number of degrees of freedom increase, which makes a trajectory parameterization very complex, especially when the vehicle must take numerous obstacles into account. This calls for a systematic approach based on mathematical optimization (as opposed to heuristic approaches such as potential field and elastic bands methods, see e.g. (Krogh 1984), (Brand 2008), with their inherent limitations, cf. (Koren and Borenstein 1991).
In the chapter at hand, we address real-time trajectory optimization\footnote{Notice, that in control theory the term trajectory planning usually implies that there is no feedback of the actual system states on the trajectory. The dynamical system is then only stabilized by a downstream trajectory tracking controller, which is not always advisable. We therefore use the term trajectory optimization instead to be independent of the utilized stabilization concept.}, a task that an automated vehicle faces when it travels through its environment, also referred to as motion planning in robotics (Latombe 1990; LaValle 2006). The focus will be on methods that engage with the longitudinal and lateral vehicle movement. However, the results can be transferred to novel warning systems that can also
benefit from an optimal trajectory prediction, see e.g. (Eichhorn et al. 2013). 
Speaking most generally, a trajectory optimization method is sought that can handle both structured (e.g. streets) and unstructured environments (parking lots), one that works amongst cluttered static obstacles and in moving traffic as well as exhibits a natural, human-like, anticipatory driving behavior.
Using more technical terms, the method should be easy to implement, parameterize, adapt, scale well with the number of vehicle states and the length of the optimization horizon, incorporate nonlinear, high fidelity vehicle models, combine the lateral and longitudinal motion, be complete \footnote{A complete algorithm always finds the solution if it exists.}, allow for both grid maps and object lists representations (with predicted future poses) of the obstacles, be numerically stable, and transparent in its convergence behavior (if applicable). Also, the calculation effort must be low to allow for short optimization cycles on (low performance) electronic control units so that the vehicle can quickly react to sudden changes in the environment.
Unfortunately, there is no such single method that has all these properties. And, most likely, there will never be one. However, different optimization methods can be combined to get as close as possible to the above requirements. The next section therefore gives a closer look into the basic principles of trajectory optimization and their application.


