\abstract{Neue vernetzte Fahrerassistenzfunktionen mit simultaner Längs- und Querführung  bei gleichzeitig  unterschiedlicher Einbindung des Fahrers in die Fahraufgabe stellen eine große Herausforderung dar. Eine aus einer überlagerten Fahrstrategie vorgegebene Trajektorie für die Fahrzeugbewegung mit dem Ziel der Erfüllung einer bestimmten Längs- und Querführungsaufgabe soll hierbei in Aktuatorstellgrößen für das Lenk-, Antriebs- und Bremssystem überführt werden. Wegen des zunehmend größeren Betriebsbereichs hinsichtlich Fahrgeschwindigkeit und Kraftschlussausnutzung in Längs- und Querrichtung sowie des zunehmenden Autonomiegrades ist ein integrierter Ansatz vorteilhaft der alle fahraktiven Fahrerassistenzfunktionen, also Parkier-, Komfort und Fahrsicherheitsfunktionen betrachtet. Der Ansatz soll geeignet sein, sowohl Einzelfunktionen der Quer- oder Längsführung als auch Funktionen mit kombinierter Längs-und Querführung mit jeweils unterschiedlichem Autonomiegrad abzudecken. Eine wesentliche Herausforderung stellt hierbei die Gestaltung der Kooperation mit dem Fahrer dar, die sowohl autonomes als auch kooperatives Fahren und die Übergänge zum manuellen Fahren z.~B. zum Aktivieren oder Deaktivieren von Funktionen durch den Fahrer adressieren soll.\\ Dieses Kapitel beschreibt detailliert einen in der industriellen Praxis bewährten Ansatz zur integrierten Quer- und Längsführung. Der Ansatz basiert auf einer funktionsübergreifend einheitlichen kaskadierten hierarchischen Reglerstruktur bei der in Anlehnung an das Drei-Ebenen-Modell die  Aufteilung der Regelungsaufgabe auf die Bahn- und Fahrzeugführungsebene erfolgt. Auf Grundlage der Fahrstrategie und der gewählten Assistenzfunktion gibt eine Trajektorienplanung einen bestimmten Sollverlauf für die Fahrzeugbewegung in Längs- und Querrichtung vor. Die Einregelung des Sollverlaufs unter besonderer Berücksichtigung der Interaktion mit dem Fahrer und Überführung in Stellgrößen für die Lenk-, Brems- und Antriebsaktuatorik erfolgt auf Grundlage der Methoden der linearen, robusten bzw. adaptiven modellbasierten Regelungstheorie und der Robotik. Grundlage für den regelungstechnischen Entwurf sind hinsichtlich der Stuktur sehr gut bekannte mathematische Modelle die die ebene Längs-, Quer- und Gierbewegung des Fahrzeugs, der Lenkung und der Bewegung relativ zu einer vorgegebenen Trajektorie beschreiben. Wegen der beladungs- und reibwertabhängigen Ungenauigkeit der mathematischen Modelle werden diese um entsprechende Unsicherheitsmodelle erweitert.  Eine mathematische Beschreibung des Betriebsbereiches, der Anforderungen hinsichtlich Regelgüte, Stabilität und Robustheit des Führungs- und Störübertragungsverhaltens werden formuliert und für exemplarische Assistenzfunktionen anhand von Experimenten gezeigt.
}

%und die Integration verschiedener Einzelfunktionen der Längs- und/oder Querführung