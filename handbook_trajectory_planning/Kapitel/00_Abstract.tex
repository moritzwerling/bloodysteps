\abstract{New active driver assistance systems that work at the road- and navigation level as well as automated driving face a challenging task. They must permanently calculate the vehicle input commands (such as those for the steering, brakes, and the engine/powertrain) to realize a desired future vehicle movement, a driving trajectory. This trajectory has to be optimal in terms of some optimization criterion (in general a trade-off between comfort, safety, energy effort, and travelling time), needs to take the vehicular dynamics into account, and must incorporate lane boundaries or the predicted free space amidst (possibly moving) obstacles. This kind of optimization can be mathematically formulated as a so-called optimal control problem. To limit the calculation effort, the optimal control problem is usually solved only on a limited prediction interval (starting with the current time) leading to a receding horizon optimization.
The chapter illustrates this practically proven approach in detail. Furthermore, the three general principles of dynamic optimization known from control theory and robotics are presented, namely calculus of variations, direct optimization, and dynamic programming. Furthermore, their application to driver assistance systems and automated driving is exemplified and the high practical relevance is supported by the given literature. Finally, the respective advantages and limitations of the optimization principles are discussed in detail proposing their combination for more involved system designs.
}
